%% -------------------------------------------------------
%%  This file was automatically generated at 2024-04-30.
%%  Manual modifications to this file will be lost!
%% -------------------------------------------------------
\documentclass[a4paper,notitlepage,openany,10pt]{book}
\usepackage{makeidx}

\usepackage{longtable}
\usepackage{lscape}
\usepackage{nameref}
\usepackage{url}
\usepackage{tabularx}
%\usepackage[table]{xcolor}
\usepackage{placeins}
\usepackage{multirow}

%\usepackage{textcomp}
%\usepackage[]{inputenc}
\usepackage{marvosym}
\usepackage[utf8]{inputenc}
\usepackage[T1]{fontenc}
\usepackage[swedish]{babel}
\usepackage[swedish]{varioref}
\usepackage{hyperref}

\usepackage[useregional]{datetime2}
%%\renewcommand{\dateseparator}{-}

%% \r{a} \"{a}  \"{o}
\usepackage[framemethod=TikZ]{mdframed}
\usepackage[figuresright]{rotating}  % PDF version
%\usepackage{rotating} % Printing version

%-------------------------------------------------------------------
% Inscribed character
%-------------------------------------------------------------------
\newcommand*\circled[1]{\tikz[baseline=(char.base)]{
	\node[shape=circle,draw,inner sep=1pt] (char) {\fontencoding{T1}\fontfamily{phv}\fontsize{7}{10}\selectfont #1};}}

%-------------------------------------------------------------------
% Horizontal figure
%-------------------------------------------------------------------
\newenvironment{lyingfigure}
{
    \begin{sidewaysfigure}[!htbp]
}
{
    \end{sidewaysfigure}
}

%-------------------------------------------------------------------
% Notabene box
%-------------------------------------------------------------------
\newenvironment{notabene}[1][]
{
    \ifstrempty{#1}%
    {\mdfsetup{%
    frametitle={%
    \tikz[baseline=(current bounding box.east),outer sep=0pt]
    \node[anchor=east,rectangle,fill=red!20]
    {\strut Notera};}}
    }%
    {\mdfsetup{%
    frametitle={%
    \tikz[baseline=(current bounding box.east),outer sep=0pt]
    \node[anchor=east,rectangle,fill=red!20]
    {\strut Notera:~#1};}}%
    }%
    \mdfsetup{innertopmargin=10pt,linecolor=red!20,%
    linewidth=2pt,topline=true,%
    frametitleaboveskip=\dimexpr-\ht\strutbox\relax
    }
    \begin{mdframed}[]\relax%
}
{
    \end{mdframed}
}

%-------------------------------------------------------------------

\date{\today}
\title{Socialförsäkringsbalk (2010:110)}
%\author{}
\date{2024-04-30}

%\makeindex

\begin{document}

%\rhead{}
\maketitle

\vfill

\newpage
%\renewcommand\contentsname{Inneh{\aa}llsf{\"o}rteckning}
%\tableofcontents
\newpage
\part*{A ÖVERGRIPANDE BESTÄMMELSER}
\chapter*{1 Innehåll m.m.}
\subsection*{1 §}
\paragraph*{}
Denna balk innehåller bestämmelser om social trygghet genom de sociala försäkringar samt andra ersättnings- och bidragssystem som behandlas i avdelningarna B-G (socialförsäkringen).
\subsection*{2 §}
\paragraph*{}
Balken är indelad i avdelningar, som betecknas med stora bokstäver.
\paragraph*{}
Avdelningarna är indelade i underavdelningar, som betecknas med romerska siffror.
\subsection*{3 §}
\paragraph*{}
Övergripande bestämmelser finns i 1-7 kap. (avdelning A).
\paragraph*{}
Vidare finns bestämmelser om
\newline - familjeförmåner i 8-22 kap. (avdelning B),
\newline - förmåner vid sjukdom eller arbetsskada i 23-47 kap. (avdelning C),
\newline - särskilda förmåner vid funktionshinder i 48-52 kap. (avdelning D),
\newline - förmåner vid ålderdom i 53-74 a kap. (avdelning E),
\newline - förmåner till efterlevande i 75-92 kap. (avdelning F), och
\newline - bostadsstöd i 93-103 e kap. (avdelning G).
\paragraph*{}
Vissa gemensamma bestämmelser om förmånerna, handläggningen och organisationen finns i 104-117 kap. (avdelning H).
Lag (2020:1239).
\subsection*{4 §}
\paragraph*{}
I avdelning A finns
\newline - allmänna bestämmelser, definitioner och förklaringar i 2 kap., och
\newline - övergripande bestämmelser om socialförsäkringsskyddet i 3-7 kap.
\paragraph*{}
Det finns definitioner och förklaringar också i andra avdelningar.
\chapter*{2 Allmänna bestämmelser, definitioner och förklaringar}
\subsection*{1 §}
\paragraph*{}
I detta kapitel finns bestämmelser om
\newline - socialförsäkringens administration, m.m. i 2 §,
\newline - socialförsäkringens finansiering, m.m. i 3 och 4 §§,
\newline - internationella förhållanden i 5 §,
\newline - prisbasbelopp m.m. i 6-10 §§,
\newline - riktålder för pension i 10 a-10 d §§, och
\newline - definitioner och förklaringar i 11-17 §§.
Lag (2019:649).
\subsection*{2 §}
\paragraph*{}
Socialförsäkringen administreras av Försäkringskassan, Pensionsmyndigheten, Fondtorgsnämnden och Skatteverket.
\paragraph*{}
Hos Försäkringskassan ska det finnas ett allmänt ombud för socialförsäkringen. Regeringen utser det allmänna ombudet.
\paragraph*{}
I inledningen av avdelningarna B-G finns bestämmelser om vilken myndighet som handlägger ärenden om förmåner som avses i den avdelningen.
Lag (2022:761).
\subsection*{3 §}
\paragraph*{}
Bestämmelser om finansiering av socialförsäkringen finns i
\newline - socialavgiftslagen (2000:980),
\newline - lagen (1994:1744) om allmän pensionsavgift,
\newline - lagen (1998:676) om statlig ålderspensionsavgift, och
\newline - lagen (2000:981) om fördelning av socialavgifter.
\paragraph*{}
Socialförsäkringen finansieras också genom avkastning på vissa av de avgifter som anges i första stycket.
\subsection*{4 §}
\paragraph*{}
Förutom enligt de lagar som anges i 3 § finansieras socialförsäkringen med medel från statsbudgeten samt, i den omfattning som anges i denna balk, genom särskilda betalningar från kommuner och enskilda.
\subsection*{5 §}
\paragraph*{}
Unionsrätten inom Europeiska unionen (EU) eller inom Europeiska ekonomiska samarbetsområdet (EES) eller avtal om social trygghet eller andra avtal som ingåtts med andra stater kan medföra begränsningar i tillämpligheten av bestämmelserna i denna balk.
Lag (2010:1312).
\subsection*{5 a §}
\paragraph*{}
Regeringen eller den myndighet som regeringen bestämmer kan med stöd av 8 kap. 7 § regeringsformen meddela föreskrifter om undantag från bestämmelserna om tillfällig föräldrapenning i 13 kap., sjukpenning och karens i 27 och 28 kap., smittbärarpenning i 46 kap. och handläggning av ärenden i 110 kap. Sådana föreskrifter kan endast meddelas vid extraordinära händelser i fredstid.
Lag (2020:189).
\subsection*{6 §}
\paragraph*{}
Vissa beräkningar som anges i denna balk ska grundas på ett prisbasbelopp eller förhöjt prisbasbelopp som beräknas för varje år.
\subsection*{7 §}
\paragraph*{}
Prisbasbeloppet räknas fram genom att bastalet 36 396 multipliceras med det jämförelsetal som anger förhållandet mellan det allmänna prisläget i juni året före det som prisbasbeloppet avser och prisläget i juni 1997. Det framräknade prisbasbeloppet avrundas till närmaste hundratal kronor.
\subsection*{8 §}
\paragraph*{}
Det förhöjda prisbasbeloppet räknas fram och avrundas på samma sätt som prisbasbeloppet. Vid framräkning av det förhöjda prisbasbeloppet används dock i stället bastalet 37 144.
\subsection*{9 §}
\paragraph*{}
Regeringen eller den myndighet som regeringen bestämmer meddelar närmare föreskrifter om prisbasbeloppet och det förhöjda prisbasbeloppet.
\subsection*{10 §}
\paragraph*{}
I 58 kap. finns bestämmelser om inkomstbasbelopp, inkomstindex, balanstal och balansindex.
\subsection*{10 a §}
\paragraph*{}
Vissa förmåner och beräkningar som anges i denna balk ska knytas till en särskild ålder (riktålder för pension) som beräknas för varje år enligt bestämmelserna i 10 b §.
Lag (2019:649).
\subsection*{10 b §}
\paragraph*{}
Riktålder för pension räknas fram genom att det till bastalet 65 läggs 2/3 av differensen mellan
\newline - den vid 65 års ålder återstående medellivslängden för befolkningen i Sverige under femårsperioden närmast före det år riktåldern beräknas, och
\newline - motsvarande värde för jämförelseåret 1994.
\paragraph*{}
Den framräknade riktåldern för pension ska avrundas till närmaste helår.
Lag (2019:649).
\subsection*{10 c §}
\paragraph*{}
Den beräknade riktåldern för pension ska gälla för det sjätte året efter beräkningsåret.
\paragraph*{}
När en gällande riktålder för pension ändras ska en åldersgräns som är knuten till den riktåldern som gällde före ändringen fortsätta att gälla för en försäkrad som före ändringen har uppnått en sådan åldersgräns.
Lag (2019:649).
\subsection*{10 d §}
\paragraph*{}
Regeringen eller den myndighet som regeringen bestämmer kan med stöd av 8 kap. 7 § regeringsformen meddela närmare föreskrifter om riktåldern för pension.
Lag (2019:649).
\subsection*{11 §}
\paragraph*{}
Med förmåner avses i denna balk dagersättningar, pensioner, livräntor, kostnadsersättningar, bidrag samt andra utbetalningar eller åtgärder som den enskilde är försäkrad för enligt 4-7 kap.
\subsection*{12 §}
\paragraph*{}
Det som i denna balk föreskrivs om den som är ogift tillämpas även i fråga om den som är änka, änkling eller frånskild, om inget annat anges.
\subsection*{13 §}
\paragraph*{}
Vad som avses med sambor framgår av 1 § sambolagen (2003:376).
\subsection*{14 §}
\paragraph*{}
Om föräldraskap till barn och rättsverkan av adoption finns bestämmelser i 1 och 4 kap. föräldrabalken.
\subsection*{15 §}
\paragraph*{}
Med blivande adoptivförälder avses i denna balk den som efter socialnämndens medgivande har tagit emot ett barn för stadigvarande vård och fostran i syfte att adoptera barnet.
Lag (2018:1290).
\subsection*{16 §}
\paragraph*{}
Med familjehemsförälder avses i denna balk den som har tagit emot ett barn för stadigvarande vård och fostran i ett enskilt hem som inte tillhör någon av barnets föräldrar eller någon annan som har vårdnaden om barnet.
\subsection*{17 §}
\paragraph*{}
Med avgiftspliktig ersättning och avgiftspliktig inkomst avses detsamma som i 2 kap. 10-11 §§ respektive 3 kap. 3-8 §§ socialavgiftslagen (2000:980).
Lag (2012:834).
\chapter*{3 Innehåll}
\subsection*{1 §}
\paragraph*{}
I denna underavdelning finns allmänna bestämmelser om socialförsäkringsskyddet i 4 kap.
\paragraph*{}
Vidare finns bestämmelser om
\newline - bosättningsbaserade förmåner i 5 kap.,
\newline - arbetsbaserade förmåner i 6 kap., och
\newline - övriga förmåner i 7 kap.
\chapter*{4 Allmänna bestämmelser om försäkringsskyddet}
\subsection*{1 §}
\paragraph*{}
I detta kapitel finns bestämmelser om
\newline - försäkringsgrenar i 2 §,
\newline - försäkrad och gällande skydd i 3 och 4 §§, och
\newline - internationella förhållanden i 5 §.
\subsection*{2 §}
\paragraph*{}
Socialförsäkringen är indelad i tre försäkringsgrenar.
Dessa avser
\newline 1. förmåner som grundas på bosättning i Sverige (bosättningsbaserade förmåner),
\newline 2. förmåner som grundas på arbete i Sverige (arbetsbaserade förmåner), och
\newline 3. förmåner som grundas på andra omständigheter än bosättning eller arbete i Sverige (övriga förmåner).
\subsection*{3 §}
\paragraph*{}
Försäkrad är den som uppfyller de krav i fråga om bosättning, arbete eller andra omständigheter som avses i 2 § samt gällande krav på försäkringstider.
\paragraph*{}
För att omfattas av socialförsäkringsskyddet ska den försäkrade dessutom uppfylla de andra villkor som gäller för respektive förmån enligt 5-7 kap.
\subsection*{4 §}
\paragraph*{}
Ytterligare bestämmelser om rätten till förmåner finns i 8-117 kap. (avdelningarna B-H).
\subsection*{5 §}
\paragraph*{}
Den som, enligt vad som följer av Europaparlamentets och rådets förordning (EG) nr 883/2004 av den 29 april 2004 om samordning av de sociala trygghetssystemen, omfattas av lagstiftningen i en annan stat är inte försäkrad för sådana förmåner enligt denna balk som motsvarar förmåner som avses i förordningen.
Lag (2010:1312).
\subsection*{5 a §}
\paragraph*{}
En person som är lokalt anställd vid en svensk utlandsmyndighet är inte försäkrad enligt denna balk när det gäller den anställningen. En sådan person ska inte anses som offentligt anställd vid tillämpning av förordning (EG) nr 883/2004 om samordning av de sociala trygghetssystemen.
Lag (2013:747).
\chapter*{5 Bosättningsbaserade förmåner}
\subsection*{1 §}
\paragraph*{}
I detta kapitel finns bestämmelser om
\newline - bosättning i Sverige i 2 och 3 §§,
\newline - särskilda personkategorier i 4-8 §§,
\newline - de bosättningsbaserade förmånerna i 9 och 10 §§,
\newline - socialförsäkringsskyddet i samband med inflyttning till Sverige i 11 och 12 §§,
\newline - förmåner vid utlandsvistelse i 13-16 §§, och
\newline - speciella försäkringssituationer i 17 och 18 §§.
Lag (2019:646).
\subsection*{2 §}
\paragraph*{}
Vid tillämpning av bestämmelserna i denna balk ska, om inget annat särskilt anges, en person anses vara bosatt i Sverige om han eller hon har sitt egentliga hemvist här i landet.
\subsection*{3 §}
\paragraph*{}
Den som kommer till Sverige och kan antas komma att vistas här under längre tid än ett år ska anses vara bosatt här i landet. Detta gäller dock inte om synnerliga skäl talar mot det. En utlänning som enligt 4 § andra stycket folkbokföringslagen (1991:481) inte ska folkbokföras ska inte heller anses vara bosatt här.
\paragraph*{}
En i Sverige bosatt person som lämnar landet ska fortfarande anses vara bosatt här i landet om utlandsvistelsen kan antas vara längst ett år.
\subsection*{4 §}
\paragraph*{}
Den som av en statlig arbetsgivare sänds till ett annat land för arbete för arbetsgivarens räkning ska anses vara bosatt i Sverige under hela utsändningstiden om han eller hon tidigare någon gång varit bosatt här i landet.
\paragraph*{}
När en annan stat som arbetsgivare sänder en person till Sverige för arbete för arbetsgivarens räkning ska den personen inte anses vara bosatt här i landet.
\subsection*{5 §}
\paragraph*{}
En person som tillhör en annan stats beskickning eller karriärkonsulat eller dess betjäning ska anses vara bosatt i Sverige endast om det är förenligt med bestämmelserna om immunitet och privilegier i de konventioner som anges i 2 och 3 §§ lagen (1976:661) om immunitet och privilegier i vissa fall. Detta gäller även en sådan persons privattjänare.
\paragraph*{}
En person som på grund av anknytning till en internationell organisation omfattas av bestämmelserna i 4 § lagen om immunitet och privilegier i vissa fall ska anses vara bosatt i Sverige endast i den mån det är förenligt med vad som följer av tillämplig stadga eller avtal som anges i bilagan till den lagen.
\subsection*{6 §}
\paragraph*{}
En i Sverige bosatt person som lämnar landet för arbete för arbetsgivarens räkning ska fortfarande anses vara bosatt här, om han eller hon är anställd av
\newline 1. en svensk ideell organisation som bedriver biståndsverksamhet, eller
\newline 2. ett svenskt trossamfund eller ett organ som är knutet till ett sådant samfund.
\paragraph*{}
Första stycket gäller endast om utlandsvistelsen kan antas vara längst fem år.
\subsection*{7 §}
\paragraph*{}
En i Sverige bosatt person som lämnar landet för att studera i ett annat land ska fortfarande anses vara bosatt här så länge han eller hon genomgår en studiestödsberättigande utbildning.
\paragraph*{}
Den som kommer till Sverige för att studera ska inte anses vara bosatt här.
Lag (2017:277).
\subsection*{8 §}
\paragraph*{}
Det som föreskrivs om personer som avses i 4-7 §§ gäller även medföljande make samt barn som inte fyllt 18 år. Med make likställs den som utan att vara gift med den utsände lever tillsammans med denne, om de tidigare har varit gifta eller gemensamt har eller har haft barn.
\subsection*{9 §}
\paragraph*{}
Den som är bosatt i Sverige är försäkrad för följande förmåner:
\paragraph*{}
Avdelning B Familjeförmåner
\newline 1. föräldrapenning på lägstanivå
och grundnivå, (11 och 12 kap.)
\newline 2. barnbidrag, (15 och 16 kap.)
\newline 3. underhållsstöd, (17-19 kap.)
\newline 4. adoptionsbidrag, (21 kap.)
\newline 5. omvårdnadsbidrag (22 kap.)
\paragraph*{}
Avdelning C Förmåner vid sjukdom eller arbetsskada
\newline 6. sjukpenning i särskilda fall, (28 a kap.)
\newline 7. rehabilitering, bidrag till arbetshjälpmedel, särskilt bidrag och rehabiliteringspenning i
särskilda fall, (29-31 a kap.)
\newline 8. sjukersättning och aktivitetsersättning i form av
garantiersättning, (33 och 35-37 kap.)
\paragraph*{}
Avdelning D Särskilda förmåner vid funktionshinder
\newline 9. merkostnadsersättning, (50 kap.)
\newline 10. assistansersättning, (51 kap.)
\newline 11. bilstöd, (52 kap.)
\paragraph*{}
Avdelning E Förmåner vid ålderdom
\newline 12. garantipension, (55, 56, 65-67 och
69-71 kap.)
\newline 13. särskilt pensionstillägg, (73 kap.)
\newline 14. äldreförsörjningsstöd, (74 kap.)
\paragraph*{}
Avdelning F Förmåner till efterlevande
\newline 15. efterlevandestöd, (77, 79 och 85 kap.)
\newline 16. garantipension till
omställningspension, (77, 81 och 85 kap.)
\paragraph*{}
Avdelning G Bostadsstöd
\newline 17. bostadsbidrag, (95-98 kap.)
\newline 18. bostadstillägg, och (100-103 kap.)
\newline 19. boendetillägg. (103 a-103 e kap.)
Lag (2018:1265).
\subsection*{10 §}
\paragraph*{}
När det gäller förmåner till efterlevande enligt 9 § 15 och 16 är det den efterlevande som ska vara försäkrad för förmånerna.
Lag (2011:1513).
\subsection*{11 §}
\paragraph*{}
Till den som bosätter sig i Sverige men inte är folkbokförd här får bosättningsbaserade förmåner inte lämnas för längre tid tillbaka än tre månader före den månad när anmälan om bosättningen gjordes till Försäkringskassan eller när Försäkringskassan på annat sätt fick kännedom om bosättningen.
\paragraph*{}
Bestämmelser om anmälan om bosättning i Sverige finns i 110 kap. 43 §.
\subsection*{12 §}
\paragraph*{}
Till den som enligt utlänningslagen (2005:716) behöver ha uppehållstillstånd i Sverige får bosättningsbaserade förmåner lämnas tidigast från och med den dag då ett sådant tillstånd börjar gälla men inte för längre tid tillbaka än tre månader före det att tillståndet beviljades. Om det finns synnerliga skäl, får förmåner lämnas även om uppehållstillstånd inte har beviljats.
\paragraph*{}
Till den som har beviljats ett tidsbegränsat uppehållstillstånd får bosättningsbaserade förmåner lämnas utan hinder av att tillståndet har upphört att gälla, om en ansökan om fortsatt uppehållstillstånd har kommit in till Migrationsverket innan det tidigare tillståndet har upphört att gälla och ansökan avser ett fortsatt tillstånd på samma grund eller på en sådan annan grund som enligt 5 kap. 18 § andra stycket 8 b eller 5 b kap. 15 § andra stycket 3 utlänningslagen kan beviljas efter inresan eller ett nytt tillstånd med stöd av någon bestämmelse i lagen (2017:353) om uppehållstillstånd för studerande på gymnasial nivå. Om ansökan avslås får förmåner lämnas till dess att utlänningens tidsfrist för frivillig avresa enligt 8 kap. 21 § första stycket utlänningslagen har löpt ut. Om avslagsbeslutet inte innehåller någon tidsfrist för frivillig avresa får förmåner lämnas till dess att beslutet har fått laga kraft.
\paragraph*{}
Förmåner enligt första stycket lämnas inte för tid då bistånd enligt lagen (1994:137) om mottagande av asylsökande m.fl. har lämnats till den försäkrade, om förmånerna är av motsvarande karaktär.
Lag (2022:304).
\subsection*{13 §}
\paragraph*{}
För tid då en försäkrad inte vistas i ett land som ingår i Europeiska ekonomiska samarbetsområdet (EES) eller i Schweiz kan förmåner som grundas på bosättning lämnas endast i de fall som avses i 14 och 15 §§.
\subsection*{14 §}
\paragraph*{}
Förmåner får lämnas om utlandsvistelsen kan antas vara längst sex månader. Äldreförsörjningsstöd enligt 9 § 14 kan dock lämnas endast om utlandsvistelsen kan antas vara längst tre månader. Sjukpenning i särskilda fall enligt 9 § 6 och rehabiliteringspenning i särskilda fall enligt 9 § 7 får för tid då en försäkrad vistas utomlands lämnas endast om Försäkringskassan medger att den försäkrade reser till utlandet.
\paragraph*{}
Förmåner enligt 9 § 8, 12, 13, 15 och 16 kan lämnas så länge den försäkrades bosättning i Sverige består.
Lag (2011:1513).
\subsection*{15 §}
\paragraph*{}
Tidsbegränsningen och kravet på medgivande i 14 § första stycket gäller inte för sådana statsanställda och deras familjemedlemmar som avses i 4 och 8 §§.
\paragraph*{}
Tidsbegränsningen och kravet på medgivande gäller inte heller för familjeförmåner enligt 9 § 1-4 till biståndsarbetare m.fl.
och studerande eller deras familjemedlemmar som avses i 6-8 §§.
Lag (2011:1513).
\subsection*{16 §}
\paragraph*{}
I 110 kap. 45 § finns bestämmelser om när den försäkrade är skyldig att anmäla till Försäkringskassan att han eller hon lämnar Sverige.
\subsection*{17 §}
\paragraph*{}
Regeringen eller den myndighet som regeringen bestämmer meddelar föreskrifter om i vilken utsträckning barn som är politiska flyktingar är försäkrade för underhållsstöd enligt 9 § 3 även om de inte är bosatta i Sverige.
\subsection*{17 a §}
\paragraph*{}
Har upphävts genom
lag (2019:646).
\subsection*{18 §}
\paragraph*{}
Har rätten till en bosättningsbaserad förmån upphört med tillämpning av bestämmelserna om bosättning i 2-8 §§ eller bestämmelserna om utlandsvistelse i 13-15 §§, får förmånen efter ansökan hos den handläggande myndigheten fortsätta att lämnas om det med hänsyn till omständigheterna skulle framstå som uppenbart oskäligt att dra in förmånen.
Lag (2019:644).
\subsection*{18 a §}
\paragraph*{}
Har upphävts genom
lag (2019:646).
\chapter*{6 Arbetsbaserade förmåner}
\subsection*{1 §}
\paragraph*{}
I detta kapitel finns bestämmelser om
\newline - arbete i Sverige i 2 §,
\newline - särskilda personkategorier i 3-5 §§,
\newline - de arbetsbaserade förmånerna i 6 och 7 §§,
\newline - försäkringstid i 8-12 §§,
\newline - socialförsäkringsskyddets tidsmässiga omfattning i samband med arbete i Sverige i 13 och 14 §§,
\newline - förmåner vid utlandsvistelse i 15-18 §§, och
\newline - speciella försäkringssituationer i 19-22 §§.
\subsection*{2 §}
\paragraph*{}
Vid tillämpning av bestämmelserna i denna balk avses med arbete i Sverige, om inget annat särskilt anges, förvärvsarbete i verksamhet här i landet.
\paragraph*{}
Om en fysisk person som bedriver näringsverksamhet har sådant fast driftställe i Sverige som avses i 2 kap. 29 § inkomstskattelagen (1999:1229) ska verksamhet som hänför sig till driftstället anses bedriven här i landet.
\subsection*{3 §}
\paragraph*{}
/Upphör att gälla U:den dag regeringen bestämmer/
Arbete som sjöman på svenskt handelsfartyg ska anses som arbete i Sverige. Detta gäller även arbete som utförs
\newline 1. i anställning på ett utländskt handelsfartyg som en svensk redare hyr i huvudsak obemannat, om anställningen sker hos redaren eller hos någon arbetsgivare som redaren har anlitat, eller
\newline 2. i anställning hos ägaren till ett svenskt handelsfartyg eller hos någon arbetsgivare som anlitats av ägaren, om fartyget hyrs ut till en utländsk redare i huvudsak obemannat.
\paragraph*{}
Med sjöman avses den som enligt 3 § sjömanslagen (1973:282) ska anses som sjöman.
\subsection*{3 §}
\paragraph*{}
/Träder i kraft I:den dag regeringen bestämmer/
Arbete som sjöman på ett svenskt handelsfartyg ska anses som arbete i Sverige. Detta gäller även arbete som utförs
\newline 1. i anställning på ett utländskt handelsfartyg som en svensk redare hyr i huvudsak obemannat, om anställningen sker hos redaren eller hos någon arbetsgivare som redaren har anlitat, eller
\newline 2. i anställning hos ägaren till ett svenskt handelsfartyg eller hos någon arbetsgivare som anlitats av ägaren, om fartyget hyrs ut till en utländsk redare i huvudsak obemannat.
\paragraph*{}
Arbete som sjöman på ett handelsfartyg från tredjeland ska också anses som arbete i Sverige, om sjömannen är bosatt i Sverige och fartyget inte uteslutande går i sådan inre fart som avses i 3 § sjömanslagen (1973:282).
\paragraph*{}
Andra stycket gäller inte fiskefartyg eller traditionsfartyg.
\paragraph*{}
Med sjöman avses den som enligt 3 § sjömanslagen ska anses som sjöman.
Lag (2012:98).
\subsection*{4 §}
\paragraph*{}
Arbete utomlands för en arbetsgivare med verksamhet i Sverige ska anses som arbete här i landet, om arbetstagaren är utsänd av arbetsgivaren och arbetet kan antas vara längst ett år.
\paragraph*{}
När en utländsk arbetsgivare under motsvarande förhållande sänder någon till Sverige för arbete ska arbete i Sverige inte anses föreligga.
\paragraph*{}
I fall som anges i 5 kap. 4 § gäller första och andra styckena även om utsändningstiden kan antas vara längre än ett år.
\subsection*{5 §}
\paragraph*{}
Arbete som utförs av en person som tillhör en annan stats beskickning eller karriärkonsulat ska anses som arbete i Sverige endast om det är förenligt med bestämmelserna om immunitet och privilegier i de konventioner som anges i 2 och 3 §§ lagen (1976:661) om immunitet och privilegier i vissa fall. Detta gäller även en sådan persons privattjänare.
\paragraph*{}
Ett sådant arbete som medför att en person på grund av anknytning till en internationell organisation omfattas av bestämmelserna i 4 § lagen om immunitet och privilegier i vissa fall ska anses som arbete i Sverige endast i den mån det är förenligt med vad som följer av tillämplig stadga eller avtal som anges i bilagan till den lagen.
\subsection*{6 §}
\paragraph*{}
Den som arbetar i Sverige är försäkrad för följande förmåner:
\paragraph*{}
Avdelning B Familjeförmåner
\paragraph*{}
1. graviditetspenning, (10 kap.)
\newline 2. föräldrapenning på grundnivå eller sjukpenningnivå och
tillfällig föräldrapenning, (11-13 kap.)
\paragraph*{}
Avdelning C Förmåner vid sjukdom eller arbetsskada
\newline 3. sjukpenning, (24-28 kap.)
\newline 4. rehabilitering, rehabiliteringsersättning och
bidrag till arbetshjälpmedel, (29-31 kap.)
\newline 5. inkomstrelaterad sjukersättning och inkomstrelaterad
aktivitetsersättning, (33, 34, 36 och 37 kap.)
\newline 6. arbetsskadeersättning, (39-42 kap.)
\newline 7. närståendepenning, (47 kap.)
\paragraph*{}
Avdelning E Förmåner vid ålderdom
\newline 8. inkomstgrundad ålderspension, (55-64 och 69-71 kap.)
8 a. inkomstpensionstillägg, (74 a kap.)
\paragraph*{}
Avdelning F Förmåner till efterlevande
\newline 9. inkomstrelaterad
efterlevandepension, (77, 78, 80 och 82-85
kap.)
\newline 10. efterlevandeförmåner från
arbetsskadeförsäkringen, och (87 och 88 kap.)
\newline 11. efterlevandeskydd i form av
premiepension. (89-92 kap.)
Lag (2020:1239).
\subsection*{7 §}
\paragraph*{}
När det gäller pensionsförmåner och skadeersättning till efterlevande enligt 6 § 9-11 är det den avlidne som ska ha varit försäkrad för förmånerna.
\subsection*{8 §}
\paragraph*{}
För arbetstagare gäller försäkringen enligt 6 § från och med den första dagen av anställningstiden. För andra gäller försäkringen från och med den dag då arbetet har påbörjats.
\paragraph*{}
Försäkringen upphör att gälla tre månader, eller för sådana förmåner som anges i 6 § 5 ett år, efter den dag då arbetet har upphört av någon annan anledning än ledighet för semester, ferier eller motsvarande uppehåll (efterskyddstid).
Försäkringen upphör tidigare än som nu sagts om den enskilde börjar arbeta i ett annat land och omfattas av motsvarande försäkring i det landet eller om det finns andra särskilda skäl.
\subsection*{9 §}
\paragraph*{}
Om en förmån som avses i 6 § lämnas när försäkringen ska upphöra enligt 8 §, fortsätter försäkringen enligt 6 § att gälla under den tid för vilken förmånen lämnas.
\subsection*{10 §}
\paragraph*{}
Försäkringen enligt 6 § fortsätter att gälla efter efterskyddstiden enligt 8 § så länge bestämmelserna i 26 kap. 11-16 §§ om sjukpenninggrundande inkomst vid förvärvsavbrott (SGI-skyddad tid) är tillämpliga på personen.
Lag (2017:585).
\subsection*{11 §}
\paragraph*{}
Försäkringen för pensioner eller skadeersättningar som avses i 6 § 6 och 8-11 gäller när rätten till en förmån enligt de bestämmelser som anges där kan härledas från ett arbete i Sverige. Detsamma gäller försäkringen för föräldrapenning på grund- eller sjukpenningnivå enligt 6 § 2.
\paragraph*{}
Försäkringen för pensioner enligt första stycket gäller också när rätten till förmånen kan härledas från sådan ersättning som anges i 19 och 20 §§.
\subsection*{12 §}
\paragraph*{}
För biståndsarbetare m.fl. enligt 5 kap. 6 § som till följd av utlandsarbete inte omfattas av den arbetsbaserade försäkringen och som efter utlandstjänstgöringens slut återvänder till Sverige ska efterskyddstiden börja löpa först efter återkomsten, om utlandstjänstgöringen varat längst fem år.
\subsection*{13 §}
\paragraph*{}
Arbetsbaserade förmåner enligt 6 § 1-5 och 7-8 a får inte lämnas för längre tid tillbaka än tre månader före den månad då Försäkringskassan fick kännedom om arbetet.
\paragraph*{}
Bestämmelser om anmälan om arbete för den som inte är bosatt i Sverige finns i 110 kap. 44 §.
Lag (2020:1239).
\subsection*{14 §}
\paragraph*{}
Den som enligt utlänningslagen (2005:716) behöver ha arbetstillstånd i Sverige eller ett uppehållstillstånd med motsvarande verkan har inte rätt till arbetsbaserade förmåner förrän ett sådant tillstånd har beviljats. Ersättning får lämnas tidigast från och med den dag då tillståndet börjar gälla men inte för längre tid tillbaka än tre månader före det att tillståndet beviljades.
\paragraph*{}
Begränsningarna i första stycket gäller inte arbetsskadeersättning enligt 6 § 6 eller arbetsskadeersättning till efterlevande enligt 6 § 10.
\subsection*{15 §}
\paragraph*{}
För tid då en försäkrad vistas utomlands kan förmåner vid sjukdom, graviditet, tillfällig vård av barn och rehabilitering enligt 6 § 1-4 lämnas endast
\newline - om ett ersättningsfall inträffar utomlands medan den försäkrade där utför sådant arbete som ska anses som arbete i Sverige, eller
\newline - om Försäkringskassan medger att den försäkrade reser till utlandet.
\subsection*{16 §}
\paragraph*{}
Förmåner enligt 6 § 5, 6, 8 och 9-11 lämnas för tid då den försäkrade vistas utomlands så länge rätten till förmånen består. Detta gäller också i fråga om föräldrapenning på grund- eller sjukpenningnivå enligt 6 § 2,
\newline 1. om barnet är bosatt i Sverige, eller
\newline 2. om Försäkringskassan medger det, när ett barn hämtas i samband med adoption.
\paragraph*{}
Första stycket första meningen gäller också i fråga om inkomstpensionstillägg enligt 6 § 8 a om den försäkrade anses vara bosatt i Sverige.
Lag (2020:1239).
\subsection*{17 §}
\paragraph*{}
När det gäller efterlevandeförmåner ska det som anges i 16 § om försäkrad gälla den försäkrades efterlevande.
\subsection*{18 §}
\paragraph*{}
I 110 kap. 45 § finns bestämmelser om när den försäkrade är skyldig att anmäla till Försäkringskassan att han eller hon lämnar Sverige.
\subsection*{19 §}
\paragraph*{}
En person som får inkomstrelaterad sjukersättning eller inkomstrelaterad aktivitetsersättning enligt 6 § 5 är försäkrad för inkomstgrundad ålderspension enligt 6 § 8 beräknad på denna ersättning.
\subsection*{20 §}
\paragraph*{}
Den som får någon av följande förmåner är försäkrad för inkomstrelaterad sjukersättning och inkomstrelaterad aktivitetsersättning enligt 6 § 5, inkomstgrundad ålderspension enligt 6 § 8 och inkomstrelaterad efterlevandeförmån enligt 6 § 9 och 11:
\newline 1. omvårdnadsbidrag enligt 5 kap. 9 § 5,
\newline 2. dagpenning från arbetslöshetskassa,
\newline 3. aktivitetsstöd till den som deltar i ett arbetsmarknadspolitiskt program,
\newline 4. ersättning till deltagare i teckenspråksutbildning för vissa föräldrar (TUFF),
\newline 5. dagpenning till totalförsvarspliktiga som tjänstgör enligt lagen (1994:1809) om totalförsvarsplikt och till andra som får dagpenning enligt de grunder som gäller för totalförsvarspliktiga, och
\newline 6. stipendium som enligt 11 kap. 46 § inkomstskattelagen (1999:1229) ska tas upp som intäkt i inkomstslaget tjänst.
Lag (2018:1265).
\subsection*{21 §}
\paragraph*{}
När det gäller inkomstgrundad ålderspension enligt 6 § 8 finns i 60 kap. dessutom bestämmelser om beräkning av pensionsgrundande belopp för
\newline - plikttjänstgöring,
\newline - studier, och
\newline - vård av småbarn (barnår).
\subsection*{22 §}
\paragraph*{}
Den som genomgår utbildning som är förenad med särskild risk för arbetsskada är försäkrad för arbetsskadeersättning enligt 6 § 6 och 10.
\paragraph*{}
Regeringen eller den myndighet som regeringen bestämmer meddelar ytterligare föreskrifter om försäkring enligt första stycket.
\chapter*{7 Övriga förmåner}
\subsection*{1 §}
\paragraph*{}
I detta kapitel finns bestämmelser om
\newline - statligt personskadeskydd i 2-6 §§,
\newline - krigsskadeersättning till sjömän i 7-9 §§, och
\newline - smittbärarersättning i 10 §.
\subsection*{2 §}
\paragraph*{}
Försäkrad för statligt personskadeskydd enligt 43 kap. är
\newline 1. den som tjänstgör enligt lagen (1994:1809) om totalförsvarsplikt, eller inställer sig till mönstring eller annan uttagning enligt den lagen eller genomgår militär utbildning inom Försvarsmakten som rekryt,
\newline 2. den som medverkar i räddningstjänst eller i övning med en kommunal organisation för räddningstjänst enligt lagen (2003:778) om skydd mot olyckor, eller i räddningstjänst enligt 10 kap. 1 § andra stycket luftfartslagen (2010:500),
\newline 3. den som är intagen för vård i kriminalvårdsanstalt, i ett hem som avses i 12 § lagen (1990:52) med särskilda bestämmelser om vård av unga eller i ett hem som avses i 22 § lagen (1988:870) om vård av missbrukare i vissa fall samt den som är häktad eller anhållen eller i annat fall intagen eller tagen i förvar i kriminalvårdsanstalt, häkte eller polisarrest,
\newline 4. den som utför samhällstjänst på grund av en föreskrift som har meddelats med stöd av 27 kap. 2 a § eller 28 kap. 2 a § brottsbalken,
\newline 5. den som utför ungdomstjänst enligt 32 kap. 2 § eller 3 § första stycket 1 brottsbalken, och
\newline 6. den som utför oavlönat arbete enligt en föreskrift som har meddelats med stöd av 8 § första stycket 2 lagen (1994:451) om intensivövervakning med elektronisk kontroll.
Lag (2010:1308).
\subsection*{3 §}
\paragraph*{}
Regeringen eller den myndighet som regeringen bestämmer meddelar föreskrifter om i vilka fall bestämmelserna i 43 kap. ska tillämpas även på den som, i annat fall än som avses i 2 §, frivilligt deltar i verksamhet inom totalförsvaret eller i verksamhet för att avvärja eller begränsa skada på människor eller egendom eller i miljön.
\subsection*{4 §}
\paragraph*{}
Personskadeskyddet börjar gälla när den första resan påbörjas till sådan verksamhet eller intagning som avses i 2 § 1-3 och upphör när den sista resan från verksamheten eller intagningen avslutats.
\paragraph*{}
Vid sådan verksamhet som avses i 2 § 4-6 börjar personskadeskyddet gälla när den dömde påbörjar resan till platsen för verksamheten och upphör när resan från platsen avslutats.
\subsection*{5 §}
\paragraph*{}
Vistas någon under tid som avses i 4 § i något annat land än Sverige, Danmark utom Färöarna och Grönland, Finland eller Norge gäller personskadeskyddet endast om staten har ordnat eller bekostat resan.
\subsection*{6 §}
\paragraph*{}
Bestämmelserna i 6 kap. 7 §, 11 § första stycket och 16 § om försäkring, försäkringstid och utlandsvistelse ska tillämpas även när det gäller personskadeersättning.
\subsection*{7 §}
\paragraph*{}
Försäkrad för krigsskadeersättning till sjömän enligt 44 kap. är den som vid en tjänstgöring på svenskt fartyg är försäkrad för arbetsskadeersättning enligt 6 kap. 6 § 6 och som anses som sjöman enligt sjömanslagen (1973:282) eller ändå följer med fartyget och utför arbete för fartygets räkning.
\paragraph*{}
Försäkringen gäller dock inte den som tjänstgör på krigsfartyg eller skadas i ett krig som Sverige befinner sig i.
\subsection*{8 §}
\paragraph*{}
Skyddet enligt 7 § gäller utomlands när sjömannen
\newline 1. är på resa till en ort där han eller hon ska tillträda tjänstgöring på svenskt fartyg,
\newline 2. under tjänstgöring på svenskt fartyg vistas på en ort där fartyget ligger, eller
\newline 3. efter avslutad tjänstgöring på svenskt fartyg väntar på eller är på hemresa.
\subsection*{9 §}
\paragraph*{}
Bestämmelserna i 6 kap. 7 §, 11 § första stycket och 16 § om försäkring, försäkringstid och utlandsvistelse ska tillämpas även när det gäller krigsskadeersättning.
\subsection*{10 §}
\paragraph*{}
Försäkrad för smittbärarersättning enligt 46 kap. i samband med åtgärder för smittskydd eller skydd för livsmedel är en sådan smittbärare som avses i 3 § samma kapitel.
\part*{B FAMILJEFÖRMÅNER}
\chapter*{8 Innehåll, definitioner och förklaringar}
\subsection*{1 §}
\paragraph*{}
I avdelning B finns bestämmelser om socialförsäkringsförmåner till föräldrar och barn (familjeförmåner).
\subsection*{2 §}
\paragraph*{}
Familjeförmåner enligt denna avdelning är
\newline - graviditetspenning till den som på grund av graviditet har nedsatt arbetsförmåga eller är förbjuden att utföra sitt förvärvsarbete,
\newline - föräldrapenningsförmåner i samband med barns födelse, vid adoption av barn eller i andra situationer när en förälder vårdar barn,
\newline - barnbidrag som generellt bidrag för barn,
\newline - underhållsstöd till ett barn vars föräldrar inte bor tillsammans, och
\newline - särskilda familjeförmåner vid vissa fall av adoption eller när ett barn lider av sjukdom eller har funktionshinder.
\subsection*{3 §}
\paragraph*{}
I detta kapitel finns inledande bestämmelser om familjeförmåner.
\paragraph*{}
Vidare finns bestämmelser om
\newline - graviditetspenning och föräldrapenningsförmåner i 9-13 kap.,
\newline - barnbidrag i 14-16 kap.,
\newline - underhållsstöd i 17-19 kap., och
\newline - särskilda familjeförmåner i 20-22 kap.
\subsection*{4 §}
\paragraph*{}
En förmån enligt denna avdelning lämnas endast till den som har ett gällande försäkringsskydd för förmånen enligt 4-6 kap.
\paragraph*{}
Bestämmelser om anmälan och ansökan samt vissa gemensamma bestämmelser om förmåner och handläggning finns i 104-117 kap. (avdelning H).
\subsection*{5 §}
\paragraph*{}
Ärenden som avser förmåner enligt denna avdelning handläggs av Försäkringskassan.
\chapter*{9 Innehåll}
\subsection*{1 §}
\paragraph*{}
I denna underavdelning finns
\newline - bestämmelser om graviditetspenning i 10 kap., och
\newline - allmänna bestämmelser om föräldrapenningsförmåner i 11 kap.
\paragraph*{}
Vidare finns bestämmelser om
\newline - föräldrapenning i 12 kap., och
\newline - tillfällig föräldrapenning i 13 kap.
\chapter*{10 Graviditetspenning}
\subsection*{1 §}
\paragraph*{}
I detta kapitel finns bestämmelser om
\newline - rätten till graviditetspenning i 2-5 §§,
\newline - förmånstiden i 6-9 §§, och
\newline - beräkning av graviditetspenning, m.m. i 10 och 11 §§.
\subsection*{2 §}
\paragraph*{}
En försäkrad som är gravid har rätt till graviditetspenning,
\newline 1. om graviditeten sätter ned hennes förmåga att utföra sitt förvärvsarbete med minst en fjärdedel, och
\newline 2. hon inte kan omplaceras till annat mindre ansträngande arbete enligt 19 § föräldraledighetslagen (1995:584).
\paragraph*{}
Som förvärvsarbete ska inte betraktas sådant förvärvsarbete som den försäkrade utför under tid för vilken hon får sjukersättning enligt bestämmelserna i 37 kap. 3 §. Om det inte går att avgöra under vilken tid den försäkrade avstår från förvärvsarbete ska frånvaron i första hand anses som frånvaro från sådant förvärvsarbete som avses i 37 kap. 3 §.
\subsection*{3 §}
\paragraph*{}
En försäkrad som är gravid har rätt till graviditetspenning om hon
\newline 1. inte får sysselsättas i sitt förvärvsarbete på grund av en föreskrift om förbud mot arbete under graviditeten, som har meddelats med stöd av 4 kap. 6 § arbetsmiljölagen (1977:1160), och
\newline 2. inte kan omplaceras till annat arbete enligt 18 § föräldraledighetslagen (1995:584).
\paragraph*{}
Bestämmelserna i 2 § andra stycket tillämpas även i fall som avses i denna paragraf.
\subsection*{3 a §}
\paragraph*{}
En gravid försäkrad som har inkomst av annat förvärvsarbete än anställning och som bedriver näringsverksamhet har rätt till graviditetspenning, om förvärvsarbetet innehåller någon risk för skadlig inverkan på hennes hälsa, graviditeten eller fostret.
\paragraph*{}
Bestämmelserna i 2 § andra stycket tillämpas även i fall som avses i denna paragraf.
Lag (2013:746).
\subsection*{4 §}
\paragraph*{}
Graviditetspenning lämnas som hel, tre fjärdedels, halv eller en fjärdedels förmån.
\subsection*{5 §}
\paragraph*{}
Graviditetspenning lämnas inte till den del den försäkrade för samma tid får
\newline - sjukpenning, eller
\newline - sjuklön eller sådan ersättning från Försäkringskassan som avses i 20 § lagen (1991:1047) om sjuklön.
\subsection*{6 §}
\paragraph*{}
Graviditetspenning vid nedsatt arbetsförmåga enligt 2 § lämnas för varje dag som nedsättningen består. Ersättningen lämnas dock tidigast från och med den sextionde dagen före den beräknade tidpunkten för barnets födelse.
\subsection*{7 §}
\paragraph*{}
Graviditetspenning vid förbud mot förvärvsarbete enligt 3 § lämnas för varje dag som förbudet gäller.
\subsection*{7 a §}
\paragraph*{}
Graviditetspenning enligt 3 a § lämnas för varje dag som kvinnan avstår från att utföra förvärvsarbetet.
Lag (2013:746).
\subsection*{8 §}
\paragraph*{}
Graviditetspenning lämnas längst till och med den elfte dagen före den beräknade tidpunkten för barnets födelse.
\subsection*{9 §}
\paragraph*{}
Graviditetspenning beviljas för en viss period.
\subsection*{10 §}
\paragraph*{}
Graviditetspenning beräknas enligt bestämmelserna om sjukpenning och sjukpenninggrundande inkomst i 25-28 kap., med de undantag som anges i 11 §.
Lag (2017:1305).
\subsection*{11 §}
\paragraph*{}
Det som i denna balk i övrigt eller i annan författning föreskrivs om sjukpenning enligt 25-28 kap. gäller i tillämpliga delar även i fråga om graviditetspenning, med undantag av bestämmelserna i
\newline - 25 kap. 5 § om bortseende från inkomst av anställning och annat förvärvsarbete överstigande 10,0 prisbasbelopp,
\newline - 27 kap. 5 § om ersättning för merutgifter,
\newline - 27 kap. 27 §, 27 a § och 28 b § första stycket om karensavdrag och karensdagar,
\newline - 27 kap. 29-33 a §§ om karenstid, och
\newline - 28 kap. 7 § 2 om beräkningsunderlag för sjukpenning på fortsättningsnivån.
\paragraph*{}
Vid beräkning av graviditetspenning ska det vid beräkningen av den sjukpenninggrundande inkomsten bortses från inkomst av anställning och annat förvärvsarbete till den del summan av dessa inkomster överstiger 7,5 prisbasbelopp. Det ska vid denna beräkning i första hand bortses från inkomst av annat förvärvsarbete.
\paragraph*{}
För hel graviditetspenning motsvarar ersättningsnivån beräkningsunderlaget för sjukpenning på normalnivån enligt 28 kap. 7 § 1 grundat på en sjukpenninggrundande inkomst beräknad enligt första och andra styckena.
Lag (2021:1240).
\chapter*{11 Allmänna bestämmelser om föräldrapenningsförmåner}
\subsection*{1 §}
\paragraph*{}
I detta kapitel finns inledande bestämmelser i 2-3 §§.
\paragraph*{}
Vidare finns bestämmelser om
\newline - föräldrabegreppet i 4-6 §§,
\newline - adoptionsbegreppet i 7 §,
\newline - rätten till föräldrapenningsförmåner i 8-13 §§,
\newline - samordning med andra förmåner i 14-16 §§, och
\newline - utbetalning till annan än förälder i 17 §.
\subsection*{2 §}
\paragraph*{}
Föräldrapenningsförmåner lämnas i följande former:
\newline 1. föräldrapenning för vård av barn med anledning av barns födelse eller vid adoption av barn (12 kap.), och
\newline 2. tillfällig föräldrapenning i särskilda situationer när någon avstår från förvärvsarbete för att vårda barn eller i samband med att ett barn har avlidit (13 kap.).
Lag (2010:2005).
\subsection*{3 §}
\paragraph*{}
I föräldraledighetslagen (1995:584) finns bestämmelser i vilka anges när en arbetstagare som får föräldrapenningsförmåner har rätt att vara ledig från sin anställning.
\subsection*{4 §}
\paragraph*{}
Vid tillämpningen av bestämmelserna om föräldrapenningsförmåner likställs med en förälder följande personer:
\newline 1. förälders make som stadigvarande sammanbor med föräldern,
\newline 2. förälders sambo,
\newline 3. särskilt förordnad vårdnadshavare som har vård om barnet, och
\newline 4. blivande adoptivförälder.
Lag (2018:1952).
\paragraph*{}
/Rubriken träder i kraft I:2024-07-01/
\subsection*{4 a §}
\paragraph*{}
/Träder i kraft I:2024-07-01/
I fråga om föräldrapenning för en försäkrad som har fått rätt till sådan förmån genom överlåtelse enligt 12 kap. 17 a § likställs den försäkrade med en förälder.
Lag (2023:905).
\subsection*{5 §}
\paragraph*{}
När det gäller tillfällig föräldrapenning likställs med en förälder även familjehemsförälder.
\paragraph*{}
Första stycket gäller inte i fall som avses i 13 kap. 8 och 9 §§.
Lag (2018:1952).
\subsection*{6 §}
\paragraph*{}
I fråga om tillfällig föräldrapenning för en försäkrad som har fått rätt till sådan förmån enligt 13 kap. 8, 9, 11 eller 31 a § likställs den försäkrade med en förälder.
\paragraph*{}
En försäkrad som enligt första stycket är likställd med en förälder får dock inte överlåta rätten till tillfällig föräldrapenning enligt 13 kap. 8 §. Inte heller får Försäkringskassan bevilja tillfällig föräldrapenning enligt 13 kap. 9 § på grundval av ett medgivande av en sådan försäkrad.
\subsection*{7 §}
\paragraph*{}
Vid tillämpning av bestämmelserna om föräldrapenningsförmåner likställs med adoption att någon efter socialnämndens medgivande har tagit emot ett barn för stadigvarande vård och fostran i syfte att adoptera barnet.
\paragraph*{}
Vid tillämpning av bestämmelserna om föräldrapenningsförmåner likställs den tidpunkt när den som adopterat ett barn har fått barnet i sin vård med tidpunkten för ett barns födelse, dock inte vid beräkning av barnets ålder.
Lag (2018:1290).
\subsection*{8 §}
\paragraph*{}
En förälder har rätt till föräldrapenningsförmåner endast för vård av barn som är bosatt i Sverige. Vid adoption ska barnet anses bosatt i Sverige om den blivande föräldern är bosatt här.
\subsection*{9 §}
\paragraph*{}
Oavsett antalet barn kan en förälder inte få mer än sammanlagt hel föräldrapenning per dag.
\subsection*{10 §}
\paragraph*{}
Föräldrapenningsförmåner får inte lämnas till båda föräldrarna för samma barn och tid i annat fall än som anges i 12 kap. 4 a § och 5 a-7 §§ samt 13 kap. 10, 11, 13, 26 och 30 §§.
\paragraph*{}
Tillfällig föräldrapenning enligt 13 kap. 31 e § får lämnas till flera föräldrar för samma barn och tid.
\paragraph*{}
För tillfällig föräldrapenning gäller även 13 kap. 3 §.
Lag (2018:1628).
\subsection*{11 §}
\paragraph*{}
Föräldrapenningsförmåner lämnas inte för dag då en förälder är semesterledig enligt semesterlagen (1977:480).
\subsection*{12 §}
\paragraph*{}
Föräldrapenning får inte lämnas för tid innan anmälan gjorts till Försäkringskassan. Detta gäller dock inte om det har funnits hinder för en sådan anmälan eller det finns särskilda skäl för att föräldrapenning ändå bör lämnas.
Lag (2018:1628).
\subsection*{13 §}
\paragraph*{}
Bestämmelserna i 27 kap. 56-58 och 60 §§ om arbetsgivarinträde och konsulärt bistånd tillämpas även i fråga om föräldrapenningsförmåner.
\subsection*{14 §}
\paragraph*{}
Föräldrapenningsförmåner lämnas inte om en förälder för samma tid får någon av följande förmåner:
\newline 1. sjuklön eller sådan ersättning från Försäkringskassan som avses i 20 § lagen (1991:1047) om sjuklön,
\newline 2. sjukpenning, och
\newline 3. ersättning som motsvarar sjukpenning enligt annan författning eller på grund av regeringens beslut i ett särskilt fall.
\paragraph*{}
Första stycket gäller även när föräldern får motsvarande förmån på grundval av utländsk lagstiftning.
\subsection*{15 §}
\paragraph*{}
Föräldrapenning med anledning av ett barns födelse lämnas inte om det för samma barn och tid lämnas en motsvarande förmån enligt utländsk lagstiftning.
\subsection*{16 §}
\paragraph*{}
Tillfällig föräldrapenning lämnas inte för sådant behov av omvårdnad eller tillsyn av ett barn som har grundat rätt till omvårdnadsbidrag.
Lag (2018:1265).
\subsection*{17 §}
\paragraph*{}
Om en förälder som inte fyllt 18 år har rätt till en föräldrapenningsförmån, får Försäkringskassan på begäran av socialnämnden besluta att föräldrapenningen helt eller delvis ska betalas ut till någon annan person eller till nämnden att användas till förälderns och familjens nytta.
\chapter*{12 Föräldrapenning}
\subsection*{1 §}
\paragraph*{}
/Upphör att gälla U:2024-07-01/
I detta kapitel finns bestämmelser om
\newline - rätten till föräldrapenning i 2-11 §§,
\newline - förmånstiden i 12-13 §§,
\newline - vem som får föräldrapenningen i 14-17 §§,
\newline - ersättningsnivåer i 18-24 §§,
\newline - beräkning av föräldrapenning på sjukpenningnivån i 25-31 §§,
\newline - beräkning av antalet dagar med rätt till föräldrapenning i 32-34 §§,
\newline - föräldrapenning för de första 180 dagarna i 35-38 §§,
\newline - föräldrapenning efter 180 dagar i 39-41 §§,
\newline - föräldrapenning för tid efter barnets fjärde levnadsår i 41 a-41 h §§, och
\newline - föräldrapenning vid flerbarnsfödsel i 42-46 §§.
Lag (2013:999).
\subsection*{1 §}
\paragraph*{}
/Träder i kraft I:2024-07-01/
I detta kapitel finns bestämmelser om
\newline - rätten till föräldrapenning i 2-11 §§,
\newline - förmånstiden i 12-13 §§,
\newline - vem som får föräldrapenningen i 14-17 c §§,
\newline - ersättningsnivåer i 18-24 §§,
\newline - beräkning av föräldrapenning på sjukpenningnivån i 25-31 §§,
\newline - beräkning av antalet dagar med rätt till föräldrapenning i 32-34 §§,
\newline - föräldrapenning för de första 180 dagarna i 35-38 §§,
\newline - föräldrapenning efter 180 dagar i 39-41 §§,
\newline - föräldrapenning för tid efter barnets fjärde levnadsår i 41 a-41 h §§, och
\newline - föräldrapenning vid flerbarnsfödsel i 42-46 §§.
Lag (2023:905).
\subsection*{2 §}
\paragraph*{}
Rätt till föräldrapenning har en försäkrad förälder som vårdar barn under tid när han eller hon inte förvärvsarbetar eller avstår från förvärvsarbete.
\paragraph*{}
Föräldrapenning lämnas i de fall och under de närmare förutsättningar som anges i detta kapitel. Förmånen kan vara bosättningsbaserad enligt 5 kap. eller arbetsbaserad enligt 6 kap.
\subsection*{3 §}
\paragraph*{}
För rätt till föräldrapenning enligt 2 § gäller också som villkor att föräldern under tid som anges där till huvudsaklig del faktiskt vårdar barnet på det sätt som krävs med hänsyn till barnets ålder.
Lag (2013:999).
\subsection*{4 §}
\paragraph*{}
I andra fall än som avses i 5-6 och 7 a §§ har en förälder som inte har barnet i sin vård rätt till föräldrapenning endast om det finns särskilda skäl.
Lag (2018:1628).
\subsection*{4 a §}
\paragraph*{}
/Upphör att gälla U:2024-07-01/
Föräldrapenning får för samma barn och tid lämnas till båda föräldrarna samtidigt i högst 30 dagar under barnets första levnadsår, räknat från barnets födelse eller därmed likställd tidpunkt. Detta gäller endast dagar för vilka det inte för någon av föräldrarna finns hinder enligt 17 § andra stycket mot att avstå föräldrapenning till förmån för den andra föräldern.
\paragraph*{}
Bestämmelser om ansökan i fall som avses i första stycket finns i 110 kap. 5 a §.
Lag (2011:1082).
\subsection*{4 a §}
\paragraph*{}
/Träder i kraft I:2024-07-01/
Föräldrapenning får för samma barn och tid lämnas till båda föräldrarna samtidigt i högst 60 dagar under 15 månader, räknat från barnets födelse eller därmed likställd tidpunkt. Detta gäller endast dagar för vilka det inte för någon av föräldrarna finns hinder enligt 17 § andra stycket mot att avstå föräldrapenning till förmån för den andra föräldern.
\paragraph*{}
Bestämmelser om ansökan i fall som avses i första stycket finns i 110 kap. 5 a §.
Lag (2023:905).
\subsection*{5 §}
\paragraph*{}
Barnets mor har rätt till föräldrapenning från och med den sextionde dagen före den beräknade tidpunkten för barnets födelse.
\paragraph*{}
Även om barnets mor inte har barnet i sin vård har hon rätt till föräldrapenning till och med den tjugonionde dagen efter förlossningsdagen.
\subsection*{5 a §}
\paragraph*{}
En förälder som besöker mödravården när det gäller ett ofött barn eller en graviditet har rätt till föräldrapenning i samband med besöket.
\paragraph*{}
Föräldrapenning i samband med besök hos mödravården kan lämnas för tid från och med den sextionde dagen före den beräknade tidpunkten för barnets födelse fram till och med förlossningsdagen.
Lag (2018:1628).
\subsection*{6 §}
\paragraph*{}
En förälder som deltar i föräldrautbildning har rätt till föräldrapenning i samband med utbildningen.
\paragraph*{}
Föräldrapenning i samband med föräldrautbildning kan lämnas före barnets födelse och även därefter till en förälder som inte har barnet i sin vård.
\subsection*{7 §}
\paragraph*{}
En förälder som besöker barnets förskola eller sådan pedagogisk verksamhet som avses i 25 kap. skollagen (2010:800), som kompletterar eller erbjuds i stället för förskola och som barnet deltar i har rätt till föräldrapenning i samband med besöket.
Lag (2010:870).
\subsection*{7 a §}
\paragraph*{}
En förälder som har ett barn som deltar i en introduktion till en verksamhet enligt skollagen (2010:800) har rätt till föräldrapenning för att delta i introduktionen, om den avser
\newline 1. förskola,
\newline 2. förskoleklass,
\newline 3. grundskola, anpassad grundskola, specialskola, sameskola eller internationell skola på grundskolenivå,
\newline 4. fritidshem som kompletterar utbildningen i en skolform enligt 2 eller 3, eller
\newline 5. verksamhet enligt 25 kap. 2, 4 eller 5 § skollagen.
\paragraph*{}
Föräldrapenning i samband med ett barns introduktion kan lämnas till en förälder som inte har barnet i sin vård.
Lag (2023:348).
\subsection*{8 §}
\paragraph*{}
Bestämmelserna om föräldrapenning gäller i tillämpliga delar också vid adoption av barn, med följande undantag:
\newline 1. Vid adoption av den andra makens eller sambons barn lämnas föräldrapenning inte utöver vad som skulle ha gällt om adoptionen inte hade ägt rum.
\newline 2. Föräldrapenning i samband med föräldrautbildning enligt 6 § till den som avser att adoptera ett barn lämnas inte före den dag då föräldern har fått barnet i sin vård.
Lag (2018:1290).
\subsection*{9 §}
\paragraph*{}
Föräldrapenning lämnas enligt följande förmånsnivåer:
\newline 1. Hel föräldrapenning lämnas för dag när föräldern inte förvärvsarbetar.
\newline 2. Tre fjärdedels föräldrapenning lämnas när föräldern förvärvsarbetar högst en fjärdedel av normal arbetstid.
\newline 3. Halv föräldrapenning lämnas när föräldern förvärvsarbetar högst hälften av normal arbetstid.
\newline 4. En fjärdedels föräldrapenning lämnas när föräldern förvärvsarbetar högst tre fjärdedelar av normal arbetstid.
\newline 5. En åttondels föräldrapenning lämnas när föräldern förvärvsarbetar högst sju åttondelar av normal arbetstid.
\paragraph*{}
Föräldrapenning får dock lämnas som hel, tre fjärdedels, halv, en fjärdedels eller en åttondels föräldrapenning på grundnivå enligt 23 § eller på lägstanivå enligt 24 § när föräldern arbetar högst sju åttondelar av normal arbetstid.
\subsection*{10 §}
\paragraph*{}
Som förvärvsarbete betraktas inte
\newline 1. vård av barn som har tagits emot för stadigvarande vård och fostran i förälderns hem, och
\newline 2. sådant förvärvsarbete som den försäkrade utför under tid för vilken han eller hon får sjukersättning enligt bestämmelserna i 37 kap. 3 §.
\paragraph*{}
Om det vid tillämpningen av första stycket 2 inte går att avgöra vilken tid den försäkrade avstår från förvärvsarbete för att vårda sitt barn ska frånvaron i första hand anses som frånvaro från sådant förvärvsarbete som avses i 37 kap. 3 §.
\subsection*{11 §}
\paragraph*{}
Föräldrapenning på sjukpenningnivå enligt 21 och 22 §§ lämnas för tid som normalt är arbetsfri för föräldern endast om han eller hon i direkt anslutning till den arbetsfria tiden får föräldrapenning på motsvarande eller högre förmånsnivå. Detta gäller dock endast för perioder av arbetsfri tid om högst fyra dagar.
\subsection*{12 §}
\paragraph*{}
Föräldrapenning med anledning av ett barns födelse lämnas under högst 480 dagar sammanlagt för föräldrarna, och vid flerbarnsfödsel under ytterligare högst 180 dagar för varje barn utöver det första.
\paragraph*{}
Om ett barn blir bosatt här i landet under barnets andra levnadsår lämnas föräldrapenning under högst 200 dagar sammanlagt för föräldrarna, och vid flerbarnsfödsel under ytterligare högst 76 dagar för varje barn utöver det första. Om barnet blir bosatt här i landet efter det andra levnadsåret lämnas föräldrapenning under högst 100 dagar sammanlagt för föräldrarna, och vid flerbarnsfödsel under ytterligare högst 38 dagar för varje barn utöver det första.
\paragraph*{}
För tid efter barnets fjärde levnadsår, räknat från barnets födelse eller därmed likställd tidpunkt, lämnas föräldrapenning dock under högst 96 dagar sammanlagt för föräldrarna, och vid flerbarnsfödsel under ytterligare högst 36 dagar för varje barn utöver det första.
Lag (2017:559).
\subsection*{12 a §}
\paragraph*{}
Föräldrapenning lämnas inte för längre tid tillbaka än 90 dagar före den dag ansökan om föräldrapenning kom in till Försäkringskassan. Detta gäller dock inte om det finns synnerliga skäl för att föräldrapenning ändå bör lämnas.
Lag (2013:999).
\subsection*{13 §}
\paragraph*{}
Föräldrapenning lämnas längst till dess barnet har fyllt tolv år eller till den senare tidpunkt då barnet har avslutat det femte skolåret i grundskolan.
Lag (2013:999).
\subsection*{14 §}
\paragraph*{}
En förälder som har ensam vårdnad om ett barn får föräldrapenning under hela den tid som anges i 12 §.
\subsection*{15 §}
\paragraph*{}
Om föräldrarna har gemensam vårdnad om ett barn får vardera föräldern föräldrapenning under hälften av den tid som anges i 12 §. Vardera föräldern får då föräldrapenning under hälften av den tid för vilken förmånen enligt 19 § lämnas på sjukpenning- eller grundnivån och hälften av den tid för vilken den lämnas på lägstanivån. Om endast en av föräldrarna har rätt till föräldrapenning, får han eller hon dock föräldrapenning under hela den tid som anges i 12 §.
Lag (2013:999).
\subsection*{15 a §}
\paragraph*{}
Om antalet kvarstående dagar för föräldrapenning omedelbart före utgången av barnets fjärde levnadsår överstiger det antal dagar för vilka föräldrapenning kan lämnas enligt 12 § tredje stycket, fördelas antalet dagar för föräldrapenning enligt nämnda lagrum mellan föräldrar som har gemensam vårdnad om ett barn på så sätt att vardera föräldern får så stor andel av antalet dagar som motsvarar den förälderns andel av de dagar som kvarstod omedelbart före utgången av barnets fjärde levnadsår. Antalet dagar som ingår i respektive beräknad andel avrundas till närmaste hel dag, varvid halv dag avrundas uppåt.
Lag (2013:999).
\subsection*{15 b §}
\paragraph*{}
Om den ena föräldern får rätt till föräldrapenning först under sådan tid som avses i 12 § tredje stycket, lämnas föräldrapenning till vardera föräldern under hälften av det antal dagar som återstår för föräldrapenning omedelbart efter utgången av barnets fjärde levnadsår. Det antal dagar under vilka den först berättigade föräldern har fått föräldrapenning för nämnda tid, fram till dess att båda föräldrarna fick rätt till föräldrapenning, ska räknas av. I första hand ska avräkning göras från den först berättigade förälderns andel.
Lag (2013:999).
\subsection*{16 §}
\paragraph*{}
Om en förälder på grund av sjukdom eller funktionshinder varaktigt saknar förmåga att vårda barnet, får den andra föräldern föräldrapenning under hela den tid som anges i 12 §.
Avstående från föräldrapenning
\subsection*{17 §}
\paragraph*{}
En förälder kan genom skriftlig anmälan till Försäkringskassan avstå rätten att få föräldrapenning till förmån för den andra föräldern.
\paragraph*{}
Detta gäller dock inte föräldrapenning på sjukpenningnivå enligt 21 och 22 §§ eller på grundnivå enligt 23 § såvitt avser en tid om
\newline 1. 90 dagar för varje barn, eller
\newline 2. 90 dagar för barnen gemensamt vid flerbarnsfödsel.
\paragraph*{}
I anmälan ska det anges vilka ersättningsnivåer enligt 18 § avståendet avser.
Lag (2022:1291).
\paragraph*{}
/Rubriken träder i kraft I:2024-07-01/
\subsection*{17 a §}
\paragraph*{}
/Träder i kraft I:2024-07-01/
En förälder kan genom skriftlig anmälan till Försäkringskassan överlåta rätten till föräldrapenning till någon annan som är försäkrad för sådan förmån. Detta gäller dock inte föräldrapenning för sådan tid som avses i 17 § andra stycket.
\paragraph*{}
I anmälan ska det anges vilka ersättningsnivåer enligt 18 § överlåtelsen avser.
\paragraph*{}
En försäkrad som har fått rätt till föräldrapenning genom en överlåtelse får inte i sin tur överlåta förmånen eller avstå den enligt 17 §.
Lag (2023:905).
\subsection*{17 b §}
\paragraph*{}
/Träder i kraft I:2024-07-01/
Högst 90 dagar för varje barn får överlåtas. Om föräldrarna har gemensam vårdnad om ett barn, har vardera föräldern rätt att överlåta högst 45 dagar.
Lag (2023:905).
\paragraph*{}
/Rubriken träder i kraft I:2024-07-01/
\subsection*{17 c §}
\paragraph*{}
/Träder i kraft I:2024-07-01/
En förälder som har avstått eller överlåtit rätten till föräldrapenning får återta avståendet eller överlåtelsen för dagar som inte har utnyttjats.
Lag (2023:905).
\subsection*{18 §}
\paragraph*{}
Föräldrapenning kan lämnas på
\newline - sjukpenningnivå,
\newline - grundnivå, eller
\newline - lägstanivå.
\subsection*{19 §}
\paragraph*{}
För de första 180 dagarna med rätt till föräldrapenning för vård av ett barn gäller att förmånen kan lämnas på antingen sjukpenningnivån eller grundnivån.
\paragraph*{}
Efter 180 dagar gäller att föräldrapenning kan lämnas för 210 dagar på antingen sjukpenningnivån eller grundnivån och för 90 dagar på lägstanivån.
\subsection*{20 §}
\paragraph*{}
När det gäller flerbarnsfödsel finns bestämmelser om ersättningsnivåerna i 42-46 §§.
Lag (2013:999).
\subsection*{21 §}
\paragraph*{}
Föräldrapenning på sjukpenningnivån kan lämnas till en förälder som är försäkrad för arbetsbaserad föräldrapenning enligt 6 kap. 6 § 2 om det enligt 25 kap. 3 § kan fastställas en sjukpenninggrundande inkomst för föräldern.
\subsection*{22 §}
\paragraph*{}
För hel föräldrapenning motsvarar sjukpenningnivån förälderns beräkningsunderlag för sjukpenning på normalnivån enligt 28 kap. 7 § 1 grundat på en sjukpenninggrundande inkomst beräknad enligt 25-31 §§.
\paragraph*{}
Om hel föräldrapenning på sjukpenningnivån inte överstiger 250 kronor om dagen, lämnas i stället föräldrapenning på grundnivån enligt 23 §.
Lag (2015:964).
\subsection*{23 §}
\paragraph*{}
Föräldrapenning på grundnivån kan lämnas till en förälder som är försäkrad för bosättningsbaserad föräldrapenning enligt 5 kap. 9 § 1 eller arbetsbaserad föräldrapenning enligt 6 kap. 6 § 2.
\paragraph*{}
För hel föräldrapenning är grundnivån 250 kronor om dagen.
Lag (2015:964).
\subsection*{24 §}
\paragraph*{}
Föräldrapenning på lägstanivån kan lämnas till en förälder som är försäkrad för bosättningsbaserad föräldrapenning enligt 5 kap. 9 § 1 och avser tid då föräldrapenning inte får lämnas på sjukpenningnivån eller grundnivån.
\paragraph*{}
För hel föräldrapenning är lägstanivån 180 kronor om dagen.
\subsection*{25 §}
\paragraph*{}
När föräldrapenning ska lämnas på sjukpenningnivån ska ersättningen beräknas enligt bestämmelserna om sjukpenning på normalnivån och sjukpenninggrundande inkomst i 25, 26 och 28 kap., med undantag av bestämmelserna i
\newline - 25 kap. 5 § om bortseende från inkomst av anställning och annat förvärvsarbete överstigande 10,0 prisbasbelopp,
\newline - 26 kap. 19-22 §§ om beräkning i vissa fall,
\newline - 28 kap. 7 § 2 om beräkningsunderlaget för sjukpenning på fortsättningsnivån, och
\newline - 28 kap. 12-18 §§ om arbetstidsberäknad sjukpenning.
Lag (2021:1240).
\subsection*{26 §}
\paragraph*{}
Vid beräkning av föräldrapenning som ska lämnas på sjukpenningnivån ska det vid beräkningen av den sjukpenninggrundande inkomsten bortses från inkomst av anställning och annat förvärvsarbete till den del summan av dessa inkomster överstiger 10 prisbasbelopp. Det ska vid denna beräkning i första hand bortses från inkomst av annat förvärvsarbete.
\subsection*{27 §}
\paragraph*{}
Om en förälders sjukpenninggrundande inkomst har sänkts sedan den tid som avses i 26 kap. 15 § (SGI-skyddad tid för föräldraledighet) har gått ut, ska föräldrapenningen till dess att barnet fyller två år beräknas lägst på grundval av
\newline 1. den sjukpenninggrundande inkomst som gällde innan sänkningen skedde, eller
\newline 2. den högre inkomst som löneavtal därefter föranleder.
\paragraph*{}
Första stycket gäller all föräldrapenning som lämnas till föräldern, när han eller hon avstår från förvärvsarbete för vård av barn under den tid som anges där.
\paragraph*{}
Skulle den sjukpenninggrundande inkomsten som gällde innan sänkningen skedde ha överstigit 10,0 prisbasbelopp, om inkomsten alltjämt hade beräknats utan begränsningen i 25 kap. 5 § andra stycket första meningen, tillämpas det beräkningssätt som anges i 26 §.
Lag (2021:1240).
\subsection*{28 §}
\paragraph*{}
För en förälder som helt eller delvis saknar anställning ska den sjukpenninggrundande inkomst som föräldrapenningen beräknas på enligt 27 § räknas om på sätt som anges i 26 kap. 28-30 §§ och 31 § första stycket.
\subsection*{29 §}
\paragraph*{}
Om en förälder är gravid på nytt innan ett barn har uppnått eller skulle ha uppnått ett år och nio månaders ålder, ska föräldrapenning för föräldrarna även fortsättningsvis beräknas på det sätt som anges i 27 och 28 §§.
\paragraph*{}
Det som anges i första stycket gäller även vid adoption av ett barn som sker högst två år och sex månader efter det att det föregående barnet har fötts eller adopterats.
\subsection*{30 §}
\paragraph*{}
Har efter tid som avses i 26 kap. 15 § (SGI-skyddad tid för föräldraledighet) sänkning inte skett av den sjukpenninggrundande inkomst som avses i 25 kap. på grund av att den årliga inkomsten överstiger 10,0 prisbasbelopp, gäller i tillämpliga delar bestämmelserna i 25-29 §§.
Lag (2021:1240).
\subsection*{31 §}
\paragraph*{}
Om en familjehemsförälder får ersättning för vården av barnet, ska det vid beräkningen av föräldrapenning bortses från den del av den sjukpenninggrundande inkomsten som grundas på ersättningen.
\subsection*{32 §}
\paragraph*{}
Vid beräkning av antalet dagar med rätt till föräldrapenning gäller följande:
\newline - En dag med hel föräldrapenning motsvarar en dag.
\newline - En dag med tre fjärdedels, halv, en fjärdedels eller en åttondels föräldrapenning motsvarar tre fjärdedelar, hälften, en fjärdedel respektive en åttondel av en dag.
\subsection*{33 §}
\paragraph*{}
Om en förälder har fått en förmån enligt utländsk lagstiftning, som motsvarar föräldrapenning med anledning av ett barns födelse, ska den tid som den utländska förmånen har lämnats för räknas av från det högsta antal dagar som föräldrapenning kan lämnas för enligt 12 §.
\paragraph*{}
Om rätten till föräldrapenning inträder först under sådan tid som avses i 12 § andra eller tredje stycket och den ena av föräldrarna eller båda har fått en sådan utländsk förmån som ska räknas av enligt första stycket, ska avräkningen göras från det högsta antal dagar som föräldrapenning hade kunnat lämnas för enligt 12 § första stycket.
Lag (2017:559).
\subsection*{34 §}
\paragraph*{}
Avräkning enligt 33 § första stycket ska i första hand göras från de dagar som föräldern själv har rätt till enligt 15 § första meningen samt, när det gäller dessa dagar, från de dagar som avses i 35-37 §§.
\paragraph*{}
För återstående dagar som ska räknas av gäller följande:
\newline 1. Om den utländska förmånen grundas på inkomst av anställning eller av annat förvärvsarbete, ska avräkningen i första hand göras från de dagar för vilka föräldrapenning kan lämnas på sjukpenningnivån.
\newline 2. Om den utländska förmånen lämnas med ett belopp som för alla förmånstagare är enhetligt och oberoende av inkomst av anställning eller av annat förvärvsarbete, ska avräkningen i första hand göras från de dagar för vilka föräldrapenning kan lämnas endast på lägstanivån.
Lag (2013:999).
\subsection*{34 a §}
\paragraph*{}
Avräkning enligt 33 § andra stycket ska i första hand göras från de dagar under vilka föräldern enligt 15 § första meningen själv skulle ha kunnat ha rätt till föräldrapenning.
Lag (2013:999).
\subsection*{35 §}
\paragraph*{}
Till en förälder som är försäkrad för både bosättningsbaserad och arbetsbaserad föräldrapenning lämnas förmånen för de första 180 dagarna enligt följande:
\newline 1. Föräldrapenning lämnas på sjukpenningnivån, om föräldern under minst 240 dagar i följd före barnets födelse eller den beräknade tidpunkten för födelsen har varit försäkrad för sjukpenning enligt 6 kap. 6 § 3 och under hela den tiden skulle ha haft rätt till en sjukpenning som överstiger lägstanivån för föräldrapenning (240-dagarsvillkoret).
\newline 2. Om förutsättningarna i 1 inte är uppfyllda eller hel föräldrapenning på sjukpenningnivån i annat fall inte överstiger 250 kronor om dagen, lämnas föräldrapenning på grundnivån.
Lag (2015:964).
\subsection*{36 §}
\paragraph*{}
Till en förälder som är försäkrad för enbart bosättningsbaserad föräldrapenning lämnas förmånen för de första 180 dagarna på grundnivån.
\subsection*{37 §}
\paragraph*{}
Till en förälder som är försäkrad för enbart arbetsbaserad föräldrapenning lämnas förmånen för de första 180 dagarna enligt 35 §.
\paragraph*{}
För att föräldrapenning ska lämnas på grundnivån krävs dock att föräldern uppfyller 240-dagarsvillkoret i 35 § 1.
\subsection*{38 §}
\paragraph*{}
För en förälder som anses bosatt i Sverige även under vistelse utomlands enligt bestämmelserna i 5 kap. 6 och 8 §§, ska det bortses från tiden för utlandsvistelsen när det bestäms om 240-dagarsvillkoret i 35 § 1 är uppfyllt.
\paragraph*{}
Vidare ska för en förälder som fått sjukersättning eller aktivitetsersättning en sjukpenninggrundande inkomst beräknad enligt 26 kap. 22 a § anses ha gällt hela den tid som föräldern fått sådan ersättning.
Lag (2015:758).
\subsection*{39 §}
\paragraph*{}
Till en förälder som är försäkrad för både bosättningsbaserad och arbetsbaserad föräldrapenning lämnas förmånen efter den 180 dagen för
\newline 1. 210 dagar på sjukpenningnivån, dock lägst på grundnivån, och
\newline 2. 90 dagar på lägstanivån.
\subsection*{40 §}
\paragraph*{}
Till en förälder som är försäkrad för enbart bosättningsbaserad föräldrapenning lämnas förmånen efter den 180 dagen för
\newline 1. 210 dagar på grundnivån, och
\newline 2. 90 dagar på lägstanivån.
\subsection*{41 §}
\paragraph*{}
Till en förälder som är försäkrad för enbart arbetsbaserad föräldrapenning lämnas förmånen efter den 180 dagen för 210 dagar på sjukpenningnivån, dock lägst på grundnivån.
\subsection*{41 a §}
\paragraph*{}
Bestämmelserna i 35-41 §§ gäller inte för föräldrapenning för tid efter barnets fjärde levnadsår, räknat från barnets födelse eller därmed likställd tidpunkt.
För sådan tid tillämpas i stället 41 b-41 h §§.
Lag (2013:999).
\subsection*{41 b §}
\paragraph*{}
Till en förälder som är försäkrad för både bosättningsbaserad och arbetsbaserad föräldrapenning lämnas förmånen på sjukpenningnivån, dock lägst på grundnivån.
Lag (2013:999).
\subsection*{41 c §}
\paragraph*{}
Till en förälder som är försäkrad för enbart bosättningsbaserad föräldrapenning lämnas förmånen på grundnivån.
Lag (2013:999).
\subsection*{41 d §}
\paragraph*{}
Till en förälder som är försäkrad för enbart arbetsbaserad föräldrapenning lämnas förmånen på sjukpenningnivån, dock lägst på grundnivån.
Lag (2013:999).
\subsection*{41 e §}
\paragraph*{}
Antalet dagar för vilka föräldrapenning kan lämnas anges i 12 § tredje stycket.
\paragraph*{}
Vid gemensam vårdnad om ett barn ska det antal dagar som det kan lämnas föräldrapenning för fördelas på det sätt som föreskrivs i 15 a och 15 b §§.
Lag (2013:999).
\subsection*{41 f §}
\paragraph*{}
Om antalet dagar för vilka en förälder kan få föräldrapenning enligt 41 e § överstiger det antal dagar för vilka förmånen kan lämnas på sjukpenningnivå eller grundnivå, lämnas förmånen på lägstanivån för det överskjutande antalet dagar, om föräldern är försäkrad för förmånen på lägstanivå.
Lag (2013:999).
\subsection*{41 g §}
\paragraph*{}
Om en förälder för tid före utgången av barnets fjärde levnadsår, räknat från barnets födelse eller därmed likställd tidpunkt, har fått föräldrapenning som avses i 17 § under färre än 90 dagar, kan han eller hon inte avstå rätten att få föräldrapenning till förmån för den andra föräldern i fråga om en tid som motsvarar de 90 dagarna efter avdrag för det antal dagar under vilka han eller hon har fått sådan föräldrapenning.
Lag (2015:674).
\subsection*{41 h §}
\paragraph*{}
Om en förälder har rätt till föräldrapenning enbart under sådan tid som avses i 12 § tredje stycket, ska det som föreskrivs i 17 § tillämpas på föräldrapenning för den tiden.
Därvid ska 17 § andra stycket dock endast omfatta det antal dagar som föräldern kan få föräldrapenning för.
Lag (2013:999).
\subsection*{42 §}
\paragraph*{}
Till en förälder som är försäkrad för både bosättningsbaserad och arbetsbaserad föräldrapenning lämnas vid flerbarnsfödsel föräldrapenning för det andra barnet för ytterligare
\newline 1. 90 dagar på sjukpenningnivån, dock lägst på grundnivån, och
\newline 2. 90 dagar på lägstanivån.
\subsection*{43 §}
\paragraph*{}
Till en förälder som är försäkrad för enbart bosättningsbaserad föräldrapenning lämnas vid flerbarnsfödsel föräldrapenning för det andra barnet för ytterligare
\newline 1. 90 dagar på grundnivån, och
\newline 2. 90 dagar på lägstanivån.
\subsection*{44 §}
\paragraph*{}
Till en förälder som är försäkrad för enbart arbetsbaserad föräldrapenning lämnas vid flerbarnsfödsel föräldrapenning för det andra barnet för ytterligare 90 dagar på sjukpenningnivån, dock lägst på grundnivån.
\subsection*{45 §}
\paragraph*{}
För varje barn utöver det andra lämnas föräldrapenning för ytterligare 180 dagar enligt den ersättningsnivå som anges i
\newline 1. 42 § 1 till en förälder som är försäkrad för såväl bosättningsbaserad som arbetsbaserad föräldrapenning,
\newline 2. 43 § 1 till en förälder som är försäkrad för enbart bosättningsbaserad föräldrapenning, eller
\newline 3. 44 § till en förälder som är försäkrad för enbart arbetsbaserad föräldrapenning.
\subsection*{45 a §}
\paragraph*{}
När det gäller föräldrapenning för sådan tid som avses i 12 § andra stycket, ska vid flerbarnsfödsel föräldrapenning lämnas på sjukpenningnivå, dock lägst på grundnivå.
Lag (2017:559).
\subsection*{46 §}
\paragraph*{}
När det gäller föräldrapenning för sådan tid som avses i 41 a §, ska vid flerbarnsfödsel
\newline - det som föreskrivs om antal dagar och ersättningsnivåer i 42 § i stället avse 36 dagar enbart på sjukpenningnivån, dock lägst på grundnivån,
\newline - det som föreskrivs om antal dagar och ersättningsnivåer i 43 § i stället avse 36 dagar enbart på grundnivån,
\newline - det som föreskrivs om antal dagar i 44 § i stället avse 18 dagar, och
\newline - det som föreskrivs om antal dagar i 45 § i stället avse 36 dagar.
Lag (2013:999).
\chapter*{13 Tillfällig föräldrapenning}
\subsection*{1 §}
\paragraph*{}
I detta kapitel finns allmänna bestämmelser om rätten till tillfällig föräldrapenning i 2-9 §§.
\paragraph*{}
Vidare finns bestämmelser om
\newline - tillfällig föräldrapenning för tid före ansökan i 9 a §,
\newline - tillfällig föräldrapenning vid barns födelse eller adoption i 10-15 §§,
\newline - vård av barn som inte har fyllt 12 år i 16-21 §§,
\newline - vård av barn som har fyllt 12 år i 22-25 §§,
\newline - vård av barn som omfattas av lagen om stöd och service till vissa funktionshindrade i 26-29 §§,
\newline - vård av allvarligt sjukt barn i 30 och 31 §§,
\newline - utvidgad rätt till tillfällig föräldrapenning vid förälders sjukdom eller smitta i 31 a-31 d §§,
\newline - tillfällig föräldrapenning i samband med att ett barn har avlidit i 31 e och 31 f §§,
\newline - beräkning av antalet dagar med rätt till tillfällig föräldrapenning i 32 §, och
\newline - beräkning av tillfällig föräldrapenning i 33-38 §§.
Lag (2018:1628).
\subsection*{2 §}
\paragraph*{}
Rätt till tillfällig föräldrapenning har en försäkrad förälder som avstår från att utföra förvärvsarbete i samband med ett barns födelse eller behov av vård eller i samband med att ett barn har avlidit.
\paragraph*{}
Tillfällig föräldrapenning lämnas i de fall och under de närmare förutsättningar som anges i detta kapitel.
Lag (2010:2005).
\subsection*{2 a §}
\paragraph*{}
Särskilda bestämmelser om rätt till tillfällig föräldrapenning när denna förmån lämnas till en försäkrad som är arbetslös finns i 36 § 2.
Lag (2010:2005).
\subsection*{3 §}
\paragraph*{}
Tillfällig föräldrapenning får förutom i de fall som anges i 11 kap. 10 § första stycket lämnas till båda föräldrarna för samma barn och tid:
\newline 1. om föräldrarna följer med sitt barn till läkare när barnet lider av allvarlig sjukdom, och
\newline 2. om föräldrarna, som en del i behandlingen av sitt barn, behöver delta i läkarbesök eller i behandling som ordinerats av läkare.
\paragraph*{}
Tillfällig föräldrapenning får förutom i de fall som anges i 11 kap. 10 § andra stycket lämnas till flera föräldrar för samma barn och tid, om föräldrarna deltar i en kurs som ordnas av en sjukvårdshuvudman för att lära sig vårda barnet.
Lag (2018:1628).
\subsection*{3 a §}
\paragraph*{}
Särskilda bestämmelser om rätt till tillfällig föräldrapenning när denna förmån lämnas på grundval av inkomst av annat förvärvsarbete finns i 35 §.
Lag (2010:423).
\subsection*{4 §}
\paragraph*{}
Försäkringskassan får i förväg pröva om förutsättningarna för tillfällig föräldrapenning enligt 22 eller 27 § är uppfyllda. Ett sådant beslut är bindande för den tid som anges i beslutet men kan omprövas om de förhållanden som har lagts till grund för beslutet ändras.
\subsection*{5 §}
\paragraph*{}
Tillfällig föräldrapenning lämnas enligt följande förmånsnivåer:
\newline 1. Hel tillfällig föräldrapenning lämnas för dag när en förälder helt avstått från förvärvsarbete.
\newline 2. Tre fjärdedels tillfällig föräldrapenning lämnas för dag när en förälder förvärvsarbetat högst en fjärdedel av den tid han eller hon annars skulle ha arbetat.
\newline 3. Halv tillfällig föräldrapenning lämnas för dag när en förälder förvärvsarbetat högst hälften av den tid han eller hon annars skulle ha arbetat.
\newline 4. En fjärdedels tillfällig föräldrapenning lämnas för dag när en förälder förvärvsarbetat högst tre fjärdedelar av den tid han eller hon annars skulle ha arbetat.
\newline 5. En åttondels tillfällig föräldrapenning lämnas för dag när en förälder förvärvsarbetat högst sju åttondelar av den tid han eller hon annars skulle ha arbetat.
\subsection*{6 §}
\paragraph*{}
Som förvärvsarbete betraktas inte
\newline 1. vård av barn som har tagits emot för stadigvarande vård och fostran i förälderns hem, och
\newline 2. sådant förvärvsarbete som den försäkrade utför under tid för vilken han eller hon får sjukersättning enligt bestämmelserna i 37 kap. 3 §.
\paragraph*{}
Om det vid tillämpningen av första stycket 2 inte går att avgöra under vilken tid den försäkrade avstår från förvärvsarbete för att vårda sitt barn ska frånvaron i första hand anses som frånvaro från sådant förvärvsarbete som avses i 37 kap. 3 §.
\subsection*{7 §}
\paragraph*{}
Om en förälder får oavkortade löneförmåner under tid då han eller hon bedriver studier, likställs avstående från studier med avstående från förvärvsarbete vid tillämpning av bestämmelserna om tillfällig föräldrapenning. Detta gäller dock endast i den utsträckning föräldern går miste om löneförmånerna.
\subsection*{8 §}
\paragraph*{}
En förälder får överlåta rätt till tillfällig föräldrapenning för vård av ett barn till någon som är försäkrad för tillfällig föräldrapenning och som i stället för föräldern avstår från förvärvsarbete för vård av barnet.
Sådan överlåtelse får göras i följande fall:
\newline 1. i samband med sjukdom eller smitta hos barnet, och
\newline 2. i samband med sjukdom eller smitta hos barnets ordinarie vårdare, när det gäller barn som avses i 16-19, 22 och 26 §§.
\subsection*{9 §}
\paragraph*{}
Försäkringskassan får, efter medgivande av en förälder, besluta att en annan person som är försäkrad för tillfällig föräldrapenning och som i stället för föräldern avstår från förvärvsarbete ska få rätt till tillfällig föräldrapenning i de fall som anges i 8 §.
\paragraph*{}
Som villkor för detta gäller
\newline 1. dels att föräldern på grund av egen sjukdom eller smitta inte kan vårda barnet,
\newline 2. dels att föräldern inte får tillfällig föräldrapenning endast av det skälet att han eller hon för samma tid får sådan förmån som avses i 11 kap. 14 § eller smittbärarpenning.
\paragraph*{}
Ytterligare bestämmelser om tillfällig föräldrapenning vid en förälders sjukdom eller smitta finns i 31 a-31 d §§.
\subsection*{9 a §}
\paragraph*{}
Tillfällig föräldrapenning lämnas inte för längre tid tillbaka än 90 dagar före den dag ansökan kom in till Försäkringskassan.
\paragraph*{}
Detta gäller dock inte om det finns synnerliga skäl för att tillfällig föräldrapenning ändå bör lämnas. Tidsbegränsningen gäller inte heller i fråga om sådan tillfällig föräldrapenning som avses i 31 e §.
Lag (2018:1628).
\subsection*{10 §}
\paragraph*{}
Rätt till tillfällig föräldrapenning har en far som avstår från förvärvsarbete i samband med sitt barns födelse för att
\newline 1. närvara vid förlossningen,
\newline 2. sköta hemmet, eller
\newline 3. vårda barn.
\paragraph*{}
Det som föreskrivs om en far i första stycket och i 11-13 §§ gäller även en förälder enligt 1 kap. 9 § föräldrabalken.
\subsection*{11 §}
\paragraph*{}
I följande fall får Försäkringskassan besluta att en annan person som är försäkrad för tillfällig föräldrapenning och som i stället för en far eller mor avstår från sitt förvärvsarbete i samband med ett barns födelse ska ha rätt till tillfällig föräldrapenning för ändamål som anges i 10 §:
\newline 1. Barnet har inte någon far som har rätt till tillfällig föräldrapenning.
\newline 2. Barnets mor är avliden.
\newline 3. Barnets far avstår från sin rätt till tillfällig föräldrapenning enligt 10 § och det skulle vara oskäligt att inte låta honom avstå.
\newline 4. Barnets far kan inte utnyttja sin rätt till tillfällig föräldrapenning enligt 10 §.
\newline 5. Barnets far kommer sannolikt inte att utnyttja sin rätt enligt 10 § på grund av kontaktförbud enligt lagen (1988:688) om kontaktförbud eller liknande eller på grund av andra särskilda omständigheter.
Lag (2011:486).
\subsection*{12 §}
\paragraph*{}
Vid adoption lämnas tillfällig föräldrapenning enligt 10 och 11 §§, dock endast om barnet inte fyllt 10 år.
\subsection*{13 §}
\paragraph*{}
Vid adoption eller när två personer enligt 6 kap. 10 a § föräldrabalken har förordnats att gemensamt utöva vårdnaden om ett barn gäller följande:
\newline 1. Rätt till tillfällig föräldrapenning enligt 10 § tillkommer båda adoptivföräldrarna eller de särskilt förordnade vårdnadshavarna.
\newline 2. Det som i 11 § föreskrivs om en far eller mor ska i stället gälla adoptivföräldrarna eller de särskilt förordnade vårdnadshavarna.
\subsection*{14 §}
\paragraph*{}
Tillfällig föräldrapenning enligt 10-13 §§ lämnas under högst tio dagar per barn, dock inte för tid efter sextionde dagen efter barnets hemkomst efter förlossningen. Vid adoption räknas tiden från den tidpunkt föräldrarna fått barnet i sin vård.
\paragraph*{}
Är det fråga om tillfällig föräldrapenning enligt 11 § eller 13 § 2, ska dock avräkning ske för dagar med tillfällig föräldrapenning som en förälder kan ha fått med stöd av 10 § och 13 § 1.
\subsection*{15 §}
\paragraph*{}
Vid adoption och för särskilt förordnade vårdnadshavare fördelas de dagar som anges i 14 § med hälften till vardera föräldern eller vårdnadshavaren om de inte kommer överens om annat.
\paragraph*{}
Om det finns endast en adoptivförälder eller särskilt förordnad vårdnadshavare med rätt till ersättning, har den föräldern eller vårdnadshavaren ensam rätt till alla dagar som anges i 14 §.
\subsection*{16 §}
\paragraph*{}
En förälder har rätt till tillfällig föräldrapenning för vård av ett barn, som inte har fyllt 12 år, om föräldern behöver avstå från förvärvsarbete i samband med
\newline 1. sjukdom eller smitta hos barnet i annat fall än som avses i 30 §,
\newline 2. sjukdom eller smitta hos barnets ordinarie vårdare,
\newline 3. besök i samhällets förebyggande barnhälsovård, eller
\newline 4. vårdbehov som uppkommer till följd av att barnets andra förälder besöker läkare med ett annat barn till någon av föräldrarna, under förutsättning att det sistnämnda barnet omfattas av bestämmelserna om tillfällig föräldrapenning.
\subsection*{17 §}
\paragraph*{}
För vård av ett barn som är yngre än 240 dagar lämnas tillfällig föräldrapenning enligt 16 § endast om tillsynen av barnet är stadigvarande ordnad. Därutöver lämnas ersättning endast om barnet vårdas på sjukhus eller får motsvarande vård i hemmet.
\subsection*{18 §}
\paragraph*{}
För vård av ett barn som är 240 dagar eller äldre lämnas tillfällig föräldrapenning enligt 16 § inte för tid när föräldrapenning annars skulle ha lämnats. Detta gäller dock inte om barnet vårdas på sjukhus.
\subsection*{19 §}
\paragraph*{}
Med vård på sjukhus enligt 17 och 18 §§ likställs tillfällig vård i övergångsboende för barn som omfattas av 1 § lagen (1993:387) om stöd och service till vissa funktionshindrade.
\subsection*{20 §}
\paragraph*{}
En förälder till ett sjukt eller funktionshindrat barn, som inte har fyllt 12 år, har rätt till tillfällig föräldrapenning när föräldern behöver avstå från förvärvsarbete i samband med
\newline 1. besök på en institution för medverkan i behandling av barnet eller för att lära sig vårda barnet,
\newline 2. deltagande i en kurs som ordnas av sjukvårdshuvudman i samma syfte som anges i 1,
\newline 3. läkarbesök på grund av att barnet lider av allvarlig sjukdom,
\newline 4. läkarbesök som är en del i behandlingen av barnet, eller
\newline 5. deltagande i någon behandling som är ordinerad av läkare i samma syfte som anges i 4.
\subsection*{21 §}
\paragraph*{}
Tillfällig föräldrapenning enligt 16-20 §§ lämnas under sammanlagt högst 60 dagar för varje barn och år.
\paragraph*{}
Om föräldern behöver avstå från förvärvsarbete av skäl som anges i 16 § 1, 3 eller 4 eller 20 § lämnas tillfällig föräldrapenning under ytterligare högst 60 dagar för varje barn och år.
\subsection*{22 §}
\paragraph*{}
En förälder har rätt till tillfällig föräldrapenning för vård av ett barn som har fyllt 12 men inte 16 år om det är styrkt att barnet är i behov av särskild tillsyn eller vård på grund av
\newline 1. sjukdom i annat fall än som avses i 30 §,
\newline 2. utvecklingsstörning, eller
\newline 3. annat funktionshinder.
\subsection*{23 §}
\paragraph*{}
En förälder har rätt till tillfällig föräldrapenning enligt 22 § endast om han eller hon behöver avstå från sitt förvärvsarbete av skäl som anges i 16 eller 20 §.
\subsection*{24 §}
\paragraph*{}
För tid när föräldrapenning annars skulle ha lämnats har en förälder rätt till tillfällig föräldrapenning enligt 22 § endast om barnet vårdas på sjukhus.
\subsection*{25 §}
\paragraph*{}
Tillfällig föräldrapenning enligt 22 § lämnas under högst 60 dagar för varje barn och år.
\paragraph*{}
Om föräldern behöver avstå från förvärvsarbete av skäl som anges i 16 § 1, 3 eller 4 eller 20 § lämnas tillfällig föräldrapenning under ytterligare högst 60 dagar för varje barn och år.
\subsection*{26 §}
\paragraph*{}
En förälder till ett barn som omfattas av 1 § lagen (1993:387) om stöd och service till vissa funktionshindrade har även rätt till tillfällig föräldrapenning för kontaktdagar från barnets födelse till dess att det fyller 16 år. Detta gäller endast om föräldern avstår från förvärvsarbete i samband med
\newline 1. deltagande i föräldrautbildning,
\newline 2. besök i barnets skola, eller
\newline 3. besök i barnets förskola eller fritidshem eller i sådan pedagogisk verksamhet som avses i 25 kap. skollagen (2010:800) och som barnet deltar i.
Lag (2010:870).
\subsection*{27 §}
\paragraph*{}
En förälder till ett barn som omfattas av 1 § lagen (1993:387) om stöd och service till vissa funktionshindrade har rätt till tillfällig föräldra-penning för vård av barnet från det att barnet fyllt 16 år till dess att det fyller 21 år. Rätt till tillfällig föräldrapenning föreligger dock endast om föräldern behöver avstå från förvärvsarbete av skäl som anges i 16 § 1.
\paragraph*{}
Om barnet efter att ha fyllt 21 år går i sådan skola som avses i 15 kap. 36 § eller 18 kap. 8 § skollagen (2010:800) har föräldern rätt till tillfällig föräldrapenning för vård av barnet till och med vårterminen det år då barnet fyller 23 år.
Lag (2010:870).
\subsection*{28 §}
\paragraph*{}
Tillfällig föräldrapenning enligt 26 § lämnas under högst 10 dagar för varje barn och år.
\subsection*{29 §}
\paragraph*{}
Tillfällig föräldrapenning enligt 27 § lämnas under högst 60 dagar för varje barn och år.
\paragraph*{}
Om föräldern behöver avstå från förvärvsarbete av skäl som anges i 16 § 1, 3 eller 4 eller 20 § lämnas tillfällig föräldrapenning under ytterligare högst 60 dagar för varje barn och år.
\subsection*{30 §}
\paragraph*{}
Föräldrar till ett allvarligt sjukt barn som inte har fyllt 18 år har rätt till tillfällig föräldrapenning när de behöver avstå från förvärvsarbete för vård av barnet.
\subsection*{31 §}
\paragraph*{}
Tillfällig föräldrapenning enligt 30 § lämnas under ett obegränsat antal dagar.
\subsection*{31 a §}
\paragraph*{}
Försäkringskassan får besluta att en annan person som är försäkrad för tillfällig föräldrapenning och som avstår från förvärvsarbete ska få rätt till tillfällig föräldrapenning för att i stället för föräldern vårda ett barn som inte har fyllt tre år. Som villkor för detta gäller
\newline 1. att föräldern på grund av egen sjukdom eller smitta inte kan vårda barnet,
\newline 2. att föräldern enligt 12 kap. 14-16 §§ har rätt att själv uppbära föräldrapenningen eller skulle ha haft rätt att själv uppbära föräldrapenningen, och
\newline 3. att föräldern inte bor tillsammans med någon som kan beviljas tillfällig föräldrapenning med stöd av bestämmelserna i 11 kap. 4 § 1 eller 2.
Lag (2018:1952).
\subsection*{31 b §}
\paragraph*{}
Vid adoption lämnas tillfällig föräldrapenning enligt 31 a § längst till dess barnet har fyllt fem år.
\subsection*{31 c §}
\paragraph*{}
Tillfällig föräldrapenning enligt 31 a § får inte lämnas till den som på annan grund kan få föräldrapenning eller tillfällig föräldrapenning för vård av barnet.
\subsection*{31 d §}
\paragraph*{}
Tillfällig föräldrapenning enligt 31 a § lämnas under högst 120 dagar för varje barn och år.
\subsection*{31 e §}
\paragraph*{}
Föräldrar till ett barn som inte har fyllt 18 år har rätt till tillfällig föräldrapenning när de avstår från att utföra förvärvsarbete i samband med att barnet har avlidit.
Lag (2010:2005).
\subsection*{31 f §}
\paragraph*{}
Tillfällig föräldrapenning enligt 31 e § lämnas under högst 10 dagar per förälder och barn. Förmånen lämnas tidigast från och med dagen efter den då barnet har avlidit och senast för den dag som infaller 90 dagar efter den dag då barnet har avlidit.
Lag (2013:999).
\subsection*{32 §}
\paragraph*{}
Vid beräkning av antal dagar med rätt till tillfällig föräldrapenning gäller följande:
\newline - En dag med hel tillfällig föräldrapenning motsvarar en dag.
\newline - En dag med tre fjärdedels, halv, en fjärdedels eller en åttondels tillfällig föräldrapenning motsvarar tre fjärdedelar, hälften, en fjärdedel respektive en åttondel av en dag.
\subsection*{33 §}
\paragraph*{}
Tillfällig föräldrapenning beräknas enligt bestämmelserna om sjukpenning på normalnivån och sjukpenninggrundande inkomst i 25-28 kap. samt 34-38 §§ i detta kapitel, dock med undantag av bestämmelserna i
\newline - 25 kap. 5 § om bortseende från inkomst av anställning och annat förvärvsarbete överstigande 10,0 prisbasbelopp,
\newline - 27 kap. 27 §, 27 a § och 28 b § första stycket om karensavdrag och karensdagar,
\newline - 27 kap. 29-33 a §§ om karenstid, och
\newline - 28 kap. 7 § 2 om beräkningsunderlag för sjukpenning på fortsättningsnivån.
\paragraph*{}
Vid beräkning av tillfällig föräldrapenning ska det vid beräkningen av den sjukpenninggrundande inkomsten bortses från inkomst av anställning och annat förvärvsarbete till den del summan av dessa inkomster överstiger 7,5 prisbasbelopp. Det ska vid denna beräkning i första hand bortses från inkomst av annat förvärvsarbete.
\paragraph*{}
För hel tillfällig föräldrapenning motsvarar ersättningsnivån förälderns beräkningsunderlag för sjukpenning på normalnivån enligt 28 kap. 7 § 1 grundat på en sjukpenninggrundande inkomst beräknad enligt första och andra styckena (beräkningsunderlaget).
Lag (2021:1240).
\subsection*{34 §}
\paragraph*{}
Om inte annat följer av 35-38 §§ ska hel tillfällig föräldrapenning beräknas på grundval av beräkningsunderlaget med tillämpning av bestämmelserna i 28 kap. 13-16 §§.
\subsection*{35 §}
\paragraph*{}
Om tillfällig föräldrapenning lämnas på grundval av inkomst av annat förvärvsarbete, ska hel tillfällig föräldrapenning för dag motsvara kvoten mellan beräkningsunderlaget och 260.
\paragraph*{}
Beloppet avrundas till närmaste hela krontal, varvid 50 öre avrundas uppåt. Tillfällig föräldrapenning lämnas under högst fem kalenderdagar per sjudagarsperiod. För det fall föräldern avstår från förvärvsarbete under fler än fem kalenderdagar under en sjudagarsperiod, lämnas tillfällig föräldrapenning för de första fem dagarna i perioden. Sjudagarsperioden ska alltid beräknas med utgångspunkt i den dag för vilken ersättning begärs, varefter de närmast föregående sex dagarna räknas med i perioden.
Lag (2010:423).
\subsection*{36 §}
\paragraph*{}
Hel tillfällig föräldrapenning ska för dag motsvara kvoten mellan beräkningsunderlaget och 365, varvid beloppet avrundas till närmaste hela krontal och 50 öre avrundas uppåt
\newline 1. när den försäkrade ska få tillfällig föräldrapenning för tid då annars graviditetspenning, föräldrapenning eller rehabiliteringspenning skulle ha lämnats, och
\newline 2. när den försäkrade är arbetslös och anmäld som arbetssökande hos den offentliga arbetsförmedlingen samt är beredd att ta ett erbjudet arbete i en omfattning som svarar mot den bestämda sjukpenninggrundande inkomsten. Om det som nu föreskrivits skulle framstå som oskäligt, får kalenderdagsberäknad tillfällig föräldrapenning ändå lämnas.
Lag (2010:2005).
\subsection*{37 §}
\paragraph*{}
Om tillfällig föräldrapenning ska lämnas på grundval av sjukpenninggrundande inkomst av såväl anställning som annat förvärvsarbete beräknas den del av förmånen som svarar mot inkomst av anställning enligt 34 §, medan den del av förmånen som svarar mot inkomst av annat förvärvsarbete beräknas enligt 35 §.
\subsection*{38 §}
\paragraph*{}
Om en familjehemsförälder får ersättning för vården av barnet, ska det bortses från den del av den sjukpenninggrundande inkomsten som grundas på ersättningen.
\paragraph*{}
Vidare ska tillfällig föräldrapenning beräknas enligt 36 § i fall som avses i 28 kap. 6 § andra stycket.
Lag (2010:423).
\chapter*{14 Innehåll}
\subsection*{1 §}
\paragraph*{}
I denna underavdelning finns bestämmelser om
\newline - rätten till barnbidrag i 15 kap., och
\newline - vem som får barnbidraget i 16 kap.
\chapter*{15 Rätten till barnbidrag}
\subsection*{1 §}
\paragraph*{}
I detta kapitel finns bestämmelser om
\newline - förmånsformer i 2 §,
\newline - allmänt barnbidrag i 3 och 4 §§,
\newline - förlängt barnbidrag i 5-7 §§,
\newline - flerbarnstillägg i 8-12 §§,
\newline - ändrade förhållanden i 13 §, och
\newline - förlust av bidrag i 14 §.
\subsection*{2 §}
\paragraph*{}
Barnbidrag för ett försäkrat barn lämnas i form av
\newline - allmänt barnbidrag,
\newline - förlängt barnbidrag, och
\newline - flerbarnstillägg.
\subsection*{3 §}
\paragraph*{}
Allmänt barnbidrag lämnas med 1 250 kronor i månaden för varje barn från och med månaden efter barnets födelse.
Lag (2017:1308).
\subsection*{4 §}
\paragraph*{}
Allmänt barnbidrag lämnas till och med det kvartal då barnet fyller 16 år.
\subsection*{5 §}
\paragraph*{}
Förlängt barnbidrag lämnas med 1 250 kronor i månaden från och med kvartalet efter den tid som anges i 4 § för ett barn som går i
\newline 1. grundskolan, sameskolan, eller internationell skola på grundskolenivå, eller
\newline 2. anpassade grundskolan, anpassade gymnasieskolan eller specialskolan.
Lag (2023:348).
\subsection*{6 §}
\paragraph*{}
Förlängt barnbidrag lämnas för ett barn som går i skolan utomlands om svenskt statsbidrag betalas till skolan och om utbildningen i huvudsak motsvarar svensk grundskola.
\subsection*{7 §}
\paragraph*{}
Förlängt barnbidrag lämnas till och med den månad då barnet slutför utbildningen eller avbryter studierna.
\subsection*{8 §}
\paragraph*{}
Om någon enligt 16 kap. får barnbidrag för två eller flera barn lämnas flerbarnstillägg med
\newline 1. 150 kronor i månaden för det andra barnet,
\newline 2. 580 kronor i månaden för det tredje barnet,
\newline 3. 1 010 kronor i månaden för det fjärde barnet, och
\newline 4. 1 250 kronor i månaden för det femte barnet och varje ytterligare barn.
\paragraph*{}
I 16 kap. 12 § finns bestämmelser om anmälan för att få flerbarnstillägg.
Lag (2016:1294).
\subsection*{9 §}
\paragraph*{}
Följande barn berättigar till flerbarnstillägg:
\newline 1. barn för vilka allmänt barnbidrag lämnas,
\newline 2. barn för vilka rätt till förlängt barnbidrag föreligger, och
\newline 3. barn som bedriver studier som ger rätt till studiehjälp enligt 2 kap. studiestödslagen (1999:1395) om barnet, bortsett från åldern, uppfyller kraven för allmänt barnbidrag.
\subsection*{10 §}
\paragraph*{}
Vid tillämpningen av 9 § beaktas inte barn i familjehem, stödboende eller hem för vård eller boende som avses i 106 kap. 6 och 7 §§.
Lag (2015:983).
\subsection*{11 §}
\paragraph*{}
Barn som avses i 9 § 2 och 3 berättigar till flerbarnstillägg från och med det kvartal då studierna påbörjas och längst till och med det andra kvartalet det år då barnet fyller 20 år.
\paragraph*{}
Ett sådant barn berättigar inte till flerbarnstillägg om barnet
\newline - deltar endast i undervisning som omfattar kortare tid än åtta veckor eller deltidsutbildning,
\newline - är gift, eller
\newline - inte stadigvarande sammanbor med den som ska få flerbarnstillägg.
\subsection*{12 §}
\paragraph*{}
Vid beräkningen av flerbarnstillägg ska de barn för vilka någon får barnbidrag räknas samman med de barn för vilka någon annan får barnbidrag om dessa bidragsmottagare
\newline 1. är gifta med varandra och stadigvarande sammanbor, eller
\newline 2. är sambor och tidigare har varit gifta med varandra eller har eller har haft barn gemensamt.
\subsection*{13 §}
\paragraph*{}
Om det inträffar något som påverkar rätten till barnbidrag ska bidraget lämnas eller upphöra att lämnas från och med månaden efter förändringen.
\subsection*{14 §}
\paragraph*{}
Rätten till barnbidrag går förlorad om bidraget ännu inte har betalats ut under året efter det år som det hänför sig till. Detta gäller dock inte om den bidragsberättigade inom den tiden gör gällande sin rätt till bidraget hos Försäkringskassan.
\chapter*{16 Vem får barnbidraget?}
\subsection*{1 §}
\paragraph*{}
I detta kapitel finns bestämmelser
\newline - om bidragsmottagare i 2 och 3 §§,
\newline - vid ensam vårdnad i 4 §,
\newline - vid gemensam vårdnad i 5-8 §§,
\newline - om det finns särskilt förordnade vårdnadshavare i 9 §,
\newline - vid adoption i 11 §,
\newline - när flerbarnstillägg ska lämnas m.m. i 12-17 §§,
\newline - om utbetalning till annan i 18 §,
\newline - om ändring av bidragsmottagare i 19 §, och
\newline - om utbetalning av barnbidrag i 20 §.
Lag (2013:1018).
\subsection*{2 §}
\paragraph*{}
För att få barnbidrag krävs, utom i fall som avses i 18 § och 106 kap. 7 §, att bidragsmottagaren är försäkrad för barnbidrag.
\subsection*{3 §}
\paragraph*{}
Om inte annat följer av detta kapitel får ett barn som är myndigt eller gift självt barnbidraget.
\subsection*{4 §}
\paragraph*{}
Den som har ensam vårdnad om ett barn får barnbidraget.
\subsection*{5 §}
\paragraph*{}
När föräldrar har gemensam vårdnad om ett barn lämnas barnbidrag med hälften till vardera föräldern. Anger föräldrarna i en gemensam anmälan till Försäkringskassan vem av dem som ska vara bidragsmottagare lämnas dock barnbidraget till den angivna mottagaren.
\paragraph*{}
Är endast en av föräldrarna försäkrad för barnbidrag lämnas hela barnbidraget till den föräldern.
\paragraph*{}
Första stycket tillämpas inte om annat följer av 6, 7 eller 18 §.
Lag (2013:1018).
\subsection*{6 §}
\paragraph*{}
Om en förälder som enligt 5 eller 7 § ska få barnbidrag under en längre tid inte kan delta i vårdnaden på grund av frånvaro, sjukdom eller något annat skäl, övergår rätten att få barnbidrag till den andra föräldern.
Lag (2013:1018).
\subsection*{7 §}
\paragraph*{}
Har föräldrar, som inte bor tillsammans, gemensam vårdnad om ett barn gäller följande.
\newline 1. När barnet bor varaktigt tillsammans med endast en förälder lämnas barnbidraget till den föräldern, om han eller hon har gjort anmälan om det.
\newline 2. När barnet bor varaktigt hos båda föräldrarna (växelvist boende), lämnas barnbidraget med hälften till vardera föräldern efter anmälan av någon av dem.
Lag (2013:1018).
\subsection*{8 §}
\paragraph*{}
Ett barn ska anses ha växelvist boende, om sådant boende görs sannolikt av den förälder som har anmält att barnbidraget ska lämnas med hälften till vardera föräldern.
Lag (2013:1018).
\subsection*{9 §}
\paragraph*{}
För barn med två särskilt förordnade vårdnadshavare gäller det som föreskrivs i 5-8 §§ om barnets föräldrar i stället barnets vårdnadshavare.
Lag (2013:1018).
\subsection*{10 §}
\paragraph*{}
Har upphävts genom
lag (2013:1018).
\subsection*{11 §}
\paragraph*{}
Blivande adoptivföräldrar vid adoption av ett barn som inte är svensk medborgare och som inte är bosatt här i landet när de får barnet i sin vård likställs med föräldrar i fråga om rätten att få barnbidrag.
\subsection*{12 §}
\paragraph*{}
Den som får allmänt barnbidrag för de barn som omfattas av flerbarnstillägget får också flerbarnstillägg för dessa barn.
\paragraph*{}
Den som vill få flerbarnstillägg med stöd av 15 kap. 9 § 2 eller 3 ska anmäla detta till Försäkringskassan. Detta gäller dock inte om förlängt barnbidrag eller studiehjälp enligt 2 kap. studiestödslagen (1999:1395) lämnas för barnet.
\paragraph*{}
Den som vill få flerbarnstillägg med stöd av 15 kap. 12 § 1 behöver inte anmäla det. Den som vill få flerbarnstillägg med stöd av 15 kap. 12 § 2 ska anmäla det till Försäkringskassan, utom i de fall där två föräldrar har ett gemensamt barn och det för barnet lämnas någon sådan förmån eller något sådant stöd som anges i 15 kap. 9 §.
Lag (2018:1628).
\subsection*{13 §}
\paragraph*{}
Om allmänt barnbidrag har upphört att lämnas för samtliga barn som flerbarnstillägget avser, får barnens vårdnadshavare flerbarnstillägget. Om samtliga barn som berättigar till flerbarnstillägget är myndiga eller annars saknar vårdnadshavare, får den eller de föräldrar som de stadigvarande sammanbor med flerbarnstillägget.
\subsection*{14 §}
\paragraph*{}
Skulle två personer som bor tillsammans och som inte får barnbidrag för samma barn kunna få flerbarnstillägget, betalas detta till den av dem som anmäls som bidragsmottagare. Anmälan ska göras till Försäkringskassan av personerna gemensamt.
\paragraph*{}
Om inte någon anmälan enligt första stycket görs betalas flerbarnstillägget till vardera föräldern och beräknas för varje mottagare för sig med tillämpning av 16 och 17 §§.
Lag (2013:1018).
\subsection*{15 §}
\paragraph*{}
Om barnbidraget delas enligt 5 eller 7 § för ett eller flera barn ska flerbarnstillägget beräknas för varje förälder för sig enligt 16 och 17 §§.
Lag (2013:1018).
\subsection*{16 §}
\paragraph*{}
För de barn för vilka det lämnas ett helt barnbidrag ska varje förälders andel av flerbarnstillägget beräknas som produkten av
\newline a. kvoten mellan
\newline - flerbarnstillägget per månad för det antal barn som barnbidrag lämnas för och
\newline - det antal barn för vilka det lämnas barnbidrag och
\newline b. det antal barn för vilka det lämnas ett helt barnbidrag.
\paragraph*{}
För de barn för vilka det lämnas ett halvt barnbidrag ska varje förälders andel av flerbarnstillägget beräknas som produkten av
\newline a. kvoten mellan
\newline - flerbarnstillägget per månad för det antal barn som barnbidrag lämnas för och
\newline - det antal barn för vilka det lämnas barnbidrag och
\newline b. produkten av
\newline - 0,5 och
\newline - det antal barn för vilka det lämnas ett halvt barnbidrag.
\subsection*{17 §}
\paragraph*{}
Summan av de belopp som erhålls vid beräkning enligt 16 § är vad respektive förälder får i flerbarnstillägg om barnbidraget delas.
\subsection*{18 §}
\paragraph*{}
Om det finns särskilda skäl får barnbidraget, i stället för vad som framgår av 2-17 §§, på begäran av socialnämnden, betalas ut till den andra av föräldrarna, någon annan lämplig person eller nämnden att användas för barnets bästa.
Lag (2013:1018).
\subsection*{19 §}
\paragraph*{}
Om förhållandena ändras på ett sätt som är avgörande för vem som får barnbidraget ska ändringen ha verkan från och med månaden efter den då ändringen ägde rum.
\subsection*{20 §}
\paragraph*{}
Barnbidrag betalas ut månadsvis. Det månadsbelopp som ska betalas ut avrundas till närmast högre hela krontal.
\chapter*{17 Innehåll, definitioner och förklaringar}
\subsection*{1 §}
\paragraph*{}
I denna underavdelning finns bestämmelser om
\newline - underhållsstödet i 18 kap., och
\newline - bidragsskyldigas betalningsskyldighet mot Försäkringskassan i 19 kap.
\subsection*{2 §}
\paragraph*{}
Underhållsstöd kan lämnas till ett barn vars föräldrar inte bor tillsammans. En förälder som är underhållsskyldig för barnet kan bli skyldig att till Försäkringskassan betala hela eller delar av det stöd som lämnats till barnet.
\subsection*{3 §}
\paragraph*{}
När det gäller underhållsstöd avses med boförälder den av föräldrarna som barnet är folkbokfört hos om barnet bor varaktigt endast hos den föräldern.
\subsection*{4 §}
\paragraph*{}
Har upphävts genom
lag (2017:1123).
\subsection*{5 §}
\paragraph*{}
När det gäller underhållsstöd avses med bidragsskyldig den som enligt 7 kap. 2 § första stycket föräldrabalken ska fullgöra sin underhållsskyldighet genom att betala underhållsbidrag till barnet.
\subsection*{6 §}
\paragraph*{}
När det gäller underhållsstöd likställs en blivande adoptivförälder vid adoption av ett barn, som inte är svensk medborgare och som inte är bosatt här i landet när han eller hon får barnet i sin vård, med en förälder som har vårdnaden om ett barn.
\chapter*{18 Underhållsstödet}
\subsection*{1 §}
\paragraph*{}
I detta kapitel finns allmänna bestämmelser om rätten till underhållsstöd i 2-4, 6 och 7 §§.
\paragraph*{}
Vidare finns bestämmelser om
\newline - undantag från rätten till underhållsstöd i 8-12 §§,
\newline - förmånstiden i 13 och 14 §§,
\newline - vem som får utbetalningen av underhållsstödet i 15 och 17-19 §§,
\newline - beräkning av underhållsstöd i 20-24, 26-30 och 32 §§,
\newline - omprövning vid ändrade förhållanden i 33-35 §§,
\newline - jämkning av underhållsstöd i 36 §, och
\newline - handläggningen i 38 och 40-42 §§.
Lag (2017:1123).
\subsection*{2 §}
\paragraph*{}
Ett barn har rätt till underhållsstöd under de förutsättningar som anges i 4 §, om föräldrarna inte bor tillsammans och en av föräldrarna är boförälder.
\paragraph*{}
Om barnet har fyllt 18 år lämnas förlängt underhållsstöd, om även villkoren i 6 § är uppfyllda.
Lag (2017:1123).
\subsection*{3 §}
\paragraph*{}
När barnets föräldrar är folkbokförda på samma adress eller är gifta med varandra ska de anses bo tillsammans.
Detta gäller dock inte om den som begär underhållsstöd eller som stöd betalas ut till visar något annat.
\paragraph*{}
Om andra omständigheter gör det sannolikt att föräldrarna bor tillsammans, måste den som begär underhållsstöd eller som stöd betalas ut till visa att de inte gör det.
\subsection*{4 §}
\paragraph*{}
Underhållsstöd lämnas endast om barnets boförälder bor här i landet. Om barnet är underårigt krävs dessutom att boföräldern är vårdnadshavare för barnet.
\subsection*{5 §}
\paragraph*{}
Har upphävts genom
lag (2017:1123).
\subsection*{6 §}
\paragraph*{}
Ett barn som har fyllt 18 år och som inte har ingått äktenskap har rätt till förlängt underhållsstöd om han eller hon bedriver studier som ger rätt till förlängt barnbidrag eller till studiehjälp enligt 2 kap. studiestödslagen (1999:1395).
\paragraph*{}
Studier som omfattar kortare tid än åtta veckor eller deltidsstudier ger dock inte rätt till förlängt underhållsstöd.
\paragraph*{}
Lämnas förlängt underhållsstöd gäller det som föreskrivs om boföräldern i 23 § andra stycket den studerande.
Barn hos särskilt förordnade vårdnadshavare
\subsection*{7 §}
\paragraph*{}
Underhållsstöd och förlängt underhållsstöd lämnas även till ett barn som varaktigt bor och är folkbokfört hos
\newline 1. en eller två särskilt förordnade vårdnadshavare som bor här i landet, eller
\newline 2. någon eller några som bor här i landet och som var särskilt förordnade vårdnadshavare för barnet, när det fyllde 18 år.
\paragraph*{}
Det som i denna underavdelning föreskrivs om boförälder gäller då den eller de särskilt förordnade vårdnadshavarna.
Bestämmelserna i 3, 8 och 9 a §§ tillämpas inte i dessa fall.
Lag (2015:755).
\subsection*{8 §}
\paragraph*{}
Underhållsstöd lämnas inte om barnets mor är boförälder och hon uppenbarligen utan giltigt skäl låter bli att vidta eller medverka till åtgärder för att få faderskapet eller föräldraskapet enligt föräldrabalken till barnet fastställt.
\subsection*{9 §}
\paragraph*{}
Underhållsstöd lämnas inte om det finns anledning att anta att en bidragsskyldig förälder i rätt ordning betalar underhåll med minst det belopp som skulle betalas ut som underhållsstöd till barnet.
\paragraph*{}
Underhållsstöd lämnas inte heller om det är uppenbart att den bidragsskyldige föräldern på något annat sätt ser till att barnet får motsvarande underhåll.
\subsection*{9 a §}
\paragraph*{}
Underhållsstöd enligt 20 § lämnas inte längre om den bidragsskyldige under minst tolv månader i följd, i rätt ordning, till Försäkringskassan har betalat det belopp som har bestämts enligt 19 kap. 16-18 och 21-27 §§.
\paragraph*{}
Trots vad som föreskrivs i första stycket ska underhållsstöd lämnas, om det finns särskilda skäl. Underhållsstöd på grund av särskilda skäl lämnas under en tidsperiod om minst sex månader och högst fyra år. Ett beslut om att lämna underhållsstöd under en bestämd tidsperiod gäller dock längst till och med den månad då rätten till underhållsstöd annars upphör.
\paragraph*{}
Efter att en tidsperiod med underhållsstöd enligt andra stycket har löpt ut lämnas underhållsstöd enligt det underliggande beslutet om att bevilja underhållsstöd (grundbeslutet) om förutsättningarna för sådant stöd fort- farande är uppfyllda.
Lag (2021:991)
\subsection*{10 §}
\paragraph*{}
Underhållsstöd lämnas inte om boföräldern trots föreläggande enligt 19 kap. 30 § utan giltigt skäl låter bli att vidta eller medverka till de åtgärder som begärs.
\subsection*{11 §}
\paragraph*{}
Underhållsstöd lämnas inte om barnet har rätt till barnpension eller efterlevandestöd efter en bidragsskyldig förälder.
\subsection*{12 §}
\paragraph*{}
Förlängt underhållsstöd lämnas inte till ett barn i fall som avses i 8-11 §§. Det som anges om boföräldern i 8 och 10 §§ ska då i stället gälla barnet.
\subsection*{13 §}
\paragraph*{}
Underhållsstöd lämnas från och med månaden efter den månad när föräldrarna har flyttat isär eller rätt till stöd annars har uppkommit, dock inte för längre tid tillbaka än en månad före ansökningsmånaden.
\paragraph*{}
Underhållsstöd lämnas till och med månaden då barnet har fyllt 18 år eller den tidigare månad när rätten till stöd annars har upphört.
Förlängt underhållsstöd
\subsection*{14 §}
\paragraph*{}
Förlängt underhållsstöd lämnas från och med månaden efter det att den studerande har fyllt 18 år eller återupptagit studier som avses i 6 §. Förlängt underhållsstöd lämnas dock tidigast från och med månaden efter den månad när föräldrarna har flyttat isär eller rätt till stöd annars har uppkommit och inte för längre tid tillbaka än en månad före ansökningsmånaden.
Förlängt underhållsstöd lämnas till och med månaden när rätten till stöd upphör, dock längst till och med juni månad det år då barnet fyller 20 år.
Lag (2011:1075).
\subsection*{15 §}
\paragraph*{}
Underhållsstödet betalas ut till den som är boförälder.
\subsection*{16 §}
\paragraph*{}
Har upphävts genom
lag (2017:1123).
\subsection*{17 §}
\paragraph*{}
Om två särskilt förordnade vårdnadshavare har utsetts att ha vårdnaden gemensamt betalas underhållsstödet ut till den kvinnliga vårdnadshavaren. Om vårdnadshavarna begär det hos Försäkringskassan betalas underhållsstödet i stället ut till den manlige vårdnadshavaren.
\paragraph*{}
Om de särskilt förordnade vårdnadshavarna är av samma kön betalas underhållsstödet ut till den äldre av dem. Om vårdnadshavarna begär det betalas underhållsstödet i stället ut till den yngre av dem.
\subsection*{18 §}
\paragraph*{}
Det förlängda underhållsstödet betalas ut till den studerande.
\subsection*{19 §}
\paragraph*{}
Om det finns synnerliga skäl får underhållsstödet, i stället för vad som framgår av 15-18 §§, på begäran av socialnämnden betalas ut till någon annan lämplig person eller till nämnden att användas för barnets bästa.
\subsection*{20 §}
\paragraph*{}
Underhållsstöd till ett barn lämnas med
\newline - 1 673 kronor i månaden till och med månaden då barnet fyller 7 år,
\newline - 1 823 kronor i månaden från och med månaden efter den då barnet har fyllt 7 år till och med månaden då barnet fyller 15 år, och
\newline - 2 223 kronor i månaden från och med månaden efter den då barnet har fyllt 15 år.
\paragraph*{}
Första stycket gäller om inte något annat följer av 21-24 och 26-30 §§.
Lag (2021:1268).
\subsection*{21 §}
\paragraph*{}
Om det finns anledning att anta att en bidragsskyldig förälder i rätt ordning betalar underhåll till barnet med minst det belopp som skulle ha fastställts som betalningsbelopp enligt 19 kap. 10-17 samt 21, 26 och 27 §§, ska det sist avsedda beloppet räknas av från underhållstödet.
Lag (2012:896).
\subsection*{22 §}
\paragraph*{}
I stället för det som föreskrivs i 21 § ska följande gälla, om
\newline 1. den bidragsskyldige bor utomlands, eller
\newline 2. den bidragsskyldige bor i Sverige och i eller från utlandet får lön eller annan sådan inkomst som avses i 7 kap. 1 § utsökningsbalken men som inte kan tas i anspråk genom utmätning på sådant sätt som anges i samma kapitel.
\paragraph*{}
Finns det anledning att anta att den bidragsskyldige i rätt ordning betalar fastställt underhållsbidrag, ska detta belopp räknas av från underhållsstödet. Har underhållsbidraget fastställts med beaktande av att den bidragsskyldige till någon del fullgör sin underhållsskyldighet genom att ha barnet hos sig, ska underhållsstödet minskas med ett belopp motsvarande den del av underhållsskyldigheten som på detta sätt får anses ha beaktats.
\subsection*{23 §}
\paragraph*{}
Inträder Försäkringskassan i barnets rätt till underhållsbidrag enligt 19 kap. 29 § och är den bidragsskyldiges underhållsbidrag fastställt med beaktande av att han eller hon till någon del fullgör sin underhållsskyldighet genom att ha barnet hos sig, ska underhållsstödet minskas med ett belopp motsvarande den del av underhållsskyldigheten som på detta sätt får anses ha beaktats.
\paragraph*{}
Om det underhållsbidrag som har blivit fastställt uppenbart understiger vad den bidragsskyldige bör betala i underhållsbidrag till barnet och detta kan läggas boföräldern till last, lämnas inte underhållsstöd med högre belopp än underhållsbidraget.
\subsection*{24 §}
\paragraph*{}
Om sökanden begär det, lämnas underhållsstöd med det belopp som ska lämnas enligt 20 § första stycket med avräkning för det betalningsbelopp som skulle ha fastställts om 19 kap. 10-17 samt 21, 26 och 27 §§ hade tillämpats på inkomsten för den av föräldrarna som inte är boförälder.
Lag (2017:995).
\subsection*{25 §}
\paragraph*{}
Har upphävts genom
lag (2017:1123).
\subsection*{26 §}
\paragraph*{}
Om ett barn bor hos en eller två särskilt förordnade vårdnadshavare eller hos någon eller några som var särskilt förordnade vårdnadshavare när barnet fyllde 18 år, lämnas dubbla underhållsstöd enligt 20 och 21 §§.
\subsection*{27 §}
\paragraph*{}
Om underhållsbidrag har fastställts enligt föräldrabalken i form av ett engångsbelopp, ska avräkning från underhållsstödet ske med vad det fastställda underhållsbidraget skäligen kan anses motsvara i underhåll per månad.
\subsection*{28 §}
\paragraph*{}
Om en bidragsskyldig förälder, innan han eller hon har delgetts beslutet om betalningsskyldighet enligt 19 kap., har betalat underhåll till barnet för en viss månad, ska motsvarande belopp dras av från underhållsstödet för den månaden.
\subsection*{29 §}
\paragraph*{}
Har den bidragsskyldige, när hans eller hennes betalningsskyldighet bestäms, enligt 19 kap. 22-25 §§ fått tillgodoräkna sig ett avdrag för barnets vistelse hos honom eller henne, ska underhållsstödet minskas med ett belopp motsvarande avdraget.
Lag (2015:755).
\subsection*{30 §}
\paragraph*{}
Om barnet har en sådan inkomst som framkommer vid en tillämpning som avser barnet av 19 kap. 10-15 §§, ska underhållsstödet minskas med hälften av inkomsten. I stället för den minskning som anges i 19 kap. 10 § ska inkomsten minskas med 60 000 kronor.
Lag (2017:995).
\subsection*{31 §}
\paragraph*{}
Har upphävts genom
lag (2017:1123).
\subsection*{32 §}
\paragraph*{}
Underhållsstöd betalas ut månadsvis i förskott.
\paragraph*{}
Om det belopp som ska betalas ut för ett barn till en sökande en viss månad är lägre än 50 kronor, bortfaller det. I övrigt avrundas belopp som slutar på öretal till närmast lägre krontal.
\subsection*{33 §}
\paragraph*{}
Försäkringskassan ska ompröva rätten till underhållsstöd, om något har inträffat som gör att underhållsstöd inte ska lämnas, eller att det ska lämnas med ett lägre belopp.
\subsection*{34 §}
\paragraph*{}
Underhållsstöd som beräknats enligt 21 eller 24 § ska omprövas när
\newline 1. det finns ett nytt beslut om slutlig skatt för den bidragsskyldige, eller
\newline 2. grunden för tillämplig procentsats enligt 19 kap. 16 och 17 §§ ändras.
\paragraph*{}
Föranleder det nya beslutet om skatt en ändring av underhållsstödet, ska ändringen gälla från och med februari året efter det år då beslutet meddelades. Vid ändring av grunden för tillämplig procentsats justeras underhållsstödets belopp från och med månaden efter den månad då Försäkringskassan fick kännedom om ändringen.
Lag (2011:1434).
\subsection*{35 §}
\paragraph*{}
Nedsättning av underhållsstöd enligt 30 § ska omprövas när det finns ett nytt beslut om slutlig skatt för barnet.
\paragraph*{}
Föranleder det nya beslutet om skatt en ändring av underhållsstödet, ska ändringen gälla från och med februari året efter det år då beslutet meddelades.
Lag (2017:1123).
\subsection*{36 §}
\paragraph*{}
Underhållsstöd som beräknats enligt 21 eller 24 § ska jämkas om det beslut om skatt som legat till grund för Försäkringskassans bedömning ändrats väsentligt.
\paragraph*{}
Jämkningen gäller från och med månaden efter den då Försäkringskassan fick kännedom om ändringen.
Lag (2011:1434).
\subsection*{37 §}
\paragraph*{}
Har upphävts genom
lag (2017:1123).
\subsection*{38 §}
\paragraph*{}
När underhållsstöd enligt 20, 21, 22, eller 24 § har sökts ska Försäkringskassan omedelbart sända meddelande om ansökan till den bidragsskyldige, under förutsättning att hans eller hennes vistelseort är känd eller går att ta reda på.
\paragraph*{}
Meddelandet ska innehålla en uppmaning till den bidragsskyldige att yttra sig muntligen eller skriftligen inom en viss tid, om han eller hon har något att invända mot ansökan eller har något att anföra i fråga om betalningsskyldighet för underhållsstöd enligt 19 kap. 2-5, 10-16 och 21-27 §§.
\subsection*{39 §}
\paragraph*{}
Har upphävts genom
lag (2017:1123).
\subsection*{40 §}
\paragraph*{}
När den bidragsskyldige begär beräkning enligt 19 kap. 22-25 §§ ska Försäkringskassan omedelbart sända meddelande till den andra föräldern. Meddelandet ska innehålla en uppmaning till den andra föräldern att yttra sig muntligen eller skriftligen inom en viss tid, om han eller hon har något att invända mot begäran.
\subsection*{40 a §}
\paragraph*{}
Inför Försäkringskassans prövning enligt 9 a § ska myndigheten sända meddelande till boföräldern och den bidragsskyldige och, vid förlängt underhållsstöd, även till den studerande. Meddelandet ska innehålla en uppmaning till mottagarna att inom en viss tid yttra sig muntligen eller skriftligen, om de har något att anföra i fråga om huruvida underhållsstöd ska lämnas fortsättningsvis. Försäkringskassan ska sända ett sådant meddelande även om myndigheten har bedömt att det finns sådana särskilda skäl som medför att underhållsstöd ändå ska lämnas.
Lag (2015:755).
\subsection*{41 §}
\paragraph*{}
När Försäkringskassan har meddelat beslut i ärendet, ska sökanden skriftligen underrättas om beslutet.
Beslut om att bevilja underhållsstöd, att fastställa betalningsskyldighet och beslut enligt 19 kap. 22-25 §§ eller 19 kap. 39 § ska delges den bidragsskyldige.
\subsection*{42 §}
\paragraph*{}
Beslut om att medge beräkning enligt 19 kap. 22-25 §§ och att minska underhållsstödet enligt 29 § ska delges boföräldern.
\subsection*{42 a §}
\paragraph*{}
När Försäkringskassan har fattat beslut med stöd av 9 a § ska boföräldern och den bidragsskyldige och, vid förlängt underhållsstöd, även den studerande skriftligen underrättas om beslutet.
Lag (2021:991).
\subsection*{43 §}
\paragraph*{}
Vid tillämpning av 41 § andra stycket och 42 § får kungörelsedelgivning enligt 47-51 §§ delgivningslagen (2010:1932) inte användas.
Lag (2010:1938).
\chapter*{19 Bidragsskyldigas betalningsskyldighet mot Försäkringskassan}
\subsection*{1 §}
\paragraph*{}
I detta kapitel finns bestämmelser om
\newline - betalningsskyldighet för underhållsstöd i 2-9 §§,
\newline - inkomstunderlag för betalningsskyldighet i 10-15 §§, - beräkning av betalningsbelopp i 16-27 §§,
\newline - när den bidragsskyldige bor utomlands eller får lön från utlandet i 28-33 §§,
\newline - omprövning och ändring av betalningsskyldighet i 34-39 §§, - anstånd i 40-44 §§,
\newline - eftergift i 45 och 46 §§,
\newline - ränta i 47 och 48 §§, och
\newline - indrivning i 49 §.
\subsection*{2 §}
\paragraph*{}
När underhållsstöd lämnas till ett barn och det finns en bidragsskyldig förälder, ska föräldern i förskott för varje månad betala ett belopp till Försäkringskassan som helt eller delvis motsvarar underhållsstödet.
\subsection*{3 §}
\paragraph*{}
Betalningsskyldighet ska fastställas av Försäkringskassan samtidigt som eller snarast efter det att ett beslut om underhållsstöd meddelas.
\subsection*{4 §}
\paragraph*{}
Ett beslut om betalningsskyldighet ska inte meddelas, om 18 kap. 21 eller 24 § tillämpas eller om underhållsskyldighet enligt föräldrabalken har fastställts i form av ett engångsbelopp.
\paragraph*{}
Om den bidragsskyldige är bosatt utomlands eller i Sverige och i eller från utlandet får lön eller annan inkomst som inte kan tas i anspråk genom utmätning enligt 7 kap. utsökningsbalken och en tillämpning av bestämmelserna i 28 och 29 §§ därför övervägs, behöver betalningsskyldighet inte fastställas.
Lag (2017:1123).
\subsection*{5 §}
\paragraph*{}
Betalningsskyldighet får inte beslutas för längre tid tillbaka än tre år före den dag då meddelande om en gjord ansökan sändes till den bidragsskyldige enligt 18 kap. 38 §.
\paragraph*{}
Betalningsskyldighet får inte beslutas för tid under vilken Försäkringskassan har trätt in i barnets rätt till underhållsbidrag enligt 29 §.
\subsection*{6 §}
\paragraph*{}
Om Försäkringskassan enligt 112 kap. 2 § har beslutat om underhållsstöd för tid till dess att slutligt beslut kan fattas, ska betalningsskyldighet fastställas för den bidragsskyldige för motsvarande tid.
\subsection*{7 §}
\paragraph*{}
Om underhållsstöd har lämnats och det pågår ett mål om fastställande av faderskap till barnet, får den man som är instämd åläggas betalningsskyldighet, om det finns sannolika skäl för att han är far till barnet. Ett beslut om betalningsskyldighet får dock inte meddelas om flera män är instämda i målet.
\paragraph*{}
Första stycket tillämpas också i fråga om föräldraskap enligt 1 kap. 9 § föräldrabalken.
\subsection*{8 §}
\paragraph*{}
Om den slutliga betalningsskyldigheten bestäms till ett lägre belopp än vad som för samma tid har betalats enligt 6 eller 7 §, ska skillnaden betalas ut till den bidragsskyldige. Bestäms den slutliga betalningsskyldigheten till ett högre belopp, ska den bidragsskyldige betala in skillnaden till Försäkringskassan.
\subsection*{9 §}
\paragraph*{}
Om betalningsskyldighet har fastställts slutligt för en man och denne senare frias från faderskapet till barnet, har han rätt att få tillbaka vad han har betalat jämte ränta enligt 5 § räntelagen (1975:635) från varje betalningsdag.
\paragraph*{}
Första stycket tillämpas också i fråga om föräldraskap enligt 1 kap. 9 § föräldrabalken.
\subsection*{10 §}
\paragraph*{}
Den bidragsskyldiges betalningsskyldighet beräknas på ett underlag som motsvarar den bidragsskyldiges inkomst enligt 11-15 §§ till den del den överstiger 120 000 kronor.
Lag (2017:995).
\subsection*{11 §}
\paragraph*{}
Den bidragsskyldiges inkomst beräknas i enlighet med det beslut om slutlig skatt enligt 56 kap. 2 § skatteförfarandelagen (2011:1244) som fattats närmast före februari det år betalningsskyldigheten avser och med utgångspunkt i
\newline 1. överskott i inkomstslaget tjänst enligt 10 kap. 16 § inkomstskattelagen (1999:1229),
\newline 2. överskott i inkomstslaget kapital beräknat enligt 13 §, och
\newline 3. överskott av en näringsverksamhet beräknad enligt 14 §.
Lag (2011:1434).
\subsection*{12 §}
\paragraph*{}
När betalningsskyldigheten fastställs för förfluten tid ska det beslut om skatt som förelåg den eller de månader som betalningsskyldigheten avser läggas till grund för inkomstberäkningen.
Lag (2011:1434).
\subsection*{13 §}
\paragraph*{}
Överskott i inkomstslaget kapital enligt 41 kap. 12 § inkomstskattelagen (1999:1229) ska ökas med gjorda avdrag i inkomstslaget, dock inte med
\newline 1. avdrag för kapitalförluster till den del de motsvarar kapitalvinster som tagits upp som intäkt enligt 42 kap. 1 § inkomstskattelagen, och
\newline 2. avdrag för uppskovsbelopp enligt 47 kap. inkomstskattelagen vid byte av bostad.
\paragraph*{}
Underskott i inkomstslaget kapital ska minskas med gjorda avdrag i inkomstslaget, dock inte med avdrag som avses i första stycket 1 och 2.
\paragraph*{}
Uppkommer ett överskott vid beräkningen ska detta minskas med schablonintäkt enligt 42 kap. 36 och 43 §§ samt 47 kap. 11 b § inkomstskattelagen.
Lag (2011:1288).
\subsection*{14 §}
\paragraph*{}
Överskott av en näringsverksamhet enligt 14 kap. 21 § inkomstskattelagen (1999:1229) ska
\paragraph*{}
ökas med
\newline 1. avdrag för underskott för tidigare beskattningsår enligt 40 kap. inkomstskattelagen,
\newline 2. avdrag enligt 16 kap. 32 § inkomstskattelagen för utgift för egen pension intill ett halvt prisbasbelopp,
\newline 3. avdrag för avsättning till periodiseringsfond enligt 30 kap. inkomstskattelagen, och
\newline 4. avdrag för avsättning till expansionsfond enligt 34 kap.
inkomstskattelagen,
samt minskas med
\newline 5. återfört avdrag för avsättning till periodiseringsfond, och
\newline 6. återfört avdrag för avsättning till expansionsfond.
\paragraph*{}
Underskottet av en näringsverksamhet som avses i första stycket ska minskas med avdrag som avses i första stycket 1-4 och ökas med återförda avdrag som avses i första stycket 5 och 6.
\subsection*{15 §}
\paragraph*{}
Till beloppen enligt 11-14 §§ ska läggas studiemedel i form av studiebidrag enligt 3 kap. studiestödslagen (1999:1395) och studiestartsstöd enligt lagen (2017:527) om studiestartsstöd, utom de delar som avser tilläggsbidrag.
Lag (2017:528).
\subsection*{16 §}
\paragraph*{}
Betalningsskyldighet ska för varje barn som har rätt till underhållsstöd bestämmas till ett visst belopp per år (betalningsbelopp). Beloppet ska motsvara det procenttal av den bidragsskyldiges inkomst som anges i 17 §.
\paragraph*{}
När procenttalet bestäms ska hänsyn tas till samtliga barn som den bidragsskyldige är underhållsskyldig för enligt 7 kap. 1 § föräldrabalken.
\subsection*{17 §}
\paragraph*{}
Om den bidragsskyldige är underhållsskyldig enligt 7 kap. 1 § föräldrabalken för ett, två eller tre barn utgör procenttalet 14, 11,5 respektive 10 procent.
\paragraph*{}
Om underhållsskyldigheten gäller fler än tre barn motsvarar procenttalet kvoten mellan
\newline - summan av antalet barn och 27 och
\newline - antalet barn.
\paragraph*{}
Procenttalet bestäms med högst två decimaler.
\subsection*{18 §}
\paragraph*{}
Underhåll som har betalats till barnet innan den bidragsskyldige har delgetts beslutet om betalningsskyldighet får, i den utsträckning som det svarar mot underhållsstödet, räknas av från vad som ska betalas till Försäkringskassan.
\subsection*{19 §}
\paragraph*{}
Har upphävts genom
lag (2015:755).
\subsection*{20 §}
\paragraph*{}
Har upphävts genom
lag (2015:755).
\subsection*{21 §}
\paragraph*{}
Betalningsskyldighet kan, i stället för vad som följer av 10-15 §§, bestämmas genom en skälighetsbedömning, om det är uppenbart att den bidragsskyldiges förvärvsförmåga väsentligt överstiger vad som motsvarar inkomsten beräknad enligt nämnda paragrafer och han eller hon inte visar godtagbar anledning till att förvärvsförmågan inte utnyttjas.
\subsection*{22 §}
\paragraph*{}
Har det i en lagakraftvunnen dom eller ett avtal som socialnämnden godkänt bestämts att den bidragsskyldige ska ha barnet hos sig under minst 30 hela dygn per kalenderår, ska på den bidragsskyldiges begäran ett avdrag tillgodoräknas när betalningsskyldigheten bestäms.
\paragraph*{}
Avdrag ska för varje helt dygn göras med 1/40 av det belopp per månad som den bidragsskyldige annars skulle betala enligt 10-17, 21, 26 och 27 §§.
\subsection*{23 §}
\paragraph*{}
Vid beräkning av antalet hela dygn enligt 22 § räknas även det dygn då barnets vistelse hos den bidragsskyldige upphör som ett helt dygn. Detta gäller dock inte om vistelsen börjar och upphör under samma dygn.
\subsection*{24 §}
\paragraph*{}
Rätt till avdrag enligt 22 § föreligger inte om det finns anledning att anta att umgänget i väsentlig mån understiger eller kommer att understiga det som fastställts i domen eller avtalet. Om det finns särskilda skäl får avdrag dock medges även i ett sådant fall.
\subsection*{25 §}
\paragraph*{}
Avdrag enligt 22 § medges från och med månaden efter den månad då anmälan om domen eller avtalet kom in till Försäkringskassan.
\subsection*{26 §}
\paragraph*{}
Betalningsbeloppet för ett barn får aldrig överstiga vad som betalas ut i underhållsstöd till barnet under den tid betalningsskyldigheten avser.
\subsection*{27 §}
\paragraph*{}
Om det belopp som ska betalas för ett barn under en månad slutar på öretal, avrundas beloppet till närmast lägre krontal. Om månadsbeloppet för ett barn blir lägre än 50 kronor, faller betalningsskyldigheten bort.
\subsection*{28 §}
\paragraph*{}
Om den bidragsskyldige är bosatt utomlands eller om den bidragsskyldige bor i Sverige och i eller från utlandet får lön eller annan inkomst som avses i 7 kap. 1 § utsökningsbalken och som inte kan tas i anspråk genom utmätning på sätt som anges i samma kapitel gäller det som föreskrivs i 29-33 §§.
\subsection*{29 §}
\paragraph*{}
Om underhållsbidrag är fastställt, inträder Försäkringskassan i barnets rätt till underhållsbidrag till den del det svarar mot utbetalt underhållsstöd.
\subsection*{30 §}
\paragraph*{}
Om underhållsbidrag inte är fastställt, kan Försäkringskassan förelägga boföräldern att vidta eller medverka till åtgärder för att få underhållsbidrag fastställt. Försäkringskassan kan också förelägga boföräldern att vidta eller medverka till åtgärder för att få fastställt ett underhållsbidrag som inte uppenbart understiger vad den bidragsskyldige bör betala.
\paragraph*{}
Om Försäkringskassan begär det ska boföräldern till Försäkringskassan ge in en skriftlig handling om fastställt underhållsbidrag som kan ligga till grund för indrivning av bidraget.
\subsection*{31 §}
\paragraph*{}
Om barnet har fyllt 18 år ska det som anges om boföräldern i 30 § i stället gälla barnet.
\subsection*{32 §}
\paragraph*{}
I fall som avses i 28 § tillämpas även 2, 18 och 40-46 §§ samt 49 § första stycket och 110 kap. 48 §. Det som där föreskrivs om betalningsskyldighet gäller då skyldighet att betala fastställt underhållsbidrag till Försäkringskassan. Betalning till Försäkringskassan ska dock ske allteftersom underhållsbidraget förfaller till betalning.
Lag (2015:755).
\subsection*{33 §}
\paragraph*{}
Försäkringskassan bör ge den som får föra barnets talan möjlighet att i samband med Försäkringskassans krav på betalning av underhållsbidrag utkräva sådan del av obetalda underhållsbidrag som överstiger underhållsstödet.
\subsection*{34 §}
\paragraph*{}
Betalningsbelopp enligt 10-17, 21, 26 och 27 §§ ska omprövas
\newline 1. när det finns ett nytt beslut om slutlig skatt,
\newline 2. när grunden för tillämplig procentsats enligt 17 § ändras, och
\newline 3. när ett högre underhållsstöd lämnas till ett barn med anledning av att barnet har fyllt 7 eller 15 år.
\paragraph*{}
En ändring av betalningsbeloppet vid omprövning enligt första stycket 1 ska gälla från och med februari året efter det år då beslutet om slutlig skatt meddelades. En ändring av betalningsbeloppet vid omprövning enligt första stycket 2 ska gälla från och med månaden efter den då grunden för tillämplig procentsats enligt 17 § ändrades, dock aldrig längre tid tillbaka än tre år före den dag då Försäkringskassan fick kännedom om ändringen. En ändring av betalningsbeloppet vid omprövning enligt första stycket 3 ska gälla från och med månaden efter den då barnet fyllde 7 eller 15 år.
Lag (2021:1268).
\subsection*{35 §}
\paragraph*{}
Betalningsskyldighet som bestämts enligt 16-27 §§ för viss tid kan på ansökan av den bidragsskyldige eller på initiativ av Försäkringskassan jämkas om det beslut om skatt som legat till grund för Försäkringskassans bedömning ändrats väsentligt. En fråga om jämkning på grund av ändrat beslut om skatt får dock aldrig tas upp till prövning efter utgången av sjätte året efter det beskattningsår som beslutet avser.
Lag (2011:1434).
\subsection*{36 §}
\paragraph*{}
Om betalningsskyldigheten enligt 35 § bestäms till ett lägre belopp än vad den bidragsskyldige i enlighet med tidigare beslut betalat för samma tid, ska skillnaden betalas ut till den bidragsskyldige. Detsamma gäller betald ränta som hänför sig till skillnadsbeloppet, om räntan uppgår till minst 100 kronor.
\paragraph*{}
Om betalningsskyldigheten bestäms till ett högre belopp än vad den bidragsskyldige i enlighet med tidigare beslut betalat för samma tid, ska den bidragsskyldige betala in skillnaden till Försäkringskassan.
\subsection*{37 §}
\paragraph*{}
Beslut om betalningsskyldighet som har bestämts enligt 16-21, 26 och 27 §§ får upphävas helt eller delvis om den bidragsskyldige bosatt sig utomlands. Detsamma gäller om den bidragsskyldige bor i Sverige och i eller från utlandet får lön eller annan inkomst som inte kan tas i anspråk genom utmätning enligt 7 kap. utsökningsbalken.
\subsection*{38 §}
\paragraph*{}
En bidragsskyldig vars betalningsskyldighet har bestämts enligt 22-25 §§ är skyldig att till Försäkringskassan omgående anmäla de ändringar i umgänget som beslutas av allmän domstol eller görs i avtal som godkänts av socialnämnden.
\paragraph*{}
När en anmälan har gjorts ska Försäkringskassan omgående göra en omprövning av betalningsskyldigheten.
\subsection*{39 §}
\paragraph*{}
Försäkringskassan ska upphäva ett beslut som har meddelats med stöd av bestämmelser i 22-25 §§, om den bidragsskyldige begär det eller om något har inträffat som medför att betalningsskyldigheten inte längre bör vara fastställd med beaktande av dom eller avtal om umgänge.
\paragraph*{}
Har ett beslut upphävts kan en ny begäran om beräkning enligt 22-25 §§ prövas tidigast två år från upphävandet.
\subsection*{40 §}
\paragraph*{}
Försäkringskassan får på ansökan av den bidragsskyldige helt eller delvis bevilja anstånd med att fullgöra betalningsskyldigheten.
\paragraph*{}
Ett beslut om anstånd gäller för högst ett år.
\subsection*{41 §}
\paragraph*{}
Anstånd enligt 40 § ska beviljas i den utsträckning som det behövs för att den bidragsskyldige ska få behålla vad som behövs för sitt eget och familjens underhåll. Bestämmelserna om förbehållsbelopp i 7 kap. 4 och 5 §§ utsökningsbalken ska då tillämpas. Försäkringskassan får vid denna prövning beakta också om den bidragsskyldige har andra lätt realiserbara tillgångar.
\paragraph*{}
Anstånd får också beviljas om det annars finns anledning till det med hänsyn till den bidragsskyldiges personliga eller ekonomiska förhållanden eller andra särskilda förhållanden.
\subsection*{42 §}
\paragraph*{}
Det belopp som beslutet om anstånd avser ska betalas före belopp som avser senare tid.
\subsection*{43 §}
\paragraph*{}
En bidragsskyldig som har en skuld till staten på grund av att han eller hon har haft anstånd ska, sedan anståndet har löpt ut, betala skulden månadsvis i delposter (avbetalning) med en tolftedel av ett belopp som motsvarar 1,5 gånger det betalningsbelopp som följer av 2-5, 10-17 och 21-27 §§ eller, om betalningsskyldigheten för något barn har upphört, skulle ha fastställts enligt dessa paragrafer, dock minst 150 kronor per barn och månad. Den bidragsskyldige behöver dock inte betala större delposter än att den bidragsskyldige får behålla vad han eller hon behöver för sitt eget och familjens underhåll enligt bestämmelserna i 7 kap. 4 och 5 §§ utsökningsbalken beräknat per månad.
\paragraph*{}
För belopp som överstiger det den bidragsskyldige är skyldig att betala enligt första stycket ska på ansökan av honom eller henne nytt beslut om anstånd meddelas.
\paragraph*{}
Om en delpost enligt första stycket inte är betald på förfallodagen, ska den och de därpå följande fem delposterna vara betalda senast fem månader efter det att den förstnämnda delposten skulle ha betalats. Om inte samtliga sex delposter är betalda inom den angivna tiden förfaller hela skulden till omedelbar betalning.
Lag (2012:896).
\subsection*{44 §}
\paragraph*{}
Försäkringskassan ska upphäva eller ändra ett beslut om anstånd, om det inträffar något som medför att anstånd inte längre bör gälla eller bör gälla i mindre omfattning.
\subsection*{45 §}
\paragraph*{}
Försäkringskassan får på eget initiativ eller på ansökan av den bidragsskyldige helt eller delvis efterge fordran i fråga om betalningsskyldighet och ränta, om det finns synnerliga skäl med hänsyn till den bidragsskyldiges personliga eller ekonomiska förhållanden.
\subsection*{46 §}
\paragraph*{}
Försäkringskassan får på ansökan av den bidragsskyldige helt eller delvis efterge fordran i fråga om betalningsskyldighet, som avser tid före den dag då hans faderskap till ett barn har fastställts genom bekräftelse eller dom, om det finns synnerliga skäl.
\subsection*{47 §}
\paragraph*{}
Om den bidragsskyldige inte betalar belopp som bestämts enligt 16-27 §§ i rätt tid, ska han eller hon betala ränta på skulden.
\paragraph*{}
Den årliga räntan enligt första stycket tas ut efter en räntesats som för varje kalenderår beräknas med ledning av emissionsräntorna för Riksgäldskontorets statsskuldväxlar och statsobligationer för de senaste tre åren. Regeringen eller den myndighet som regeringen bestämmer kan med stöd av 8 kap. 7 § regeringsformen meddela närmare föreskrifter om ränta enligt denna paragraf.
Lag (2015:755).
\subsection*{48 §}
\paragraph*{}
De belopp som den bidragsskyldige betalar in ska i första hand avräknas på upplupen ränta.
\subsection*{49 §}
\paragraph*{}
Försäkringskassan ska utan dröjsmål vidta åtgärder för att driva in fordringen, om den bidragsskyldige inte fullgör sin betalningsskyldighet.
\paragraph*{}
Beslut om betalningsskyldighet enligt 16-27 och 43 §§ samt ränta enligt 47 och 48 §§ får verkställas enligt bestämmelserna i utsökningsbalken.
Lag (2012:896).
\chapter*{20 Innehåll}
\subsection*{1 §}
\paragraph*{}
I denna underavdelning finns bestämmelser om
\newline - adoptionsbidrag i 21 kap., och
\newline - omvårdnadsbidrag i 22 kap.
Lag (2018:1265).
\chapter*{21 Adoptionsbidrag}
\subsection*{1 §}
\paragraph*{}
I detta kapitel finns bestämmelser om
\newline - rätten till adoptionsbidrag i 2-6 §§, och
\newline - vem av föräldrarna som får adoptionsbidraget i 7 §.
\subsection*{2 §}
\paragraph*{}
Adoptionsbidrag kan lämnas till försäkrade föräldrar för kostnader vid adoption av barn som inte är svenska medborgare och som inte är bosatta här i landet när föräldrarna får dem i sin vård.
\paragraph*{}
Bidraget lämnas med 75 000 kronor för varje barn.
Lag (2016:1294).
\subsection*{3 §}
\paragraph*{}
För att adoptionsbidrag ska lämnas krävs
\newline 1. att föräldrarna var bosatta här i landet såväl när de fick barnet i sin vård som när adoptionen blev giltig här, och
\newline 2. att ansökan om bidraget görs inom ett år från det att adoptionen blev giltig här i landet.
\subsection*{4 §}
\paragraph*{}
Adoptionsbidrag lämnas endast till den eller de föräldrar som har adopterat enligt beslut av svensk domstol.
\paragraph*{}
Med svensk domstols beslut likställs
\newline 1. ett utomlands meddelat beslut om adoption som gäller i Sverige enligt lagen (1997:191) med anledning av Sveriges tillträde till Haagkonventionen om skydd av barn och samarbete vid internationella adoptioner, och
\newline 2. ett utomlands meddelat beslut om adoption som gäller i Sverige enligt lagen (2018:1289) om adoption i internationella situationer.
Lag (2018:1290).
\subsection*{5 §}
\paragraph*{}
Adoptionsbidrag lämnas endast för barn som inte hade fyllt tio år när föräldrarna fick barnet i sin vård.
\paragraph*{}
Bidrag lämnas inte för adoption av en makes eller sambos barn.
Lag (2018:1290).
\subsection*{6 §}
\paragraph*{}
Bidrag lämnas endast för adoption som har förmedlats av en sammanslutning som är auktoriserad enligt lagen (1997:192) om internationell adoptionsförmedling.
\subsection*{7 §}
\paragraph*{}
Föräldrarna får hälften var av adoptionsbidraget om de inte begär en annan fördelning.
\chapter*{22 Omvårdnadsbidrag}
\subsection*{1 §}
\paragraph*{}
I detta kapitel finns bestämmelser om
\newline - personer som likställs med förälder i 2 §,
\newline - rätten till omvårdnadsbidrag i 3-7 §§,
\newline - förmånstiden i 8-12 §§,
\newline - beräkning av omvårdnadsbidrag i 13 §,
\newline - ansökan i 14 §,
\newline - omprövning i 15 och 16 §§, och
\newline - utbetalning i 17 §.
Lag (2018:1265).
\subsection*{2 §}
\paragraph*{}
Följande personer likställs med förälder när det gäller omvårdnadsbidrag:
\newline 1. förälders make som stadigvarande sammanbor med föräldern,
\newline 2. förälders sambo som tidigare har varit gift med eller har eller har haft barn med föräldern,
\newline 3. särskilt förordnad vårdnadshavare som har vård om barnet, och
\newline 4. blivande adoptivförälder vid adoption av ett barn som inte är svensk medborgare och inte är bosatt här i landet när den blivande adoptivföräldern får barnet i sin vård.
Lag (2018:1265).
\subsection*{3 §}
\paragraph*{}
En försäkrad förälder har rätt till omvårdnadsbidrag för ett försäkrat barn, om barnet på grund av funktionsnedsättning kan antas vara i behov av omvårdnad och tillsyn under minst sex månader i sådan omfattning som anges i 4 §.
\paragraph*{}
En försäkrad person som likställs med förälder enligt 2 § 1 eller 2 har rätt till omvårdnadsbidrag endast om föräldern inte har ansökt om eller beviljats bidraget. En försäkrad person som likställs med förälder enligt 2 § 3 har rätt till omvårdnadsbidrag i stället för föräldern.
\paragraph*{}
Omvårdnadsbidrag kan endast beviljas för högst två personer för samma barn och tid.
Lag (2018:1265).
\subsection*{4 §}
\paragraph*{}
Omvårdnadsbidraget lämnas som
\newline 1. en fjärdedels förmån om barnet har mer än måttliga behov av omvårdnad och tillsyn på grund av funktionsnedsättning,
\newline 2. halv förmån om barnet har stora behov av omvårdnad och tillsyn på grund av funktionsnedsättning,
\newline 3. tre fjärdedels förmån om barnet har mycket stora behov av omvårdnad och tillsyn på grund av funktionsnedsättning, och
\newline 4. hel förmån om barnet har särskilt stora behov av omvårdnad och tillsyn på grund av funktionsnedsättning.
Lag (2018:1265).
\subsection*{5 §}
\paragraph*{}
Vid bedömningen enligt 4 § av behovet av omvårdnad och tillsyn ska det bortses från de behov som en vårdnadshavare normalt ska tillgodose enligt föräldrabalken med hänsyn till barnets ålder, utveckling och övriga omständigheter. Vid denna bedömning ska det även bortses från behov som tillgodoses genom annat samhällsstöd.
Lag (2018:1265).
\subsection*{6 §}
\paragraph*{}
Om en förälder har flera barn som på grund av funktionsnedsättning har behov av omvårdnad och tillsyn, ska rätten till omvårdnadsbidrag grundas på en bedömning av de sammanlagda behoven.
\paragraph*{}
Första stycket gäller inte om det finns särskilda skäl mot en sådan bedömning.
Lag (2018:1265).
\subsection*{7 §}
\paragraph*{}
Om ett barns båda föräldrar ansöker om och har rätt till omvårdnadsbidrag, ska bidraget lämnas med hälften till vardera föräldern.
\paragraph*{}
Om föräldrarna begär en annan fördelning av omvårdnadsbidraget än den som anges i första stycket, ska bidraget lämnas till dem enligt deras begäran. Om föräldrarna är oense om fördelningen av omvårdnadsbidraget, ska bidraget fördelas mellan dem i fjärdedelar med hänsyn till var barnet bor eller vistas. En tillfällig förändring av ett barns vistelse ska dock inte påverka fördelningen av bidraget.
Lag (2018:1265).
\subsection*{8 §}
\paragraph*{}
Omvårdnadsbidrag lämnas från och med den månad när rätten till bidraget har inträtt, dock inte för längre tid tillbaka än tre månader före ansökningsmånaden. Omvårdnadsbidrag lämnas inte retroaktivt för en sådan månad för vilken det redan har lämnats bidrag för barnet utom till den del det avser en ökning av bidraget.
Lag (2018:1265).
\subsection*{9 §}
\paragraph*{}
Omvårdnadsbidrag lämnas till och med juni det år när barnet fyller 19 år eller den tidigare månad när rätten till förmånen annars upphör.
\paragraph*{}
Rätten till omvårdnadsbidraget får begränsas till att omfatta viss tid.
Lag (2018:1265).
\subsection*{10 §}
\paragraph*{}
Om föräldern tillfälligt är förhindrad att ge barnet omvårdnad, lämnas omvårdnadsbidrag under avbrott i omvårdnaden som varar högst sex månader.
\paragraph*{}
Om det finns särskilda skäl kan bidraget lämnas även under ett avbrott som varar ytterligare högst sex månader.
Lag (2018:1265).
\subsection*{11 §}
\paragraph*{}
Om det lämnas mer än en fjärdedels omvårdnadsbidrag för ett barn och barnet avlider, lämnas omvårdnadsbidraget till och med den åttonde månaden efter dödsfallet eller den tidigare månad när bidraget annars skulle ha upphört.
Lag (2018:1265).
\subsection*{12 §}
\paragraph*{}
Under den tid som anges i 11 § lämnas omvårdnadsbidrag som
\newline 1. halv förmån om bidraget vid tiden för dödsfallet lämnades som hel eller tre fjärdedels förmån, och
\newline 2. en fjärdedels förmån om bidraget vid tiden för dödsfallet lämnades som halv förmån.
\paragraph*{}
Om bidraget före dödsfallet lämnades till barnets båda föräldrar, ska det fortfarande lämnas i enlighet med den fördelning som redan är beslutad för föräldrarna.
Lag (2018:1265).
\subsection*{13 §}
\paragraph*{}
Helt omvårdnadsbidrag för ett år motsvarar 250 procent av prisbasbeloppet. Partiell förmån motsvarar tillämplig nivå enligt 4 § av helt bidrag.
Lag (2018:1265).
\subsection*{14 §}
\paragraph*{}
Ansökan om omvårdnadsbidrag kan göras gemensamt av föräldrarna, av var och en av dem för sig eller av endast den ena föräldern.
Lag (2018:1265).
\subsection*{15 §}
\paragraph*{}
Rätten till omvårdnadsbidrag ska omprövas
\newline 1. minst vartannat år, om det inte finns skäl för omprövning med längre mellanrum, eller
\newline 2. när förhållanden som påverkar rätten till bidraget ändras.
Lag (2018:1265).
\paragraph*{}
Omprövning enligt första stycket 2 ska dock inte göras vid enbart tillfälliga förändringar.
\subsection*{16 §}
\paragraph*{}
Ändring av omvårdnadsbidrag ska gälla från och med månaden närmast efter den månad när anledningen till ändringen uppkom. Gäller det ökning efter ansökan ska dock även 8 § beaktas.
Lag (2018:1265).
\subsection*{17 §}
\paragraph*{}
Omvårdnadsbidrag ska betalas ut månadsvis. När bidraget beräknas för månad ska det bidrag för år räknat som beräkningen utgår från avrundas till närmaste krontal som är jämnt delbart med tolv.
Lag (2018:1265).
\part*{C FÖRMÅNER VID SJUKDOM ELLER ARBETSSKADA}
\chapter*{23 Innehåll, definitioner och förklaringar}
\subsection*{1 §}
\paragraph*{}
I avdelning C finns bestämmelser om socialförsäkringsförmåner vid sjukdom eller arbetsskada.
\subsection*{2 §}
\paragraph*{}
Förmåner vid sjukdom eller arbetsskada enligt denna avdelning är
\newline - sjukpenning som lämnas vid sjukdom och nedsatt arbetsförmåga,
\newline - rehabiliteringsåtgärder vid sjukdom och nedsatt arbetsförmåga,
\newline - rehabiliteringsersättning i samband med rehabiliteringsåtgärder,
\newline - sjukersättning eller aktivitetsersättning när arbetsförmågan är långvarigt nedsatt,
\newline - arbetsskadeersättning vid skada i samband med förvärvsarbete,
\newline - statlig personskadeersättning vid skada i samband med vissa statliga eller kommunala verksamheter,
\newline - krigsskadeersättning till sjömän vid skada utomlands,
\newline - smittbärarersättning för inkomstförlust m.m., och
\newline - närståendepenning i samband med ledighet för vård av en svårt sjuk person.
\subsection*{3 §}
\paragraph*{}
I detta kapitel finns inledande bestämmelser om förmåner vid sjukdom och arbetsskada.
\paragraph*{}
Vidare finns bestämmelser om
\newline - sjukpenning m.m. i 24-28 a kap.,
\newline - rehabilitering och rehabiliteringsersättning i 29-31 a kap.,
\newline - sjukersättning och aktivitetsersättning i 32-37 kap.,
\newline - förmåner vid arbetsskada m.m. i 38-44 kap., och
\newline - särskilda förmåner vid smitta, sjukdom eller skada i 45-47 kap.
Lag (2011:1513).
\subsection*{4 §}
\paragraph*{}
En förmån enligt denna avdelning lämnas endast till den som har ett gällande försäkringsskydd för förmånen enligt 4-7 kap.
\paragraph*{}
Bestämmelser om anmälan och ansökan samt vissa gemensamma bestämmelser om förmåner och handläggning finns i 104-117 kap. (avdelning H).
\subsection*{5 §}
\paragraph*{}
Ärenden som avser förmåner enligt denna avdelning handläggs av Försäkringskassan.
\chapter*{24 Innehåll och inledande bestämmelser}
\subsection*{1 §}
\paragraph*{}
I denna underavdelning finns allmänna bestämmelser om sjukpenninggrundande inkomst och årsarbetstid i 25 kap.
\paragraph*{}
Vidare finns bestämmelser om
\newline - bestämmande och behållande av sjukpenninggrundande inkomst samt beräkning av årsarbetstiden i vissa situationer i 26 kap.,
\newline - sjukpenning i 27 och 28 kap., och
\newline - sjukpenning i särskilda fall i 28 a kap.
Lag (2011:1513).
\subsection*{2 §}
\paragraph*{}
Sjukpenning kan lämnas till en försäkrad som har nedsatt arbetsförmåga på grund av sjukdom. Ersättningens storlek är beroende av den försäkrades sjukpenninggrundande inkomst (SGI) och den omfattning i vilken dennes arbetsförmåga är nedsatt.
\paragraph*{}
Sjukpenning i särskilda fall enligt 28 a kap. kan dock lämnas även till en försäkrad för vilken det inte kan fastställas någon sjukpenninggrundande inkomst.
Lag (2011:1513).
\subsection*{3 §}
\paragraph*{}
En försäkrads sjukpenninggrundande inkomst ligger också till grund för beräkning av följande förmåner enligt denna balk:
\newline - graviditetspenning,
\newline - föräldrapenning på sjukpenningnivå,
\newline - tillfällig föräldrapenning,
\newline - rehabiliteringspenning enligt 31 kap.,
\newline - skadelivränta,
\newline - smittbärarpenning, och
\newline - närståendepenning.
Lag (2011:1513).
\chapter*{25 Allmänna bestämmelser om sjukpenninggrundande inkomst och årsarbetstid}
\subsection*{1 §}
\paragraph*{}
I detta kapitel finns grundläggande bestämmelser om sjukpenninggrundande inkomst i 2-6 §§.
\paragraph*{}
Vidare finns bestämmelser om
\newline - inkomst av anställning i 7 och 7 a §§,
\newline - inkomst av annat förvärvsarbete i 8-15 §§,
\newline - undantag för vissa inkomster i 16-24 §§, och
\newline - årsarbetstid i 25-31 §§.
Lag (2018:670).
\subsection*{2 §}
\paragraph*{}
Sjukpenninggrundande inkomst är den årliga inkomst i pengar som en försäkrad kan antas komma att tills vidare få för eget arbete antingen
\newline 1. som arbetstagare i allmän eller i enskild tjänst (inkomst av anställning), eller
\newline 2. på annan grund (inkomst av annat förvärvsarbete).
\subsection*{3 §}
\paragraph*{}
För att sjukpenninggrundande inkomst ska kunna fastställas för en person krävs att han eller hon är försäkrad för arbetsbaserade förmåner enligt 4 och 6 kap.
\paragraph*{}
För att sjukpenninggrundande inkomst ska kunna fastställas krävs dessutom att den försäkrades årliga inkomst
\newline 1. kommer från arbete i Sverige,
\newline 2. kommer från arbete som kan antas vara under minst sex månader i följd eller vara årligen återkommande, och
\newline 3. kan antas uppgå till minst 24 procent av prisbasbeloppet.
\subsection*{4 §}
\paragraph*{}
Av ett beslut om fastställande av sjukpenninggrundande inkomst ska det framgå hur stor den sjukpenninggrundande inkomsten är och i vilken utsträckning denna avser inkomst av anställning eller inkomst av annat förvärvsarbete.
\subsection*{5 §}
\paragraph*{}
När sjukpenninggrundande inkomst beräknas ska inkomst av anställning och inkomst av annat förvärvsarbete var för sig beräknas och avrundas till närmast lägre hundratal kronor.
\paragraph*{}
Vid beräkningen ska det bortses från inkomst av anställning och inkomst av annat förvärvsarbete till den del summan av dessa överstiger 10,0 prisbasbelopp. Det ska därvid i första hand bortses från inkomst av annat förvärvsarbete.
Lag (2021:1240).
\subsection*{6 §}
\paragraph*{}
Om den försäkrades förhållanden inte är kända för Försäkringskassan, ska beräkningen av den försäkrades sjukpenninggrundade inkomst grundas på
\newline 1. de upplysningar som Försäkringskassan kan få av den försäkrade eller dennes arbetsgivare, eller
\newline 2. den uppskattning av den försäkrades inkomster som gjorts vid inkomstbeskattningen.
Lag (2011:1434).
\subsection*{7 §}
\paragraph*{}
Ersättning med minst 1 000 kronor om året för utfört arbete för någon annans räkning ska räknas som inkomst av anställning. Detta gäller även om betalningsmottagaren inte är anställd av den som betalar ersättningen. Den som har betalat ut sådan ersättning ska då anses som arbetsgivare, och den som har utfört arbetet ska anses som arbetstagare. När utbetalningen avser statlig ersättning för arbete i etableringsjobb ska dock den för vars räkning arbetet har utförts anses som arbetsgivare.
\paragraph*{}
Det som anges i första stycket gäller inte om
\newline 1. ersättningen betalas ut i form av pension,
\newline 2. ersättningen ska räknas som inkomst av annat förvärvsarbete enligt 10-15 §§, eller
\newline 3. ersättningen omfattas av undantag enligt 16-24 §§.
Lag (2020:475).
\subsection*{7 a §}
\paragraph*{}
För en försäkrad som har inkomst av anställning, bedriver näringsverksamhet i aktiebolagsform och som är företagare i den mening som avses i 34-34 c §§ lagen (1997:238) om arbetslöshetsförsäkring ska, under en tid om 36 månader räknat från den månad då anmälan för registrering enligt 7 kap. 2 § skatteförfarandelagen (2011:1244) har gjorts eller borde ha gjorts (uppbyggnadsskedet), den sjukpenninggrundande inkomsten av anställningen i aktiebolaget beräknas till minst vad som motsvarar skälig avlöning för liknande arbete för annans räkning.
\paragraph*{}
Om den försäkrade tidigare har bedrivit samma verksamhet i en sådan företagsform som avses i 9 § första stycket ska uppbyggnadsskedet räknas från den tidpunkt som följer av den paragrafen.
Lag (2018:670).
\subsection*{8 §}
\paragraph*{}
Inkomst av näringsverksamhet ska räknas som inkomst av annat förvärvsarbete. Detsamma gäller sådan inkomst av arbete för egen räkning som utgör inkomst av tjänst.
\paragraph*{}
Det som anges i första stycket gäller endast i den utsträckning inkomsten inte ska räknas som inkomst av anställning. Vidare gäller det som anges i 9-17 och 19-24 §§.
\subsection*{9 §}
\paragraph*{}
För en försäkrad som har inkomst av annat förvärvsarbete och som bedriver näringsverksamhet i någon form ska, under en tid om 36 månader räknat från den månad då en anmälan för registrering enligt 7 kap. 2 § skatteförfarandelagen (2011:1244) har gjorts eller borde ha gjorts (uppbyggnadsskedet), den sjukpenninggrundande inkomsten från näringsverksamheten beräknas till minst vad som motsvarar skälig avlöning för liknande arbete för annans räkning. I det fall den försäkrade inte är skyldig att göra en anmälan för registrering, ska uppbyggnadsskedet räknas som om en sådan skyldighet hade funnits.
\paragraph*{}
Om den försäkrade efter det att anmälan om registrering enligt första stycket gjordes, borde ha gjorts eller ska anses som gjord, börjar bedriva samma verksamhet i en annan sådan företagsform som avses i första stycket ska uppbyggnadsskedet räknas från det att den första näringsverksamheten påbörjades.
\paragraph*{}
Om den försäkrade tidigare har bedrivit samma verksamhet i aktiebolagsform ska uppbyggnadsskedet räknas från den tidpunkt som följer av 7 a § första stycket.
Lag (2018:670).
\subsection*{10 §}
\paragraph*{}
Som inkomst av annat förvärvsarbete räknas ersättning för arbete om ersättningen betalas ut till en mottagare som är godkänd för F-skatt när ersättningen bestäms eller när den betalas ut.
\paragraph*{}
Om mottagaren är godkänd för F-skatt med villkor enligt 9 kap. 3 § skatteförfarandelagen (2011:1244), räknas ersättningen som inkomst av annat förvärvsarbete bara om godkännandet skriftligen åberopas.
Lag (2011:1434).
\subsection*{11 §}
\paragraph*{}
Den som i en handling som upprättas i samband med uppdraget har lämnat uppgift om godkännande för F-skatt ska anses ha ett sådant godkännande om handlingen även innehåller följande uppgifter:
\newline 1. utbetalarens och betalningsmottagarens namn och adress eller andra för identifiering godtagbara uppgifter, och
\newline 2. betalningsmottagarens personnummer, samordningsnummer eller organisationsnummer.
\paragraph*{}
Uppgiften om godkännande för F-skatt gäller även som sådant skriftligt åberopande av godkännandet som avses i 10 § andra stycket.
Lag (2011:1434).
\subsection*{12 §}
\paragraph*{}
Det som anges i 11 § gäller dock inte om den som betalar ut ersättningen känner till att uppgiften om godkännande för F-skatt är oriktig.
\paragraph*{}
Bestämmelser om påföljd för den som lämnar oriktig uppgift finns i skattebrottslagen (1971:69).
Lag (2011:1434).
\subsection*{13 §}
\paragraph*{}
Som inkomst av annat förvärvsarbete räknas ersättning för arbete om ersättningen betalas ut till en mottagare som inte är godkänd för F-skatt eller är godkänd för F-skatt med villkor enligt 9 kap. 3 § skatteförfarandelagen (2011:1244), om
\newline 1. utbetalaren är en fysisk person eller ett dödsbo,
\newline 2. ersättningen inte är en utgift i en näringsverksamhet som bedrivs av utbetalaren,
\newline 3. ersättningen tillsammans med annan ersättning för arbete från samma utbetalare under beskattningsåret kan antas bli mindre än 10 000 kronor,
\newline 4. utbetalaren och mottagaren inte har träffat en överenskommelse om att ersättningen ska anses som inkomst av anställning, och
\newline 5. det inte är fråga om sådan ersättning för arbete som avses i 12 kap. 16 § föräldrabalken.
Lag (2011:1434).
\subsection*{14 §}
\paragraph*{}
Som inkomst av annat förvärvsarbete räknas ersättning för arbete om ersättningen betalas ut från 1. ett handelsbolag till en delägare i handelsbolaget, eller 2. en europeisk ekonomisk intressegruppering till en medlem i intressegrupperingen.
\subsection*{15 §}
\paragraph*{}
Har upphävts genom
lag (2012:834).
\subsection*{17 §}
\paragraph*{}
Som sjukpenninggrundande inkomst räknas inte semesterlön till den del lönen överstiger vad som skulle ha betalats i lön för utfört arbete under motsvarande tid. Motsvarande begränsning gäller för semesterersättning.
\subsection*{18 §}
\paragraph*{}
Som sjukpenninggrundande inkomst räknas inte inkomst som avses i 10 kap. 3 § 1-3 inkomstskattelagen (1999:1229).
\subsection*{19 §}
\paragraph*{}
Som sjukpenninggrundande inkomst av anställning räknas inte ersättning som en idrottsutövare får från en ideell förening som har till ändamål att främja idrott och som uppfyller kraven i 7 kap. 4-6 och 10 §§ inkomstskattelagen (1999:1229), om ersättningen från föreningen under året kan antas bli mindre än ett halvt prisbasbelopp.
Lag (2013:949).
\subsection*{20 §}
\paragraph*{}
Som sjukpenninggrundande inkomst räknas inte följande ersättningar:
\newline 1. ersättning som anges i 1 § första stycket 1-5 och fjärde stycket lagen (1990:659) om särskild löneskatt på vissa förvärvsinkomster, och
\newline 2. ersättning enligt gruppsjukförsäkring eller trygghetsförsäkring vid arbetsskada som enligt 2 § första stycket lagen om särskild löneskatt på vissa förvärvsinkomster utgör underlag för nämnda skatt.
\paragraph*{}
Ersättning från vinstandelsstiftelse 21 § Som sjukpenninggrundande inkomst räknas inte ersättning från en stiftelse som har till väsentligt ändamål att tillgodose ekonomiska intressen hos dem som är eller har varit anställda hos en arbetsgivare som har lämnat bidrag till stiftelsen (vinstandelsstiftelse) om följande förutsättningar är uppfyllda: - ersättningen avser en sådan anställd som omfattas av ändamålet med vinstandelsstiftelsen,
\newline - ersättningen avser inte betalning för den anställdes arbete åt vinstandelsstiftelsen, och
\newline - de bidrag arbetsgivaren har lämnat till vinstandelsstiftelsen har varit avsedda att vara bundna under minst tre kalenderår och att på likartade villkor tillkomma en betydande andel av de anställda.
\paragraph*{}
Detta gäller även ersättning från en annan juridisk person med motsvarande ändamål som en vinstandelsstiftelse.
\subsection*{22 §}
\paragraph*{}
Om arbetsgivaren är ett fåmansföretag eller ett fåmanshandelsbolag gäller det som föreskrivs i 21 § inte ersättning som den juridiska personen lämnar till företagsledare eller delägare i företaget eller till en person som är närstående till någon av dem.
Med fåmansföretag, fåmanshandelsbolag, företagsledare och närstående avses detsamma som i inkomstskattelagen (1999:1229).
\subsection*{23 §}
\paragraph*{}
Som sjukpenninggrundande inkomst räknas inte sådan ersättning från en vinstandelsstiftelse som härrör från bidrag som arbetsgivaren har lämnat under något av åren 1988-1991.
\subsection*{24 §}
\paragraph*{}
Som sjukpenninggrundande inkomst räknas inte inkomst på grund av förvärvsarbete som den försäkrade utför under tid för vilken han eller hon får sjukersättning enligt bestämmelserna i 37 kap. 3 §.
\subsection*{25 §}
\paragraph*{}
Årsarbetstid ska beräknas för en försäkrad som har en sjukpenninggrundande inkomst som helt eller delvis avser anställning. Årsarbetstiden beräknas när den enligt särskilda bestämmelser har betydelse för beräkningen av en förmån. 26 § Årsarbetstiden är det antal timmar eller dagar per år som en försäkrad tills vidare kan antas komma att ha som ordinarie arbetstid eller motsvarande normal arbetstid i sitt förvärvsarbete.
\subsection*{27 §}
\paragraph*{}
Årsarbetstiden beräknas i dagar endast när den schablonberäknas. Regeringen eller den myndighet som regeringen bestämmer meddelar föreskrifter om schablonberäkning av årsarbetstiden.
\subsection*{28 §}
\paragraph*{}
När årsarbetstiden beräknas ska följande ledigheter likställas med förvärvsarbete: 1. ledighet för semester, dock inte om den försäkrade under ledigheten får semesterlön enligt semesterlagen (1977:480) och, enligt 15 § samma lag, kan begära att dagar då han eller hon är oförmögen till arbete på grund av sjukdom inte räknas som semesterdag,
\newline 2. ledighet under studietid för vilken oavkortade löneförmåner lämnas,
\newline 3. ledighet under tid då den försäkrade får ersättning för att delta i teckenspråksutbildning för vissa föräldrar (TUFF), och
\newline 4. ledighet för ferier eller för motsvarande uppehåll för lärare som är anställda inom utbildningsväsendet.
\subsection*{29 §}
\paragraph*{}
Årsarbetstiden avrundas till närmast hela timtal, varvid halv timme avrundas uppåt.
\subsection*{30 §}
\paragraph*{}
När den försäkrades förhållanden inte är kända för Försäkringskassan ska beräkningen av årsarbetstiden grundas på upplysningar som Försäkringskassan kan inhämta från den försäkrade eller dennes arbetsgivare eller uppdragsgivare.
\subsection*{31 §}
\paragraph*{}
I 26 kap. samt 28 kap. 8 och 9 §§ finns ytterligare bestämmelser om beräkning av årsarbetstid i vissa situationer.
\chapter*{26 Bestämmande och behållande av sjukpenninggrundande inkomst samt beräkning av årsarbetstiden i vissa situationer}
\subsection*{1 §}
\paragraph*{}
I detta kapitel finns bestämmelser om
\newline - när sjukpenninggrundande inkomst bestäms i 2 och 3 §§,
\newline - ändring av den sjukpenninggrundande inkomsten i 4-8 §§,
\newline - sjukpenninggrundande inkomst vid förvärvsavbrott (SGI- skyddad tid) i 9-18 §§,
\newline - sjukpenninggrundande inkomst och årsarbetstid i vissa situationer i 19-27 §§, och
\newline - årlig omräkning av sjukpenninggrundande inkomst vid förvärvsavbrott i vissa fall (SGI-skyddad tid) i 28-31 §§.
\subsection*{2 §}
\paragraph*{}
Sjukpenninggrundande inkomst bestäms för en försäkrad i samband med att han eller hon begär att få en förmån som beräknas på grundval av sjukpenninggrundande inkomst eller att det annars behövs för handläggningen av ett ärende.
\subsection*{3 §}
\paragraph*{}
Försäkringskassan ska på begäran av en försäkrad bestämma den försäkrades sjukpenninggrundande inkomst även om något ersättningsärende inte är aktuellt.
\subsection*{4 §}
\paragraph*{}
Den sjukpenninggrundande inkomsten ska ändras om Försäkringskassan har fått reda på att den försäkrades inkomstförhållanden eller andra omständigheter har ändrats på ett sätt som har betydelse antingen för rätten till en förmån som redan lämnas eller för storleken på förmånen.
\subsection*{5 §}
\paragraph*{}
Ändring enligt 4 § får inte avse ändring av den försäkrades inkomstförhållanden på grund av sådant förvärvsarbete som avses i 37 kap. 3 §.
\subsection*{6 §}
\paragraph*{}
En ändring av den fastställda sjukpenninggrundande inkomsten ska gälla från och med den dag då anledning till ändringen uppkom.
\paragraph*{}
Den ändrade sjukpenninggrundande inkomsten får dock läggas till grund för ersättning tidigast från och med första dagen i den ersättningsperiod som infaller i anslutning till att Försäkringskassan fått kännedom om inkomständringen.
\subsection*{7 §}
\paragraph*{}
Det som föreskrivs i 6 § andra stycket tillämpas inte om ändringen föranleds av att
\newline 1. sjukersättning eller aktivitetsersättning har beviljats den försäkrade eller att sådan ersättning som redan lämnats har ändrats på grund av att den försäkrades arbetsförmåga har ändrats,
\newline 2. ett beslut om vilandeförklaring av sjukersättning eller aktivitetsersättning enligt 36 kap. 13-15 §§ har upphört,
\newline 3. livränta enligt bestämmelserna i denna avdelning har beviljats den försäkrade eller sådan livränta som redan lämnas har ändrats, eller
\newline 4. tjänstepension har beviljats den försäkrade.
\subsection*{8 §}
\paragraph*{}
Bestämmelsen i 7 § 1 ska tillämpas även när den försäkrade skulle ha fått sådan ersättning som avses där i form av garantiersättning om han eller hon hade haft rätt till sådan ersättning enligt bestämmelserna i 35 kap. 4-15 §§ om försäkringstid.
Lag (2014:239).
\subsection*{9 §}
\paragraph*{}
SGI-skydd innebär att den sjukpenninggrundande inkomsten för tid då den försäkrade avbryter eller inskränker sitt förvärvsarbete av något skäl som anges i 11-18 a §§ (SGI-skyddad tid) beräknas med utgångspunkt i förhållandena närmast före avbrottet eller inskränkningen, om den sjukpenninggrundande inkomsten därigenom blir högre än om den hade beräknats med beaktande av förhållandena under avbrottet eller inskränkningen.
\paragraph*{}
Det som föreskrivs i denna paragraf gäller dock inte när 7 § 1, 3 eller 4 är tillämplig.
Lag (2013:949).
\subsection*{10 §}
\paragraph*{}
I 28-31 §§ finns bestämmelser om omräkning av den sjukpenninggrundande inkomsten under SGI-skyddad tid i vissa fall.
\subsection*{11 §}
\paragraph*{}
SGI-skydd gäller under tid då den försäkrade bedriver studier, för vilka han eller hon får studiestöd enligt studiestödslagen (1999:1395). SGI-skydd gäller även under tid då den försäkrade utan att få studiestöd bedriver studier på minst halvtid för vilka studiemedel får lämnas enligt studiestödslagen under förutsättning att studierna bedrivs på eftergymnasial nivå eller bedrivs från och med det andra kalenderhalvåret det år den försäkrade fyller 20 år.
\paragraph*{}
Regeringen eller den myndighet som regeringen bestämmer kan med stöd av 8 kap. 7 § regeringsformen meddela ytterligare föreskrifter om vilka studier som medför att skyddet är tillämpligt.
Lag (2018:670).
\subsection*{12 §}
\paragraph*{}
SGI-skydd gäller under tid då den försäkrade får periodiskt ekonomiskt stöd enligt särskilda avtal mellan arbetsmarknadens parter. Regeringen eller den myndighet som regeringen bestämmer meddelar ytterligare föreskrifter om förutsättningarna för att avtalen ska medföra att skyddet är tillämpligt.
\subsection*{13 §}
\paragraph*{}
SGI-skydd gäller under tid då den försäkrade
\newline 1. deltar i ett arbetsmarknadspolitiskt program och får aktivitetsstöd, utvecklingsersättning eller etableringsersättning, eller
\newline 2. står till arbetsmarknadens förfogande.
\paragraph*{}
Regeringen eller den myndighet som regeringen bestämmer kan med stöd av 8 kap. 7 § regeringsformen meddela
\newline 1. föreskrifter om undantag från kravet på att den som deltar i ett arbetsmarknadspolitiskt program ska få aktivitetsstöd, utvecklingsersättning eller etableringsersättning, och
\newline 2. föreskrifter om de villkor som gäller för att den försäkrade ska anses stå till arbetsmarknadens förfogande.
Lag (2017:585).
\subsection*{14 §}
\paragraph*{}
SGI-skydd gäller under tid då den försäkrade är gravid och avbryter eller inskränker sitt förvärvsarbete tidigast sex månader före barnets födelse eller den beräknade tidpunkten för födelsen.
\subsection*{15 §}
\paragraph*{}
SGI-skydd gäller under tid då den försäkrade helt eller delvis avstår från förvärvsarbete för vård av barn, om den försäkrade är förälder till barnet eller likställs med förälder enligt 1 § föräldraledighetslagen (1995:584) och barnet inte fyllt ett år.
\paragraph*{}
Det som föreskrivs i första stycket gäller även vid adoption av barn som inte fyllt tio år eller vid mottagande av sådant barn i avsikt att adoptera det, om mindre än ett år har förflutit sedan den försäkrade fick barnet i sin vård.
\subsection*{16 §}
\paragraph*{}
SGI-skydd gäller under tid då den försäkrade fullgör tjänstgöring enligt lagen (1994:1809) om totalförsvarsplikt eller genomgår militär utbildning inom Försvarsmakten som rekryt.
Lag (2010:467).
\subsection*{16 a §}
\paragraph*{}
Har upphävts genom
lag (2017:585).
\subsection*{17 §}
\paragraph*{}
/Upphör att gälla U:2025-12-01/
SGI-skydd gäller under tid då den försäkrade inte förvärvsarbetar av någon anledning som ger rätt till ersättning i form av
\newline 1. sjukpenning,
\newline 2. ersättning för arbetsresor i stället för sjukpenning,
\newline 3. rehabiliteringsersättning, eller
\newline 4. ersättning från arbetsskadeförsäkringen enligt 38-42 kap. som motsvarar ersättning enligt 1-3 eller någon annan jämförbar ekonomisk förmån.
\paragraph*{}
Om det inte finns skäl som talar emot det gäller SGI-skydd också under tid då den försäkrade har ansökt om sådan ersättning som anges i första stycket 1 och väntar på ett slutligt beslut av Försäkringskassan i ärendet.
\paragraph*{}
Första och andra styckena gäller endast för tid före 66 års ålder.
Lag (2022:878).
\subsection*{17 §}
\paragraph*{}
/Träder i kraft I:2025-12-01/
SGI-skydd gäller under tid då den försäkrade inte förvärvsarbetar av någon anledning som ger rätt till ersättning i form av
\newline 1. sjukpenning,
\newline 2. ersättning för arbetsresor i stället för sjukpenning,
\newline 3. rehabiliteringsersättning, eller
\newline 4. ersättning från arbetsskadeförsäkringen enligt 38-42 kap. som motsvarar ersättning enligt 1-3 eller någon annan jämförbar ekonomisk förmån.
\paragraph*{}
Om det inte finns skäl som talar emot det gäller SGI-skydd också under tid då den försäkrade har ansökt om sådan ersättning som anges i första stycket 1 och väntar på ett slutligt beslut av Försäkringskassan i ärendet.
\paragraph*{}
Första och andra styckena gäller endast för tid före uppnådd riktålder för pension.
Lag (2022:879).
\subsection*{18 §}
\paragraph*{}
SGI-skydd gäller under högst tre månader i följd för en försäkrad som avbryter sitt förvärvsarbete, oavsett om avsikten är att förvärvsavbrottet ska pågå längre tid.
\paragraph*{}
Första stycket gäller endast för tid före uppnådd riktålder för pension.
Lag (2022:879).
\subsection*{18 a §}
\paragraph*{}
SGI-skydd gäller under tid då den försäkrade deltar i korttidsarbete som berättigar arbetsgivaren till preliminärt stöd enligt lagen (2013:948) om stöd vid korttidsarbete.
Lag (2013:949).
\subsection*{19 §}
\paragraph*{}
Vid sjukdom gäller det som föreskrivs i andra stycket för en försäkrad som
\newline 1. bedriver studier som avses i 11 §,
\newline 2. får periodiskt ekonomiskt stöd som avses i 12 §, eller
\newline 3. deltar i ett arbetsmarknadspolitiskt program och får aktivitetsstöd, utvecklingsersättning eller etableringsersättning.
\paragraph*{}
Under den tid som avses i första stycket ska sjukpenningen för den försäkrade beräknas på en sjukpenninggrundande inkomst som har fastställts på grundval av enbart den inkomst av eget arbete som den försäkrade kan antas få under den tiden. Om den sjukpenninggrundande inkomsten då helt eller delvis är inkomst av anställning, ska årsarbetstiden beräknas på grundval av enbart det antal arbetstimmar som den försäkrade kan antas ha i det förvärvsarbetet under den aktuella tiden.
Lag (2017:585).
\subsection*{20 §}
\paragraph*{}
Det som föreskrivs i 19 § andra stycket tillämpas även vid sjukdom för en försäkrad som fullgör plikttjänstgöring eller militär utbildning inom Försvarsmakten som avses i 16 §.
Detta gäller dock endast vid utbildning som är längre än 60 dagar.
Lag (2010:467).
Behandling eller rehabilitering
\subsection*{21 §}
\paragraph*{}
För en försäkrad som får sådan behandling eller rehabilitering som avses i 27 kap. 6 § eller 29 kap. 2 § och då får livränta från arbetsskadeförsäkringen enligt 41 kap.
eller motsvarande ersättning enligt annan författning, gäller vid sjukdom det som anges i andra stycket.
Under den tid som livräntan betalas ut ska sjukpenningen beräknas på en sjukpenninggrundande inkomst som har fastställts på grundval av enbart den inkomst av eget arbete som den försäkrade kan antas få under denna tid.
\subsection*{22 §}
\paragraph*{}
För en försäkrad som avses i 28 kap. 6 § första stycket 1 eller 2, eller i den paragrafens tredje stycke, och som under studieuppehåll mellan vår- och hösttermin inte får studiesociala förmåner för studier som avses i 11 §, gäller vid sjukdom det som anges i andra stycket.
Under studieuppehållet ska sjukpenning beräknas på den sjukpenninggrundande inkomst som följer av 4-7, 9 och 10 §§ om sjukpenningen då blir högre än sjukpenning beräknad på den sjukpenninggrundande inkomsten enligt 19 §.
Lag (2010:2005).
\subsection*{22 a §}
\paragraph*{}
/Upphör att gälla U:2025-12-01/
Vid utgången av en period då en försäkrad helt eller delvis har fått sjukersättning, aktivitetsersättning eller livränta enligt 41, 43 eller 44 kap., ska den sjukpenninggrundande inkomsten motsvara den sjukpenninggrundande inkomst som den försäkrade skulle ha varit berättigad till omedelbart före en eller flera sådana perioder. Om ett år eller längre tid har förflutit från den tidpunkt när sjukersättningen, aktivitetsersättningen eller livräntan enligt 41, 43 eller 44 kap. började lämnas ska den sjukpenninggrundande inkomsten räknas om enligt 31 § för varje helt år som har förflutit.
\paragraph*{}
Första stycket gäller längst till och med månaden före den när den försäkrade fyller 66 år, om inte annat anges i 22 b §.
Lag (2022:878).
\subsection*{22 a §}
\paragraph*{}
/Träder i kraft I:2025-12-01/
Vid utgången av en period då en försäkrad helt eller delvis har fått sjukersättning, aktivitetsersättning eller livränta enligt 41, 43 eller 44 kap., ska den sjukpenninggrundande inkomsten motsvara den sjukpenninggrundande inkomst som den försäkrade skulle ha varit berättigad till omedelbart före en eller flera sådana perioder. Om ett år eller längre tid har förflutit från den tidpunkt när sjukersättningen, aktivitetsersättningen eller livräntan enligt 41, 43 eller 44 kap. började lämnas ska den sjukpenninggrundande inkomsten räknas om enligt 31 § för varje helt år som har förflutit.
\paragraph*{}
Första stycket gäller längst till och med månaden före den när den försäkrade uppnår riktåldern för pension, om inte annat anges i 22 b §.
Lag (2022:879).
\subsection*{22 b §}
\paragraph*{}
/Upphör att gälla U:2025-12-01/
Om livränta har lämnats till följd av arbetsskada som inträffat tidigast den månad då den försäkrade fyllde 66 år, gäller det som föreskrivs i 22 a § första stycket längst till och med månaden före den när den försäkrade fyller 69 år.
Lag (2022:878).
\subsection*{22 b §}
\paragraph*{}
/Träder i kraft I:2025-12-01/
Om livränta har lämnats till följd av arbetsskada som inträffat tidigast den månad då den försäkrade uppnådde riktåldern för pension, gäller det som föreskrivs i 22 a § första stycket längst till och med månaden före den när den försäkrade fyller 69 år.
Lag (2022:879).
\subsection*{23 §}
\paragraph*{}
Har upphävts genom
lag (2015:758).
\subsection*{24 §}
\paragraph*{}
För sådana personer som enligt 5 kap. 6 och 8 §§ anses bosatta i Sverige även under vistelse utomlands, ska den sjukpenninggrundande inkomsten vid återkomsten till Sverige motsvara lägst det belopp som utgjorde deras sjukpenninggrundande inkomst omedelbart före utlandsresan.
\subsection*{25 §}
\paragraph*{}
/Upphör att gälla U:2025-12-01/
Om en försäkrad får tjänstepension i form av ålderspension eller därmed likställd pension före utgången av den månad han eller hon fyller 66 år, ska sjukpenninggrundande inkomst fastställas endast om den försäkrade har ett förvärvsarbete som beräknas pågå under minst sex månader i följd. Den sjukpenninggrundande inkomsten ska då beräknas enligt bestämmelserna i 25 kap.
Lag (2022:878).
\subsection*{25 §}
\paragraph*{}
/Träder i kraft I:2025-12-01/
Om en försäkrad får tjänstepension i form av ålderspension eller därmed likställd pension före utgången av den månad då han eller hon uppnår riktåldern för pension, ska sjukpenninggrundande inkomst fastställas endast om den försäkrade har ett förvärvsarbete som beräknas pågå under minst sex månader i följd. Den sjukpenninggrundande inkomsten ska då beräknas enligt bestämmelserna i 25 kap.
Lag (2022:879).
\subsection*{26 §}
\paragraph*{}
Om en försäkrad som får sådan tjänstepension som avses i 25 § är helt eller delvis arbetslös och är arbetssökande, ska sjukpenninggrundande inkomst fastställas endast om tjänstepensionen understiger 60 procent av lönen närmast före pensionsavgången.
\subsection*{27 §}
\paragraph*{}
För en försäkrad som avses i 26 § ska den sjukpenninggrundande inkomsten beräknas på grundval av lönen närmast före pensionsavgången. Om den försäkrade inte har för avsikt att förvärvsarbeta i samma omfattning som tidigare, ska den sjukpenninggrundande inkomsten beräknas på grundval av den inkomst han eller hon kan antas få av arbete som svarar mot arbetsutbudet. Vid beräkningen ska det iakttas att den sjukpenninggrundande inkomsten inte får överstiga 10,0 prisbasbelopp.
\paragraph*{}
Vid beräkningen ska den försäkrades inkomst minskas med tjänstepensionen. Minskningen får dock inte leda till att en försäkrad som är endast delvis arbetslös får lägre sjukpenninggrundande inkomst än om beräkningen skulle ha gjorts enligt 25 §.
Lag (2021:1240).
\subsection*{28 §}
\paragraph*{}
När Försäkringskassan ska bestämma den sjukpenninggrundande inkomsten för en försäkrad som omfattas av bestämmelserna i 11-18 §§ och som helt eller delvis saknar anställning ska den sjukpenninggrundande inkomsten räknas om enligt 29-31 §§.
\paragraph*{}
Det som anges i första stycket gäller även en försäkrad vars anställning upphör under en pågående ersättningsperiod.
\subsection*{29 §}
\paragraph*{}
Den sjukpenninggrundande inkomsten ska räknas om när minst ett år har förflutit från den tidpunkt när den anställning upphörde som senast föranlett eller kunnat föranleda beräkning av sjukpenninggrundande inkomst. Därefter ska den sjukpenninggrundande inkomsten räknas om årligen, räknat ett år från den senaste omräkningen.
\paragraph*{}
Med den tidpunkt när anställningen upphörde likställs den tidpunkt då den försäkrade helt upphörde med annat förvärvsarbete än arbete som anställd.
\paragraph*{}
Efter omräkning enligt 22 a § likställs utgången av det senaste hela år för vilket omräkning skett med den tidpunkt när anställningen upphörde.
\subsection*{30 §}
\paragraph*{}
Den sjukpenninggrundande inkomsten av annat förvärvsarbete, grundad på annan inkomst än som avses i 25 kap. 15 §, ska räknas om för en försäkrad som inte har upphört med förvärvsarbetet.
\paragraph*{}
Omräkning ska göras under en pågående ersättningsperiod efter det att ett år har förflutit från ersättningsperiodens början. Därefter ska den sjukpenninggrundande inkomsten räknas om årligen, räknat ett år från den senaste omräkningen.
\subsection*{31 §}
\paragraph*{}
Omräkningen ska göras med den procentuella förändringen i det allmänna prisläget räknad från det senast fastställda talet för konsumentprisindex jämfört med motsvarande tal tolv månader dessförinnan. En omräkning som innebär en sänkning av den sjukpenninggrundande inkomsten ska inte beaktas.
\paragraph*{}
Den sjukpenninggrundande inkomsten får aldrig fastställas till ett belopp som överstiger 10,0 prisbasbelopp.
Lag (2021:1240).
\chapter*{27 Allmänna bestämmelser om sjukpenning}
\subsection*{1 §}
\paragraph*{}
I detta kapitel finns bestämmelser om
\newline - rätten till sjukpenning i 2-8 §§,
\newline - samordning med sjuklön i 9 §,
\newline - sjukpenning för anställda och vissa andra vid kortare sjukdomsfall, m.m. i 10-16 a §§,
\newline - sjukanmälan i 17 och 18 §§,
\newline - ersättningsnivåer i 19 §,
\newline - förmånstiden och karens i 20-38 §§,
\newline - allmänt högriskskydd i 39 och 39 a §§,
\newline - särskilt högriskskydd i 40-44 §§,
\newline - förmånsnivåer och arbetsförmåga i 45 §,
\newline - bedömning av arbetsförmågans nedsättning (rehabiliteringskedjan) i 46-55 b §§, och
\newline - arbetsgivarinträde m.m. i 56-61 §§.
Lag (2021:1240).
\subsection*{2 §}
\paragraph*{}
En försäkrad har rätt till sjukpenning vid sjukdom som sätter ned hans eller hennes arbetsförmåga med minst en fjärdedel.
\paragraph*{}
Med sjukdom likställs ett tillstånd av nedsatt arbetsförmåga som orsakats av sjukdom för vilken det lämnats sjukpenning, om tillståndet fortfarande kvarstår efter det att sjukdomen upphört.
\subsection*{3 §}
\paragraph*{}
Vid bedömningen av om den försäkrade är sjuk ska det bortses från arbetsmarknadsmässiga, ekonomiska, sociala och liknande förhållanden.
\subsection*{4 §}
\paragraph*{}
Sjukpenning lämnas som hel, tre fjärdedels, halv eller en fjärdedels förmån enligt 45 §.
\subsection*{5 §}
\paragraph*{}
För att underlätta den försäkrades återgång till arbete i anslutning till ett sjukdomsfall får, i stället för den sjukpenning som annars skulle ha lämnats, skälig ersättning lämnas för den försäkrades merutgifter för resor till och från arbetet.
\paragraph*{}
Ersättning lämnas endast om merutgifterna beror på att den försäkrades hälsotillstånd inte tillåter honom eller henne att utnyttja det färdsätt som han eller hon normalt använder för att ta sig till sitt arbete.
\subsection*{6 §}
\paragraph*{}
En försäkrad har rätt till sjukpenning även när han eller hon genomgår en medicinsk behandling eller medicinsk rehabilitering som syftar till att
\newline 1. förebygga sjukdom,
\newline 2. förkorta sjukdomstiden, eller
\newline 3. helt eller delvis förebygga eller häva nedsättning av arbetsförmågan.
\paragraph*{}
Som villkor för att sjukpenning ska lämnas gäller att den medicinska behandlingen eller medicinska rehabiliteringen har
\newline - ordinerats av läkare, och
\newline - ingår i en av Försäkringskassan godkänd plan.
\subsection*{7 §}
\paragraph*{}
Om sjukpenning lämnas enligt 6 § ska arbetsförmågan anses nedsatt i den utsträckning som den försäkrade på grund av behandlingen eller rehabiliteringen är förhindrad att förvärvsarbeta.
\subsection*{8 §}
\paragraph*{}
Om en familjehemsförälder får ersättning för vården av barn som omfattas av uppdraget för tid när sjukpenning kommer i fråga, bedöms rätten till sjukpenning med bortseende från ersättningen.
\subsection*{9 §}
\paragraph*{}
Sjukpenning lämnas inte på grundval av anställningsförmåner för tid som ingår i en sjuklöneperiod när den försäkrades arbetsgivare ska svara för sjuklön enligt lagen (1991:1047) om sjuklön.
\paragraph*{}
Sjukpenning lämnas inte heller på grundval av statlig ersättning för arbete i etableringsjobb för tid som ingår i en sjuklöneperiod när den försäkrades arbetsgivare i etableringsjobbet ska svara för sjuklön enligt samma lag.
Lag (2020:475).
\subsection*{9 a §}
\paragraph*{}
Regeringen eller den myndighet som regeringen bestämmer kan med stöd av 8 kap. 7 § regeringsformen meddela föreskrifter om att ersättning i form av sjukpenning ska lämnas till en försäkrad som får ett karensavdrag enligt 6 § andra stycket lagen (1991:1047) om sjuklön. Sådana föreskrifter kan endast meddelas vid extraordinära händelser i fredstid.
Lag (2020:189).
\subsection*{10 §}
\paragraph*{}
För de första 14 dagarna i en sjukperiod lämnas sjukpenning som svarar mot sjukpenninggrundande inkomst av anställning endast under förutsättning att den försäkrade skulle ha förvärvsarbetat om han eller hon inte hade varit sjuk.
\paragraph*{}
Med tid för förvärvsarbete enligt första stycket likställs tid som avses i 25 kap. 28 §.
\subsection*{11 §}
\paragraph*{}
Kan det inte utredas hur den försäkrade skulle ha förvärvsarbetat under sjukperiodens första 14 dagar gäller följande. Sjukpenningen får lämnas efter vad som kan anses skäligt med ledning av hur den försäkrade har förvärvsarbetat före sjukperioden, om det kan antas att den försäkrade skulle ha förvärvsarbetat i motsvarande omfattning under de första 14 dagarna i sjukperioden.
\subsection*{11 a §}
\paragraph*{}
Särskilda bestämmelser om rätt till sjukpenning under de första 14 dagarna i en sjukperiod när denna förmån lämnas till en försäkrad som är helt eller delvis arbetslös finns i 28 kap. 6 § tredje stycket.
Lag (2010:2005).
\subsection*{12 §}
\paragraph*{}
Det som föreskrivs i 10 § gäller även för tid efter de första 14 dagarna av sjukperioden för en försäkrad som bedriver sådana studier som avses i 26 kap. 11 §.
\subsection*{13 §}
\paragraph*{}
Det som föreskrivs i 10 § gäller även för tid efter de första 14 dagarna av sjukperioden för en försäkrad som får periodiskt ekonomiskt stöd enligt sådana särskilda avtal mellan arbetsmarknadens parter som avses i 26 kap. 12 §.
\subsection*{14 §}
\paragraph*{}
Det som föreskrivs i 10 § gäller även för tid efter de första 14 dagarna av sjukperioden för en försäkrad som, på det sätt som avses i 26 kap. 13 §,
\newline 1. deltar i ett arbetsmarknadspolitiskt program och får aktivitetsstöd, utvecklingsersättning eller etableringsersättning, eller
\newline 2. står till arbetsmarknadens förfogande.
Lag (2017:585).
\subsection*{15 §}
\paragraph*{}
Det som föreskrivs i 10 § gäller även för tid efter de första 14 dagarna av sjukperioden för en försäkrad som får sådan behandling som avses i 6 § eller 31 kap. 3 § och som under denna tid får livränta vid arbetsskada eller annan skada som avses i 41-44 kap.
\subsection*{16 §}
\paragraph*{}
Det som föreskrivs i 10 § gäller även för tid efter de första 14 dagarna för en försäkrad som fullgör tjänstgöring enligt lagen (1994:1809) om totalförsvarsplikt, om tjänstgöringen avser grundutbildning som är längre än 60 dagar.
\subsection*{16 a §}
\paragraph*{}
Med behovsanställd avses i detta kapitel en försäkrad som vid behov kallas in för att arbeta i en tidsbegränsad anställning eller som är anställd för att arbeta vid behov.
\paragraph*{}
Sjukpenning till en försäkrad i egenskap av behovsanställd lämnas under de första 90 dagarna i sjukperioden endast under förutsättning att det kan antas att den försäkrade skulle ha förvärvsarbetat om han eller hon inte hade varit sjuk.
Lag (2021:1240).
\subsection*{17 §}
\paragraph*{}
Sjukpenning får inte lämnas för längre tid tillbaka än sju dagar före den dag då anmälan om sjukdomsfallet gjordes till Försäkringskassan. Detta gäller dock inte om det har funnits hinder mot att göra en sådan anmälan eller det finns särskilda skäl för att förmånen ändå bör lämnas.
Lag (2013:747).
\subsection*{18 §}
\paragraph*{}
Om den försäkrades arbetsgivare ska anmäla sjukdomsfallet enligt 12 § första stycket lagen (1991:1047) om sjuklön, ska sjukpenning som grundas på inkomst av anställning lämnas utan hinder av det som anges i 17 §.
\subsection*{19 §}
\paragraph*{}
Sjukpenning lämnas på
\newline - normalnivå, eller
\newline - fortsättningsnivå.
\paragraph*{}
Sjukpenning på normalnivån beräknas på ett beräkningsunderlag enligt 28 kap. 7 § 1 och sjukpenning på fortsättningsnivån beräknas på ett beräkningsunderlag enligt 28 kap. 7 § 2.
\subsection*{20 §}
\paragraph*{}
Sjukpenning kan lämnas för dagar i en sjukperiod så länge den försäkrade uppfyller förutsättningarna för rätt till sjukpenning inom den tid som anges i 21-24 §§.
Lag (2015:963).
\subsection*{21 §}
\paragraph*{}
Sjukpenning på normalnivån lämnas för högst 364 dagar under en ramtid som omfattar de 450 närmast föregående dagarna. Om ett avdrag enligt 27 § ska göras och sjukperioden inleds med sjukpenning, ska den första dagen med sjukpenning inte räknas in i de 364 dagar som sjukpenning får lämnas för på normalnivån under ramtiden.
Lag (2018:647).
\subsection*{22 §}
\paragraph*{}
Om den försäkrade inom ramtiden redan har fått sjukpenning för maximalt antal dagar på normalnivån, kan sjukpenning lämnas enligt bestämmelserna i 24 §. Vid beräkningen av antalet dagar med sjukpenning på normalnivån anses som sådana dagar även dagar med
\newline 1. sjukpenning på fortsättningsnivån, och
\newline 2. rehabiliteringspenning enligt 31 kap.
\paragraph*{}
Som dagar med sjukpenning på normalnivån räknas vidare tretton dagar under sådana perioder som avses i 26 § andra stycket.
Lag (2018:647).
\subsection*{23 §}
\paragraph*{}
Om den försäkrade har en allvarlig sjukdom lämnas sjukpenning på normalnivån trots att sådan sjukpenning redan har lämnats för maximalt antal dagar under ramtiden. I sådant fall tillämpas inte bestämmelserna i 22 och 24 §§.
\paragraph*{}
Det som föreskrivs i första stycket gäller när den försäkrades arbetsförmåga till minst en fjärdedel är nedsatt till följd av en allvarlig sjukdom.
Lag (2018:647).
\subsection*{24 §}
\paragraph*{}
Sjukpenning på fortsättningsnivån lämnas om sjukpenning på normalnivån inte kan lämnas på grund av det som föreskrivs i 21 §. Detta gäller även för dagar i en ny sjukperiod.
Lag (2015:963).
\subsection*{24 a §}
\paragraph*{}
Har upphävts genom
lag (2015:963).
\subsection*{25 §}
\paragraph*{}
Den försäkrade ska styrka nedsättningen av arbetsförmågan på grund av sjukdom senast från och med den sjunde dagen efter sjukperiodens första dag genom att lämna in ett läkarintyg till Försäkringskassan.
\paragraph*{}
Regeringen eller den myndighet som regeringen bestämmer meddelar föreskrifter dels om undantag från skyldigheten att lämna läkarintyg enligt första stycket när ett sådant intyg inte behövs, dels om att skyldigheten enligt första stycket ska gälla från och med en annan dag.
Lag (2013:747).
\subsection*{26 §}
\paragraph*{}
Som sjukperiod anses tid då en försäkrad
\newline 1. i oavbruten följd lider av sjukdom som avses i 2 §,
\newline 2. har rätt till sjukpenning enligt 6 §,
\newline 3. har rätt till rehabiliteringsersättning enligt 31 kap. 2 och 3 §§, eller
\newline 4. får sjukpenning eller sjukpenning i särskilda fall enligt 112 kap. 2 a §.
\paragraph*{}
Om rätt till sjukpenning för den försäkrade uppkommer i omedelbar anslutning till en period med lön enligt 34 § sjömanslagen (1973:282), en sjuklöneperiod enligt lagen (1991:1047) om sjuklön eller en period när en arbetsgivare för sjömän som avses i sjömanslagen har betalat ut lön vid sjukdom med stöd av sådant kollektivavtal som avses i 56 §, ska sjukperioden enligt denna lag anses omfatta också sådana perioder.
Lag (2017:1305).
\subsection*{27 §}
\paragraph*{}
Om sjukperioden inleds med sjukpenning som svarar mot inkomst av anställning och inte något annat följer av 27 a §, 28 §, 28 b §, 39 §, 39 a § eller 40-44 §§ ska ett karensavdrag göras från och med den första dagen med sjukpenning.
\paragraph*{}
Om sjukpenningen ska kalenderdagsberäknas under de första 14 dagarna i en sjukperiod, görs ett karensavdrag som motsvarar ersättningen för en hel dag med sjukpenning.
\paragraph*{}
Om sjukpenningen ska arbetstidsberäknas under de första 14 dagarna i en sjukperiod, görs ett karensavdrag som motsvarar 20 procent av den sjukpenning som den försäkrade genomsnittligen beräknas få under en vecka.
Lag (2018:647).
\subsection*{27 a §}
\paragraph*{}
För sjukpenning som svarar mot inkomst av annat förvärvsarbete gäller att sjukpenning inte lämnas under de första sju dagarna i en sjukperiod (karensdagar) om inte karenstid har anmälts enligt 29 § första stycket.
\paragraph*{}
Bestämmelser om karenstid i stället för karensdagar finns i 29-31 §§.
Lag (2018:647).
\subsection*{28 §}
\paragraph*{}
Till en försäkrad som har rätt till sjukpenning under medicinsk behandling eller medicinsk rehabilitering enligt 6 § lämnas sjukpenning på normalnivån eller fortsättningsnivån utan beaktande av sådan karens som avses i 27 och 27 a §§.
\paragraph*{}
Detsamma gäller för den som betalar egenavgift och som gjort anmälan om karenstid på 1 dag enligt 29 §, för den första dagen i en sjukperiod.
Lag (2018:647).
\subsection*{28 a §}
\paragraph*{}
Till en sjöman på fartyg som inte uteslutande går i inre fart lämnas sjukpenning utan beaktande av sådan karens som avses i 27 och 27 a §§.
\paragraph*{}
Med inre fart avses detsamma som i 64 kap. 6 § inkomstskattelagen (1999:1229).
Lag (2018:647).
\subsection*{28 b §}
\paragraph*{}
För en försäkrad som omfattas av bestämmelserna i 27 a § och som blir arbetslös, gäller i stället bestämmelserna om karensavdrag i 27 §. Detsamma gäller för en försäkrad som inte längre uppfyller villkoren för karenstid i 29-31 §§ på grund av att han eller hon är arbetslös.
\paragraph*{}
Till en försäkrad som omfattas av första stycket lämnas sjukpenning utan beaktande av bestämmelserna om karens enligt 28 § första stycket.
Lag (2018:647).
\subsection*{29 §}
\paragraph*{}
En försäkrad som har inkomst av annat förvärvsarbete och som betalar egenavgift har rätt att anmäla till Försäkringskassan att han eller hon vill ha sjukpenning med en karenstid på 1 dag, eller 14, 30, 60 eller 90 dagar.
\paragraph*{}
Om den som betalar egenavgift inte gör någon sådan anmälan, lämnas sjukpenning efter de karensdagar som anges i 27 a §.
Lag (2018:647).
\subsection*{30 §}
\paragraph*{}
Om den som betalar egenavgift gör en sådan anmälan som anges i 29 §, ska sjukpenning som svarar mot inkomst av annat förvärvsarbete inte lämnas för den första dagen, eller de första 14, 30, 60 eller 90 dagarna av varje sjukperiod, den dag sjukdomsfallet inträffade inräknad (karenstid).
Lag (2012:932).
\subsection*{31 §}
\paragraph*{}
Den som betalar egenavgift och som gjort anmälan om karenstid enligt 29 § får efter uppsägningstid övergå till sjukpenning med kortare eller ingen karenstid, om han eller hon inte har fyllt 55 år vid anmälan till Försäkringskassan om ändrad karenstid. Uppsägningstiden är det antal dagar med vilka karenstiden förkortas.
\paragraph*{}
Uppsägningstid enligt första stycket gäller även för en försäkrad som omfattas av bestämmelserna i 27 a § och som gör anmälan om karenstid på 1 dag enligt 29 §.
\paragraph*{}
Ändringen börjar gälla efter det att uppsägningstiden löpt ut. Ändringen får dock inte tillämpas vid sjukdom som inträffat innan ändringen har börjat gälla.
Lag (2018:647).
\subsection*{32 §}
\paragraph*{}
Om en sjukperiod börjar inom fem dagar från det att en tidigare sjukperiod har avslutats ska bestämmelserna i 20-24, 27 och 27 a §§ tillämpas som om den senare sjukperioden är en fortsättning på den tidigare sjukperioden.
Lag (2018:647).
\subsection*{33 §}
\paragraph*{}
Om en sjukperiod börjar inom 20 dagar efter föregående sjukperiods slut ska bestämmelsen i 30 § om karenstid för den som betalar egenavgift tillämpas på så sätt att de båda perioderna ska anses som en sjukperiod.
\paragraph*{}
För den som betalar egenavgift och som gjort anmälan om karenstid på 1 dag enligt 29 § gäller dock i stället 32 §.
Lag (2012:932).
\subsection*{33 a §}
\paragraph*{}
För en försäkrad som omfattas av bestämmelserna i 28 b § första stycket gäller 32 §.
Lag (2012:932).
\subsection*{34 §}
\paragraph*{}
En försäkrad har inte rätt till sjukpenning om han eller hon får hel sjukersättning eller hel aktivitetsersättning.
\paragraph*{}
En försäkrad har inte heller rätt till sjukpenning om han eller hon under månaden närmast före den då han eller hon började få hel ålderspension fick hel sjukersättning.
\subsection*{35 §}
\paragraph*{}
Bestämmelserna i 34 § om sjukersättning och aktivitetsersättning ska tillämpas även när den försäkrade skulle ha haft sådan ersättning i form av garantiersättning om han eller hon hade haft rätt till sådan ersättning enligt bestämmelserna i 35 kap. 4-15 §§ om försäkringstid.
Lag (2014:239).
\subsection*{35 a §}
\paragraph*{}
Har upphävts genom
lag (2015:963).
\paragraph*{}
/Rubriken upphör att gälla U:2025-12-01/
\subsection*{36 §}
\paragraph*{}
/Upphör att gälla U:2025-12-01/
Har den försäkrade fått sjukpenning för 180 dagar efter ingången av den månad när han eller hon fyllde 66 år, får Försäkringskassan besluta att sjukpenning inte längre ska lämnas till den försäkrade.
Lag (2022:878).
\subsection*{36 §}
\paragraph*{}
/Träder i kraft I:2025-12-01/
Har den försäkrade fått sjukpenning för 180 dagar efter ingången av den månad när han eller hon uppnådde riktåldern för pension, får Försäkringskassan besluta att sjukpenning inte längre ska lämnas till den försäkrade.
Lag (2022:879).
\subsection*{37 §}
\paragraph*{}
För tid efter ingången av den månad då den försäkrade fyllt 71 år får sjukpenning lämnas under högst 180 dagar.
Lag (2022:878).
\subsection*{38 §}
\paragraph*{}
När antalet dagar beräknas enligt 36 eller 37 § ska varje dag som sjukpenning har lämnats för räknas som en dag.
\subsection*{39 §}
\paragraph*{}
Om den försäkrade gått miste om sjukpenning som svarar mot inkomst av anställning till följd av bestämmelsen i 27 § vid sammanlagt tio tillfällen under de senaste tolv månaderna, kan sjukpenning lämnas utan karensavdrag som avses i 27 § (allmänt högriskskydd). Om den försäkrade gått miste om sjukpenning som svarar mot inkomst av annat förvärvsarbete till följd av bestämmelsen i 27 a § under fem sjukperioder under de senaste tolv månaderna, kan sjukpenning lämnas även för dagar som avses i 27 a § från och med den sjukperiod som inträder efter det att den försäkrade gått miste om sjukpenning för sammanlagt minst 21 dagar.
\paragraph*{}
Om den som betalar egenavgift och som gjort anmälan om karenstid på 1 dag enligt 29 § gått miste om sjukpenning som svarar mot inkomst av annat förvärvsarbete till följd av bestämmelserna i 27 a § eller 30 § under minst tio sjukperioder under de senaste tolv månaderna, kan sjukpenning lämnas även för den första dagen i en sjukperiod.
Lag (2018:647).
\subsection*{39 a §}
\paragraph*{}
För en försäkrad som omfattas av bestämmelserna i 28 b § första stycket gäller, i fråga om allmänt högriskskydd, 39 § andra stycket.
Lag (2012:932).
\subsection*{40 §}
\paragraph*{}
Efter ansökan av den försäkrade får Försäkringskassan besluta att sjukpenning kan lämnas utan beaktande av sådan karens som avses i 27 och 27 a §§ (särskilt högriskskydd).
\paragraph*{}
Försäkringskassan får även efter ansökan av den som betalar egenavgift och som gjort anmälan om karenstid på 1 dag enligt 29 § besluta att sjukpenning kan lämnas även för den första dagen i en sjukperiod.
\paragraph*{}
Detsamma gäller i förhållande till en försäkrad som omfattas av bestämmelserna i 28 b § första stycket.
Lag (2018:647).
\subsection*{41 §}
\paragraph*{}
Ett beslut om särskilt högriskskydd får meddelas om den försäkrade har en sjukdom som under en tolvmånadersperiod kan antas medföra ett större antal sjukperioder.
\subsection*{42 §}
\paragraph*{}
Ett beslut om särskilt högriskskydd får även meddelas för en sjukperiod när den försäkrade, som givare av biologiskt material enligt lagen (1995:831) om transplantation m.m., har rätt till sjukpenning till följd av ingrepp för att ta till vara det biologiska materialet eller förberedelser för sådant ingrepp.
\subsection*{43 §}
\paragraph*{}
Ett beslut om särskilt högriskskydd enligt 41 § gäller från och med den kalendermånad när ansökan gjordes, om inte annat anges i beslutet.
\paragraph*{}
Högriskskyddet ska gälla för viss tid som anges i beslutet eller, om det finns särskilda skäl, tills vidare.
\subsection*{44 §}
\paragraph*{}
Ett beslut om särskilt högriskskydd enligt 41 § ska upphävas om det villkor som anges där inte längre är uppfyllt.
\subsection*{45 §}
\paragraph*{}
Sjukpenning lämnas enligt följande förmånsnivåer:
\newline 1. Hel sjukpenning lämnas för dag när den försäkrade saknar arbetsförmåga.
\newline 2. Tre fjärdedels sjukpenning lämnas när den försäkrades arbetsförmåga är nedsatt med minst tre fjärdedelar men inte saknas helt.
\newline 3. Halv sjukpenning lämnas när den försäkrades arbetsförmåga är nedsatt med minst hälften men inte med tre fjärdedelar.
\newline 4. En fjärdedels sjukpenning lämnas när den försäkrades arbetsförmåga är nedsatt med minst en fjärdedel men inte med hälften.
\paragraph*{}
Bedömning av arbetsförmågans nedsättning - (Rehabiliteringskedjan)
\subsection*{46 §}
\paragraph*{}
Vid bedömningen av om arbetsförmågan är nedsatt ska det beaktas om den försäkrade på grund av sjukdomen inte kan utföra sitt vanliga arbete eller annat lämpligt arbete som arbetsgivaren tillfälligt erbjuder honom eller henne.
\paragraph*{}
Om den försäkrade på grund av sjukdomen behöver avstå från förvärvsarbete under minst en fjärdedel av sin normala arbetstid en viss dag, ska hans eller hennes arbetsförmåga anses nedsatt i minst motsvarande grad den dagen.
\paragraph*{}
Vid bedömningen av i vilken omfattning arbetsförmågan är nedsatt enligt 45 § 2-4 gäller att arbetstiden varje dag ska minskas i motsvarande grad eller, om arbetstiden förläggs på ett annat sätt, att arbetstidens förläggning inte får försämra möjligheterna till återgång i arbete.
Lag (2021:1240).
\subsection*{47 §}
\paragraph*{}
Från och med den tidpunkt då den försäkrade har haft nedsatt arbetsförmåga under 90 dagar ska det även beaktas om han eller hon kan försörja sig efter en omplacering till annat arbete hos arbetsgivaren.
\paragraph*{}
Bedömningen av arbetsförmågans nedsättning ska göras i förhållande till högst ett heltidsarbete.
\subsection*{48 §}
\paragraph*{}
Från och med den tidpunkt då den försäkrade har haft nedsatt arbetsförmåga under 180 dagar ska det dessutom beaktas om den försäkrade har sådan förmåga att han eller hon kan försörja sig själv genom
\newline 1. förvärvsarbete i en sådan angiven yrkesgrupp som innehåller arbeten som är normalt förekommande på arbetsmarknaden, eller
\newline 2. annat lämpligt arbete är tillgängligt för honom eller henne.
\paragraph*{}
Vid bedömningen tillämpas 47 § andra stycket.
\paragraph*{}
Den försäkrades arbetsförmåga ska, trots det som sägs i första stycket, från och med den tidpunkt då han eller hon har haft nedsatt arbetsförmåga under 180 dagar bedömas enligt 46 och 47 §§ om
\newline 1. övervägande skäl talar för att den försäkrade kan återgå till arbete som avses i 46 och 47 §§ före den tidpunkt då han eller hon har haft nedsatt arbetsförmåga i 365 dagar,
\newline 2. det finns särskilda skäl som grundas på att den försäkrade kan förväntas återgå till arbete som avses i 46 och 47 §§ före den tidpunkt då han eller hon har haft nedsatt arbetsförmåga i 550 dagar, eller
\newline 3. det kan anses oskäligt att bedöma den försäkrades arbetsförmåga enligt första stycket.
Lag (2021:1243).
\subsection*{49 §}
\paragraph*{}
Från och med den tidpunkt då den försäkrade har haft nedsatt arbetsförmåga under 365 dagar ska det beaktas om han eller hon har sådan förmåga som avses i 48 § första stycket.
\paragraph*{}
Vid bedömningen tillämpas 47 § andra stycket.
\paragraph*{}
Den försäkrades arbetsförmåga ska, trots det som sägs i första stycket, från och med den tidpunkt då han eller hon har haft nedsatt arbetsförmåga under 365 dagar bedömas enligt 46 och 47 §§ om
\newline 1. det finns särskilda skäl som grundas på att den försäkrade kan förväntas återgå till arbete som avses i 46 och 47 §§ före den tidpunkt då han eller hon har haft nedsatt arbetsförmåga i 550 dagar, eller
\newline 2. det kan anses oskäligt att bedöma den försäkrades arbetsförmåga enligt 48 § första stycket.
Lag (2021:1240).
\subsection*{49 a §}
\paragraph*{}
Har upphävts genom
lag (2022:1854).
\subsection*{49 b §}
\paragraph*{}
För en behovsanställd försäkrad ska arbetsförmågan bedömas mot sådant arbete som avses i 46 § första stycket.
\paragraph*{}
Från och med den tidpunkt då den behovsanställde har haft nedsatt arbetsförmåga under 90 dagar ska det dessutom beaktas om den behovsanställde har sådan förmåga att han eller hon kan försörja sig själv genom
\newline 1. förvärvsarbete i en sådan angiven yrkesgrupp som innehåller arbeten som är normalt förekommande på arbetsmarknaden, eller
\newline 2. annat lämpligt arbete som är tillgängligt för honom eller henne.
\paragraph*{}
Vid bedömningen tillämpas 47 § andra stycket.
Lag (2022:939).
\subsection*{49 c §}
\paragraph*{}
Om den behovsanställde omfattas av en överenskommelse om att utföra arbete under en särskilt angiven tid, gäller för den tiden bestämmelserna om bedömning av arbetsförmåga i 46-49 §§ i stället för vad som anges i 49 b §.
Lag (2021:1240).
\subsection*{49 d §}
\paragraph*{}
Bestämmelserna i 48 och 49 §§ gäller inte för en försäkrad som har ett förvärvsarbete och som har uppnått den ålder då inkomstgrundad ålderspension tidigast kan lämnas.
\paragraph*{}
För en försäkrad som avses i första stycket gäller bestämmelserna om bedömning av arbetsförmåga i 46 och 47 §§. Från och med den tidpunkt då den försäkrade har haft nedsatt arbetsförmåga under 180 dagar ska det vid bedömningen dessutom beaktas om den försäkrade har sådan förmåga att han eller hon kan försörja sig själv genom annat lämpligt arbete som är tillgängligt för honom eller henne.
\paragraph*{}
Sjukpenning med stöd av andra stycket kan som längst lämnas fram till den tidpunkt då den försäkrade tidigast kan ta ut garantipension.
Lag (2021:1240).
\subsection*{49 e §}
\paragraph*{}
Den som har uppnått den ålder då inkomstgrundad ålderspension tidigast kan lämnas ska omfattas av regleringen i 49 d § även om tidpunkten för uttag av sådan pension därefter skulle senareläggas genom en författningsändring.
Lag (2021:1240).
\subsection*{50 §}
\paragraph*{}
I de fall den försäkrade är i behov av någon medicinsk behandling eller medicinsk rehabilitering som avses i 27 kap. 6 § eller rehabiliteringsåtgärd som avses i 29-31 kap., ska bedömningen enligt 46-49 §§ göras med beaktande av den försäkrades arbetsförmåga efter en sådan åtgärd.
\subsection*{51 §}
\paragraph*{}
Vid beräkningen av hur lång tid den försäkrade har haft nedsatt arbetsförmåga enligt 46-49 §§ ska dagar i sjukperioder läggas samman om färre än 90 dagar förflutit mellan sjukperioderna.
Lag (2015:963).
\subsection*{52 §}
\paragraph*{}
Vid prövning av rätt till sjukpenning för tid när den försäkrade annars skulle ha fått föräldrapenning, ska arbetsförmågan anses nedsatt endast i den utsträckning som den försäkrades förmåga att vårda barn är nedsatt på grund av sjukdom.
\subsection*{53 §}
\paragraph*{}
Vid prövningen av den försäkrades rätt till sjukpenning ska det vid bedömningen av hans eller hennes arbetsförmåga bortses från den nedsättning av förmågan eller möjligheten att bereda sig arbetsinkomst som ligger till grund för ersättning till den försäkrade i form av
\newline 1. sjukersättning eller aktivitetsersättning, eller
\newline 2. livränta vid arbetsskada eller annan skada som avses i 41-44 kap.
\subsection*{54 §}
\paragraph*{}
Bestämmelsen i 53 § 1 ska tillämpas även när den försäkrade skulle ha haft sjukersättning eller aktivitetsersättning i form av garantiersättning om han eller hon hade haft rätt till sådan ersättning enligt bestämmelserna i 35 kap. 4-15 §§ om försäkringstid.
Lag (2014:239).
\subsection*{55 §}
\paragraph*{}
För en försäkrad som förvärvsarbetar under tid som han eller hon får sjukersättning enligt bestämmelserna i 37 kap. 3 § ska nedsättningen av arbetsförmågan, om det inte går att avgöra till vilken tid och till vilket förvärvsarbete nedsättningen hänför sig, i första hand anses hänföra sig till sådant förvärvsarbete som avses i den paragrafen.
\paragraph*{}
För en försäkrad som avses i första stycket ska bedömningen enligt 53 § alltid göras som om sjukersättning och livränta lämnas med oavkortade belopp.
\subsection*{55 a §}
\paragraph*{}
Har upphävts genom
lag (2020:427).
\subsection*{55 b §}
\paragraph*{}
För en försäkrad som på grund av sjukdom har lämnat ett arbetsmarknadspolitiskt program, och som har formell möjlighet att återinträda i ett sådant program, ska det vid bedömningen av nedsättningen av arbetsförmågan även beaktas den försäkrades förmåga att delta i ett sådant program.
\subsection*{56 §}
\paragraph*{}
Genom ett kollektivavtal som på arbetstagarsidan har slutits eller godkänts av en central arbetstagarorganisation får det bestämmas att en arbetsgivare, som har betalat ut lön till en arbetstagare under sjukdom, har rätt till arbetstagarens sjukpenning. En arbetsgivare som har betalat ut lön till en arbetstagare under sjukdom enligt sjömanslagen (1973:282) har dock alltid rätt till ersättning i den utsträckning som följer av 61 §.
Lag (2011:1075).
\subsection*{57 §}
\paragraph*{}
En arbetsgivare som är bunden av ett kollektivavtal enligt 56 § får tillämpa avtalet även på en arbetstagare som inte tillhör den avtalsslutande arbetstagarorganisationen, om arbetstagaren sysselsätts i arbete som avses med avtalet och inte omfattas av något annat tillämpligt kollektivavtal.
\subsection*{58 §}
\paragraph*{}
Regeringen eller den myndighet som regeringen bestämmer meddelar ytterligare föreskrifter om sjukpenningberäkning och hand-läggning av sjukdomsfall för arbetstagare hos staten som omfattas av sådant avtal som avses i 56 §.
\paragraph*{}
Regeringen eller den myndighet som regeringen bestämmer meddelar även föreskrifter om sjukpenningberäkning för arbetstagare med statligt reglerad anställning som är anställda hos en annan arbetsgivare än staten och som omfattas av sådant kollektivavtal som anges i 56 §.
\subsection*{59 §}
\paragraph*{}
Sjukpenning som enligt bestämmelserna i 56-58 §§ betalas ut till en arbetsgivare ska minskas med sådan lön under sjukdom som arbetsgivaren lämnar till arbetstagaren för samma tid som sjukpenningen avser, dock endast med den del av lönen under sjukdom som överstiger
\newline 1. 90 procent i fråga om sjukpenning på normalnivån, och
\newline 2. 85 procent i fråga om sjukpenning på fortsättningsnivån.
\paragraph*{}
Vid beräkningen tillämpas bestämmelserna i 28 kap. 20 och 21 §§.
\subsection*{60 §}
\paragraph*{}
Om en försäkrad har blivit sjuk utomlands och då fått ekonomisktstöd av utrikesförvaltningen kan förvaltningen få rätt till den försäkrades sjukpenning. Detta gäller dock endast i den utsträckning sjukpenningen inte överstiger vad som lämnats som ekonomiskt stöd.
\paragraph*{}
Regeringen eller den myndighet som regeringen bestämmer meddelar närmare föreskrifter om utrikesförvaltningens rätt enligt första stycket.
\subsection*{61 §}
\paragraph*{}
Regeringen eller den myndighet som regeringen bestämmer meddelar föreskrifter om i vilken utsträckning ersättning som en arbetsgivare för sjömän som avses i sjömanslagen (1973:282) har rätt till enligt 56-59 §§ ska lämnas för varje dag och med viss procent av utbetald lön och andra kostnader som arbetsgivaren haft för den anställde.
\chapter*{28 Beräkning av sjukpenning}
\subsection*{1 §}
\paragraph*{}
I detta kapitel finns grundläggande bestämmelser i 2-6 §§.
\paragraph*{}
Vidare finns bestämmelser om
\newline - ersättningsnivå och beräkningsunderlag i 7-9 §§,
\newline - kalenderdagsberäknad sjukpenning i 10 och 11 §§,
\newline - arbetstidsberäknad sjukpenning i 12-18 §§, och
\newline - samordning av sjukpenning med samtidig lön i 19-21 §§.
\paragraph*{}
Grundläggande bestämmelser Beräkningsmetoder
\subsection*{2 §}
\paragraph*{}
Sjukpenning lämnas som
\newline - kalenderdagsberäknad sjukpenning, eller
\newline - arbetstidsberäknad sjukpenning.
\subsection*{3 §}
\paragraph*{}
Kalenderdagsberäknad sjukpenning lämnas för alla dagar i veckan oavsett om den försäkrade skulle ha utfört förvärvsarbete eller inte.
Arbetstidsberäknad sjukpenning lämnas bara för timmar eller dagar när den försäkrade skulle ha förvärvsarbetat.
När kalenderdagsberäknas respektive arbetstidsberäknas sjukpenningen?
\subsection*{4 §}
\paragraph*{}
Sjukpenning ska kalenderdagsberäknas om inte annat följer av 5 §.
\subsection*{5 §}
\paragraph*{}
Sjukpenning ska arbetstidsberäknas
\newline 1. under de första 14 dagarna i en sjukperiod enligt 27 kap. 10 och 11 §§, om inte annat följer av 6 § tredje stycket,
\newline 2. under studietid som avses i 27 kap. 12 §,
\newline 3. under tid med periodiskt ekonomiskt understöd som avses i 27 kap. 13 §,
\newline 4. under deltagande i arbetsmarknadspolitiskt program som avses i 27 kap. 14 §,
\newline 5. under behandling eller rehabilitering som avses i 27 kap. 15 §, och
\newline 6. under plikttjänstgöring som avses i 27 kap. 16 §.
\paragraph*{}
Det som föreskrivs i första stycket gäller endast till den del sjukpenningen motsvarar sjukpenninggrundande inkomst av anställning. Om den försäkrade har inkomst även av annat förvärvsarbete, ska sjukpenningen kalenderdagsberäknas i den delen.
Lag (2010:2005).
\subsection*{6 §}
\paragraph*{}
Sjukpenning ska alltid kalenderdagsberäknas när den försäkrade
\newline 1. är helt eller delvis arbetslös, om inte annat följer av tredje stycket,
\newline 2. får sjukpenning för tid då han eller hon annars skulle ha fått graviditetspenning, föräldrapenning eller rehabiliteringspenning, eller
\newline 3. är egenföretagare och har en sjukpenninggrundande inkomst som består av endast inkomst av annat förvärvsarbete.
\paragraph*{}
Om sjukpenning till en familjehemsförälder ska beräknas på grundval av en sjukpenninggrundande inkomst som omfattar ersättning för vården, ska sjukpenning som motsvarar denna ersättning kalenderdagsberäknas.
För en försäkrad som avses i första stycket 1 lämnas kalenderdagsberäknad sjukpenning under de första 14 dagarna i en sjukperiod endast om den försäkrade är anmäld som arbetssökande hos den offentliga arbetsförmedlingen samt är beredd att ta ett erbjudet arbete i en omfattning som svarar mot den bestämda sjukpenninggrundande inkomsten. Om det som nu föreskrivits skulle framstå som oskäligt, får kalenderdagsberäknad sjukpenning ändå lämnas under de första 14 dagarna i sjukperioden.
Lag (2010:2005).
\subsection*{7 §}
\paragraph*{}
Den försäkrades sjukpenning ska beräknas på ett underlag (beräkningsunderlag) som för
\newline 1. sjukpenning på normalnivån motsvarar 80 procent av den försäkrades sjukpenninggrundande inkomst sedan denna har multiplicerats med talet 0,97, och
\newline 2. sjukpenning på fortsättningsnivån motsvarar 75 procent av den försäkrades sjukpenninggrundande inkomst sedan denna har multiplicerats med talet 0,97.
\subsection*{8 §}
\paragraph*{}
Om en arbetsgivare ska svara för sjuklön för samma dag som sjukpenning kommer i fråga, ska sjukpenningens storlek beräknas på grundval av en sjukpenninggrundande inkomst som inte omfattar anställningsförmåner från den arbetsgivaren. Den sjukpenninggrundande inkomsten ska inte heller omfatta statlig ersättning som lämnas för arbete i etableringsjobb hos den arbetsgivare som ska svara för sjuklön.
\paragraph*{}
Årsarbetstiden beräknas i de fall som anges i första stycket på grundval av beräknat antal timmar i förvärvsarbete hos arbetsgivare som inte ska svara för sjuklön.
Lag (2020:475).
\subsection*{9 §}
\paragraph*{}
För en familjehemsförälder som får ersättning för vården för tid då sjukpenning kommer i fråga, ska sjukpenningens storlek och årsarbetstiden beräknas på en sjukpenninggrundande inkomst respektive ett beräknat antal timmar i förvärvsarbete som inte omfattar ersättningen.
\subsection*{10 §}
\paragraph*{}
För dagar i en sjukperiod gäller att hel kalenderdagsberäknad sjukpenning motsvarar kvoten mellan
\newline - beräkningsunderlaget enligt 7 § 1 eller 2 och
\newline - 365.
\paragraph*{}
Sjukpenningen avrundas till närmaste hela krontal, varvid 50 öre avrundas uppåt.
\subsection*{11 §}
\paragraph*{}
Till den del den försäkrade är arbetslös lämnas hel sjukpenning med högst 543 kronor om dagen. Detta gäller dock inte sjukpenning som avses i 27 kap. 6 eller 16 a §.
Lag (2021:1240).
\subsection*{12 §}
\paragraph*{}
När arbetstidsberäknad sjukpenning ska beräknas tillämpas
\newline - 13 och 16 §§, om sjukpenning ska lämnas för endast en dag, och
\newline - 14-16 §§, om sjukpenning ska lämnas för mer än en dag.
\subsection*{13 §}
\paragraph*{}
Om sjukpenning ska lämnas för endast en dag, ska hel sjukpenning beräknas enligt följande:
\newline 1. Först ska beräkningsunderlaget enligt 7 § 1 eller 2 divideras med årsarbetstiden, varefter kvoten avrundas till närmaste hela krontal.
\newline 2. Därefter ska kvoten som fås i 1 multipliceras med antalet timmar av ordinarie arbetstid eller motsvarande normal arbetstid.
\subsection*{14 §}
\paragraph*{}
Om sjukpenning på samma förmånsnivå lämnas för mer än en dag, ska hel sjukpenning för dag beräknas för dessa dagar enligt följande:
\newline 1. Först ska kvoten enligt 13 § 1 beräknas.
\newline 2. Därefter ska kvoten enligt 1 multipliceras med det sammanlagda antalet timmar av ordinarie arbetstid eller motsvarande normal arbetstid som avser dessa dagar. 3. Slutligen ska den produkt som fås i 2 divideras med antalet dagar med sjukpenning.
\subsection*{15 §}
\paragraph*{}
Om sjukpenning lämnas på olika förmånsnivåer för mer än en dag ska de timmar som avser samma nivå adderas var för sig. Sjukpenning ska beräknas för varje sådan period för sig.
\subsection*{16 §}
\paragraph*{}
Om antalet timmar eller det sammanlagda antalet timmar enligt 13-15 §§ inte uppgår till ett helt timtal, ska avrundning göras till närmaste hela timtal, varvid halv timme avrundas uppåt.
Sjukpenning avrundas till närmaste hela krontal, varvid 50 öre avrundas uppåt.
\subsection*{17 §}
\paragraph*{}
Om den försäkrades sjukpenning i de fall som avses i 5 § motsvarar sjukpenninggrundande inkomst av såväl anställning som annat förvärvsarbete beräknas beloppet av hel sjukpenning för dag enligt följande:
\newline 1. Den del av sjukpenningen som motsvarar inkomst av anställning beräknas enligt 12-16 §§.
\newline 2. Den del av sjukpenningen som motsvarar inkomst av annat förvärvsarbete beräknas enligt 10 och 11 §§.
\subsection*{18 §}
\paragraph*{}
Regeringen eller den myndighet som regeringen bestämmer meddelar föreskrifter om schablonberäkning av ordinarie arbetstid och motsvarande normal arbetstid.
\subsection*{19 §}
\paragraph*{}
Om den försäkrade får lön av arbetsgivaren under sjukdom för samma tid som sjukpenningen avser, ska sjukpenningen minskas med det belopp som lönen under sjukdomen överstiger 10 procent av vad den försäkrade skulle ha fått i lön om han eller hon hade arbetat.
\paragraph*{}
Till den del lönen under sjukdom lämnas i förhållande till lön i arbete som för år räknat överstiger den högsta sjukpenninggrundande inkomst som kan beräknas enligt 25 kap. 5 § andra stycket, ska minskning dock endast göras med belopp som överstiger
\newline 1. 90 procent av lönen i arbete i fråga om sjukpenning på normalnivån, och
\newline 2. 85 procent av lönen i arbete i fråga om sjukpenning på fortsättningsnivån.
\subsection*{20 §}
\paragraph*{}
Vid tillämpning av bestämmelserna i 19 § ska ersättning som lämnas till den försäkrade på grund av förmån av fri gruppsjukförsäkring enligt grunder som fastställs i kollektivavtal anses som lön under sjukdom från arbetsgivare.
\subsection*{21 §}
\paragraph*{}
Det belopp varmed minskning enligt 19 § ska göras, avrundas till närmast lägre hela krontal.
\paragraph*{}
Avräkning ska i första hand göras vid utbetalning av sjukpenning som avser samma tid som den lön under sjukdom som föranlett minskningen. Avräkningen får också göras vid närmast följande utbetalning av sjukpenning.
\chapter*{28 a Sjukpenning i särskilda fall}
\subsection*{1 §}
\paragraph*{}
I detta kapitel finns en inledande bestämmelse i 2 §.
\paragraph*{}
Vidare finns bestämmelser om
\newline - rätten till sjukpenning i särskilda fall i 3-7 §§,
\newline - bedömning av arbetsförmågans nedsättning i 8 §,
\newline - ersättningsnivåer i 9 §,
\newline - beräkning av sjukpenning i särskilda fall i 10-12 §§,
\newline - sjukpenning i särskilda fall vid sjukersättning i 13 §,
\newline - sjukpenning i särskilda fall vid livränta enligt 41 eller 43 kap. i 14 §,
\newline - förmånstiden i 15 §,
\newline - karensavdrag och karensdagar i 16 och 17 §§,
\newline - behållande av rätten till sjukpenning i särskilda fall i 18 §,
\newline - upphörande av rätten till sjukpenning i särskilda fall i 19 §, och
\newline - arbetsgivarinträde m.m. i 20 §.
Lag (2018:647).
\subsection*{2 §}
\paragraph*{}
Bestämmelserna i 27 och 28 kap. gäller även i fråga om sjukpenning i särskilda fall, om inte annat följer av detta kapitel.
Lag (2011:1513).
\subsection*{3 §}
\paragraph*{}
En försäkrad som helt eller delvis har fått tidsbegränsad sjukersättning under det högsta antalet månader som sådan ersättning kan betalas ut enligt 4 kap. 31 § lagen (2010:111) om införande av socialförsäkringsbalken har i de fall och under de närmare förutsättningar som anges i detta kapitel rätt till sjukpenning i särskilda fall. Detta gäller även för en försäkrad vars rätt till aktivitetsersättning upphör på grund av att han eller hon fyller 30 år.
Lag (2011:1514).
\subsection*{4 §}
\paragraph*{}
En försäkrad som omfattas av 3 § och som på grund av bestämmelserna om sjukpenninggrundande inkomst i denna balk inte skulle ha rätt till hel sjukpenning som på fortsättningsnivån medför en sjukpenning om minst 160 kronor per kalenderdag, har rätt till sjukpenning i särskilda fall vid sjukdom som sätter ned hans eller hennes arbetsförmåga.
\paragraph*{}
Med sjukdom likställs, förutom vad som föreskrivs i 27 kap. 2 § andra stycket, även annan nedsättning av den fysiska eller psykiska prestationsförmågan.
Lag (2011:1513).
\subsection*{5 §}
\paragraph*{}
Rätten till sjukpenning i särskilda fall inträder från och med dagen efter den då rätten till sådan tidsbegränsad sjukersättning eller aktivitetsersättning som avses i 3 § har upphört.
Lag (2011:1514).
\subsection*{6 §}
\paragraph*{}
Sjukpenning i särskilda fall lämnas inte när den försäkrade
\newline 1. bedriver studier, för vilka han eller hon uppbär studiestöd enligt studiestödslagen (1999:1395), studiestartsstöd enligt lagen (2017:527) om studiestartsstöd eller omställningsstudiestöd enligt lagen (2022:856) om omställningsstudiestöd, eller
\newline 2. deltar i ett arbetsmarknadspolitiskt program och får aktivitetsstöd.
\paragraph*{}
Sjukpenning i särskilda fall lämnas inte heller när den försäkrade deltar i ett arbetsmarknadspolitiskt program men är avstängd från rätt till aktivitetsstöd.
Lag (2022:858).
\subsection*{7 §}
\paragraph*{}
/Upphör att gälla U:2025-12-01/
Sjukpenning i särskilda fall lämnas längst till och med månaden före den när den försäkrade fyller 66 år.
Lag (2022:878).
\subsection*{7 §}
\paragraph*{}
/Träder i kraft I:2025-12-01/
Sjukpenning i särskilda fall lämnas längst till och med månaden före den när den försäkrade uppnår riktåldern för pension.
Lag (2022:879).
\subsection*{8 §}
\paragraph*{}
Vid bedömningen av om arbetsförmågan är nedsatt ska det beaktas om den försäkrade på grund av sjukdomen har sådan förmåga att han eller hon kan försörja sig själv genom
\newline 1. förvärvsarbete i en sådan angiven yrkesgrupp som innehåller arbeten som är normalt förekommande på arbetsmarknaden, eller
\newline 2. annat lämpligt arbete som är tillgängligt för honom eller henne.
\paragraph*{}
Bedömningen av arbetsförmågans nedsättning ska göras i förhållande till ett heltidsarbete.
Lag (2021:990).
\subsection*{9 §}
\paragraph*{}
Sjukpenning i särskilda fall lämnas med högst
\newline - 160 kronor per dag vid hel förmån,
\newline - 120 kronor per dag vid tre fjärdedels förmån,
\newline - 80 kronor per dag vid halv förmån, och
\newline - 40 kronor per dag vid en fjärdedels förmån.
Lag (2011:1513).
\subsection*{10 §}
\paragraph*{}
Sjukpenning i särskilda fall lämnas för sju dagar per vecka.
Lag (2011:1513).
\subsection*{11 §}
\paragraph*{}
Sjukpenning i särskilda fall beräknas, om inte annat följer av 12 eller 13 §, enligt följande:
\newline 1. För en försäkrad för vilken en sjukpenninggrundande inkomst inte kan fastställas, uppgår sjukpenning i särskilda fall per dag för respektive förmånsnivå till de belopp som anges i 9 §.
\newline 2. För annan försäkrad än den som avses i 1, motsvarar sjukpenning i särskilda fall per dag för respektive förmånsnivå differensen mellan
\newline - de belopp som anges i 9 § och
\newline - den sjukpenning som lämnas per dag enligt 27 och 28 kap.
Lag (2011:1513).
\subsection*{12 §}
\paragraph*{}
Om sjukpenning ska lämnas under de första 14 dagarna i en sjukperiod med stöd av 27 kap. 10 eller 11 §, lämnas sjukpenning i särskilda fall endast till den del denna ersättning, sammanlagt under fjortondagarsperioden, överstiger vad som annars ska lämnas i sjukpenning under samma period.
Lag (2011:1513).
\subsection*{13 §}
\paragraph*{}
I det fall en försäkrad får sjukersättning lämnas hel sjukpenning i särskilda fall med högst
\newline - 120 kronor per dag när sjukersättning lämnas som en fjärdedels förmån,
\newline - 80 kronor per dag när sjukersättning lämnas som en halv förmån, och
\newline - 40 kronor per dag när sjukersättning lämnas som tre fjärdedels förmån.
\paragraph*{}
När sjukpenning i särskilda fall ska lämnas som partiell förmån, lämnas ersättning med högst tre fjärdedelar, hälften eller en fjärdedel av beloppen i första stycket.
\paragraph*{}
För en försäkrad för vilken en sjukpenninggrundande inkomst kan fastställas, motsvarar sjukpenning i särskilda fall per dag differensen mellan
\newline - de belopp som framgår av första eller andra styckena och
\newline - den sjukpenning som lämnas per dag enligt 27 och 28 kap.
Lag (2011:1513).
\paragraph*{}
Sjukpenning i särskilda fall vid livränta enligt 41 eller 43 kap.
\subsection*{14 §}
\paragraph*{}
I det fall en försäkrad får livränta enligt 41 eller 43 kap., minskas sjukpenning i särskilda fall per dag enligt följande.
\paragraph*{}
När sjukpenning i särskilda fall har beräknats enligt 11 §, ska det framräknade beloppet per dag minskas med ett belopp som motsvarar det livräntebelopp enligt 41 eller 43 kap. som gäller vid tiden för beslutet delat med 365. Livräntebeloppet avrundas till närmaste hela krontal och 50 öre avrundas nedåt.
\paragraph*{}
Det som föreskrivs i första och andra styckena gäller endast om den försäkrade inte får sjukersättning för samma tid som livränta enligt 41 eller 43 kap.
Lag (2011:1513).
\subsection*{15 §}
\paragraph*{}
När det enligt 27 kap. 20-24 §§ kan lämnas sjukpenning på normalnivån eller fortsättningsnivån, ska även sjukpenning i särskilda fall kunna lämnas.
Lag (2015:963).
\subsection*{16 §}
\paragraph*{}
För en försäkrad för vilken en sjukpenninggrundande inkomst inte kan fastställas eller för vilken en sjukpenninggrundande inkomst har fastställts eller skulle ha kunnat fastställas som svarar mot inkomst av anställning, tillämpas vad som föreskrivs om karensavdrag i 27 kap. 27 §.
\paragraph*{}
För en försäkrad för vilken en sjukpenninggrundande inkomst har fastställts eller skulle ha kunnat fastställas som svarar mot endast inkomst av annat förvärvsarbete, och som inte är arbetslös, tillämpas vad som föreskrivs om karensdagar i 27 kap. 27 a §.
Lag (2018:647).
\subsection*{17 §}
\paragraph*{}
Sjukpenning i särskilda fall kan lämnas utan karensavdrag eller avdrag för karensdagar i motsvarande fall som det enligt 27 kap. 28 § kan lämnas sjukpenning på normalnivån eller fortsättningsnivån.
Lag (2018:647).
\subsection*{18 §}
\paragraph*{}
En försäkrad behåller sin rätt till sjukpenning i särskilda fall under tid då
\newline 1. han eller hon förvärvsarbetar,
\newline 2. en sådan situation föreligger som anges i 26 kap. 11, 12, eller 14-18 §§ som grund för SGI-skydd, eller
\newline 3. han eller hon deltar i ett arbetsmarknadspolitiskt program och får aktivitetsstöd eller står till arbetsmarknadens förfogande.
\paragraph*{}
Rätten enligt första stycket gäller endast om den inte upphör på grund av 19 §.
\paragraph*{}
Regeringen eller den myndighet som regeringen bestämmer kan med stöd av 8 kap. 7 § regeringsformen meddela
\newline 1. föreskrifter om undantag från kravet på att den som deltar i ett arbetsmarknadspolitiskt program ska få aktivitetsstöd, och
\newline 2. föreskrifter om de villkor som gäller för att den försäkrade ska anses stå till arbetsmarknadens förfogande.
Lag (2015:119).
\subsection*{19 §}
\paragraph*{}
Rätten till sjukpenning i särskilda fall upphör när det för den försäkrade har bestämts eller skulle ha kunnat bestämmas en sjukpenninggrundande inkomst som uppgår till minst 80 300 kronor.
Lag (2011:1513).
\subsection*{20 §}
\paragraph*{}
Bestämmelserna om arbetsgivarinträde m.m. i 27 kap. 56-59 och 61 §§ gäller inte sjukpenning i särskilda fall.
Lag (2011:1513).
\chapter*{29 Innehåll och inledande bestämmelser}
\subsection*{1 §}
\paragraph*{}
I denna underavdelning finns bestämmelser om
\newline - rehabilitering i 30 kap.,
\newline - rehabiliteringsersättning i 31 kap., och
\newline - rehabiliteringspenning i särskilda fall i 31 a kap.
Lag (2011:1513).
\subsection*{2 §}
\paragraph*{}
Rehabilitering enligt bestämmelserna i denna underavdelning ska syfta till att en försäkrad som har drabbats av sjukdom ska få tillbaka sin arbetsförmåga och få förutsättningar att försörja sig själv genom förvärvsarbete (arbetslivsinriktad rehabilitering).
\subsection*{3 §}
\paragraph*{}
Under den tid som den arbetslivsinriktade rehabiliteringen pågår kan rehabiliteringsersättning lämnas enligt bestämmelserna i 31 och 31 a kap.
Lag (2011:1513).
\chapter*{30 Rehabilitering}
\subsection*{1 §}
\paragraph*{}
I detta kapitel finns bestämmelser om rehabilitering i 2-14 §§.
Lag (2017:1306).
\subsection*{2 §}
\paragraph*{}
En försäkrad har rätt till rehabiliteringsåtgärder enligt bestämmelserna i detta kapitel.
\subsection*{3 §}
\paragraph*{}
Rehabiliteringsåtgärder ska planeras i samråd med den försäkrade och utgå från hans eller hennes individuella förutsättningar och behov.
\subsection*{4 §}
\paragraph*{}
Regeringen eller den myndighet som regeringen bestämmer meddelar ytterligare föreskrifter om rehabilitering av försäkrade som inte är bosatta här i landet.
\subsection*{5 §}
\paragraph*{}
Regeringen eller den myndighet som regeringen bestämmer meddelar föreskrifter om bidrag till sådana arbetshjälpmedel som en förvärvsarbetande försäkrad behöver som ett led i sin rehabilitering.
\subsection*{6 §}
\paragraph*{}
Om det kan antas att den försäkrades arbetsförmåga kommer att vara nedsatt på grund av sjukdom under minst 60 dagar ska arbetsgivaren senast den dag när den försäkrades arbetsförmåga har varit nedsatt under 30 dagar ha upprättat en plan för återgång i arbete. Om det har antagits att den försäkrades arbetsförmåga inte kommer att vara nedsatt under minst 60 dagar och det senare framkommer att nedsättningen kan antas komma att fortgå under minst 60 dagar ska dock en sådan plan omgående upprättas.
\paragraph*{}
En plan för återgång i arbete behöver inte upprättas om det med hänsyn till den försäkrades hälsotillstånd klart framgår att den försäkrade inte kan återgå i arbete. Om hälsotillståndet senare förbättras ska dock en plan omgående upprättas. Planen ska i den utsträckning som det är möjligt upprättas i samråd med den försäkrade.
\paragraph*{}
Arbetsgivaren ska fortlöpande se till att planen för återgång i arbete följs och att det vid behov görs ändringar i den.
Lag (2017:1306).
\subsection*{6 a §}
\paragraph*{}
Den försäkrades arbetsgivare ska efter samråd med den försäkrade lämna de upplysningar till Försäkringskassan som behövs för att den försäkrades behov av rehabilitering snarast ska kunna klarläggas och även i övrigt medverka till det. Arbetsgivaren ska också svara för att de åtgärder vidtas som behövs för en effektiv rehabilitering.
\paragraph*{}
Bestämmelser om arbetsgivarens skyldigheter avseende arbetsanpassning och rehabilitering finns även i arbetsmiljölagen (1977:1160).
Lag (2017:1306).
\subsection*{7 §}
\paragraph*{}
Den försäkrade ska
\newline - lämna de upplysningar som behövs för att klarlägga hans eller hennes behov av rehabilitering, och
\newline - efter bästa förmåga aktivt medverka i rehabiliteringen.
\subsection*{8 §}
\paragraph*{}
Försäkringskassan samordnar och utövar tillsyn över de insatser som behövs för rehabiliteringsverksamheten.
\subsection*{9 §}
\paragraph*{}
Försäkringskassan ska i samråd med den försäkrade se till att
\newline - den försäkrades behov av rehabilitering snarast klarläggs, och
\newline - de åtgärder vidtas som behövs för en effektiv rehabilitering av den försäkrade.
\subsection*{10 §}
\paragraph*{}
Försäkringskassan ska, om den försäkrade medger det, i arbetet med rehabiliteringen samverka med
\newline - den försäkrades arbetsgivare och arbetstagarorganisation,
\newline - hälso- och sjukvården,
\newline - socialtjänsten,
\newline - Arbetsförmedlingen, och
\newline - andra myndigheter som berörs av rehabiliteringen av den försäkrade.
\paragraph*{}
Försäkringskassan ska verka för att de organisationer och myndigheter som anges i första stycket, var och en inom sitt verksamhetsområde, vidtar de åtgärder som behövs för en effektiv rehabilitering av den försäkrade.
\subsection*{11 §}
\paragraph*{}
Försäkringskassan ska se till att rehabiliteringsåtgärder påbörjas så snart det är möjligt av medicinska och andra skäl.
\subsection*{12 §}
\paragraph*{}
Om den försäkrade behöver en rehabiliteringsåtgärd, för vilken rehabiliteringsersättning kan lämnas, ska Försäkringskassan upprätta en rehabiliteringsplan. Planen ska i den utsträckning det är möjligt upprättas i samråd med den försäkrade.
\subsection*{13 §}
\paragraph*{}
En rehabiliteringsplan ska ange
\newline 1. de rehabiliteringsåtgärder som ska komma i fråga,
\newline 2. vem som har ansvaret för rehabiliteringsåtgärderna,
\newline 3. en tidsplan för rehabiliteringen, och
\newline 4. de uppgifter i övrigt som behövs för att genomföra rehabiliteringen.
Lag (2013:747).
\subsection*{14 §}
\paragraph*{}
Försäkringskassan ska fortlöpande se till att en rehabiliteringsplan följs och att det vid behov görs ändringar i den.
\chapter*{31 Rehabiliteringsersättning}
\subsection*{1 §}
\paragraph*{}
I detta kapitel finns bestämmelser om
\newline - ersättningsformer i 2 §,
\newline - rätten till rehabiliteringsersättning i 3-7 §§,
\newline - rehabiliteringspenning i 8-13 §§, och
\newline - särskilt bidrag i 14 §.
\subsection*{2 §}
\paragraph*{}
Rehabiliteringsersättning lämnas i följande former:
\newline 1. Rehabiliteringspenning till en försäkrad som deltar i arbetslivsinriktad rehabilitering.
\newline 2. Särskilt bidrag till en försäkrad för kostnader som uppstår i samband med arbetslivsinriktad rehabilitering.
\subsection*{3 §}
\paragraph*{}
Vid sjukdom som sätter ned en försäkrads arbetsförmåga med minst en fjärdedel har den försäkrade rätt till rehabiliteringsersättning under tid då han eller hon deltar i arbetslivsinriktad rehabilitering som avser att
\newline 1. förkorta sjukdomstiden, eller
\newline 2. helt eller delvis förebygga eller häva nedsättning av arbetsförmågan.
\subsection*{4 §}
\paragraph*{}
/Upphör att gälla U:2025-12-01/
Rehabiliteringsersättning lämnas längst till och med månaden före den när den försäkrade fyller 66 år.
Lag (2022:878).
\subsection*{4 §}
\paragraph*{}
/Träder i kraft I:2025-12-01/
Rehabiliteringsersättning lämnas längst till och med månaden före den när den försäkrade uppnår riktåldern för pension.
Lag (2022:879).
\subsection*{5 §}
\paragraph*{}
En försäkrad som får rehabiliteringsersättning får behålla ersättningen
\newline 1. vid kortvarig ledighet för enskild angelägenhet av vikt,
\newline 2. vid ledighet på grund av uppehåll i rehabiliteringen enligt föreskrifter som regeringen eller den myndighet som regeringen bestämmer kan meddela med stöd av 8 kap. 7 § regeringsformen, och
\newline 3. när den försäkrade på grund av sjukdom är helt eller delvis ur stånd att delta i rehabiliteringen.
\paragraph*{}
Rehabiliteringsersättningen får lämnas enligt första stycket 3 för högst 30 kalenderdagar i följd. Efter den sjunde dagen lämnas ersättningen endast om den försäkrade genom att lämna in ett läkarintyg till Försäkringskassan styrker nedsättningen av förmågan att delta i rehabiliteringen på grund av sjukdom.
\paragraph*{}
Regeringen eller den myndighet som regeringen bestämmer kan med stöd av 8 kap. 7 § regeringsformen meddela föreskrifter dels om undantag från kravet att lämna ett läkarintyg enligt andra stycket när ett sådant intyg inte behövs, dels om att kravet ska gälla för ersättning från och med en annan dag än den som följer av andra stycket.
Lag (2020:427).
\subsection*{6 §}
\paragraph*{}
Regeringen eller den myndighet som regeringen bestämmer meddelar ytterligare föreskrifter om rehabiliteringsersättning vid utbildning.
\subsection*{7 §}
\paragraph*{}
Om en familjehemsförälder får ersättning för vården av barn som omfattas av uppdraget för tid när rehabiliteringspenning kommer i fråga, bedöms rätten till rehabiliteringspenning med bortseende från ersättningen.
\subsection*{8 §}
\paragraph*{}
Rehabiliteringspenning lämnas enligt följande förmånsnivåer:
\newline 1. Hel rehabiliteringspenning lämnas för dag när den försäkrade saknar arbetsförmåga.
\newline 2. Tre fjärdedels rehabiliteringspenning lämnas för dag när den försäkrades arbetsförmåga är nedsatt med minst tre fjärdedelar men inte saknas helt.
\newline 3. Halv rehabiliteringspenning lämnas för dag när den försäkrades arbetsförmåga är nedsatt med minst hälften men inte med tre fjärdedelar.
\newline 4. En fjärdedels rehabiliteringspenning lämnas för dag när den försäkrades arbetsförmåga är nedsatt med minst en fjärdedel men inte med hälften.
\paragraph*{}
Arbetsförmågan ska under tiden för rehabiliteringsåtgärden anses nedsatt i den utsträckning den försäkrade på grund av åtgärden är förhindrad att förvärvsarbeta.
\subsection*{9 §}
\paragraph*{}
Vid tillämpningen av 8 § första stycket ska det bortses från sådan arbetsförmåga som den försäkrade utnyttjar i samband med förvärvsarbete som utförs med stöd av 37 kap. 3 §. Om det inte går att avgöra till vilken tid och till vilket förvärvsarbete nedsättningen av arbetsförmågan hänför sig ska denna i första hand anses hänföra sig till sådant förvärvsarbete som avses i 37 kap. 3 §.
\subsection*{10 §}
\paragraph*{}
Hel rehabiliteringspenning för dag motsvarar kvoten mellan
\newline - den försäkrades beräkningsunderlag enligt 28 kap. 7 § 1 eller 2 och
\newline - 365.
\paragraph*{}
Vid beräkningen ska bestämmelserna i 27 kap. 21-24, 26 och 32 §§ samt 28 kap. 10 och 11 §§ tillämpas.
Lag (2015:963).
\subsection*{11 §}
\paragraph*{}
I fall som avses i 7 § ska rehabiliteringspenningens storlek beräknas på en sjukpenninggrundande inkomst som inte omfattar den ersättning som den försäkrade får i egenskap av familjehemsförälder.
\subsection*{12 §}
\paragraph*{}
Rehabiliteringspenningen ska minskas med det belopp den försäkrade för samma tid får som
\newline 1. föräldrapenningsförmån,
\newline 2. sjukpenning,
\newline 3. sjukpenning eller livränta vid arbetsskada eller annan skada som avses i 40-44 kap. eller motsvarande ersättning enligt någon annan författning, dock endast till den del ersättningen avser samma inkomstbortfall som rehabiliteringspenningen är avsedd att täcka, eller
\newline 4. studiestöd enligt studiestödslagen (1999:1395), studiestartsstöd enligt lagen (2017:527) om studiestartsstöd, omställningsstudiestöd enligt lagen (2022:856) om omställningsstudiestöd eller ersättning till deltagare i teckenspråksutbildning för vissa föräldrar (TUFF), dock inte till den del studiestödet eller omställningsstudiestödet är återbetalningspliktigt.
\paragraph*{}
Det som föreskrivs i första stycket gäller även för motsvarande förmån som lämnas till den försäkrade på grundval av utländsk lagstiftning.
Lag (2022:858).
\subsection*{13 §}
\paragraph*{}
Bestämmelserna om arbetsgivarinträde och inträde av staten i vissa fall i 27 kap. 56-58 och 60 §§ ska tillämpas även när det gäller rehabiliteringspenning.
\subsection*{14 §}
\paragraph*{}
Särskilt bidrag lämnas under rehabiliteringstiden för kostnader som uppstår för den försäkrade i samband med rehabiliteringen. Regeringen eller den myndighet som regeringen bestämmer meddelar ytterligare föreskrifter om sådant bidrag.
\chapter*{31 a Rehabiliteringspenning i särskilda fall}
\subsection*{1 §}
\paragraph*{}
I detta kapitel finns en inledande bestämmelse i 2 §.
\paragraph*{}
Vidare finns bestämmelser om
\newline - rätten till rehabiliteringspenning i särskilda fall i 3-5 §§,
\newline - bedömning av arbetsförmågans nedsättning i 6 §,
\newline - ersättningsnivåer i 7 §,
\newline - beräkning av rehabiliteringspenning i särskilda fall i 8 och 9 §§,
\newline - rehabiliteringspenning i särskilda fall vid sjukersättning i 10 §,
\newline - rehabiliteringspenning i särskilda fall vid livränta enligt 41 eller 43 kap. i 11 §,
\newline - förmånstiden i 12 §,
\newline - behållande av rätten till rehabiliteringspenning i särskilda fall i 13 §,
\newline - upphörande av rätten till rehabiliteringspenning i särskilda fall i 14 §, och
\newline - arbetsgivarinträde m.m. i 15 §.
Lag (2011:1513).
\subsection*{2 §}
\paragraph*{}
Bestämmelserna i 31 kap. gäller även i fråga om rehabiliteringspenning i särskilda fall, om inte annat följer av detta kapitel.
Lag (2011:1513).
\subsection*{3 §}
\paragraph*{}
En försäkrad som helt eller delvis har fått tidsbegränsad sjukersättning under det högsta antalet månader som sådan ersättning kan betalas ut enligt 4 kap. 31 § lagen (2010:111) om införande av socialförsäkringsbalken har i de fall och under de närmare förutsättningar som anges i detta kapitel rätt till rehabiliteringspenning i särskilda fall. Detta gäller även för en försäkrad vars rätt till aktivitetsersättning upphör på grund av att han eller hon fyller 30 år.
Lag (2011:1514).
\subsection*{4 §}
\paragraph*{}
En försäkrad som omfattas av 3 § och som på grund av bestämmelserna om sjukpenninggrundande inkomst i denna balk inte skulle ha rätt till hel rehabiliteringspenning motsvarande vad som för sjukpenning på fortsättningsnivån medför en sjukpenning om minst 160 kronor per kalenderdag, har rätt till rehabiliteringspenning i särskilda fall vid sjukdom som sätter ned hans eller hennes arbetsförmåga under tid som anges i 31 kap. 3 §.
\paragraph*{}
Med sjukdom likställs, förutom vad som föreskrivs i 27 kap. 2 § andra stycket, även annan nedsättning av den fysiska eller psykiska prestationsförmågan.
Lag (2011:1513).
\subsection*{5 §}
\paragraph*{}
Rätten till rehabiliteringspenning i särskilda fall inträder från och med dagen efter den då rätten till sådan tidsbegränsad sjukersättning eller aktivitetsersättning som avses i 3 § har upphört.
Lag (2011:1514).
\subsection*{6 §}
\paragraph*{}
Bedömningen av arbetsförmågans nedsättning ska göras i förhållande till ett heltidsarbete.
Lag (2011:1513).
\subsection*{7 §}
\paragraph*{}
Rehabiliteringspenning i särskilda fall lämnas med högst
\newline - 160 kronor per dag vid hel förmån,
\newline - 120 kronor per dag vid tre fjärdedels förmån,
\newline - 80 kronor per dag vid halv förmån, och
\newline - 40 kronor per dag vid en fjärdedels förmån.
Lag (2011:1513).
\subsection*{8 §}
\paragraph*{}
Rehabiliteringspenning i särskilda fall lämnas för sju dagar per vecka.
Lag (2011:1513).
\subsection*{9 §}
\paragraph*{}
Rehabiliteringspenning i särskilda fall beräknas, om inte annat följer av 10 §, enligt följande:
\newline 1. För en försäkrad för vilken en sjukpenninggrundande inkomst inte kan fastställas, uppgår rehabiliteringspenning i särskilda fall per dag för respektive förmånsnivå till de belopp som anges i 7 §.
\newline 2. För annan försäkrad än den som avses i 1, motsvarar rehabiliteringspenning i särskilda fall per dag för respektive förmånsnivå differensen mellan
\newline - de belopp som anges i 7 § och
\newline - den rehabiliteringspenning som lämnas per dag enligt 31 kap.
Lag (2011:1513).
\subsection*{10 §}
\paragraph*{}
I det fall en försäkrad får sjukersättning lämnas hel rehabiliteringspenning i särskilda fall med högst
\newline - 120 kronor per dag när sjukersättning lämnas som en fjärdedels förmån,
\newline - 80 kronor per dag när sjukersättning lämnas som en halv förmån, och
\newline - 40 kronor per dag när sjukersättning lämnas som tre fjärdedels förmån.
\paragraph*{}
När rehabiliteringspenning i särskilda fall ska lämnas som partiell förmån, lämnas ersättning med högst tre fjärdedelar, hälften eller en fjärdedel av beloppen i första stycket.
\paragraph*{}
För en försäkrad för vilken en sjukpenninggrundande inkomst kan fastställas, motsvarar rehabiliteringspenning i särskilda fall per dag differensen mellan
\newline - de belopp som framgår av första eller andra styckena och
\newline - den rehabiliteringspenning som lämnas per dag enligt 31 kap.
Lag (2011:1513).
\paragraph*{}
Rehabiliteringspenning i särskilda fall vid livränta enligt 41 eller 43 kap.
\subsection*{11 §}
\paragraph*{}
I det fall en försäkrad får livränta enligt 41 eller 43 kap., minskas rehabiliteringspenning i särskilda fall per dag enligt följande.
\paragraph*{}
När rehabiliteringspenning i särskilda fall har beräknats enligt 9 §, ska det framräknade beloppet per dag minskas med ett belopp som motsvarar det livräntebelopp enligt 41 eller 43 kap. som gäller vid tiden för beslutet delat med 365.
Livräntebeloppet avrundas till närmaste hela krontal och 50 öre avrundas nedåt.
\paragraph*{}
Det som föreskrivs i första och andra styckena gäller endast om den försäkrade inte får sjukersättning för samma tid som livränta enligt 41 eller 43 kap.
Lag (2011:1513).
\subsection*{12 §}
\paragraph*{}
När det kan lämnas rehabiliteringspenning på den nivå som föreskrivs i 27 kap. 21-24 §§, ska även rehabiliteringspenning i särskilda fall kunna lämnas.
Lag (2015:963).
\subsection*{13 §}
\paragraph*{}
En försäkrad behåller sin rätt till rehabiliteringspenning i särskilda fall under tid då
\newline 1. han eller hon förvärvsarbetar,
\newline 2. en sådan situation föreligger som anges i 26 kap. 11, 12, eller 14-18 §§ som grund för SGI-skydd, eller
\newline 3. han eller hon deltar i ett arbetsmarknadspolitiskt program och får aktivitetsstöd eller står till arbetsmarknadens förfogande.
\paragraph*{}
Rätten enligt första stycket gäller endast om den inte upphör på grund av 14 §.
\paragraph*{}
Regeringen eller den myndighet som regeringen bestämmer kan med stöd av 8 kap. 7 § regeringsformen meddela
\newline 1. föreskrifter om undantag från kravet på att den som deltar i ett arbetsmarknadspolitiskt program ska få aktivitetsstöd, och
\newline 2. föreskrifter om de villkor som gäller för att den försäkrade ska anses stå till arbetsmarknadens förfogande.
Lag (2015:119).
\subsection*{14 §}
\paragraph*{}
Rätten till rehabiliteringspenning i särskilda fall upphör när det för den försäkrade har bestämts eller skulle ha kunnat bestämmas en sjukpenninggrundande inkomst som uppgår till minst 80 300 kronor.
Lag (2011:1513).
\subsection*{15 §}
\paragraph*{}
Bestämmelserna om arbetsgivarinträde m.m. i 31 kap. 13 § gäller inte rehabiliteringspenning i särskilda fall.
Lag (2011:1513).
\chapter*{32 Innehåll}
\subsection*{1 §}
\paragraph*{}
I denna underavdelning finns allmänna bestämmelser om sjukersättning och aktivitetsersättning i 33 kap.
\paragraph*{}
Vidare finns
\newline - bestämmelser om inkomstrelaterad sjukersättning och inkomstrelaterad aktivitetsersättning i 34 kap.,
\newline - bestämmelser om sjukersättning och aktivitetsersättning i form av garantiersättning i 35 kap.,
\newline - gemensamma bestämmelser om sjukersättning och aktivitetsersättning i 36 kap., och
\newline - bestämmelser om försäkrade som senast för juli 2008 beviljats icke tidsbegränsad sjukersättning i 37 kap.
\chapter*{33 Allmänna bestämmelser om sjukersättning och aktivitetsersättning}
\subsection*{1 §}
\paragraph*{}
I detta kapitel finns inledande bestämmelser i 2-4 §§.
\paragraph*{}
Vidare finns bestämmelser om
\newline - rätten till sjukersättning eller aktivitetsersättning i 5-8 §§,
\newline - förmånsnivåer i 9-13 §§,
\newline - förmånstiden i 14-20 §§,
\newline - aktiviteter under tid med aktivitetsersättning i 21-25 §§, och
\newline - särskilda insatser för försäkrade med tre fjärdedels ersättning i 26-28 §§.
\subsection*{2 §}
\paragraph*{}
Sjukersättning eller aktivitetsersättning kan lämnas till en försäkrad vars arbetsförmåga är långvarigt nedsatt.
\subsection*{3 §}
\paragraph*{}
Sjukersättning och aktivitetsersättning lämnas i form av
\newline 1. inkomstrelaterad ersättning enligt 34 kap., och
\newline 2. garantiersättning enligt 35 kap.
\subsection*{4 §}
\paragraph*{}
Sjukersättning lämnas tills vidare, medan aktivitetsersättning lämnas för viss tid.
\subsection*{5 §}
\paragraph*{}
En försäkrad vars arbetsförmåga är nedsatt med minst en fjärdedel på grund av sjukdom eller annan nedsättning av den fysiska eller psykiska prestationsförmågan och som var försäkrad vid försäkringsfallet har, enligt närmare bestämmelser i denna underavdelning, rätt till sjukersättning eller aktivitetsersättning.
\paragraph*{}
Om försäkringsfallet har inträffat före ingången av det år då den försäkrade fyllde 18 år, gäller inte kravet att den försäkrade ska vara försäkrad vid försäkringsfallet.
\subsection*{6 §}
\paragraph*{}
För rätt till sjukersättning krävs att arbetsförmågan kan anses stadigvarande nedsatt och att åtgärder som avses i 27 kap. 6 § samt i 29-31 kap. inte bedöms kunna leda till att den försäkrade återfår någon arbetsförmåga i ett sådant arbete som avses i 10 eller 10 a §.
Lag (2022:1222).
\subsection*{7 §}
\paragraph*{}
För rätt till aktivitetsersättning krävs att nedsättningen kan antas bestå under minst ett år.
\subsection*{8 §}
\paragraph*{}
En försäkrad som på grund av funktionshinder ännu inte har avslutat sin skolgång på grundskolenivå och gymnasial nivå vid ingången av juli det år då han eller hon fyller 19 år har rätt till aktivitetsersättning.
\subsection*{9 §}
\paragraph*{}
Sjukersättning och aktivitetsersättning lämnas enligt följande förmånsnivåer:
\newline 1. Hel sjukersättning eller aktivitetsersättning lämnas när den försäkrades arbetsförmåga är helt eller i det närmaste helt nedsatt.
\newline 2. Tre fjärdedels sjukersättning eller aktivitetsersättning lämnas när den försäkrades arbetsförmåga är nedsatt i mindre grad än som anges i 1 men med minst tre fjärdedelar.
\newline 3. Halv sjukersättning eller aktivitetsersättning lämnas när den försäkrades arbetsförmåga är nedsatt med mindre än tre fjärdedelar men med minst hälften.
\newline 4. En fjärdedels sjukersättning eller aktivitetsersättning lämnas när den försäkrades arbetsförmåga är nedsatt med mindre än hälften men med minst en fjärdedel.
\subsection*{10 §}
\paragraph*{}
När det bedöms hur nedsatt arbetsförmågan är ska Försäkringskassan beakta den försäkrades förmåga att försörja sig själv genom förvärvsarbete på arbetsmarknaden.
\subsection*{10 a §}
\paragraph*{}
För en försäkrad som har uppnått den ålder då det återstår som mest fem år till den tidpunkt då sjukersättning som längst kan lämnas enligt 16 § och som har erfarenhet av sådant förvärvsarbete som är normalt förekommande på arbetsmarknaden ska Försäkringskassan, när det bedöms hur nedsatt arbetsförmågan är, beakta den försäkrades förmåga att försörja sig själv genom
\newline 1. sådant förvärvsarbete som är normalt förekommande på arbetsmarknaden och som den försäkrade har erfarenhet av, eller
\newline 2. annat lämpligt arbete som är tillgängligt för honom eller henne.
\paragraph*{}
Med förvärvsarbeten som den försäkrade har erfarenhet av avses arbeten som den försäkrade har haft under en tidsperiod på femton år före den månad för vilken han eller hon ansöker om sjukersättning, före ett beviljande enligt 36 kap. 25 § eller före en omprövning av rätten till sjukersättning enligt 36 kap. 19 §.
Lag (2022:1222).
\subsection*{10 b §}
\paragraph*{}
Den som har uppnått den ålder då det återstår som mest fem år till den tidpunkt då sjukersättning som längst kan lämnas enligt 16 § ska omfattas av regleringen i 10 a § även om tidpunkten då sjukersättning som längst kan lämnas därefter skulle senareläggas genom en författningsändring.
Lag (2022:1222).
\subsection*{11 §}
\paragraph*{}
Bedömningen enligt 10 och 10 a §§ ska göras
\newline - efter samma grunder oavsett på vilket sätt prestationsförmågan är nedsatt, och
\newline - i förhållande till ett heltidsarbete.
\paragraph*{}
Med inkomst av arbete likställs i skälig omfattning värdet av arbete med skötsel av hemmet.
Lag (2022:1222).
\subsection*{12 §}
\paragraph*{}
Under tid som den försäkrade genomgår behandling eller rehabilitering som avses i 27 kap. 6 § eller 31 kap. 3 §, ska arbetsförmågan anses nedsatt i den utsträckning som behandlingen eller rehabiliteringen hindrar honom eller henne från att förvärvsarbeta.
\subsection*{13 §}
\paragraph*{}
En funktionshindrad som avses i 8 § får hel aktivitetsersättning oberoende av arbetsförmågans nedsättning.
\subsection*{14 §}
\paragraph*{}
Sjukersättning eller aktivitetsersättning lämnas från och med den månad då rätten till förmånen har uppkommit, dock inte för längre tid tillbaka än tre månader före ansökningsmånaden.
\paragraph*{}
När sjukersättning eller aktivitetsersättning lämnas utan ansökan enligt 36 kap. 25 §, lämnas ersättningen dock från och med månaden efter den då beslutet om förmånen meddelats.
\subsection*{15 §}
\paragraph*{}
Sjukersättning och aktivitetsersättning lämnas, om inte något annat är särskilt föreskrivet, till och med den månad när rätten till förmånen upphör.
\subsection*{16 §}
\paragraph*{}
/Upphör att gälla U:2025-12-01/
Hel sjukersättning kan lämnas tidigast från och med juli det år då den försäkrade fyller 19 år och längst till och med månaden före den månad då han eller hon fyller 66 år.
\paragraph*{}
Tre fjärdedels, halv eller en fjärdedels sjukersättning kan lämnas tidigast från och med den månad då den försäkrade fyller 30 år och längst till och med månaden före den månad då han eller hon fyller 66 år.
Lag (2022:878).
\subsection*{16 §}
\paragraph*{}
/Träder i kraft I:2025-12-01/
Hel sjukersättning kan lämnas tidigast från och med juli det år då den försäkrade fyller 19 år och längst till och med månaden före den månad då han eller hon uppnår riktåldern för pension.
\paragraph*{}
Tre fjärdedels, halv eller en fjärdedels sjukersättning kan lämnas tidigast från och med den månad då den försäkrade fyller 30 år och längst till och med månaden före den månad då han eller hon uppnår riktåldern för pension.
Lag (2022:879).
\subsection*{17 §}
\paragraph*{}
/Upphör att gälla U:2025-12-01/
Försäkringskassan ska senast tre år räknat från ett beslut om sjukersättning göra en uppföljning av den försäkrades arbetsförmåga. Försäkringskassan ska därefter, så länge den försäkrade har rätt till sjukersättning, minst vart tredje år på nytt följa upp den försäkrades arbetsförmåga.
\paragraph*{}
Om den försäkrade har fyllt 61 år, behöver någon uppföljning inte göras.
Lag (2022:878).
\subsection*{17 §}
\paragraph*{}
/Träder i kraft I:2025-12-01/
Försäkringskassan ska senast tre år räknat från ett beslut om sjukersättning göra en uppföljning av den försäkrades arbetsförmåga. Försäkringskassan ska därefter, så länge den försäkrade har rätt till sjukersättning, minst vart tredje år på nytt följa upp den försäkrades arbetsförmåga.
\paragraph*{}
Om den försäkrade har fem eller färre år kvar tills riktåldern för pension uppnås, behöver någon uppföljning inte göras.
Lag (2022:879).
\subsection*{18 §}
\paragraph*{}
Aktivitetsersättning kan tidigast lämnas från och med juli det år då den försäkrade fyller 19 år och längst till och med månaden före den månad då han eller hon fyller 30 år.
\subsection*{19 §}
\paragraph*{}
Ett beslut om aktivitetsersättning får inte avse längre tid än tre år.
\subsection*{20 §}
\paragraph*{}
En funktionshindrad som avses i 8 § har rätt till aktivitetsersättning under den tid som skolgången varar.
\subsection*{21 §}
\paragraph*{}
I samband med ett beslut om att bevilja en försäkrad aktivitetsersättning ska Försäkringskassan undersöka om han eller hon under den tid ersättningen ska lämnas kan delta i aktiviteter som kan antas ha en gynnsam inverkan på hans eller hennes sjukdomstillstånd eller fysiska eller psykiska prestationsförmåga.
\subsection*{22 §}
\paragraph*{}
Om den försäkrade bedöms kunna delta i aktiviteter ska Försäkringskassan närmare planera vilka aktiviteter som är lämpliga för honom eller henne. Planeringen ska ske i samråd med den försäkrade och Försäkringskassan ska i möjligaste mån tillgodose den försäkrades önskemål. Om Försäkringskassan och den försäkrade kommer överens ska Försäkringskassan upprätta en plan för aktiviteterna.
\subsection*{23 §}
\paragraph*{}
Försäkringskassan ska verka för att planerade aktiviteter kommer till stånd. Försäkringskassan ska samordna de insatser som behövs och se till att åtgärder vidtas för att underlätta för den försäkrade att delta i aktiviteterna.
\subsection*{24 §}
\paragraph*{}
Som aktiviteter enligt 21-23 §§ räknas inte sådana åtgärder som avses i 110 kap. 14 § 4.
\subsection*{25 §}
\paragraph*{}
Särskild ersättning kan lämnas för den försäkrades kostnader med anledning av de aktiviteter som han eller hon deltar i. Regeringen eller den myndighet som regeringen bestämmer meddelar ytterligare föreskrifter om sådan ersättning.
\subsection*{26 §}
\paragraph*{}
För den som får tre fjärdedels sjukersättning eller aktivitetsersättning ska särskilda insatser göras för att han eller hon ska kunna få en anställning motsvarande den återstående arbetsförmågan.
\subsection*{27 §}
\paragraph*{}
Försäkringskassan ska med den försäkrades samtycke se till att sådana insatser som avses i 26 § inleds.
Lag (2020:1256).
\subsection*{28 §}
\paragraph*{}
Regeringen eller den myndighet som regeringen bestämmer kan med stöd av 8 kap. 7 § regeringsformen meddela ytterligare föreskrifter om sådana insatser som avses i 26 §.
Lag (2020:1256).
\chapter*{34 Inkomstrelaterad sjukersättning och inkomstrelaterad aktivitetsersättning}
\subsection*{1 §}
\paragraph*{}
I detta kapitel finns bestämmelser om
\newline - rätten till inkomstrelaterad ersättning i 2 §,
\newline - ramtid i 3 §,
\newline - beräkningsunderlag för inkomstrelaterad ersättning i 4-11 §§, och
\newline - beräkning av inkomstrelaterad ersättning i 12-14 §§.
\subsection*{2 §}
\paragraph*{}
Rätt till inkomstrelaterad sjukersättning eller aktivitetsersättning har en försäkrad som avses i 33 kap. om det för honom eller henne har fastställts pensionsgrundande inkomst enligt 59 kap. för minst ett år under en viss tidsperiod (ramtid) närmast före det år då försäkringsfallet inträffade.
\subsection*{3 §}
\paragraph*{}
Ramtiden är, om inte något annat följer av 10 och 11 §§,
\newline - 5 år för den som fyller 53 år eller mer det år då försäkringsfallet inträffar,
\newline - 6 år för den som fyller minst 50 år och högst 52 år det år då försäkringsfallet inträffar,
\newline - 7 år för den som fyller minst 47 år och högst 49 år det år då försäkringsfallet inträffar, och
\newline - 8 år för den som fyller högst 46 år det år då försäkringsfallet inträffar.
\subsection*{4 §}
\paragraph*{}
Inkomstrelaterad sjukersättning och aktivitetsersättning beräknas på grundval av en antagandeinkomst. Denna beräknas enligt 6-11 §§ med ledning av den försäkrades bruttoårsinkomster inom ramtiden.
\subsection*{5 §}
\paragraph*{}
Antagandeinkomsten anknyts till prisbasbeloppet för det år då ersättningen ska börja lämnas och räknas om vid förändringar av detta belopp.
\subsection*{6 §}
\paragraph*{}
Med bruttoårsinkomst avses
\newline 1. pensionsgrundande inkomst enligt 59 kap. med tillägg för debiterade allmänna pensionsavgifter för respektive år, och
\newline 2. pensionsgrundande belopp enligt 60 kap. som tillgodoräknats med anledning av sjukersättning eller aktivitetsersättning.
\subsection*{7 §}
\paragraph*{}
När bruttoårsinkomsten beräknas ska det bortses från pensionsgrundande inkomster och pensionsgrundande belopp som sammantagna överstiger 7,5 gånger det prisbasbelopp som gäller för beskattningsåret.
Lag (2011:1434).
\subsection*{8 §}
\paragraph*{}
Bruttoårsinkomsten räknas om med hänsyn till förändringar i prisbasbeloppet på följande sätt:
\paragraph*{}
Bruttoårsinkomsten multipliceras med kvoten mellan
\newline - prisbasbeloppet för det år sjukersättningen eller aktivitetsersättningen ska börja lämnas och
\newline - prisbasbeloppet för det år bruttoårsinkomsten avser.
\subsection*{9 §}
\paragraph*{}
Antagandeinkomsten motsvarar genomsnittet av de tre högsta enligt 8 § omräknade bruttoårsinkomsterna inom ramtiden.
\paragraph*{}
Om endast en eller två bruttoårsinkomster kan tillgodoräknas inom ramtiden, ska två respektive en bruttoårsinkomst om noll kronor tas med i beräkningen.
\subsection*{10 §}
\paragraph*{}
Vid beräkning av inkomstrelaterad sjukersättning till och med månaden före den månad då den försäkrade fyller 30 år eller vid beräkning av inkomstrelaterad aktivitetsersättning får, om det medför en högre antagandeinkomst, vid tillämpning av 9 § i stället användas de två högsta enligt 8 § omräknade bruttoårsinkomsterna inom en ramtid av tre år.
\paragraph*{}
Om endast en bruttoårsinkomst kan tillgodoräknas inom ramtiden, ska en bruttoårsinkomst om noll kronor tas med i beräkningen.
Lag (2016:1291).
\subsection*{11 §}
\paragraph*{}
Om den försäkrade, vid beräkning av antagandeinkomst enligt 10 §, kan tillgodoräknas en bruttoårsinkomst året före det år då försäkringsfallet inträffar men denna är lägre än den sjukpenninggrundande inkomst enligt 25 och 26 kap. som den försäkrade skulle ha haft vid tidpunkten för försäkringsfallet, får den sjukpenninggrundande inkomsten, till den del den inte överstiger 7,5 prisbasbelopp, i stället utgöra bruttoårsinkomst.
\subsection*{12 §}
\paragraph*{}
Hel inkomstrelaterad sjukersättning och aktivitetsersättning lämnas för år räknat med 64,7 procent av den försäkrades antagandeinkomst.
Lag (2015:453).
\subsection*{13 §}
\paragraph*{}
Partiell inkomstrelaterad sjukersättning och aktivitetsersättning lämnas för år räknat med så stor andel av hel sådan ersättning som motsvarar den andel av sjukersättning eller aktivitetsersättning som den försäkrade har rätt till enligt 33 kap. 9 §.
\subsection*{14 §}
\paragraph*{}
Från inkomstrelaterad sjukersättning och aktivitetsersättning ska sådana förmåner räknas av som lämnas till den försäkrade enligt utländsk lagstiftning och som motsvarar sjukersättning eller aktivitetsersättning eller som utgör pension vid invaliditet.
\chapter*{35 Sjukersättning och aktivitetsersättning i form av garantiersättning}
\subsection*{1 §}
\paragraph*{}
I detta kapitel finns bestämmelser om
\newline - rätten till garantiersättning i 2 och 3 §§,
\newline - försäkringstiden i 4-15 §§,
\newline - ersättningsnivåer i 18-20 §§,
\newline - beräkningsunderlag för garantiersättning i 21 och 22 §§, och
\newline - beräkning av garantiersättning i 23-25 §§.
Lag (2014:239).
\subsection*{2 §}
\paragraph*{}
Rätt till sjukersättning och aktivitetsersättning i form av garantiersättning har en försäkrad som avses i 33 kap. och
\newline 1. som saknar inkomstrelaterad sjukersättning eller aktivitetsersättning enligt 34 kap., eller
\newline 2. vars inkomstrelaterade ersättning enligt samma kapitel understiger en viss nivå (garantinivå).
\paragraph*{}
Garantiersättningen är beroende av en särskilt beräknad försäkringstid.
\subsection*{3 §}
\paragraph*{}
Garantiersättning lämnas endast till den som kan tillgodoräknas en försäkringstid om minst tre år.
\subsection*{4 §}
\paragraph*{}
/Upphör att gälla U:2025-12-01/
Försäkringstid för rätt till garantiersättning tillgodoräknas en försäkrad
\newline 1. enligt 6-11 §§ under tiden från och med det år då han eller hon fyllde 16 år till och med året före försäkringsfallet (faktisk försäkringstid),
\newline 2. enligt 12 och 13 §§ för tiden därefter till och med det år då han eller hon fyller 65 år (framtida försäkringstid), och
\newline 3. enligt 14 och 15 §§ om försäkringsfallet har inträffat före 18 års ålder.
Lag (2022:878).
\subsection*{4 §}
\paragraph*{}
/Träder i kraft I:2025-12-01/
Försäkringstid för rätt till garantiersättning tillgodoräknas en försäkrad
\newline 1. enligt 6-11 §§ under tiden från och med det år då han eller hon fyllde 16 år till och med året före försäkringsfallet (faktisk försäkringstid),
\newline 2. enligt 12 och 13 §§ för tiden därefter till och med året före det år då han eller hon uppnår den vid försäkringsfallet gällande riktåldern för pension (framtida försäkringstid), och
\newline 3. enligt 14 och 15 §§ om försäkringsfallet har inträffat före 18 års ålder.
Lag (2022:879).
\subsection*{5 §}
\paragraph*{}
Faktisk försäkringstid och framtida försäkringstid sätts ned var för sig till närmaste antal hela månader.
\paragraph*{}
Den sammanlagda försäkringstiden utgör summan av faktisk försäkringstid och framtida försäkringstid. Den sammanlagda försäkringstiden ska sättas ned till närmaste antal hela år.
\subsection*{6 §}
\paragraph*{}
Som faktisk försäkringstid räknas tid när en person har varit försäkrad på grund av bosättning i Sverige enligt 5 kap.
\subsection*{7 §}
\paragraph*{}
Som faktisk försäkringstid räknas även tid när en person före tidpunkten för bosättning i Sverige oavbrutet har vistats här i landet efter att ha ansökt om uppehållstillstånd.
\subsection*{8 §}
\paragraph*{}
Om den försäkrade har beviljats uppehållstillstånd i Sverige som flykting enligt 4 kap. 1 § eller som alternativt skyddsbehövande enligt 2 § utlänningslagen (2005:716) eller enligt motsvarande äldre bestämmelser eller beviljats statusförklaring enligt 4 kap. 3 c § utlänningslagen eller motsvarande förklaring enligt äldre bestämmelser räknas som faktisk försäkringstid även tid då han eller hon har varit bosatt i sitt tidigare hemland från och med det år då han eller hon fyllde 16 år till tidpunkten då han eller hon först kom till Sverige.
Lag (2021:767).
\subsection*{9 §}
\paragraph*{}
Vid beräkningen enligt 8 § ska en så stor andel av tiden i hemlandet räknas in som svarar mot kvoten mellan
\newline - den tid när den försäkrade har varit bosatt i Sverige, inklusive den tid som avses i 7 §, från den första ankomsten till landet till och med året före försäkringsfallet och
\newline - hela tidrymden från det att den försäkrade första gången kom till landet till och med året före försäkringsfallet.
\paragraph*{}
Vid beräkningen ska det bortses från tid för vilken den försäkrade, vid bosättning i Sverige, har rätt till sådan ersättning från det andra landet som inte enligt 22 § ska ligga till grund för beräkning av garantiersättning.
\subsection*{10 §}
\paragraph*{}
Vid tillämpning av 8 och 9 §§ likställs med tid för bosättning i hemlandet tid då den försäkrade före den första ankomsten till Sverige har befunnit sig i ett annat land där han eller hon beretts en tillfällig fristad.
\subsection*{11 §}
\paragraph*{}
När faktisk försäkringstid beräknas för en person som enligt 5 kap. 6 § anses bosatt i Sverige även under vistelse utomlands ska det bortses från tid för vilken den utsände, vid bosättning i Sverige, har rätt till sådan ersättning från det andra landet som inte enligt 22 § ska ligga till grund för beräkning av garantiersättning.
\subsection*{12 §}
\paragraph*{}
/Upphör att gälla U:2025-12-01/
Om den faktiska försäkringstiden enligt 6-11 §§ utgör minst fyra femtedelar av tiden från och med det år då den försäkrade fyllde 16 år till och med året före försäkringsfallet räknas hela tiden därefter till och med det år då den försäkrade fyller 65 år som framtida försäkringstid.
Lag (2022:878).
\subsection*{12 §}
\paragraph*{}
/Träder i kraft I:2025-12-01/
Om den faktiska försäkringstiden enligt 6-11 §§ utgör minst fyra femtedelar av tiden från och med det år då den försäkrade fyllde 16 år till och med året före försäkringsfallet, räknas hela tiden därefter till och med året före det år då den försäkrade uppnår den vid försäkringsfallet gällande riktåldern för pension som framtida försäkringstid.
Lag (2022:879).
\subsection*{13 §}
\paragraph*{}
/Upphör att gälla U:2025-12-01/
Om den faktiska försäkringstiden motsvarar mindre än fyra femtedelar av tiden från och med det år då den försäkrade fyllde 16 år till och med året före försäkringsfallet gäller följande.
\paragraph*{}
Som framtida försäkringstid räknas en så stor andel av tiden från och med året för försäkringsfallet, till och med det år då den försäkrade fyller 65 år, som motsvarar kvoten mellan den faktiska försäkringstiden och fyra femtedelar av tiden från och med det år då den försäkrade fyllde 16 år till och med året före försäkringsfallet.
Lag (2022:878).
\subsection*{13 §}
\paragraph*{}
/Träder i kraft I:2025-12-01/
Om den faktiska försäkringstiden motsvarar mindre än fyra femtedelar av tiden från och med det år då den försäkrade fyllde 16 år till och med året före försäkringsfallet gäller följande.
\paragraph*{}
Som framtida försäkringstid räknas en så stor andel av tiden från och med året för försäkringsfallet, till och med året före det år då den försäkrade uppnår den vid försäkringsfallet gällande riktåldern för pension, som motsvarar kvoten mellan den faktiska försäkringstiden och fyra femtedelar av tiden från och med det år då den försäkrade fyllde 16 år till och med året före försäkringsfallet.
Lag (2022:879).
\subsection*{14 §}
\paragraph*{}
Om ett försäkringsfall har inträffat före det år då den försäkrade fyllde 18 år, får försäkringstiden i stället för vad som följer av 12 och 13 §§ beräknas enligt 15 §.
Lag (2014:239).
\subsection*{15 §}
\paragraph*{}
/Upphör att gälla U:2025-12-01/
För en försäkrad som avses i 14 § ska som försäkringstid räknas tiden från och med det år då han eller hon fyllde 16 år till och med det år då han eller hon fyller 65 år. Hänsyn ska dock bara tas till tid då den försäkrade efter fyllda 16 år har uppfyllt förutsättningarna för tillgodoräknande av försäkringstid enligt 6-11 §§.
Lag (2022:878).
\subsection*{15 §}
\paragraph*{}
/Träder i kraft I:2025-12-01/
För en försäkrad som avses i 14 § ska som försäkringstid räknas tiden från och med det år då han eller hon fyllde 16 år till och med året före det år då han eller hon uppnår den riktålder för pension som gällde när han eller hon fyllde 17 år. Hänsyn ska dock bara tas till tid då den försäkrade efter fyllda 16 år har uppfyllt förutsättningarna för tillgodoräknande av försäkringstid enligt 6-11 §§.
Lag (2022:879).
\subsection*{16 §}
\paragraph*{}
Har upphävts genom
lag (2014:239).
\subsection*{17 §}
\paragraph*{}
Har upphävts genom
lag (2014:239).
\subsection*{18 §}
\paragraph*{}
Garantinivån för hel sjukersättning från och med den månad då den försäkrade fyller 30 år motsvarar för år räknat 2,78 prisbasbelopp.
Lag (2021:1240).
\subsection*{19 §}
\paragraph*{}
Garantinivån för hel sjukersättning till och med månaden före den månad då den försäkrade fyller 30 år samt för hel aktivitetsersättning motsvarar för år räknat
\newline - 2,48 prisbasbelopp till och med månaden före den månad då den försäkrade fyller 21 år,
\newline - 2,53 prisbasbelopp från och med den månad då den försäkrade fyller 21 år till och med månaden före den månad då han eller hon fyller 23 år,
\newline - 2,58 prisbasbelopp från och med den månad då den försäkrade fyller 23 år till och med månaden före den månad då han eller hon fyller 25 år,
\newline - 2,63 prisbasbelopp från och med den månad då den försäkrade fyller 25 år till och med månaden före den månad då han eller hon fyller 27 år,
\newline - 2,68 prisbasbelopp från och med den månad då den försäkrade fyller 27 år till och med månaden före den månad då han eller hon fyller 29 år, samt
\newline - 2,73 prisbasbelopp från och med den månad då den försäkrade fyller 29 år till och med månaden före den månad då han eller hon fyller 30 år.
Lag (2021:1240).
\subsection*{20 §}
\paragraph*{}
För den som inte kan tillgodoräknas 40 års försäkringstid gäller följande:
\paragraph*{}
Garantinivån enligt 18 eller 19 § ska avkortas till så stor andel av denna som svarar mot kvoten mellan - försäkringstiden enligt 5 § andra stycket och
\newline - 40.
\subsection*{21 §}
\paragraph*{}
Till grund för beräkning av garantiersättning ska ligga den årliga inkomstrelaterade sjukersättning eller aktivitetsersättning enligt 34 kap. som den försäkrade har rätt till för det år som garantiersättningen avser, före minskning med livränta enligt 36 kap. 3-8 §§.
\subsection*{22 §}
\paragraph*{}
Med inkomstrelaterad sjukersättning och aktivitetsersättning avses i detta kapitel även sådana utländska förmåner som
\newline - ska avräknas från inkomstrelaterad ersättning enligt 34 kap. 14 §, och
\newline - inte är att likställa med garantiersättning enligt detta kapitel.
\subsection*{23 §}
\paragraph*{}
För den som inte har inkomstrelaterad sjukersättning eller aktivitetsersättning motsvarar årlig hel garantiersättning garantinivån.
\subsection*{24 §}
\paragraph*{}
För den vars hela inkomstrelaterade sjukersättning eller aktivitetsersättning för år räknat understiger garantinivån gäller följande:
\paragraph*{}
Årlig garantiersättning motsvarar differensen mellan
\newline - garantinivån och
\newline - hela den årliga inkomstrelaterade sjukersättningen eller aktivitetsersättningen enligt 21 och 22 §§.
\paragraph*{}
Beräkningen görs efter det att garantinivån minskats enligt 20 §.
\subsection*{25 §}
\paragraph*{}
Partiell garantiersättning lämnas för år räknat med så stor andel av hel sådan ersättning enligt 23 och 24 §§ som motsvarar den andel av sjukersättning eller aktivitetsersättning som den försäkrade har rätt till enligt 33 kap. 9 §.
\chapter*{36 Gemensamma bestämmelser om sjukersättning och aktivitetsersättning}
\subsection*{1 §}
\paragraph*{}
I detta kapitel finns bestämmelser om
\newline - samordning med andra förmåner 2-8 §§,
\newline - förvärvsarbete som hinder för rätt till förmån i 9 §,
\newline - aktivitetsersättning under prövotid i 9 a §,
\newline - vilande sjukersättning eller aktivitetsersättning i 10-18 §§,
\newline - aktivitetsersättning efter tid för vilandeförklaring i 18 a-18 d §§,
\newline - omprövning vid ändrade förhållanden i 19-24 §§,
\newline - ersättning utan ansökan i vissa fall i 25-27 §§,
\newline - ändring av ersättning i 28 §, och
\newline - utbetalning av ersättning i 29 och 30 §§.
Lag (2016:1291).
\subsection*{2 §}
\paragraph*{}
Om en försäkrad för samma månad har rätt till såväl sjukersättning i form av garantiersättning eller aktivitetsersättning i form av garantiersättning som efterlevandestöd, lämnas endast den till beloppet största av förmånerna.
Lag (2016:1291).
\subsection*{3 §}
\paragraph*{}
Inkomstrelaterad sjukersättning och sjukersättning i form av garantiersättning ska minskas om den försäkrade
\newline 1. har rätt till livränta på grund av obligatorisk försäkring enligt den upphävda lagen (1954:243) om yrkesskadeförsäkring eller någon annan författning,
\newline 2. enligt någon annan författning eller enligt särskilt beslut av regeringen har rätt till annan livränta, som bestäms eller betalas ut av Försäkringskassan, eller
\newline 3. får livränta enligt utländsk lagstiftning om yrkesskadeförsäkring.
\subsection*{4 §}
\paragraph*{}
Minskning enligt 3 § får inte göras på grund av livränta enligt 41-44 kap.
\subsection*{5 §}
\paragraph*{}
Om en skada, som livränta har börjat lämnas för, återigen medför sjukdom som berättigar till sjukpenning, ska det anses som om livränta hade lämnats under sjukdomstiden.
\subsection*{6 §}
\paragraph*{}
Summan av inkomstrelaterad sjukersättning och garantiersättning ska minskas med tre fjärdedelar av varje livränta vars årsbelopp överstiger en sjättedel av prisbasbeloppet och som den sjukersättningsberättigade har rätt till som skadad. Minskningen ska i första hand göras på garantiersättningen.
\subsection*{7 §}
\paragraph*{}
Om livränta, del av livränta eller livränta för viss tid har bytts ut mot ett engångsbelopp, ska det vid beräkningen enligt 6 § anses som om den livränta som lämnas har höjts med ett belopp som motsvarar engångsbeloppet enligt de försäkringstekniska grunder som tillämpades vid utbytet.
\subsection*{8 §}
\paragraph*{}
Sammanlagd hel sjukersättning får aldrig, på grund av bestämmelserna i 3-7 §§, för månad räknat understiga 5 procent av prisbasbeloppet.
Förvärvsarbete som hinder för rätt till förmån
\subsection*{9 §}
\paragraph*{}
En försäkrad som förvärvsarbetar med utnyttjande av en arbetsförmåga som han eller hon antogs sakna när beslutet om sjukersättning eller aktivitetsersättning fattades har inte rätt att få sådan ersättning för samma tid och i den omfattning som förvärvsarbetet utförs. Bestämmelser om återkrav finns i 108 kap.
\subsection*{9 a §}
\paragraph*{}
Försäkringskassan får efter ansökan av en försäkrad som under minst tolv månader har fått aktivitetsersättning på grund av nedsatt arbetsförmåga besluta att han eller hon får studera utan att aktivitetsersättningen minskas med hänsyn till studierna (aktivitetsersättning under prövotid).
\paragraph*{}
Under hela den tidsperiod som aktivitetsersättning kan lämnas enligt 33 kap. 18 § kan aktivitetsersättning under prövotid betalas ut under sammanlagt högst sex månader.
\paragraph*{}
Om studierna avbryts under prövotiden eller beslutet avser kortare tid än sex månader, får Försäkringskassan senare fatta ett nytt beslut om aktivitetsersättning under prövotid. Ett sådant beslut får tillsammans med redan lämnad aktivitetsersättning under prövotid sammanlagt uppgå till högst sex månader.
Lag (2016:1291).
\subsection*{10 §}
\paragraph*{}
Försäkringskassan får efter ansökan av den försäkrade besluta att hans eller hennes sjukersättning eller aktivitetsersättning i den omfattning som anges i 13-15 §§ ska förklaras vilande när den försäkrade förvärvsarbetar eller studerar med utnyttjande av en arbetsförmåga som han eller hon antogs sakna när beslutet om förmånen fattades. Ett sådant beslut får fattas endast om den försäkrade under minst tolv månader omedelbart dessförinnan har fått sjukersättning eller aktivitetsersättning.
\subsection*{11 §}
\paragraph*{}
Sjukersättning och aktivitetsersättning som har förklarats vilande får inte betalas ut för den tid som vilandeförklaringen avser.
\subsection*{12 §}
\paragraph*{}
Vilandeförklaring får avse hel sjukersättning eller aktivitetsersättning eller en sådan andel av ersättningen som anges i 33 kap. 9 §. När det bedöms hur stor del av förmånen som ska förklaras vilande ska Försäkringskassan beakta omfattningen av det förvärvsarbete som den försäkrade avser att utföra. Vid studier ska dock alltid den beviljade förmånen i sin helhet förklaras vilande.
\subsection*{13 §}
\paragraph*{}
Försäkringskassan får besluta att sjukersättningen eller aktivitetsersättningen ska förklaras vilande från och med den månad som anges i ansökan.
\subsection*{14 §}
\paragraph*{}
Sjukersättningen får förklaras vilande under högst tjugofyra månader, dock längst till utgången av tjugofjärde månaden från och med den första månad som beslutet omfattar.
\subsection*{15 §}
\paragraph*{}
Aktivitetsersättningen får förklaras vilande under högst 24 månader, dock längst till utgången av tjugofjärde månaden från och med den första månad som beslutet omfattar.
\paragraph*{}
Om den försäkrade under en period om tolv månader före den första månad som vilandeförklaringen avser har fått aktivitetsersättning under prövotid, får ett beslut om vilandeförklaring tillsammans med prövotidsperioden uppgå till högst 24 månader.
\paragraph*{}
Beslut om vilandeförklaring får avse en period som är längre än den period som återstår enligt beslutet om aktivitetsersättning.
\paragraph*{}
Ett beslut om vilandeförklaring enligt tredje stycket får fattas senast under månaden före den sista månad som beslutet om aktivitetsersättning omfattar.
Lag (2016:1291).
\subsection*{15 a §}
\paragraph*{}
Sedan tiden för ett beslut om vilandeförklaring enligt 15 § tredje stycket har löpt ut eller beslutet har upphävts enligt 16 § får ett nytt beslut om vilandeförklaring av aktivitetsersättning fattas endast om den försäkrade under minst tolv månader omedelbart dessförinnan har fått aktivitetsersättning.
Lag (2016:1291).
\subsection*{15 b §}
\paragraph*{}
Om aktivitetsersättning har förklarats vilande enligt 15 § tredje stycket och den försäkrade fortfarande förvärvsarbetar eller studerar vid utgången av den period som beslutet om aktivitetsersättning omfattar, ska perioden med aktivitetsersättning förlängas med den tid som motsvarar den återstående tiden för vilandeförklaringen. Förlängningen av aktivitetsersättningen får dock endast avse den del av arbetsförmågan som fortfarande används för förvärvsarbete. Vid studier ska alltid aktivitetsersättningen förlängas i sin helhet.
Lag (2016:1291).
\subsection*{16 §}
\paragraph*{}
Ett beslut om vilandeförklaring ska upphävas om den försäkrade begär det.
\subsection*{17 §}
\paragraph*{}
Försäkringskassan får utan att den försäkrade har begärt det upphäva ett beslut om vilandeförklaring om den försäkrade insjuknar och beräknas bli långvarigt sjuk.
\paragraph*{}
Detsamma gäller om den försäkrade helt eller delvis avbryter det arbetsförsök eller de studier som legat till grund för beslutet om vilandeförklaring för att i stället få
\newline - graviditetspenning,
\newline - föräldrapenning, eller
\newline - tillfällig föräldrapenning.
\subsection*{18 §}
\paragraph*{}
Försäkringskassan får besluta att en försäkrad som förvärvsarbetar, när sjukersättning eller aktivitetsersättning helt eller delvis har förklarats vilande, varje månad ska erhålla ett belopp som motsvarar 25 procent av den sjukersättning eller aktivitetsersättning som har förklarats vilande. Beloppet får betalas ut för varje månad under en period om 24 månader.
Lag (2016:1291).
\subsection*{18 a §}
\paragraph*{}
När tiden för ett beslut om vilandeförklaring av aktivitetsersättning har löpt ut eller beslutet har upphävts enligt 16 § lämnas aktivitetsersättning för en period om tre månader i den omfattning som ersättningen tidigare var förklarad vilande. Ersättningen lämnas utan någon ny prövning av i vilken mån arbetsförmågan är nedsatt. Ersättning lämnas inte till en försäkrad till den del denne förvärvsarbetar eller till en försäkrad som studerar.
Lag (2012:933).
\subsection*{18 b §}
\paragraph*{}
Ett beslut om aktivitetsersättning enligt 18 a § ersätter ett beslut om förlängning enligt 15 b §.
Lag (2012:933).
\subsection*{18 c §}
\paragraph*{}
Vid andra fall av vilandeförklaring än sådant där aktivitetsersättningen har förlängts enligt 15 b § gäller vad som sägs i 18 a § endast om det inte återstår någon tid med aktivitetsersättning när tiden för vilandeförklaring upphör eller om den återstående tiden med sådan ersättning är kortare än tre kalendermånader.
\paragraph*{}
Perioden om tre månader med aktivitetsersättning enligt 18 a § ersätter i förekommande fall den kortare tid som återstår av den tidigare beviljade aktivitetsersättningen i den omfattning som denna ersättning har varit förklarad vilande.
Lag (2012:933).
\subsection*{18 d §}
\paragraph*{}
Den sammanlagda tiden med vilande aktivitetsersättning och aktivitetsersättning enligt 18 a § får aldrig överstiga 24 månader.
\paragraph*{}
Aktivitetsersättning enligt 18 a § lämnas längst till och med månaden före den månad då den försäkrade fyller 30 år.
Lag
Lag (2012:933).
\subsection*{19 §}
\paragraph*{}
Om arbetsförmågan förbättras för en försäkrad som får sjukersättning, ska rätten till förmånen omprövas.
\paragraph*{}
En försäkrad som har uppvisat en arbetsförmåga som han eller hon antogs sakna när beslutet om sjukersättning fattades ska, om inte annat framkommer, antas ha en förbättrad arbetsförmåga. Om arbetsförmågan för en försäkrad som har uppnått den ålder som avses i 33 kap. 10 a § fortfarande är nedsatt i förhållande till arbeten som avses i den paragrafen, ska dock rätten till sjukersättning inte omprövas.
Lag (2022:1222).
\subsection*{20 §}
\paragraph*{}
Om rätten till sjukersättning ska omprövas enligt 19 § får förmånen lämnas till dess att den försäkrade har fått ett arbete som motsvarar den förbättring som har uppkommit.
Sjukersättningen får dock i sådana fall lämnas för längst sex månader.
\subsection*{21 §}
\paragraph*{}
Vid omprövning enligt 19 § får sådan sjukersättning som har förklarats vilande enligt 13 § inte ändras med anledning av att den försäkrade under den tid och i den omfattning som anges i beslutet genom förvärvsarbete eller studier har uppvisat en förbättrad arbetsförmåga.
\subsection*{22 §}
\paragraph*{}
Om arbetsförmågan väsentligt förbättras för en försäkrad som får aktivitetsersättning, ska rätten till förmånen omprövas.
\paragraph*{}
En försäkrad som regelbundet och under en längre tid har visat en arbetsförmåga som han eller hon antogs sakna när beslutet om aktivitetsersättning fattades ska antas ha en väsentligt förbättrad arbetsförmåga, om inte något annat framkommer.
\subsection*{23 §}
\paragraph*{}
Vid omprövning enligt 22 § får sådan aktivitetsersättning som har förklarats vilande enligt 13 § inte ändras med anledning av att den försäkrade under den tid och i den omfattning som anges i beslutet genom förvärvsarbete eller studier har visat en väsentligt förbättrad arbetsförmåga.
\subsection*{24 §}
\paragraph*{}
Vid omprövning enligt 22 § får aktivitetsersättningen inte ändras på grund av att den försäkrade deltar i sådan aktivitet som avses i 33 kap. 21-23 §§.
\subsection*{25 §}
\paragraph*{}
Om en försäkrad får sjukpenning eller rehabiliteringspenning enligt denna balk får Försäkringskassan bevilja honom eller henne sjukersättning eller aktivitetsersättning även om han eller hon inte ansökt om det.
\paragraph*{}
Detsamma ska gälla då en försäkrad får sjukpenning, ersättning för sjukhusvård eller livränta enligt 40-44 kap.
eller motsvarande ersättning som lämnas enligt annan författning eller på grund av särskilt beslut av regeringen.
\subsection*{26 §}
\paragraph*{}
Om en försäkrad får aktivitetsersättning, får tiden för förmånen förlängas även om han eller hon inte ansökt om det.
\subsection*{27 §}
\paragraph*{}
Det som föreskrivs i 25 § gäller även i fråga om ökning av sjukersättning och aktivitetsersättning.
\subsection*{28 §}
\paragraph*{}
Ändring av sjukersättning och aktivitetsersättning ska gälla från och med månaden efter den när anledningen till ändringen uppkommit. Om ökningen kräver ansökan av den som är berättigad till sådan ersättning tillämpas 33 kap. 14 § första stycket.
Utbetalning av ersättning 29 § Sjukersättning och aktivitetsersättning ska betalas ut månadsvis. Om det årliga beloppet av sådana förmåner uppgår till högst 2 400 kronor ska dock, om det inte finns särskilda skäl, utbetalning ske i efterskott en eller två gånger per år. Efter överenskommelse med den försäkrade får utbetalning även i annat fall ske en eller två gånger per år. 30 § När månadsbelopp för sjukersättning och aktivitetsersättning beräknas ska det belopp för år räknat som beräkningen utgår från avrundas till närmaste hela krontal som är jämnt delbart med tolv. Är det årliga beloppet av sjukersättning och aktivitetsersättning lägre än tolv kronor, faller ersättningen bort för den månad som beräkningen avser. Avrundning ska i första hand göras på garantiersättning enligt 35 kap.
\chapter*{37 Försäkrade som senast för juli 2008 beviljats icke tidsbegränsad sjukersättning}
\subsection*{1 §}
\paragraph*{}
I detta kapitel finns allmänna bestämmelser i 2-4 §§.
Vidare finns beräkningsregler i 5-9 §§.
\paragraph*{}
Slutligen finns särskilda handläggningsregler i 10-23 §§.
\subsection*{2 §}
\paragraph*{}
För en försäkrad som för juni 2008 hade rätt till sjukersättning enligt 7 kap. 1 § i den upphävda lagen (1962:381) om allmän försäkring och vars arbetsförmåga har ansetts varaktigt nedsatt (icke tidsbegränsad sjukersättning) ska bestämmelserna i detta kapitel gälla i stället för bestämmelserna i 36 kap. 9-21 §§ samt 110 kap. 50-52 §§ om inte annat följer av 4 eller 23 §. Detsamma gäller den som före den 1 juli 2008 beviljats icke tidsbegränsad sjukersättning för tiden från och med juli 2008.
\paragraph*{}
Det som anges i första stycket ska inte gälla för en försäkrad som får sjukersättning med en högre förmånsnivå efter juni eller, för en försäkrad som avses i första stycket andra meningen, juli 2008.
\subsection*{3 §}
\paragraph*{}
Försäkringskassan får efter ansökan av den försäkrade besluta att hans eller hennes sjukersättning, i den omfattning som anges i 5-7 §§, ska betalas ut när den försäkrade förvärvsarbetar med utnyttjande av en arbetsförmåga som han eller hon antogs sakna när beslutet om sjukersättning fattades.
\paragraph*{}
Ansökan ska göras hos Försäkringskassan varje år innan sådant förvärvsarbete som avses i första stycket påbörjas för året.
\subsection*{4 §}
\paragraph*{}
En ansökan som utan giltig anledning görs efter den tid som anges i 3 § andra stycket ska avvisas. I sådant fall ska övriga bestämmelser i detta kapitel inte tillämpas. I stället ska bestämmelserna i 36 kap. 9-21 §§ samt 110 kap. 50-52 §§ tillämpas.
\paragraph*{}
Det som anges i första stycket sista meningen ska gälla även för den som förvärvsarbetar utan att ha gjort en ansökan.
\subsection*{5 §}
\paragraph*{}
Sjukersättning enligt 3 § första stycket ska lämnas med ett preliminärt belopp och bestämmas slutligt i efterhand på grundval av den försäkrades reduceringsinkomst.
\subsection*{6 §}
\paragraph*{}
Sjukersättningen ska minskas med 50 procent av reduceringsinkomsten till den del den överstiger ett visst fribelopp, som för den som får
\newline - hel sjukersättning är 1 prisbasbelopp, - tre fjärdedels sjukersättning är 2,6 prisbasbelopp,
\newline - två tredjedels sjukersättning är 3,6 prisbasbelopp, - halv sjukersättning är 4,2 prisbasbelopp, och - en fjärdedels sjukersättning är 5,8 prisbasbelopp.
\paragraph*{}
Avdraget enligt första stycket ska i första hand göras från inkomstrelaterad sjukersättning.
\subsection*{7 §}
\paragraph*{}
Sjukersättning lämnas inte till den del summan av den beräknade sjukersättningen och reduceringsinkomsten överstiger 8 prisbasbelopp.
\subsection*{8 §}
\paragraph*{}
Reduceringsinkomsten beräknas med utgångspunkt i den pensionsgrundande inkomst som fastställts enligt bestämmelserna om pensionsgrundande inkomst i 59 kap.
\subsection*{9 §}
\paragraph*{}
Den försäkrades reduceringsinkomst beräknas enligt följande:
\newline 1. Först ska från den fastställda pensionsgrundande inkomsten avdrag göras för inkomst enligt
\newline - 59 kap. 13 § 1, i form av föräldrapenning på lägstanivån,
\newline - 59 kap. 13 § 2, i form av omvårdnadsbidrag,
\newline - 59 kap. 13 § 5, i form av inkomstrelaterad sjukersättning och inkomstrelaterad aktivitetsersättning, samt
\newline - 59 kap. 13 § 6, i form av livränta enligt 41-44 kap.
\newline 2. Därefter ska det avdrag som enligt 59 kap. 37 § ska göras för debiterad allmän pensionsavgift läggas till i den utsträckning avgiften hänför sig till de i reduceringsinkomsten inräknade inkomsterna.
\paragraph*{}
Utländska inkomster ska ingå i reduceringsinkomsten, om de är av motsvarande slag som dem som anges i första stycket och skulle ha varit pensionsgrundande enligt 59 kap. om de tjänats in i Sverige.
Lag (2018:1265).
\subsection*{10 §}
\paragraph*{}
För en försäkrad som ansöker om att bestämmelserna om beräkning av sjukersättning i detta kapitel ska tillämpas, ska Försäkringskassan fatta ett beslut om preliminär sjukersättning.
\paragraph*{}
Ett beslut om preliminär sjukersättning får avse längst 12 månader.
\subsection*{11 §}
\paragraph*{}
Preliminär sjukersättning beräknas efter en uppskattad reduceringsinkomst och ska så nära som möjligt motsvara den slutliga sjukersättning som kan antas komma att bestämmas enligt bestämmelserna i detta kapitel.
\paragraph*{}
Preliminär sjukersättning betalas ut med ett månadsbelopp.
När månadsbelopp för preliminär sjukersättning beräknas ska det belopp för år räknat som beräkningen utgår från avrundas till närmaste hela krontal som är jämnt delbart med tolv. Är det årliga beloppet av preliminär sjukersättning lägre än tolv kronor, faller ersättningen bort för den månad som beräkningen avser. Avrundning ska i första hand göras på garantiersättning enligt 35 kap.
Lag (2013:747).
\subsection*{12 §}
\paragraph*{}
Försäkringskassan ska fatta ett beslut om slutlig sjukersättning för varje kalenderår som ett beslut om preliminär sjukersättning har fattats.
\subsection*{13 §}
\paragraph*{}
Slutlig sjukersättning bestäms efter den tidpunkt då pensionsgrundande inkomst har fastställts enligt bestämmelserna i 59 kap. om pensionsgrundande inkomst.
\subsection*{14 §}
\paragraph*{}
Bestäms den slutliga sjukersättningen till högre belopp än den som för samma år har betalats ut i preliminär sjukersättning, ska skillnaden betalas ut. Bestäms den slutliga sjukersättningen till lägre belopp än den som för samma år har betalats ut i preliminär ersättning, ska skillnaden betalas tillbaka enligt 108 kap. 9, 11-14 och 22 §§.
\paragraph*{}
Belopp under 1 200 kr ska varken betalas ut eller betalas tillbaka.
Lag (2014:470).
\subsection*{15 §}
\paragraph*{}
Belopp som ska betalas ut enligt 14 § första stycket ska ökas med ett tillägg. Tillägget på det överskjutande beloppet beräknas med ledning av den statslåneränta som gällde vid utgången av det år som sjukersättningen avser.
\paragraph*{}
På belopp som ska betalas tillbaka enligt 14 § första stycket ska en avgift betalas. Avgiften på återbetalningsbeloppet beräknas med ledning av den statslåneränta som gällde vid utgången av det år som sjukersättningen avser.
\paragraph*{}
Regeringen eller den myndighet som regeringen bestämmer meddelar närmare föreskrifter om beräkningen av tillägget och avgiften.
\subsection*{16 §}
\paragraph*{}
Har upphävts genom
lag (2014:470).
\subsection*{17 §}
\paragraph*{}
Har upphävts genom
lag (2014:470).
\subsection*{18 §}
\paragraph*{}
Har upphävts genom
lag (2014:470).
\subsection*{19 §}
\paragraph*{}
Har upphävts genom
lag (2014:470).
\subsection*{20 §}
\paragraph*{}
Har upphävts genom
lag (2014:470).
\subsection*{21 §}
\paragraph*{}
Den preliminära sjukersättningen ska omprövas om något har inträffat som påverkar storleken av ersättningen.
Försäkringskassan får avstå från att besluta om ändring, om det som har inträffat endast i liten utsträckning påverkar ersättningen.
\subsection*{22 §}
\paragraph*{}
Om den försäkrades pensionsgrundande inkomst ändras efter det att slutlig sjukersättning bestämts och ändringen innebär att sjukersättningen skulle ha varit högre eller lägre, ska en ny slutlig sjukersättning bestämmas, om den försäkrade begär det inom ett år från det att beslutet om ändring av den pensionsgrundande inkomsten meddelades eller om Försäkringskassan självmant tar upp frågan.
\paragraph*{}
En fråga om ny slutlig sjukersättning enligt bestämmelserna i första stycket får inte tas upp efter utgången av femte året efter det år beslutet om pensionsgrundande inkomst meddelades.
\subsection*{23 §}
\paragraph*{}
Efter ansökan av den försäkrade får Försäkringskassan besluta att dra in eller minska sjukersättning som avses i detta kapitel i den omfattning som den försäkrade ansökt om.
Bestämmelserna i 36 kap. 19-21 §§ ska då tillämpas. För sådan sjukersättning som återstår efter minskningen ska bestämmelserna i detta kapitel fortfarande gälla.
\chapter*{38 Innehåll}
\subsection*{1 §}
\paragraph*{}
I denna underavdelning finns allmänna bestämmelser om arbetsskada i 39 kap.
\paragraph*{}
Vidare finns bestämmelser om
\newline - ersättning vid sjukdom i 40 kap, och
\newline - ersättning vid bestående nedsättning av arbetsförmågan i 41 kap.
\paragraph*{}
Dessutom finns särskilda bestämmelser om arbetsskadeersättning och handläggning i 42 kap.
\paragraph*{}
Slutligen finns bestämmelser om
\newline - statligt personskadeskydd i 43 kap., och
\newline - krigsskadeersättning till sjömän i 44 kap.
\chapter*{39 Allmänna bestämmelser om arbetsskada Innehåll}
\subsection*{1 §}
\paragraph*{}
I detta kapitel finns inledande bestämmelser i 2 §.
\paragraph*{}
Vidare finns bestämmelser om
\newline - arbetsskadebegreppet i 3-7 §§, och
\newline - skadetidpunkt i 8 §.
\subsection*{2 §}
\paragraph*{}
Från arbetsskadeförsäkringen kan ersättning lämnas till försäkrade förvärvsarbetande och vissa studerande.
\paragraph*{}
I 86-88 kap. finns bestämmelser om att ersättning från arbetsskadeförsäkringen kan lämnas även till efterlevande.
\subsection*{3 §}
\paragraph*{}
Med arbetsskada avses en skada till följd av olycksfall eller annan skadlig inverkan i arbetet. En skada ska anses ha uppkommit av sådan orsak, om övervägande skäl talar för det.
\subsection*{4 §}
\paragraph*{}
Med skada avses en personskada eller en skada på en protes eller annan liknande anordning som användes för avsett ändamål när skadan inträffade.
\subsection*{5 §}
\paragraph*{}
Som arbetsskada anses inte en skada av psykisk eller psykosomatisk natur som är en följd av en företagsnedläggelse, bristande uppskattning av den försäkrades arbetsinsatser, vantrivsel med arbetsuppgifter eller arbetskamrater eller därmed jämförliga förhållanden.
\subsection*{6 §}
\paragraph*{}
Regeringen eller den myndighet som regeringen bestämmer meddelar föreskrifter om i vilken utsträckning en skada, som inte beror på ett olycksfall men som har framkallats av smitta, ska anses som arbetsskada.
\subsection*{7 §}
\paragraph*{}
Olycksfall vid färd till eller från arbetsstället räknas som olycksfall i arbetet, om färden föranleddes av och stod i nära samband med arbetet.
\subsection*{8 §}
\paragraph*{}
En skada som beror på ett olycksfall anses ha inträffat dagen för olycksfallet.
\paragraph*{}
En skada som beror på annan skadlig inverkan än ett olycksfall anses ha inträffat den dag när den först visade sig.
\chapter*{40 Ersättning vid sjukdom}
\subsection*{1 §}
\paragraph*{}
I detta kapitel finns allmänna bestämmelser i 2 och 3 §§.
\paragraph*{}
Vidare finns bestämmelser om
\newline - arbetsskadesjukpenning i 4-9 §§,
\newline - återinsjuknande i 10 §, och
\newline - sjukvårdsersättning i 11 och 12 §§.
\subsection*{2 §}
\paragraph*{}
En försäkrad har vid arbetsskada rätt till samma förmåner enligt denna balk som han eller hon har rätt till vid annan sjukdom.
\subsection*{3 §}
\paragraph*{}
Om någon som avses i 2 § inte är försäkrad enligt 5 kap. 9 § 7 eller 6 kap. 6 § 3 eller 4, har han eller hon vid arbetsskada rätt till motsvarande förmåner från arbetsskadeförsäkringen. Motsvarande gäller den som på grund av att han eller hon inte är bosatt i Sverige inte har rätt till vårdförmåner eller ersättning för kostnader i samband med vård.
Lag (2011:1513).
\subsection*{4 §}
\paragraph*{}
En försäkrad som beviljas ersättning för inkomstförlust till följd av arbetsskada har rätt till arbetsskadesjukpenning för inkomstförlust som avser två sjukdagar.
\paragraph*{}
För en sjukdag motsvarar sjukpenningen kvoten mellan
\newline - 80 procent av det livränteunderlag enligt 41 kap. som gäller vid tiden för beslutet och
\newline - 365.
\subsection*{5 §}
\paragraph*{}
Om den försäkrade kan visa att han eller hon har haft fler än två sjukdagar med inkomstförlust under sjukdomsperioder till följd av arbetsskada har han eller hon rätt till arbetsskadesjukpenning för dessa ytterligare dagar.
\paragraph*{}
För en sjukdag motsvarar sjukpenningen 80 procent av den faktiska inkomstförlusten, dock högst vad den försäkrade fått för en sjukdag enligt 4 §.
\subsection*{6 §}
\paragraph*{}
Arbetsskadesjukpenningen avrundas för dag till närmaste hela krontal, varvid 50 öre avrundas uppåt.
\subsection*{7 §}
\paragraph*{}
Till en försäkrad som genomgår utbildning som avses i 6 kap. 22 § eller som i annat fall genomgår yrkesutbildning när skadan inträffar kan arbetsskadesjukpenning lämnas. Detta gäller vid sjukdom som fortfarande 180 dagar efter det att skadan inträffade sätter ned den försäkrades förmåga att skaffa sig arbetsinkomst med minst en fjärdedel.
\paragraph*{}
Sjukpenningen lämnas på grundval av den försäkrades livränteunderlag som anges i 41 kap. 14-16 och 18 §§ samt beräknas enligt 28 eller 31 kap.
\paragraph*{}
Sjukpenningen lämnas endast i den utsträckning som den överstiger den sjukpenning som den försäkrade är berättigad till enligt 2 eller 3 §.
\subsection*{8 §}
\paragraph*{}
Den försäkrade har rätt till arbetsskadesjukpenning under tid när han eller hon avhåller sig från arbete på uppmaning av Försäkringskassan eller med dess samtycke i syfte att förebygga att en arbetsskada uppstår, återuppstår eller förvärras.
\paragraph*{}
Sjukpenningen lämnas med skäligt belopp och får motsvara högst hel sjukpenning enligt 2 eller 3 § samt 7 §.
\subsection*{9 §}
\paragraph*{}
Om någon som avses i 2 § inte är försäkrad för sjukpenning enligt 6 kap. 6 § 3, ska arbetsskadesjukpenning lämnas med det belopp som sjukpenningen skulle ha uppgått till om den sjukpenninggrundande inkomsten hade beräknats med bortseende från bestämmelserna i 25 kap. 3 § andra stycket 2 och 3.
\subsection*{10 §}
\paragraph*{}
Om en arbetsskada som har gett den försäkrade rätt till livränta enligt 41 kap. på nytt medför sjukdom, har den försäkrade rätt till sjukpenning enligt 2 och 7 §§, om sjukdomen sätter ned hans eller hennes kvarstående förmåga att skaffa sig inkomst genom arbete.
\subsection*{11 §}
\paragraph*{}
I den utsträckning som ersättning inte lämnas enligt 2 och 3 §§, ersätter arbetsskadeförsäkringen nödvändiga kostnader för
\newline 1. sjukvård utomlands,
\newline 2. tandvård, och
\newline 3. särskilda hjälpmedel.
\paragraph*{}
Som kostnader räknas även nödvändiga utgifter för resor.
\subsection*{12 §}
\paragraph*{}
Ersättning enligt 11 § första stycket 2 lämnas endast för tandvård som ges av en vårdgivare vars vård kan berättiga till ersättning enligt lagen (2008:145) om statligt tandvårdsstöd.
\chapter*{41 Ersättning vid bestående nedsättning av arbetsförmågan}
\subsection*{1 §}
\paragraph*{}
I detta kapitel finns bestämmelser om
\newline - rätten till livränta i 2-4 §§,
\newline - förmånstiden i 5-7 §§,
\newline - beräkning av livränta i 8-10 §§,
\newline - livränteunderlaget i 11-18 §§,
\newline - vilande livränta i 19 och 20 §§,
\newline - indexering av livränta i 21 §,
\newline - omprövning vid ändrade förhållanden i 22 och 23 §§, och
\newline - omräkning av livränteunderlaget i 24 §.
\subsection*{2 §}
\paragraph*{}
Livränta lämnas till en försäkrad som till följd av arbetsskada har fått sin förmåga att skaffa sig inkomst genom arbete nedsatt med minst en femtondel.
\paragraph*{}
Detta gäller endast om
\newline 1. nedsättningen av förmågan att skaffa inkomst genom arbete kan antas bestå under minst ett år, och
\newline 2. inkomstförlusten för år räknat uppgår till minst en fjärdedel av prisbasbeloppet för det år när livräntan ska börja lämnas.
\subsection*{3 §}
\paragraph*{}
Om en försäkrad, som genom arbetsskada har fått sin arbetsförmåga nedsatt med mindre än en femtondel, senare drabbas av ytterligare arbetsskada, bestäms rätten till livränta på grundval av båda skadorna.
\subsection*{4 §}
\paragraph*{}
Under tid när den försäkrade genomgår behandling eller rehabilitering som avses i 27 kap. 6 § eller 31 kap. 3 § ska hans eller hennes förmåga att skaffa sig inkomst genom arbete anses nedsatt även i den utsträckning som åtgärden hindrar honom eller henne från att förvärvsarbeta.
\subsection*{5 §}
\paragraph*{}
/Upphör att gälla U:2025-12-01/
Livränta lämnas längst till och med månaden före den då den försäkrade fyller 66 år, om inte något annat anges i 6 §.
Lag (2022:878).
\subsection*{5 §}
\paragraph*{}
/Träder i kraft I:2025-12-01/
Livränta lämnas längst till och med månaden före den då den försäkrade uppnår riktåldern för pension, om inte något annat anges i 6 §.
Lag (2022:879).
\subsection*{6 §}
\paragraph*{}
/Upphör att gälla U:2025-12-01/
Livränta lämnas längst till och med månaden före den då den försäkrade fyller 69 år, om
\newline - skadan inträffar den månad då den försäkrade fyller 66 år eller senare, och
\newline - en sjukpenninggrundande inkomst kan fastställas för den försäkrade enligt 25 kap. 3 §.
Lag (2022:878).
\subsection*{6 §}
\paragraph*{}
/Träder i kraft I:2025-12-01/
Livränta lämnas längst till och med månaden före den då den försäkrade fyller 69 år, om
\newline - skadan inträffar den månad då den försäkrade uppnår riktåldern för pension eller senare, och
\newline - en sjukpenninggrundande inkomst kan fastställas för den försäkrade enligt 25 kap. 3 §.
Lag (2022:879).
\subsection*{7 §}
\paragraph*{}
Livränta lämnas för viss tid eller tills vidare.
\paragraph*{}
Försäkringskassan ska i samband med beslut om livränta också bedöma om förnyad utredning av förmågan att skaffa sig inkomst genom arbete ska göras efter viss tid.
\subsection*{8 §}
\paragraph*{}
Livränta lämnas med så stor andel av den försäkrades livränteunderlag enligt 11-18 §§ som motsvarar graden av nedsättning av hans eller hennes förmåga att skaffa sig inkomst genom arbete.
\subsection*{9 §}
\paragraph*{}
När förvärvsförmågan enligt 8 § bedöms ska det beaktas vad som rimligen kan begäras med hänsyn till den försäkrades arbetsskada, utbildning och tidigare verksamhet samt ålder, bosättningsförhållanden och andra sådana omständigheter.
\subsection*{10 §}
\paragraph*{}
Förvärvsförmågan ska bestämmas utan hänsyn till det allmänna läget på arbetsmarknaden.
\paragraph*{}
I fråga om en äldre försäkrad ska hänsyn främst tas till hans eller hennes förmåga och möjlighet att skaffa sig fortsatt inkomst genom sådant arbete som han eller hon har utfört tidigare, eller genom annat lämpligt arbete som är tillgängligt för honom eller henne.
\subsection*{11 §}
\paragraph*{}
Livränteunderlaget motsvarar, om inte annat följer av 12-18 §§, den försäkrades sjukpenninggrundande inkomst enligt 25 och 26 kap. vid den tidpunkt från vilken livräntan första gången ska lämnas.
\paragraph*{}
Om den skadade inte är försäkrad för sjukpenning enligt 6 kap. 6 § 3 ska beräkningen enligt första stycket göras med bortseende från bestämmelserna i 25 kap. 3 § andra stycket 2 och 3.
\subsection*{12 §}
\paragraph*{}
När livränteunderlaget beräknas gäller följande avvikelser från bestämmelserna i 25 och 26 kap.:
\newline 1. Andra skattepliktiga förmåner än pengar ska beaktas och ska värderas på det sätt som är föreskrivet för beräkning av pensionsgrundande inkomst enligt 59 kap. 35 §.
\newline 2. Med inkomst av anställning likställs kostnadsersättning som inte enligt 10 kap. 3 § 9 eller 10 skatteförfarandelagen (2011:1244) undantas vid beräkning av skatteavdrag.
\newline 3. Semesterlön och semesterersättning ska räknas med utan den begränsning som anges i 25 kap. 17 §.
\newline 4. Inkomst av sådant förvärvsarbete som avses i 37 kap. 3 § ska medräknas.
\paragraph*{}
Vid beräkning av livränteunderlaget får även beaktas ett löneavtal som har träffats efter den tidpunkt från vilken livränta första gången ska lämnas om avtalet omfattar denna tidpunkt.
Lag (2011:1434).
\subsection*{13 §}
\paragraph*{}
Ska livränta börja lämnas avsevärd tid efter det att den försäkrade utsattes för skadlig inverkan i arbetet, får någon annan högre inkomst av förvärvsarbete än den som anges i 11 och 12 §§ användas som livränteunderlag, om det finns särskilda skäl.
\subsection*{14 §}
\paragraph*{}
Hade den försäkrade inte fyllt 25 år när skadan inträffade, motsvarar livränteunderlaget, för tiden efter 25 års ålder, den inkomst som han eller hon med hänsyn till sin sysselsättning när skadan inträffade sannolikt skulle ha haft vid 25 års ålder, om skadan inte hade inträffat.
\paragraph*{}
Motsvarande gäller för en försäkrad som
\newline 1. när skadan inträffade inte hade fyllt 21 år, i fråga om livränteunderlaget för tiden mellan 21 och 25 års ålder, eller
\newline 2. när skadan inträffade inte hade fyllt 18 år, i fråga om livränteunderlaget för tiden mellan 18 och 21 års ålder.
\subsection*{15 §}
\paragraph*{}
För en försäkrad som genomgår utbildning som avses i 6 kap. 22 § eller som i annat fall genomgick yrkesutbildning när skadan inträffade motsvarar livränteunderlaget för den beräknade utbildningstiden lägst den inkomst som han eller hon sannolikt skulle ha haft, om han eller hon när skadan inträffade hade avbrutit utbildningen och börjat förvärvsarbeta.
\paragraph*{}
Livränteunderlaget för tid efter utbildningstidens slut motsvarar lägst den inkomst av förvärvsarbete som den försäkrade då sannolikt skulle ha haft, om skadan inte hade inträffat.
\subsection*{16 §}
\paragraph*{}
Ett livränteunderlag enligt 15 § beräknas till minst
\newline - 2 prisbasbelopp för tid före 21 års ålder,
\newline - 2,5 prisbasbelopp för tid mellan 21 och 25 års ålder, och
\newline - 3 prisbasbelopp för tid från och med 25 års ålder.
\subsection*{17 §}
\paragraph*{}
Om det redan vid den tidpunkt från vilken livränta första gången ska lämnas är sannolikt att livränteunderlaget skulle bli väsentligen högre eller lägre vid en senare tidpunkt får ett nytt livränteunderlag fastställas och läggas till grund för beräkning av livränta vid ett sådant senare tillfälle.
\subsection*{18 §}
\paragraph*{}
När livränteunderlaget beräknas ska det bortses från belopp som överstiger 7,5 gånger det prisbasbelopp som fastställts för året när livräntan ska börja lämnas.
\subsection*{19 §}
\paragraph*{}
Ett beslut om vilandeförklaring av sjukersättning eller aktivitetsersättning enligt 36 kap. ska också omfatta sådan livränta som enligt 42 kap. 2 och 4 §§ samordnats med ersättningen.
\subsection*{20 §}
\paragraph*{}
För den tid när en vilandeförklaring gäller enligt 19 § ska livränta beräknas utifrån
\newline - dels det livränteunderlag som legat till grund för bestämmande av livräntan före vilandeförklaringen,
\newline - dels de inkomstförhållanden som råder under tiden med vilandeförklaring.
\subsection*{21 §}
\paragraph*{}
En fastställd livränta ska för varje år räknas om med ledning av ett särskilt tal. Detta tal ska visa den årliga procentuella förändringen av halva den reala inkomstförändringen med tillägg för den procentuella förändringen i det allmänna prisläget i juni två år före det år livräntan omräknas och det allmänna prisläget i juni året närmast före det året. Den reala inkomstförändringen ska beräknas på det sätt som anges i 58 kap. 12 § med den ändringen att när inkomsterna beräknas ska den årliga förändringen i det allmänna prisläget under samma period, räknat från juni till juni, räknas bort.
\paragraph*{}
Livräntan får räknas om till högst ett belopp som utgör samma andel av 7,5 gånger det för året fastställda prisbasbeloppet som den andel av livränteunderlaget som svarar mot nedsättningen av förvärvsförmågan enligt 8 §.
Lag (2015:676).
\subsection*{22 §}
\paragraph*{}
Rätten till livränta ska omprövas om det har skett någon ändring av betydelse i de förhållanden som var avgörande för beslutet, eller om den försäkrades möjlighet att skaffa sig inkomst genom arbete väsentligen har förbättrats. Livränta för förfluten tid får dock inte ändras till den försäkrades nackdel. I 108 kap. finns bestämmelser om återkrav i vissa fall.
\subsection*{23 §}
\paragraph*{}
Utan hinder av bestämmelserna i 22 § ska följande gälla.
\paragraph*{}
Ett beslut om sjukersättning enligt 37 kap. ska också omfatta sådan livränta som enligt 42 kap. 2 § är samordnad med sjukersättningen. Det som där föreskrivs om sjukersättning ska då även avse livränta.
\subsection*{24 §}
\paragraph*{}
Om en livränta ska beräknas på nytt efter det år den första gången ska lämnas eller om den ska omprövas enligt 22 § ska det livränteunderlag som anges i 11-13 och 17 §§ räknas om med ledning av det särskilda tal som anges i 21 § för varje år efter det år livräntan första gången ska lämnas.
\paragraph*{}
Vid omräkning enligt första stycket ska även bestämmelsen i 18 § om högsta livränteunderlag beaktas.
\chapter*{42 Särskilda bestämmelser om arbetsskadeersättning och handläggning}
\subsection*{1 §}
\paragraph*{}
I detta kapitel finns bestämmelser om
\newline - samordning med andra socialförsäkringsförmåner i 2-4 §§,
\newline - arbetsgivares rätt till ersättning i 5 §,
\newline - förlust av ersättning i 6 §,
\newline - underrättelse till arbetsgivare om skada i 7-9 §§,
\newline - skadeanmälan till Försäkringskassan i 10 och 11 §§, och
\newline - prövning av frågan om arbetsskada i 12 §.
\subsection*{2 §}
\paragraph*{}
Livränta ska minskas om den försäkrade, med anledning av den inkomstförlust som föranlett livräntan, samtidigt har rätt till sjukersättning eller aktivitetsersättning.
\paragraph*{}
Minskningen görs genom att livräntan betalas ut endast i den utsträckning som den överstiger dessa förmåner.
\subsection*{3 §}
\paragraph*{}
Livräntan ska även minskas på det sätt som anges i 2 § andra stycket i fråga om pension som enligt ett utländskt system för social trygghet lämnas med anledning av arbetsskadan.
\subsection*{4 §}
\paragraph*{}
Om den försäkrade får livränta och vid en senare tidpunkt blir berättigad till sjukersättning eller aktivitetsersättning för en sjukdom som inte är arbetsrelaterad gäller följande.
\paragraph*{}
Livräntan ska minskas med den andel av ersättningen som motsvarar graden av nedsättningen av förvärvsförmågan till följd av skadan jämfört med hel förvärvsförmåga när livräntan bestämdes.
\subsection*{4 a §}
\paragraph*{}
Livränta vid arbetsskada eller annan skada som avses i 41-44 kap. ska minskas med det belopp den försäkrade för samma tid får som omställningsstudiestöd enligt lagen (2022:856) om omställningsstudiestöd, dock endast till den del omställningsstudiestödet avser samma inkomstbortfall som livräntan är avsedd att täcka.
Lag (2022:858).
\subsection*{5 §}
\paragraph*{}
En arbetsgivare som enligt författning är skyldig att lämna ersättning vid arbetsskada har rätt att få sådan ersättning från arbetsskadeförsäkringen som den försäkrade annars skulle ha haft rätt till för motsvarande ändamål.
Ersättningen får dock inte överstiga vad arbetsgivaren har betalat ut.
\subsection*{6 §}
\paragraph*{}
Rätten till ersättning går förlorad om en ansökan om arbetsskadeersättning inte görs inom sex år
\newline - för sjukpenning och livränta, från den dag ersättningen avser, och
\newline - för annan ersättning, från den dag då den försäkrade betalade det belopp för vilket ersättning begärs.
\subsection*{7 §}
\paragraph*{}
Om en arbetstagare drabbas av en arbetsskada ska den arbetsgivare som han eller hon var anställd hos när skadan inträffade underrättas omedelbart.
\paragraph*{}
Om en skada till följd av annat än olycksfall har visat sig först när den försäkrade har upphört att vara utsatt för den inverkan som har orsakat skadan, är det den arbetsgivare hos vilken den försäkrade senast var utsatt för sådan inverkan som ska underrättas.
\subsection*{8 §}
\paragraph*{}
Om en arbetsgivare enligt 7 § ska underrättas om en skada får underrättelsen i stället lämnas till en person som på arbetsgivarens vägnar förestår arbetet.
\subsection*{9 §}
\paragraph*{}
När det gäller en försäkrad som genomgår utbildning som avses i 6 kap. 22 § likställs med en arbetsgivare en rektor eller någon annan som förestår utbildningen.
\subsection*{10 §}
\paragraph*{}
En arbetsgivare eller arbetsföreståndare som har fått kännedom om en inträffad arbetsskada är skyldig att omedelbart anmäla skadan till Försäkringskassan.
\paragraph*{}
Om den försäkrade inte är arbetstagare ska han eller hon själv anmäla arbetsskadan till Försäkringskassan. Om den försäkrade har avlidit till följd av skadan, ska anmälan göras av den som har rätt att företräda dödsboet.
\subsection*{11 §}
\paragraph*{}
Regeringen eller den myndighet som regeringen bestämmer meddelar föreskrifter om ersättning för nödvändiga utgifter för läkarutlåtanden.
\subsection*{12 §}
\paragraph*{}
Frågan om den försäkrade har fått en arbetsskada ska prövas endast i den utsträckning det behövs för att bestämma arbetsskadeersättning enligt denna balk eller sjukpenning enligt 28 kap.
\chapter*{43 Statligt personskadeskydd}
\subsection*{1 §}
\paragraph*{}
I detta kapitel finns allmänna bestämmelser i 2 §.
\paragraph*{}
Vidare finns bestämmelser om
\newline - skadebegreppet i 3-6 §§,
\newline - förmånerna i 7-10 §§,
\newline - totalförsvarspliktiga m.fl. i 11-18 §,
\newline - intagna m.fl. i 19 och 20 §§,
\newline - frivilliga inom totalförsvaret m.fl. i 21 §,
\newline - underrättelse om skada i 22 §, och
\newline - skadeanmälan till Försäkringskassan, m.m. i 23 §.
\subsection*{2 §}
\paragraph*{}
Det statliga personskadeskyddet gäller för skador som uppkommer under sådan skyddstid som anges i 7 kap. 4-6 §§.
\paragraph*{}
I 86-88 kap. finns bestämmelser om att ersättning från personskadeskyddet kan lämnas även till efterlevande.
\subsection*{3 §}
\paragraph*{}
Bestämmelserna om personskada, protesskada och skadetidpunkt i 39 kap. 4 och 8 §§ tillämpas i fråga om skada enligt detta kapitel.
\subsection*{4 §}
\paragraph*{}
Om den försäkrade under skyddstiden har varit utsatt för ett olycksfall ska en skada som han eller hon har fått anses vara orsakad av olycksfallet, om övervägande skäl talar för det.
\subsection*{5 §}
\paragraph*{}
Om en skada, som inte beror på ett olycksfall, visar sig under skyddstiden ska skadan anses ha uppkommit under denna tid.
\paragraph*{}
Detta gäller dock inte om det finns skälig anledning att anta att skadan har orsakats av annat än verksamheten eller intagningen i fråga och att verksamheten eller intagningen inte väsentligt har bidragit till skadan.
\subsection*{6 §}
\paragraph*{}
Visar sig en skada, som inte beror på ett olycksfall, efter skyddstiden, ska den anses ha uppkommit under denna tid om det skäligen kan antas att verksamheten eller intagningen väsentligt har bidragit till skadan.
\subsection*{7 §}
\paragraph*{}
Om den försäkrade även enligt 40-42 kap. har rätt till arbetsskadeersättning för en skada, lämnas ersättning enligt detta kapitel endast i den utsträckning som ersättningen därigenom blir högre.
\subsection*{8 §}
\paragraph*{}
Ersättning lämnas vid sjukdom och bestående nedsättning av arbetsförmågan. Därvid tillämpas 40 och 41 kap. samt 42 kap. 2-6 §§, om inte något annat följer av 9-21 §§.
\subsection*{9 §}
\paragraph*{}
Sjukvårdsersättning lämnas endast i den utsträckning som staten inte tillhandahåller motsvarande förmån på annat sätt.
\subsection*{10 §}
\paragraph*{}
Sjukpenning och livränta enligt detta kapitel lämnas inte under skyddstiden. Sjukpenning lämnas aldrig för den dag då skadan inträffade.
\subsection*{11 §}
\paragraph*{}
Bestämmelserna i 12-18 §§ gäller den som enligt 7 kap. 2 § 1 är försäkrad på grund av tjänstgöring inom totalförsvaret eller liknande verksamhet och tillämpas utöver vad som följer av 8 §.
\subsection*{12 §}
\paragraph*{}
För en försäkrad som har skadats under militär utbildning inom Försvarsmakten utgör sjukpenningunderlaget vid tillämpning av 8 § minst
\newline - 4 prisbasbelopp för tid före 21 års ålder,
\newline - 4,5 prisbasbelopp för tid mellan 21 och 25 års ålder, och
\newline - 5 prisbasbelopp för tid från och med 25 års ålder.
Lag (2010:467).
\subsection*{13 §}
\paragraph*{}
För en försäkrad som har skadats under militär utbildning inom Försvarsmakten utgör livränteunderlaget vid tillämpning av 8 § minst 7 prisbasbelopp.
Lag (2010:467).
\subsection*{14 §}
\paragraph*{}
En försäkrad har rätt till särskild sjukpenning, om han eller hon under militär utbildning inom Försvarsmakten som är längre än 60 dagar har drabbats av en sjukdom som efter skyddstidens slut sätter ned hans eller hennes förmåga att skaffa sig inkomst genom arbete med minst en fjärdedel. Detta gäller även vid sjukdom som har uppkommit under utryckningsmånaden eller månaden efter denna.
Lag (2010:467).
\subsection*{15 §}
\paragraph*{}
Beräkningsunderlaget för särskild sjukpenning är
\newline - 4 prisbasbelopp för tid före 21 års ålder,
\newline - 4,5 prisbasbelopp för tid mellan 21 och 25 års ålder, och
\newline - 5 prisbasbelopp för tid från och med 25 års ålder.
\subsection*{16 §}
\paragraph*{}
För en dag är hel särskild sjukpenning kvoten mellan
\newline - 80 procent av beräkningsunderlaget och
\newline - 365.
\subsection*{17 §}
\paragraph*{}
Särskild sjukpenning lämnas endast i den utsträckning som den överstiger annan ersättning enligt detta kapitel eller sjukpenning enligt 27, 28 och 40 kap. som den försäkrade har rätt till för samma tid.
\subsection*{18 §}
\paragraph*{}
En försäkrad som har skadats under militär utbildning inom Försvarsmakten har under högst tre års tid efter skyddstiden rätt till ersättning för nödvändiga kostnader för
\newline 1. läkarvård,
\newline 2. sjukvårdande behandling,
\newline 3. sjukhusvård, och
\newline 4. läkemedel.
\paragraph*{}
Som kostnader för vård räknas även nödvändiga utgifter för resor.
Lag (2010:467).
\subsection*{19 §}
\paragraph*{}
För den som är försäkrad på grund av intagning m.m.
enligt 7 kap. 2 § 3 och som när skadan inträffade hade varit intagen längre tid än sex månader, ska sjukpenningunderlag och livränteunderlag vid tillämpning av 8 § beräknas till lägst de belopp som anges i 41 kap. 16 §.
\subsection*{20 §}
\paragraph*{}
Ersättning till den som avses i 7 kap. 2 § 3 får dras in eller sättas ned även i annat fall än som avses i 110 kap. 52-58 §§, om särskilda skäl talar för det.
\subsection*{21 §}
\paragraph*{}
Sjukpenningunderlag och livränteunderlag ska, i den utsträckning som framgår av föreskrifter som meddelas av regeringen eller den myndighet som regeringen bestämmer, beräknas till lägst de belopp som anges i 41 kap. 16 § för den som är försäkrad på grund av frivilligt deltagande i verksamhet inom totalförsvaret eller i räddningsarbete enligt 7 kap. 3 §.
\subsection*{22 §}
\paragraph*{}
Om den försäkrade drabbas av en skada ska en chef eller någon annan omedelbart underrättas. Regeringen eller den myndighet som regeringen bestämmer meddelar föreskrifter om till vem sådan underrättelse ska ske.
\paragraph*{}
Om skadan har visat sig efter skyddstiden, ska i stället Försäkringskassan underrättas.
\subsection*{23 §}
\paragraph*{}
Bestämmelserna i 42 kap. 10 och 11 §§ om anmälan och ersättning för utgifter för läkarintyg gäller även i ärenden enligt detta kapitel.
\chapter*{44 Krigsskadeersättning till sjömän}
\subsection*{1 §}
\paragraph*{}
I detta kapitel finns allmänna bestämmelser i 2 §.
\paragraph*{}
Vidare finns bestämmelser om
\newline - förmånerna i 3-5 §§, och
\newline - övriga frågor i 6 §.
\subsection*{2 §}
\paragraph*{}
Krigsskadeersättning kan lämnas till en försäkrad sjöman som till följd av olycksfall på grund av en krigshändelse drabbas av en personskada utomlands under sådan tid som anges i 7 kap. 8 och 9 §§.
\paragraph*{}
I 86-88 kap. finns bestämmelser om att krigsskadeersättning kan lämnas även till efterlevande.
\subsection*{3 §}
\paragraph*{}
Krigsskadeersättning lämnas med tillämpning av bestämmelserna i 40 och 41 kap. samt 42 kap. 2-6 §§.
\subsection*{4 §}
\paragraph*{}
Den försäkrade har inte rätt till krigsskadeersättning om han eller hon för samma tid har rätt till arbetsskadeersättning enligt 39-42 kap.
\subsection*{5 §}
\paragraph*{}
Vid skada har den försäkrade, utöver livränta, rätt till särskild ersättning med hänsyn till skadans beskaffenhet.
\paragraph*{}
Vid förlust av arbetsförmågan motsvarar ersättningen 6 gånger det prisbasbelopp som gällde för det år då skadan inträffade.
\paragraph*{}
Vid nedsättning av arbetsförmågan betalas så stor andel av summan ut som svarar mot graden av nedsättningen.
\subsection*{6 §}
\paragraph*{}
Bestämmelserna i 39 kap. och 42 kap. 7-12 §§ tillämpas i övrigt på motsvarande sätt i fråga om krigsskadeersättning.
VI Särskilda förmåner vid smitta, sjukdom eller skada
\chapter*{45 Innehåll}
\subsection*{1 §}
\paragraph*{}
I denna underavdelning finns bestämmelser om
\newline - smittbärarersättning i 46 kap., och - närståendepenning i 47 kap.
\chapter*{46 Smittbärarersättning}
\subsection*{1 §}
\paragraph*{}
I detta kapitel finns allmänna bestämmelser om smittbärarersättning i 2-4 §§.
\paragraph*{}
Vidare finns bestämmelser om - rätten till smittbärarpenning i 5-10 §§,
\newline - förmånsnivåer och förvärvsarbete i 11-13 §§, - beräkning av smittbärarpenning i 14-17 §§,
\newline - samordning med andra förmåner i 18 och 19 §§, och
\newline - resekostnadsersättning i 20 och 21 §§.
\subsection*{2 §}
\paragraph*{}
Smittbärarersättning kan lämnas till en smittbärare i samband med åtgärder för smittskydd eller skydd för livsmedel.
\subsection*{3 §}
\paragraph*{}
Med smittbärare avses i detta kapitel
\newline 1. den som har eller kan antas ha en smittsam sjukdom utan att ha förlorat sin arbetsförmåga till följd av sjukdomen,
\newline 2. den som för eller kan antas föra smitta utan att vara sjuk i en smittsam sjukdom, och
\newline 3. den som i annat fall har eller kan antas ha varit utsatt för smitta av en samhällsfarlig sjukdom som avses i smittskyddslagen (2004:168) utan att vara sjuk i en sådan sjukdom.
\subsection*{4 §}
\paragraph*{}
Smittbärarersättning lämnas i form av smittbärarpenning och resekostnadsersättning.
\subsection*{5 §}
\paragraph*{}
En smittbärare har rätt till smittbärarpenning om han eller hon måste avstå från förvärvsarbete på grund av
\newline 1. beslut enligt smittskyddslagen (2004:168) eller livsmedelslagen (2006:804) eller föreskrifter som har meddelats med stöd av sistnämnda lag, eller
\newline 2. läkarundersökning eller hälsokontroll som smittbäraren genomgår utan föregående beslut enligt 1 och som syftar till att klarlägga om han eller hon är smittad av en allmänfarlig sjukdom eller har en sjukdom, en smitta, ett sår eller en annan skada, som kan göra livsmedel som han eller hon hanterar otjänligt som människoföda.
\newline 1. hälsokontroll vid inresa enligt 3 kap. 8 § smittskyddslagen, eller
\newline 2. beslut som avser avspärrning enligt 3 kap. 10 § smittskyddslagen.
\paragraph*{}
Första stycket gäller inte när det är fråga om
\subsection*{6 §}
\paragraph*{}
Smittbärarpenning får dras in eller sättas ned om smittbäraren
\newline 1. inte följer förhållningsregler som har beslutats med stöd av smittskyddslagen (2004:168) eller överträder förbud enligt 3 kap. 9 § den lagen, eller
\newline 2. inte följer särskilda villkor i samband med ett beslut enligt livsmedelslagen (2006:804) eller föreskrifter som har meddelats med stöd av den lagen.
\subsection*{7 §}
\paragraph*{}
Smittbärarpenning lämnas från och med första dagen i ersättningsperioden och därefter så länge den försäkrade uppfyller förutsättningarna för rätt till smittbärarpenning.
\subsection*{8 §}
\paragraph*{}
Bestämmelserna om sjukpenning i samband med pension, sjukersättning eller aktivitetsersättning i 27 kap. 34 § och sjukpenning i samband med 71-årsdagen i 27 kap. 37 § tillämpas även i fråga om smittbärarpenning.
Lag (2022:878).
\subsection*{9 §}
\paragraph*{}
För en försäkrad som får smittbärarpenning ska bestämmelserna i 30 kap. 8-14 §§ om Försäkringskassans åtgärder för rehabilitering tillämpas.
\subsection*{10 §}
\paragraph*{}
Bestämmelserna om arbetsgivarinträde, m.m. i 27 kap. 56-60 §§ tilllämpas även i fråga om smittbärarpenning.
Förmånsnivåer och förvärvsarbete
\subsection*{11 §}
\paragraph*{}
Smittbärarpenning lämnas enligt följande förmånsnivåer:
\newline 1. Hel smittbärarpenning lämnas när den försäkrade helt måste avstå från förvärvsarbete. 2. Tre fjärdedels smittbärarpenning lämnas när den försäkrade arbetar högst en fjärdedel av den tid han eller hon annars skulle ha arbetat. 3. Halv smittbärarpenning lämnas när den försäkrade arbetar högst hälften av den tid han eller hon annars skulle ha arbetat.
\newline 4. En fjärdedels smittbärarpenning lämnas när den försäkrade arbetar högst tre fjärdedelar av den tid han eller hon annars skulle ha arbetat.
\subsection*{12 §}
\paragraph*{}
Vid tillämpning av 11 § ska som förvärvsarbete inte betraktas sådant förvärvsarbete som utförs under tid då den försäkrade förvärvsarbetar med stöd av 37 kap. 3 §.
\paragraph*{}
Om det inte går att avgöra under vilken tid den försäkrade avstår från förvärvsarbete ska frånvaron i första hand anses som frånvaro från sådant förvärvsarbete som avses i 37 kap. 3 §.
\subsection*{13 §}
\paragraph*{}
Med tid för förvärvsarbete likställs följande:
\newline 1. Ledighet för semester, dock inte om den försäkrade under ledigheten får semesterlön enligt semesterlagen (1977:480) och, enligt 15 § samma lag, kan begära att dag då han eller hon inte kan arbeta på grund av sjukdom inte ska räknas som semesterdag.
\newline 2. Ledighet under studietid för vilken oavkortade löneförmåner lämnas.
\newline 3. Ledighet under tid då den försäkrade får ersättning för att delta i teckenspråksutbildning för vissa föräldrar (TUFF).
\newline 4. Ledighet för ferier eller motsvarande uppehåll för lärare som är anställda inom utbildningsväsendet.
\subsection*{14 §}
\paragraph*{}
Smittbärarpenning lämnas med ett belopp som motsvarar den försäkrades sjukpenning på normalnivån enligt 28 kap.
\subsection*{15 §}
\paragraph*{}
Är smittbäraren inte försäkrad för sjukpenning enligt 6 kap. 6 § 3, ska smittbärarpenningen lämnas med det belopp som sjukpenningen skulle ha uppgått till om den sjukpenninggrundande inkomsten hade beräknats med bortseende från bestämmelserna i 25 kap. 3 § andra stycket 1 och 2.
\subsection*{16 §}
\paragraph*{}
I fall som avses i 15 § ska som inkomst av anställning även räknas ersättning för eget arbete från en arbetsgivare som är bosatt utomlands eller som är en utländsk juridisk person, när arbetet har utförts i arbetsgivarens verksamhet utomlands.
\subsection*{17 §}
\paragraph*{}
Smittbärarpenning enligt 15 § ska beräknas per kalenderdag enligt 28 kap. 10 och 11 §§.
\subsection*{18 §}
\paragraph*{}
Smittbärarpenningen ska minskas med följande förmåner i den utsträckning förmånerna lämnas för samma tid:
\newline 1. graviditetspenning,
\newline 2. föräldrapenningsförmåner,
\newline 3. sjuklön eller sådan ersättning från Försäkringskassan som avses i 20 § lagen (1991:1047) om sjuklön,
\newline 4. sjukpenning enligt denna balk eller motsvarande äldre lag,
\newline 5. rehabiliteringspenning,
\newline 6. livränta enligt denna balk eller motsvarande äldre författning på grund av smitta,
\newline 7. närståendepenning, och
\newline 8. statlig ersättning som lämnas för arbete i etableringsjobb.
\paragraph*{}
Det som föreskrivs i första stycket gäller även motsvarande förmån som lämnas till smittbäraren på grundval av utländsk lagstiftning.
Lag (2020:475).
\subsection*{19 §}
\paragraph*{}
Om smittbäraren har inkomst av anställning från en arbetsgivare som avses i 1 eller 2 § lagen (1994:260) om offentlig anställning för tid som avses i 7 §, lämnas smittbärarpenning endast till den del den överstiger inkomsten.
\paragraph*{}
När anställningen avser arbete i etableringsjobb ska även statlig ersättning som lämnas för det arbetet räknas med i den inkomst som avses i första stycket.
Lag (2020:475).
\subsection*{20 §}
\paragraph*{}
En smittbärare har rätt till skälig ersättning för resekostnader i samband med läkarundersökning, hälsokontroll, vård, behandling eller annan motsvarande åtgärd som sker på grund av bestämmelserna i
\newline 1. smittskyddslagen (2004:168), eller
\newline 2. livsmedelslagen (2006:804) eller föreskrifter som har meddelats med stöd av sistnämnda lag.
\subsection*{21 §}
\paragraph*{}
Ersättning enligt 20 § lämnas inte till den del ersättning för resan kan lämnas enligt
\newline 1. andra bestämmelser i denna balk,
\newline 2. någon äldre författning som motsvarar bestämmelserna i 39-43 kap., eller
\newline 3. bestämmelser i någon annan författning om ersättning av allmänna medel för sjukresor och sjuktransporter.
\chapter*{47 Närståendepenning}
\subsection*{1 §}
\paragraph*{}
I detta kapitel finns inledande bestämmelser i 2 §.
\paragraph*{}
Vidare finns bestämmelser om
\newline - rätten till närståendepenning i 3-6 §§,
\newline - förlust av ersättning i 7 §,
\newline - förmånstiden i 8-10 §§,
\newline - förmånsnivåer och förvärvsarbete i 11-13 §§,
\newline - beräkning av närståendepenning i 14-16 §§, och
\newline - samordning med andra förmåner i 17 §.
Lag (2013:747).
\subsection*{2 §}
\paragraph*{}
Bestämmelser om närståendepenning i samband med att en svårt sjuk person vårdas av en närstående finns i detta kapitel.
\paragraph*{}
I lagen (1988:1465) om ledighet för närståendevård finns bestämmelser i vilka anges när en arbetstagare som vårdar en närstående har rätt att vara ledig från sin anställning.
Lag (2010:1307).
\subsection*{3 §}
\paragraph*{}
En försäkrad som vårdar en närstående som är svårt sjuk har rätt till närståendepenning för tid då han eller hon avstår från förvärvsarbete i samband med vården, om
\newline 1. den sjuke är försäkrad enligt någon bestämmelse i 4-7 kap.,
\newline 2. den sjuke vårdas här i landet, och 3. den sjuke har gett sitt samtycke till vården.
\paragraph*{}
Om den sjuke på grund av sitt tillstånd inte kan ge samtycke till vården enligt första stycket 3, ska i stället detta framgå.
\subsection*{4 §}
\paragraph*{}
Med en svårt sjuk person likställs den som har fått infektion av hiv (humant immunbristvirus) genom smitta
\newline 1. vid användning av blod eller blodprodukter vid behandling inom den svenska hälso- och sjukvården, eller
\newline 2. av en sådan person som avses i 1 och som är hans eller hennes nuvarande eller före detta make eller sambo, under förutsättning att smittan ägt rum innan den först smittade fått kännedom om sin infektion.
\subsection*{5 §}
\paragraph*{}
Om någon vårdar mer än en person under samma tid ger det inte rätt till ytterligare ersättning.
\subsection*{6 §}
\paragraph*{}
Närståendepenning för vård av en person får inte lämnas till flera vårdare för samma tid.
\subsection*{7 §}
\paragraph*{}
Rätten till närståendepenning går förlorad om en ansökan om denna förmån inte görs inom tre månader från den dag ersättningen avser. Detta gäller dock inte om det har funnits hinder för att göra en ansökan inom denna tid eller det finns särskilda skäl för att förmånen ändå bör lämnas.
Lag (2013:747).
\subsection*{8 §}
\paragraph*{}
Närståendepenning lämnas för högst 100 dagar sammanlagt för varje person som vårdas. Ersättningen lämnas från och med första vårddagen.
\subsection*{9 §}
\paragraph*{}
Närståendepenning lämnas för högst 240 dagar sammanlagt vid vård av en närstående som har fått infektion av hiv på sätt som anges i 4 §. Ersättningen lämnas från och med första vårddagen.
\subsection*{10 §}
\paragraph*{}
Vid beräkning av antalet dagar med rätt till ersättning gäller följande:
\newline - En dag med hel närståendepenning motsvarar en dag.
\newline - En dag med tre fjärdedels, halv eller en fjärdedels närståendepenning motsvarar tre fjärdedelar, hälften respektive en fjärdedel av en dag.
Lag (2018:670).
\subsection*{11 §}
\paragraph*{}
Närståendepenning lämnas enligt följande förmånsnivåer:
\newline 1. Hel närståendepenning lämnas för dag när en vårdare helt avstått från förvärvsarbete.
\newline 2. Tre fjärdedels närståendepenning lämnas för dag när en vårdare förvärvsarbetat högst en fjärdedel av den tid han eller hon annars skulle ha arbetat.
\newline 3. Halv närståendepenning lämnas för dag när en vårdare förvärvsarbetat högst hälften av den tid han eller hon annars skulle ha arbetat.
\newline 4. En fjärdedels närståendepenning lämnas för dag när en vårdare förvärvsarbetat högst tre fjärdedelar av den tid han eller hon annars skulle ha arbetat.
Lag (2017:1124).
\subsection*{12 §}
\paragraph*{}
Vid tillämpningen av 11 § ska som förvärvsarbete inte betraktas sådant förvärvsarbete som utförs under tid då vårdaren förvärvsarbetar med stöd av 37 kap. 3 §.
\paragraph*{}
Om det inte går att avgöra under vilken tid vårdaren avstår från förvärvsarbete ska frånvaron i första hand anses som frånvaro från sådant förvärvsarbete som avses i 37 kap. 3 §.
\subsection*{13 §}
\paragraph*{}
Om en vårdare får oreducerade löneförmåner under tid då han eller hon bedriver studier, likställs avstående från studier med avstående från förvärvsarbete i den utsträckning vårdaren går miste om löneförmånerna.
\subsection*{14 §}
\paragraph*{}
Närståendepenning lämnas med ett belopp som motsvarar vårdarens sjukpenning på normalnivån enligt 28 kap. med de avvikelser som följer av 15 och 16 §§.
\subsection*{15 §}
\paragraph*{}
Närståendepenning ska arbetstidsberäknas enligt 28 kap. 12-18 §§ för hela den tid som förmånen avser när ersättning lämnas på grundval av sjukpenninggrundande inkomst av anställning. Detta gäller dock inte i fall som avses i 6 § samma kapitel, då närståendepenning i stället ska kalenderdagsberäknas enligt 10 och 11 §§ i det kapitlet. Vad som föreskrivs i 28 kap. 6 § tredje stycket om när kalenderdagsberäknad sjukpenning lämnas till en arbetslös försäkrad under de första 14 dagarna i en sjukperiod, tillämpas i fråga om närståendepenning för hela den tid som förmånen avser.
Lag (2010:2005).
\subsection*{16 §}
\paragraph*{}
För en vårdare som studerar får närståendepenningen vid ett studieuppehåll inte beräknas på grundval av den sjukpenninggrundande inkomst som följer av bestämmelserna i 26 kap. 22 §.
\subsection*{17 §}
\paragraph*{}
Närståendepenning lämnas inte i den utsträckning vårdaren för samma tid får
\newline 1. föräldrapenningsförmåner,
\newline 2. sjuklön eller sådan ersättning från Försäkringskassan som avses i 20 § lagen (1991:1047) om sjuklön,
\newline 3. sjukpenning,
\newline 4. rehabiliteringspenning, eller
\newline 5. ersättning som motsvarar sjukpenning enligt någon annan författning eller på grund av regeringens beslut i ett särskilt fall.
\part*{D SÄRSKILDA FÖRMÅNER VID FUNKTIONSHINDER}
\chapter*{48 Innehåll, definitioner och förklaringar}
\subsection*{1 §}
\paragraph*{}
I avdelning D finns bestämmelser om särskilda socialförsäkringsförmåner vid funktionshinder.
\subsection*{2 §}
\paragraph*{}
Förmåner enligt denna avdelning är
\newline - merkostnadsersättning till en person med nedsatt funktionsförmåga eller till en förälder som har ett barn med nedsatt funktionsförmåga,
\newline - assistansersättning till en funktionshindrad som behöver personlig assistans för sina grundläggande behov, och
\newline - bilstöd till personer med funktionshinder för att skaffa eller anpassa motorfordon.
Lag (2018:1265).
\subsection*{3 §}
\paragraph*{}
I detta kapitel finns inledande bestämmelser om särskilda förmåner vid funktionshinder.
\paragraph*{}
Vidare finns bestämmelser om merkostnadsersättning, assistansersättning och bilstöd till en funktionshindrad i 49-52 kap.
Lag (2018:1265).
\subsection*{4 §}
\paragraph*{}
En förmån enligt denna avdelning lämnas endast till den som har ett gällande försäkringsskydd för förmånen enligt 4 och 5 kap.
\paragraph*{}
Bestämmelser om anmälan och ansökan samt vissa gemensamma bestämmelser om förmåner och handläggning finns i 104-117 kap. (avdelning H).
\subsection*{5 §}
\paragraph*{}
Ärenden som avser förmåner enligt denna avdelning handläggs av Försäkringskassan.
\chapter*{49 Innehåll}
\subsection*{1 §}
\paragraph*{}
I denna underavdelning finns bestämmelser om
\newline - merkostnadsersättning i 50 kap.,
\newline - assistansersättning i 51 kap., och
\newline - bilstöd i 52 kap.
Lag (2018:1265).
\chapter*{50 Merkostnadsersättning}
\subsection*{1 §}
\paragraph*{}
I detta kapitel finns bestämmelser om
\newline - vad som avses med merkostnader och i vilka fall en person ska anses blind eller gravt hörselskadad i 2 §,
\newline - personer som likställs med förälder i 3 §,
\newline - rätten till merkostnadsersättning i 4-8 §§,
\newline - ersättningsberättigande merkostnader i 9 §,
\newline - förmånstiden i 10 och 11 §§,
\newline - ersättningsnivå i 12 §,
\newline - ansökan i 13 §,
\newline - omprövning i 14 och 15 §§, och
\newline - utbetalning i 16 §.
Lag (2018:1265).
\subsection*{2 §}
\paragraph*{}
När det gäller merkostnadsersättning avses med en merkostnad en skälig kostnad som uppkommer på grund av en persons funktionsnedsättning och som går utöver en kostnad som är normal för en person utan funktionsnedsättning i motsvarande ålder.
\paragraph*{}
När det gäller merkostnadsersättning ska en person anses blind om hans eller hennes synförmåga, sedan ljusbrytningsfel har rättats, är så nedsatt att han eller hon saknar ledsyn.
\paragraph*{}
När det gäller merkostnadsersättning ska en person anses gravt hörselskadad om han eller hon även med hörapparat saknar möjlighet eller har stora svårigheter att uppfatta tal.
Lag (2018:1265).
\subsection*{3 §}
\paragraph*{}
Följande personer likställs med förälder när det gäller merkostnadsersättning:
\newline 1. särskilt förordnad vårdnadshavare som har vård om ett barn, och
\newline 2. blivande adoptivförälder vid adoption av ett barn som inte är svensk medborgare eller bosatt här i landet när den blivande adoptivföräldern får barnet i sin vård.
Lag (2018:1265).
\subsection*{4 §}
\paragraph*{}
/Upphör att gälla U:2025-12-01/
Rätt till merkostnadsersättning har en försäkrad person för merkostnader, i sådan omfattning som anges i 12 §, till följd av att han eller hon före 66 års ålder har fått sin funktionsförmåga nedsatt, om det kan antas att nedsättningen kommer att bestå under minst ett år.
\paragraph*{}
En försäkrad som är blind eller gravt hörselskadad har dock alltid rätt till merkostnadsersättning, om blindheten eller den grava hörselskadan har inträtt före 66 års ålder och det kan antas att denna funktionsnedsättning kommer att bestå under minst ett år.
\paragraph*{}
Det som anges i andra stycket gäller dock inte för ett försäkrat barn som har en förälder som är underhållsskyldig enligt 7 kap. föräldrabalken eller som har en sådan särskilt förordnad vårdnadshavare som avses i 3 § 1.
Lag (2022:878).
\subsection*{4 §}
\paragraph*{}
/Träder i kraft I:2025-12-01/
Rätt till merkostnadsersättning har en försäkrad person för merkostnader, i sådan omfattning som anges i 12 §, till följd av att han eller hon före uppnådd riktålder för pension har fått sin funktionsförmåga nedsatt, om det kan antas att nedsättningen kommer att bestå under minst ett år.
\paragraph*{}
En försäkrad som är blind eller gravt hörselskadad har dock alltid rätt till merkostnadsersättning, om blindheten eller den grava hörselskadan har inträtt före uppnådd riktålder för pension och det kan antas att denna funktionsnedsättning kommer att bestå under minst ett år.
\paragraph*{}
Det som anges i andra stycket gäller dock inte för ett försäkrat barn som har en förälder som är underhållsskyldig enligt 7 kap. föräldrabalken eller som har en sådan särskilt förordnad vårdnadshavare som avses i 3 § 1.
Lag (2022:879).
\subsection*{5 §}
\paragraph*{}
När det gäller rätt till merkostnadsersättning för ett försäkrat barn som har en förälder som är underhållsskyldig enligt 7 kap. föräldrabalken tillämpas inte 4 § första stycket. Rätt till merkostnadsersättning har i sådana fall föräldern för merkostnader, i den omfattning som anges i 12 §, till följd av att barnet har fått sin funktionsförmåga nedsatt, om det kan antas att nedsättningen kommer att bestå under minst sex månader.
\paragraph*{}
En person som avses i 3 § 1 har rätt till ersättning i stället för en förälder.
Lag (2018:1265).
\subsection*{6 §}
\paragraph*{}
Vid bedömningen av rätten till merkostnadsersättning ska det bortses från merkostnader för behov som tillgodoses genom annat samhällsstöd.
Lag (2018:1265).
\subsection*{7 §}
\paragraph*{}
Om en förälder har merkostnader för flera barn, ska bedömningen av rätten till merkostnadsersättning grundas på en bedömning av de sammanlagda merkostnaderna för barnen.
\paragraph*{}
Första stycket gäller inte om det finns särskilda skäl mot en bedömning av de sammanlagda merkostnaderna.
Lag (2018:1265).
\subsection*{8 §}
\paragraph*{}
Om ett barns båda föräldrar ansöker om och har rätt till merkostnadsersättning, ska ersättningen lämnas med hälften till vardera föräldern.
\paragraph*{}
Om föräldrarna begär en annan fördelning av merkostnadsersättningen än den som anges i första stycket, ska ersättningen lämnas till dem enligt deras begäran. Om föräldrarna är oense om fördelningen av merkostnadsersättningen, ska ersättningen fördelas mellan dem i fjärdedelar utifrån en bedömning av respektive förälders merkostnader för barnet.
Lag (2018:1265).
\subsection*{9 §}
\paragraph*{}
Merkostnadsersättning lämnas för merkostnader för:
\newline 1. hälsa, vård och kost,
\newline 2. slitage och rengöring,
\newline 3. resor,
\newline 4. hjälpmedel,
\newline 5. hjälp i den dagliga livsföringen,
\newline 6. boende, och
\newline 7. övriga ändamål.
Lag (2018:1265).
\subsection*{10 §}
\paragraph*{}
Merkostnadsersättning lämnas från och med den månad när rätt till förmånen har inträtt, dock inte för längre tid tillbaka än tre månader före ansökningsmånaden. Merkostnadsersättning lämnas dock inte retroaktivt för en sådan månad för vilken det redan har lämnats ersättning för ett barn utom till den del det avser en ökning av ersättningen.
\paragraph*{}
Rätten till merkostnadsersättning får begränsas till att omfatta viss tid.
Lag (2018:1265).
\subsection*{11 §}
\paragraph*{}
Om ett barn som avses i 5 § avlider, lämnas merkostnadsersättning till och med sex månader efter dödsfallet eller den tidigare månad när ersättningen annars skulle ha upphört.
Lag (2018:1265).
\subsection*{12 §}
\paragraph*{}
Merkostnadsersättning ska lämnas med ett belopp som för år räknat motsvarar
\newline 1. 30 procent av prisbasbeloppet om merkostnaderna uppgår till 25 procent men inte 35 procent av prisbasbeloppet,
\newline 2. 40 procent av prisbasbeloppet om merkostnaderna uppgår till 35 procent men inte 45 procent av prisbasbeloppet,
\newline 3. 50 procent av prisbasbeloppet om merkostnaderna uppgår till 45 procent men inte 55 procent av prisbasbeloppet,
\newline 4. 60 procent av prisbasbeloppet om merkostnaderna uppgår till 55 procent av prisbasbeloppet men inte 65 procent av prisbasbeloppet, eller
\newline 5. 70 procent av prisbasbeloppet om merkostnaderna uppgår till 65 procent av prisbasbeloppet eller mer.
\paragraph*{}
Merkostnadsersättning ska dock för en försäkrad som är blind alltid lämnas med ett belopp som för år räknat motsvarar 70 procent av prisbasbeloppet. Om han eller hon för samma tid får hel sjukersättning, hel aktivitetsersättning eller hel ålderspension ska merkostnadsersättning, från och med den månad när den andra förmånen lämnas, lämnas med 40 procent av prisbasbeloppet, om inte den försäkrades merkostnader är av sådan omfattning att högre ersättning ska lämnas enligt första stycket.
\paragraph*{}
Merkostnadsersättning ska dock för en försäkrad som är gravt hörselskadad alltid lämnas med ett belopp som för år räknat motsvarar 40 procent av prisbasbeloppet, om inte den försäkrades merkostnader är av sådan omfattning att högre ersättning ska lämnas enligt första stycket.
Lag (2018:1265).
\subsection*{13 §}
\paragraph*{}
Ansökan om merkostnadsersättning för ett barn som avses i 5 § kan göras gemensamt av föräldrarna, av var och en av dem för sig eller av endast den ena föräldern.
Lag (2018:1265).
\subsection*{14 §}
\paragraph*{}
Rätten till merkostnadsersättning ska omprövas
\newline 1. minst vart fjärde år, om det inte finns skäl för omprövning med längre mellanrum, eller
\newline 2. när förhållanden som påverkar rätten till merkostnadsersättningen ändras.
\paragraph*{}
Omprövning enligt första stycket 2 ska dock inte göras vid enbart tillfälliga förändringar.
Lag (2018:1265).
\subsection*{15 §}
\paragraph*{}
Ändring av merkostnadsersättning ska gälla från och med månaden närmast efter den månad när anledningen till ändringen uppkom. Gäller det ökning efter ansökan ska dock även 10 § första stycket beaktas.
Lag (2018:1265).
\subsection*{16 §}
\paragraph*{}
Merkostnadsersättning ska betalas ut månadsvis. När ersättningsbeloppet beräknas för månad ska den ersättning för år räknat som beräkningen utgår från avrundas till närmaste krontal som är jämnt delbart med tolv.
Lag (2018:1265).
\chapter*{51 Assistansersättning}
\subsection*{1 §}
\paragraph*{}
I detta kapitel finns bestämmelser om
\newline - rätten till assistansersättning i 2-6 §§,
\newline - förmånstiden i 7 och 8 §§,
\newline - beräkning av assistansersättning i 9-11 a §§,
\newline - omprövning vid ändrade förhållanden i 12 och 13 §§,
\newline - utbetalning av assistansersättning i 14-19 §§,
\newline - återbetalning av assistansersättning i 20 §,
\newline - samverkan med kommunen i 21-23 §§, och
\newline - uppgiftsskyldighet i 24 §.
Lag (2018:122).
\subsection*{2 §}
\paragraph*{}
En försäkrad som omfattas av 1 § lagen (1993:387) om stöd och service till vissa funktionshindrade kan för sin dagliga livsföring få assistansersättning för kostnader för sådan personlig assistans som avses i 9 a § samma lag.
\paragraph*{}
De bestämmelser i lagen om stöd och service till vissa funktionshindrade som avser utförandet av insatsen personlig assistans tillämpas också på personlig assistans enligt detta kapitel. Avser den personliga assistansen stöd- och serviceinsatser åt barn med funktionshinder ska bestämmelserna i lagen (2010:479) om registerkontroll av personal som utför vissa insatser åt barn med funktionshinder tillämpas.
Lag (2010:482).
\subsection*{3 §}
\paragraph*{}
För rätt till assistansersättning krävs att den försäkrade behöver personlig assistans i genomsnitt mer än 20 timmar i veckan för sådana grundläggande behov som avses i 9 a § lagen (1993:387) om stöd och service till vissa funktionshindrade.
\subsection*{4 §}
\paragraph*{}
Assistansersättning lämnas endast under förutsättning att den används för köp av personlig assistans eller för kostnader för personliga assistenter.
\subsection*{5 §}
\paragraph*{}
Assistansersättning lämnas inte för sjukvårdande insatser enligt hälso- och sjukvårdslagen (2017:30).
\paragraph*{}
I 106 kap. 24-25 a §§ finns bestämmelser om rätten till assistansersättning när den funktionshindrade vårdas på en institution, bor i en gruppbostad eller vistas i eller deltar i barnomsorg, skola eller daglig verksamhet enligt 9 § 10 lagen (1993:387) om stöd och service till vissa funktionshindrade.
Lag (2020:440).
\subsection*{6 §}
\paragraph*{}
När behovet av personlig assistans bedöms för ett barn ska det bortses från det hjälpbehov som en vårdnadshavare normalt ska tillgodose enligt föräldrabalken med hänsyn till barnets ålder, utveckling och övriga omständigheter.
\paragraph*{}
Detta ska göras genom schablonavdrag (föräldraavdrag) från barnets behov av hjälp med grundläggande behov och andra personliga behov enligt 9 a § lagen (1993:387) om stöd och service till vissa funktionshindrade. Föräldraavdrag ska fastställas med hänsyn till barnets ålder och göras dels från grundläggande behov, dels från andra personliga behov. Avdrag ska dock inte göras till den del hjälpbehovet avser
\newline 1. sådant stöd som avses i 9 a § första stycket 1 eller 7 lagen om stöd och service till vissa funktionshindrade,
\newline 2. åtgärder som är direkt nödvändiga för att hjälp enligt 9 a § första stycket 1 eller 7 lagen om stöd och service till vissa funktionshindrade ska kunna ges,
\newline 3. måltider i form av sondmatning,
\newline 4. åtgärder som är direkt nödvändiga för förberedelse och efterarbete i samband med sådana måltider,
\newline 5. grundläggande behov från och med den månad då barnet fyller 12 år,
\newline 6. andra personliga behov före den månad då barnet fyller ett år, eller
\newline 7. andra personliga behov från och med den månad då barnet fyller 18 år.
\paragraph*{}
Regeringen kan med stöd av 8 kap. 7 § regeringsformen meddela föreskrifter om föräldraavdragens storlek.
\paragraph*{}
Om omvårdnadsbidrag lämnas får det inte påverka bedömningen enligt 3 §.
Lag (2022:1251).
\subsection*{7 §}
\paragraph*{}
Assistansersättning får inte lämnas för längre tid tillbaka än en månad före den månad när ansökan gjorts eller det kommit in en anmälan från kommunen att det kan antas att den enskilde har rätt till assistansersättning.
Assistansersättning som avser assistans som utförs innan beslut har fattats i ett ärende lämnas endast om den enskilde månadsvis under handläggningstiden redovisar till Försäkringskassan att assistansen utförs i enlighet med kraven och förutsättningarna i denna balk.
Lag (2012:935).
\subsection*{8 §}
\paragraph*{}
/Upphör att gälla U:2025-12-01/
Assistansersättning kan lämnas för tid efter det att den försäkrade har fyllt 66 år endast om
\newline 1. ersättning har beviljats innan han eller hon har fyllt 66 år, eller
\newline 2. ansökan kommer in till Försäkringskassan senast dagen före 66-årsdagen och därefter blir beviljad.
Lag (2022:878).
\subsection*{8 §}
\paragraph*{}
/Träder i kraft I:2025-12-01/
Assistansersättning kan lämnas för tid efter det att den försäkrade har uppnått riktåldern för pension endast om
\newline 1. ersättning har beviljats innan han eller hon har uppnått riktåldern för pension, eller
\newline 2. ansökan kommer in till Försäkringskassan senast dagen före den dag när riktåldern för pension uppnås och därefter blir beviljad.
Lag (2022:879).
\subsection*{9 §}
\paragraph*{}
Assistansersättning ska beviljas för ett visst antal timmar per vecka, månad eller längre tid, dock längst sex månader, när den försäkrade har behov av personlig assistans för sin dagliga livsföring (beviljade assistanstimmar).
\subsection*{10 §}
\paragraph*{}
/Upphör att gälla U:2025-12-01/
Antalet assistanstimmar får inte utökas efter det att den försäkrade har fyllt 66 år.
Lag (2022:878).
\subsection*{10 §}
\paragraph*{}
/Träder i kraft I:2025-12-01/
Antalet assistanstimmar får inte utökas efter det att den försäkrade har uppnått riktåldern för pension.
Lag (2022:879).
\subsection*{11 §}
\paragraph*{}
Assistansersättning lämnas med ett särskilt angivet belopp per timme. Det eller de belopp som assistansersättning lämnas med ska för varje år bestämmas som schablonbelopp som beräknas med ledning av de uppskattade kostnaderna för att få assistans.
\paragraph*{}
Om det finns särskilda skäl kan ersättning till en försäkrad efter ansökan lämnas med ett högre belopp än det schablonbelopp med vilket assistansersättning ska lämnas till den försäkrade enligt vad som följer av första stycket. Ersättningen får dock inte överstiga schablonbeloppet med mer än 12 procent.
\paragraph*{}
Assistansersättning som betalas ut enligt 16 § andra stycket lämnas med skäligt belopp.
Lag (2018:122).
\subsection*{11 a §}
\paragraph*{}
Regeringen eller den myndighet som regeringen bestämmer kan med stöd av 8 kap. 7 § regeringsformen
\newline 1. meddela ytterligare föreskrifter om sådana belopp som avses i 11 § första och andra styckena, och
\newline 2. meddela föreskrifter om i vilken omfattning tid som avses i 9 a § fjärde stycket 1 och 2 lagen (1993:387) om stöd och service till vissa funktionshindrade berättigar till assistansersättning.
Lag (2022:1226).
\subsection*{12 §}
\paragraph*{}
Rätten till assistansersättning ska omprövas i den utsträckning som denna rätt har minskat i omfattning på grund av väsentligt ändrade förhållanden som är hänförliga till den försäkrade.
Lag (2018:122).
\subsection*{13 §}
\paragraph*{}
En ändring av assistansersättning vid väsentligt ändrade förhållanden ska gälla från och med den månad när anledningen till ändring uppkom.
\subsection*{14 §}
\paragraph*{}
Assistansersättning betalas ut månadsvis med visst belopp för det antal beviljade assistanstimmar som assistans har lämnats.
\paragraph*{}
Regeringen eller den myndighet som regeringen bestämmer kan med stöd av 8 kap. 7 § regeringsformen meddela föreskrifter om att ersättning med ett högre belopp än schablonbeloppet enligt 11 § andra stycket ska betalas ut på annat sätt än månadsvis.
Lag (2018:555).
\subsection*{15 §}
\paragraph*{}
Utbetalningen av assistansersättning för en viss månad får grundas på ett beräknat antal assistanstimmar för den månaden. Avräkning för större avvikelser ska då göras senast andra månaden efter den då den preliminära utbetalningen skett. Slutavräkning ska göras senast två månader efter utgången av varje tidsperiod för vilken assistansersättning beviljats.
\subsection*{16 §}
\paragraph*{}
Assistansersättning enligt 14 § betalas inte ut om assistansen har utförts av någon
\newline 1. som inte har fyllt 18 år,
\newline 2. på arbetstid som överstiger den tid som anges i 2-4 §§ lagen (1970:943) om arbetstid m.m. i husligt arbete, 5-10 b §§ arbetstidslagen (1982:673) eller kollektivavtal som uppfyller kraven i 3 § arbetstidslagen, eller
\newline 3. som till följd av ålderdom, sjukdom eller liknande orsak saknar förmåga att utföra arbete som personlig assistent.
\paragraph*{}
Om assistansen har utförts av någon som är bosatt utanför Europeiska ekonomiska samarbetsområdet, betalas assistansersättning ut enligt 14 § endast om det finns särskilda skäl.
Lag (2012:935).
\subsection*{16 a §}
\paragraph*{}
När assistansersättning har beviljats och assistansen utförs av någon som är närstående till eller lever i hushållsgemenskap med den försäkrade och som inte är anställd av kommunen ska Inspektionen för vård och omsorg få tillträde till bostaden för att inspektera assistansen enligt 26 d § lagen (1993:387) om stöd och service till vissa funktionshindrade. Vid sådan inspektion gäller inte 26 e § lagen om stöd och service till vissa funktionshindrade.
\paragraph*{}
Med närstående enligt första stycket avses make, sambo, barn, förälder och syskon samt deras makar, sambor och barn.
Lag (2012:962).
\subsection*{16 b §}
\paragraph*{}
Assistansersättning enligt 14 § betalas inte ut för assistans som har utförts i en yrkesmässig enskild verksamhet utan tillstånd enligt 23 § lagen (1993:387) om stöd och service till vissa funktionshindrade.
\paragraph*{}
Även om assistansen har utförts utan tillstånd betalas assistansersättning ut om
\newline 1. verksamheten har haft tillstånd under avtalstiden och assistansen har utförts senast två veckor efter den dag då den försäkrade underrättades om att tillståndet har upphört att gälla, eller
\newline 2. det finns särskilda skäl.
Lag (2021:878).
\subsection*{17 §}
\paragraph*{}
Om den försäkrade har fått biträde av personlig assistent genom kommunen får Försäkringskassan besluta att assistansersättningen ska betalas ut till kommunen i den utsträckning som den motsvarar avgiften till kommunen för sådan hjälp.
\subsection*{18 §}
\paragraph*{}
Om den försäkrade till följd av ålderdomssvaghet, sjuklighet, långvarigt missbruk av beroendeframkallande medel eller någon annan liknande orsak är ur stånd att själv ta hand om assistansersättningen, får Försäkringskassan besluta att ersättningen ska betalas ut till någon annan person eller till en kommunal myndighet för att användas till kostnader för personlig assistans till den försäkrade.
\subsection*{19 §}
\paragraph*{}
Utöver det som följer av 17 och 18 §§, får Försäkringskassan på begäran av den försäkrade besluta att assistansersättningen ska betalas ut till en kommun eller till någon annan som har tillstånd enligt 23 § lagen (1993:387) om stöd och service till vissa funktionshindrade att bedriva verksamhet med personlig assistans.
Lag (2012:935).
\subsection*{20 §}
\paragraph*{}
Den försäkrade eller den som för den försäkrades räkning tagit emot assistansersättning ska utan uppmaning betala tillbaka sådan ersättning som inte har använts för köp av personlig assistans eller för kostnader för personliga assistenter. Återbetalning ska göras senast i samband med slutavräkningen enligt 15 §. Om det inte finns särskilda skäl, ska ersättningen betalas tillbaka av den försäkrades förmyndare i stället för av den försäkrade om denne är under 18 år. Om det finns flera förmyndare svarar de solidariskt för skyldigheten.
\paragraph*{}
Om återbetalning inte görs får Försäkringskassan besluta om återbetalning enligt bestämmelserna i 108 kap.
Lag (2012:935).
\subsection*{21 §}
\paragraph*{}
När en försäkrad ansöker om eller beviljas assistansersättning ska den kommun som enligt 16, 16 c eller 16 d § lagen (1993:387) om stöd och service till vissa funktionshindrade har ansvar för insatser åt den försäkrade höras i ärendet, om det inte är obehövligt.
Beslut om assistansersättning och beslut enligt 17-19 §§ ska sändas till kommunen.
Lag (2011:332).
\subsection*{22 §}
\paragraph*{}
För den som beviljats assistansersättning ska den kommun som enligt 16, 16 c eller 16 d § lagen (1993:387) om stöd och service till vissa funktionshindrade har ansvar för insatser åt den försäkrade ersätta kostnaderna för de första 20 assistanstimmarna per vecka.
Lag (2011:332).
\subsection*{23 §}
\paragraph*{}
Försäkringskassan ska underrätta kommunen om det belopp som kommunen ska betala enligt 22 §. Beloppet ska betalas månadsvis till Försäkringskassan, om inte annat följer av föreskrifter som regeringen har meddelat med stöd av tredje stycket.
\paragraph*{}
Kommunen har rätt att ta del av den slutavräkning som ska göras enligt 15 §. Har kommunen betalat ersättning med ett för högt belopp ska beloppet återbetalas från Försäkringskassan.
\paragraph*{}
Regeringen får meddela föreskrifter om skyldighet för kommunen att betala
\newline 1. en del av ersättningen enligt 22 § på annat sätt än månadsvis, och
\newline 2. dröjsmålsränta på en skuld som avser betalningsskyldighet enligt 22 §.
Lag (2018:555).
\subsection*{24 §}
\paragraph*{}
Den som är arbetsgivare för eller uppdragsgivare åt en personlig assistent ska lämna följande uppgifter till Försäkringskassan:
\newline 1. Uppgifter som visar om assistenten är närstående till eller lever i hushållsgemenskap med den assistansberättigade, om assistenten har fyllt 18 år och om assistenten är bosatt inom eller utanför EES-området.
\paragraph*{}
Uppgifterna ska lämnas innan assistansen börjar utföras och vid ändrade förhållanden.
\newline 2. Uppgifter som visar den arbetstid som assistenten har arbetat hos en assistansberättigad. Uppgifterna ska lämnas månadsvis i efterhand.
\newline 3. Uppgifter som visar att något förhållande som anges i 16 § första stycket 3 inte föreligger. Uppgifterna ska lämnas på begäran av Försäkringskassan.
Lag (2012:935).
\chapter*{52 Bilstöd}
\subsection*{1 §}
\paragraph*{}
I detta kapitel finns bestämmelser om
\newline - rätten till bilstöd i 2-14 §§,
\newline - beräkning av bilstöd i 15-22 §§,
\newline - funktionskontroll i 22 a §, och
\newline - återbetalning av bilstöd i 23-25 §§.
Lag (2016:1066).
\subsection*{2 §}
\paragraph*{}
Bilstöd kan lämnas till en försäkrad som på grund av ett varaktigt funktionshinder har väsentliga svårigheter att förflytta sig på egen hand eller att anlita allmänna kommunikationer.
\subsection*{3 §}
\paragraph*{}
Bilstöd kan lämnas även till en försäkrad förälder till ett försäkrat barn som har ett sådant funktionshinder som avses i 2 §.
\subsection*{4 §}
\paragraph*{}
Med en förälder likställs när det gäller bilstöd följande personer:
\newline 1. en vårdnadshavare som inte är förälder och som har vård om barnet,
\newline 2. den med vilken föräldern är eller har varit gift eller har eller har haft barn, om de stadigvarande sammanbor,
\newline 3. den som tagit emot ett barn för stadigvarande vård och fostran i syfte att adoptera det, och
\newline 4. den som i annat fall tagit emot ett barn för stadigvarande vård och fostran i sitt hem, om placeringen kan förväntas pågå i minst tre år.
\subsection*{5 §}
\paragraph*{}
Bilstöd lämnas inom ramen för anslagna medel och i form av
\newline 1. grundbidrag,
\newline 2. anskaffningsbidrag,
\newline 3. tilläggsbidrag,
\newline 4. anpassningsbidrag som avser ett fordon, och
\newline 5. bidrag för körkortsutbildning.
\paragraph*{}
Anpassningsbidrag enligt första stycket 4 lämnas för sådana åtgärder som anges i 8 § första stycket 2-4.
Lag (2016:1066).
\subsection*{6 §}
\paragraph*{}
Grundbidrag, anskaffningsbidrag och tilläggsbidrag lämnas endast för fordon som anskaffats efter det att ett beslut om rätt till sådant bidrag har meddelats.
Lag (2016:1066).
\subsection*{7 §}
\paragraph*{}
Om en försäkrad har fått grundbidrag, anskaffningsbidrag eller tilläggsbidrag får sådant bidrag beviljas på nytt tidigast nio år efter det senaste beslutet att bevilja något av dessa bidrag.
\paragraph*{}
Nytt bidrag får dock lämnas tidigare om
\newline 1. det finns skäl för det från trafiksäkerhetssynpunkt eller medicinsk synpunkt, eller
\newline 2. fordonet har framförts minst 18 000 mil sedan grundbidrag, anskaffningsbidrag eller tilläggsbidrag senast beviljades.
Lag (2016:1066).
\subsection*{8 §}
\paragraph*{}
Bilstöd lämnas för
\newline 1. anskaffning av personbil klass I med en utvändig höjd som inte överstiger 2 050 millimeter, motorcykel eller moped,
\newline 2. ändring av ett fordon som avses i 1 och kostnader i samband därmed samt justering och reparation av ändringen,
\newline 3. anskaffning av en särskild anordning på ett fordon som avses i 1 och kostnader i samband därmed samt justering och reparation av anordningen,
\newline 4. kostnader för körträning som behövs före och efter sådana åtgärder som anges i 2 och 3, och kostnader i samband därmed, eller
\newline 5. körkortsutbildning i samband med anskaffning av motorfordon.
\paragraph*{}
Vad som avses med personbil klass I, motorcykel och moped anges i lagen (2001:559) om vägtrafikdefinitioner.
Lag (2020:509).
\subsection*{9 §}
\paragraph*{}
Om det på grund av funktionshindrets art eller andra omständigheter finns särskilda skäl för det, lämnas bilstöd enligt 8 § första stycket 1-3 för ett annat motorfordon än som avses i 8 § första stycket 1.
Lag (2020:509).
\subsection*{10 §}
\paragraph*{}
/Upphör att gälla U:2025-12-01/
Bilstöd enligt 8 § första stycket 1-3 lämnas till
\newline 1. en försäkrad som är under 66 år och är beroende av ett sådant fordon som avses i 8 eller 9 § för att genom arbete få sin försörjning eller ett väsentligt tillskott till sin försörjning, eller för att genomgå arbetslivsinriktad utbildning eller genomgå rehabilitering under vilken han eller hon får rehabiliteringsersättning enligt 31 kap. eller aktivitetsstöd enligt föreskrifter som meddelas av regeringen,
\newline 2. en försäkrad som är under 66 år och, efter att ha beviljats bidrag enligt 1, har beviljats sjukersättning eller aktivitetsersättning,
\newline 3. en försäkrad som fyllt 18 år men inte 50 år, och
\newline 4. en försäkrad som har barn som inte fyllt 18 år.
\paragraph*{}
Bilstöd lämnas i fall som avses i 3 § till en försäkrad som har barn som har ett sådant funktionshinder som avses i 2 §.
Lag (2022:878).
\subsection*{10 §}
\paragraph*{}
/Träder i kraft I:2025-12-01/
Bilstöd enligt 8 § första stycket 1-3 lämnas till
\newline 1. en försäkrad som inte har uppnått riktåldern för pension och är beroende av ett sådant fordon som avses i 8 eller 9 § för att genom arbete få sin försörjning eller ett väsentligt tillskott till sin försörjning, eller för att genomgå arbetslivsinriktad utbildning eller genomgå rehabilitering under vilken han eller hon får rehabiliteringsersättning enligt 31 kap. eller aktivitetsstöd enligt föreskrifter som meddelas av regeringen,
\newline 2. en försäkrad som inte har uppnått riktåldern för pension och, efter att ha beviljats bidrag enligt 1, har beviljats sjukersättning eller aktivitetsersättning,
\newline 3. en försäkrad som fyllt 18 år men inte 50 år, och
\newline 4. en försäkrad som har barn som inte fyllt 18 år.
\paragraph*{}
Bilstöd lämnas i fall som avses i 3 § till en försäkrad som har barn som har ett sådant funktionshinder som avses i 2 §.
Lag (2022:879).
\subsection*{11 §}
\paragraph*{}
Bilstöd enligt 10 § första stycket 4 och andra stycket lämnas under förutsättning att föräldern
\newline 1. sammanbor med barnet, och
\newline 2. behöver ett fordon för att förflytta sig tillsammans med barnet.
\subsection*{12 §}
\paragraph*{}
Om anpassningsbidrag tidigare har lämnats med stöd av 10 § första stycket 1-3 får bidrag för reparation av anpassningen lämnas trots att de förutsättningar som anges där inte längre är uppfyllda.
\subsection*{13 §}
\paragraph*{}
Bilstöd enligt 10 § första stycket 1 och 2 kan lämnas även till en försäkrad som inte själv ska köra fordonet, under förutsättning att någon annan kan anlitas som förare vid resorna.
\paragraph*{}
Bilstöd enligt 10 § första stycket 3 och 4 samt andra stycket lämnas under förutsättning att den försäkrade själv ska köra fordonet.
\subsection*{14 §}
\paragraph*{}
Bilstöd till körkortsutbildning lämnas till en försäkrad som har beviljats bilstöd till anskaffning av ett motorfordon och som uppfyller förutsättningarna i 10 § första stycket 1 samt är eller riskerar att bli arbetslös, om det bedöms att körkortsutbildningen kan leda till att han eller hon får ett stadigvarande arbete.
\subsection*{15 §}
\paragraph*{}
Vid anskaffning av motorfordon lämnas grundbidrag med högst 30 000 kronor.
\paragraph*{}
Vid anskaffning av motorcykel eller moped lämnas dock grundbidrag med högst 12 000 respektive 3 000 kronor.
Lag (2016:1066).
\subsection*{16 §}
\paragraph*{}
Anskaffningsbidrag lämnas med högst 40 000 kronor.
\paragraph*{}
Helt sådant bidrag lämnas till den vars årliga bruttoinkomst understiger 121 000 kronor.
Lag (2016:1066).
\subsection*{17 §}
\paragraph*{}
Till den vars årliga bruttoinkomst uppgår till 121 000 kronor eller mer, dock högst 210 000 kronor, lämnas anskaffningsbidrag med högst det belopp som återstår efter det att det belopp som anges i 16 § första stycket har reducerats med 400 kronor för varje tusental kronor som inkomsten överstiger 120 000 kronor.
\paragraph*{}
Till den vars årliga bruttoinkomst överstiger 210 000 kronor men inte 220 000 kronor lämnas anskaffningsbidrag med högst 4 000 kronor.
\paragraph*{}
Överstiger den årliga bruttoinkomsten 220 000 kronor lämnas inte något anskaffningsbidrag.
Lag (2016:1066).
\subsection*{18 §}
\paragraph*{}
Med årlig bruttoinkomst avses inkomst enligt 102 kap. 7-15 §§.
\paragraph*{}
För en försäkrad som avses i 10 § första stycket 1-3 ska bruttoinkomsten räknas fram utan hänsyn till makes inkomst.
\paragraph*{}
För föräldrar som avses i 10 § första stycket 4 och andra stycket räknas inkomsterna samman vid beräkning av anskaffningsbidragets storlek.
\subsection*{18 a §}
\paragraph*{}
Tilläggsbidrag för anskaffning av en personbil klass I lämnas med högst 50 000 kronor till en försäkrad som, för att kunna bruka bilen, har behov av en sådan ändring eller anordning för vilken anpassningsbidrag kan lämnas.
Lag (2020:509).
\subsection*{18 b §}
\paragraph*{}
Tilläggsbidrag för anskaffning av en personbil klass I lämnas, utöver vad som anges i 18 a §, med högst 60 000 kronor till en försäkrad som, för att kunna bruka bilen, har behov av en personbil som är
\newline 1. särskilt lämpad för personer som behöver färdas i bilen sittande i en rullstol eller göra överflyttning från en rullstol till ett bilsäte inne i bilen, eller
\newline 2. särskilt lämpad för att i annat fall än som avses i 1 medföra en motordriven rullstol eller ett annat jämförbart hjälpmedel för förflyttning.
Lag (2020:509).
\subsection*{18 c §}
\paragraph*{}
Tilläggsbidrag för anskaffning av en personbil klass I lämnas med särskilt belopp för kostnader som följer av att bilen har särskilda originalmonterade anordningar.
\paragraph*{}
Bidrag enligt första stycket lämnas endast om anordningen behövs för att den försäkrade ska kunna bruka bilen.
Lag (2016:1066).
\subsection*{18 d §}
\paragraph*{}
Regeringen eller den myndighet som regeringen bestämmer kan med stöd av 8 kap. 7 § regeringsformen meddela ytterligare föreskrifter om
\newline - vilka egenskaper ett fordon ska ha för att uppfylla kraven i 18 b §,
\newline - vilka slags anordningar som berättigar till tilläggsbidrag enligt 18 c § första stycket, och
\newline - beräkningen av tilläggsbidrag enligt 18 c § första stycket.
Lag (2016:1066).
\subsection*{19 §}
\paragraph*{}
Anpassningsbidraget motsvarar kostnaden för sådana åtgärder som anges i 8 § första stycket 2-4 och som behövs för att den försäkrade ska kunna bruka fordonet.
\paragraph*{}
Försäkringskassan får besluta att inte medge anpassningsbidrag om det fordon som den försäkrade valt är olämpligt med hänsyn till den form av anpassning som behövs. Detsamma gäller om fordonet är olämpligt med hänsyn till sitt skick.
\paragraph*{}
Om fordonet är äldre än fyra år eller har framförts mer än 6 000 mil, lämnas anpassningsbidrag för ändring av fordonet eller anskaffning av en särskild anordning på fordonet endast om det finns särskilda skäl. Sådana särskilda skäl krävs dock endast när det till den försäkrade för första gången ska lämnas anpassningsbidrag för fordonet.
Lag (2020:509).
\subsection*{19 a §}
\paragraph*{}
Anpassningsbidrag lämnas inte för kostnader för anordningar för vilka det har lämnats bidrag enligt 18 c § första stycket.
\paragraph*{}
Anpassningsbidrag lämnas inte heller för kostnader för åtgärder som hade kunnat undvikas om den försäkrade utnyttjat rätten till bidrag enligt 18 b eller 18 c §, för anordningar som är att anse som standardutrustning i bilen eller för normalt förekommande tilläggsutrustning till bilen. Detta gäller dock inte om det finns särskilda skäl för att ändå lämna anpassningsbidrag.
Lag (2020:509).
\subsection*{20 §}
\paragraph*{}
Ett anskaffningsbidrag ska minskas med sådant bidrag till inköp av ett fordon som den försäkrade får från en kommun eller en region eller i form av försäkringsersättning.
Lag (2019:843).
\subsection*{21 §}
\paragraph*{}
Grundbidrag, anskaffningsbidrag och tilläggsbidrag får tillsammans inte överstiga fordonets anskaffningskostnad. Avräkning ska i första hand göras på grundbidraget och i andra hand på anskaffningsbidraget.
Lag (2016:1066).
\subsection*{22 §}
\paragraph*{}
Om grundbidrag, anskaffningsbidrag eller tilläggsbidrag enligt 7 § andra stycket 1 lämnas tidigare än nio år efter det senaste beslutet att bevilja rätt till bilstöd, ska det tidigare bidraget räknas av från det nya bidraget.
\paragraph*{}
Det belopp som enligt första stycket ska räknas av ska dock minskas med en niondel för varje helt år som har förflutit sedan det tidigare bidraget betalades ut.
Lag (2016:1066).
\subsection*{22 a §}
\paragraph*{}
Om det inte är uppenbart obehövligt, ska en kontroll göras av om utförda anpassningar fungerar för den försäkrade på ett trafiksäkert sätt (funktionskontroll).
\paragraph*{}
En funktionskontroll ska göras så snart det är möjligt och senast sex månader efter det att anpassningarna har utförts.
Lag (2016:1066).
\subsection*{23 §}
\paragraph*{}
Grundbidrag, anskaffningsbidrag och tilläggsbidrag ska betalas tillbaka
\newline 1. om den försäkrade säljer eller på annat sätt gör sig av med fordonet inom nio år från det att bidraget beviljades, eller
\newline 2. om ett barn som avses i 10 § andra stycket efter att ha uppnått 18 års ålder själv beviljas bidrag.
Lag (2016:1066).
\subsection*{24 §}
\paragraph*{}
Det återbetalningspliktiga beloppet enligt 23 § ska minskas med en niondel för varje helt år som har förflutit sedan bidraget betalades ut. Om det finns särskilda skäl, får den försäkrade helt eller delvis befrias från återbetalningsskyldigheten.
\subsection*{25 §}
\paragraph*{}
Om den försäkrade vägrar att medverka till en funktionskontroll enligt 22 a §, ska Försäkringskassan besluta att återbetalning ska göras med 10 procent av det beviljade anpassningsbidraget, dock högst med ett belopp som motsvarar ett prisbasbelopp.
\paragraph*{}
Finns det särskilda skäl får Försäkringskassan helt eller delvis efterge kravet på återbetalning enligt första stycket.
Lag (2016:1066).
\part*{E FÖRMÅNER VID ÅLDERDOM}
\chapter*{53 Innehåll, definitioner och förklaringar}
\subsection*{1 §}
\paragraph*{}
I avdelning E finns bestämmelser om socialförsäkringsförmåner vid ålderdom.
\subsection*{2 §}
\paragraph*{}
Förmåner vid ålderdom enligt denna avdelning är
\newline 1. allmän ålderspension i form av
a) inkomstgrundad ålderspension, och
b) garantipension,
\newline 2. särskilt pensionstillägg som tillägg till allmän ålderspension,
\newline 3. äldreförsörjningsstöd som tillägg till eller i stället för allmän ålderspension, och
\newline 4. inkomstpensionstillägg som tillägg till inkomstgrundad ålderspension.
Lag (2020:1239).
\subsection*{3 §}
\paragraph*{}
I detta kapitel finns inledande bestämmelser om förmåner vid ålderdom.
\paragraph*{}
Vidare finns
\newline - övergripande bestämmelser om allmän ålderspension i 54-56 kap.,
\newline - bestämmelser om inkomstgrundad ålderspension i 57-64 kap.,
\newline - bestämmelser om garantipension i 65-67 kap.,
\newline - vissa gemensamma bestämmelser om allmän ålderspension i 68-71 kap., och
\newline - bestämmelser om särskilda förmåner vid ålderdom i 72-74 a kap.
Lag (2020:1239).
\subsection*{4 §}
\paragraph*{}
En förmån enligt denna avdelning lämnas endast till den som har ett gällande försäkringsskydd för förmånen enligt 4-6 kap.
\paragraph*{}
Bestämmelser om anmälan och ansökan samt vissa gemensamma bestämmelser om förmåner och handläggning finns i 104-117 kap. (avdelning H).
\subsection*{5 §}
\paragraph*{}
Ärenden som avser förmåner enligt denna avdelning handläggs av Pensionsmyndigheten. Ärenden om pensionsgrundande inkomst handläggs dock av Skatteverket.
\paragraph*{}
Bestämmelserna i första stycket gäller inte om något annat följer av annan bestämmelse i denna balk eller annan författning.
\chapter*{54 Innehåll}
\subsection*{1 §}
\paragraph*{}
I denna underavdelning finns
\newline - allmänna bestämmelser om allmän ålderspension i 55 kap., och
\newline - bestämmelser om uttag av allmän ålderspension m.m. i 56 kap.
\chapter*{55 Allmänna bestämmelser om allmän ålderspension}
\subsection*{1 §}
\paragraph*{}
I detta kapitel finns bestämmelser om
\newline - inkomstgrundad ålderspension i 2-7 §§, och
\newline - garantipension i 8-10 §§.
\subsection*{2 §}
\paragraph*{}
Försäkringen för inkomstgrundad ålderspension består av ett fördelningssystem och ett premiepensionssystem.
\subsection*{3 §}
\paragraph*{}
Inkomstgrundad ålderspension från fördelningssystemet lämnas som inkomstpension eller tilläggspension.
\paragraph*{}
Inkomstgrundad ålderspension från premiepensionssystemet lämnas som premiepension.
\subsection*{4 §}
\paragraph*{}
Den myndighet som regeringen bestämmer ska för varje år upprätta en redovisning av det inkomstgrundade ålderspensionssystemets finansiella ställning och utveckling.
Försäkringskassan, Pensionsmyndigheten samt Första-Fjärde och Sjätte AP-fonderna ska lämna de uppgifter som behövs för detta till myndigheten.
\paragraph*{}
Vem kan få inkomstpension, tilläggspension och premiepension?
\subsection*{5 §}
\paragraph*{}
En försäkrad som är född 1937 eller tidigare kan få inkomstgrundad ålderspension i form av tilläggspension.
\subsection*{6 §}
\paragraph*{}
En försäkrad som är född 1938 eller senare kan få inkomstgrundad ålderspension i form av inkomstpension och premiepension. En försäkrad som är född något av åren 1938-1953 kan dessutom få inkomstgrundad ålderspension i form av tilläggspension.
\subsection*{7 §}
\paragraph*{}
Pensionsrätt för inkomstgrundad ålderspension i form av premiepension kan överföras till den försäkrades make enligt bestämmelserna i 61 kap. 11-16 §§.
\paragraph*{}
I 89, 91 och 92 kap. (avdelning F) finns bestämmelser om att premiepension kan lämnas även som efterlevandeskydd.
\subsection*{8 §}
\paragraph*{}
Garantipension är grundskyddet i den allmänna ålderspensionen.
\subsection*{9 §}
\paragraph*{}
Garantipension för försäkrade födda 1937 eller tidigare kan lämnas som kompensation för bortfall av följande förmåner enligt äldre lagstiftning:
\newline - folkpension i form av ålderspension,
\newline - pensionstillskott, och
\newline - särskilt grundavdrag för folkpensionärer vid inkomsttaxeringen.
\subsection*{10 §}
\paragraph*{}
Garantipension för försäkrade födda 1938 eller senare är beroende av försäkringstid och kan lämnas till
\newline - den som saknar inkomstgrundad ålderspension, och
\newline - den vars inkomstgrundade ålderspension inte överstiger ett visst belopp.
\chapter*{56 Uttag av allmän ålderspension, m.m.}
\subsection*{1 §}
\paragraph*{}
I detta kapitel finns bestämmelser om
\newline - förmånsnivåer i 2 §,
\newline - förmånstiden i 3-9 §§,
\newline - återkallelse eller minskning av pensionsuttag i 10 §,
\newline - samordning av pensionsuttag i 11 och 12 §§, och
\newline - omräkning av allmän ålderspension i 13 §.
\subsection*{2 §}
\paragraph*{}
Uttag av inkomstgrundad ålderspension får begränsas till någon av följande förmånsnivåer: tre fjärdedelar, hälften eller en fjärdedel av hel pension.
\paragraph*{}
För garantipension finns särskilda bestämmelser om förmånsnivåer i 66 och 67 kap.
\subsection*{3 §}
\paragraph*{}
/Upphör att gälla U:2025-12-01/
Inkomstgrundad ålderspension lämnas tidigast från och med den månad då den försäkrade fyller 63 år.
\paragraph*{}
För garantipension finns särskilda bestämmelser om förmånstid i 66 och 67 kap.
Lag (2022:878).
\subsection*{3 §}
\paragraph*{}
/Träder i kraft I:2025-12-01/
Inkomstgrundad ålderspension lämnas tidigast från och med tre år före den månad då den försäkrade uppnår riktåldern för pension.
\paragraph*{}
För garantipension finns särskilda bestämmelser om förmånstid i 66 och 67 kap.
Lag (2022:879).
\subsection*{4 §}
\paragraph*{}
Allmän ålderspension lämnas från och med den månad som anges i ansökan, dock tidigast från och med den månad ansökan kom in till Pensionsmyndigheten.
\subsection*{4 a §}
\paragraph*{}
/Upphör att gälla U:2025-12-01/
Allmän ålderspension får lämnas utan ansökan från och med den månad den försäkrade fyller 66 år om den försäkrade fick hel sjukersättning omedelbart före den månaden.
Lag (2022:878).
\subsection*{4 a §}
\paragraph*{}
/Träder i kraft I:2025-12-01/
Allmän ålderspension får lämnas utan ansökan från och med den månad den försäkrade uppnår riktåldern för pension om den försäkrade fick hel sjukersättning omedelbart före den månaden.
Lag (2022:879).
\subsection*{5 §}
\paragraph*{}
/Upphör att gälla U:2025-12-01/
Allmän ålderspension får lämnas för högst tre månader före ansökningsmånaden om den försäkrade fick sjukersättning omedelbart före 66 års ålder.
\paragraph*{}
Bestämmelserna i första stycket tillämpas bara i fråga om pension som lämnas tidigast från och med den månad då den försäkrade fyller 66 år.
Lag (2022:878).
\subsection*{5 §}
\paragraph*{}
/Träder i kraft I:2025-12-01/
Allmän ålderspension får lämnas för högst tre månader före ansökningsmånaden om den försäkrade fick sjukersättning omedelbart före det att han eller hon uppnår riktåldern för pension.
\paragraph*{}
Bestämmelserna i första stycket tillämpas bara i fråga om pension som lämnas tidigast från och med den månad då den försäkrade uppnår riktåldern för pension.
Lag (2022:879).
\subsection*{6 §}
\paragraph*{}
En försäkrad som är född något av åren 1938-1953 och som får inkomstpension utan att ha rätt till tilläggspension ska, utan att behöva ansöka särskilt, få tilläggspension från och med den månad då han eller hon har fått rätt till sådan pension.
\subsection*{7 §}
\paragraph*{}
Allmän ålderspension lämnas till och med den månad då rätten till pensionen upphör.
\subsection*{8 §}
\paragraph*{}
/Upphör att gälla U:2025-12-01/
Pensionsmyndigheten ska utreda om den försäkrade vill ta ut allmän ålderspension om han eller hon inte har ansökt om sådan pension månaden före den då han eller hon fyller 66 år och för den månaden har fått sjukersättning. Detta gäller inte i de fall 4 a § tillämpas.
Lag (2022:878).
\subsection*{8 §}
\paragraph*{}
/Träder i kraft I:2025-12-01/
Pensionsmyndigheten ska utreda om den försäkrade vill ta ut allmän ålderspension om han eller hon inte har ansökt om sådan pension månaden före den då han eller hon uppnår riktåldern för pension och för den månaden har fått sjukersättning. Detta gäller inte i de fall 4 a § tillämpas.
Lag (2022:879).
\subsection*{9 §}
\paragraph*{}
Det som föreskrivs i 4 och 6 §§ gäller också i fråga om ökat uttag av allmän ålderspension.
Lag (2018:772).
\subsection*{10 §}
\paragraph*{}
Om den försäkrade vill minska eller helt återkalla uttaget av allmän ålderspension, ska han eller hon skriftligen anmäla detta till Pensionsmyndigheten.
\paragraph*{}
En anmälan om minskning eller hel återkallelse av uttag av allmän ålderspension gäller från och med den månad som anges i anmälan, dock tidigast från och med månaden efter den månad då anmälan kom in till Pensionsmyndigheten.
\subsection*{11 §}
\paragraph*{}
En försäkrad som är född under något av åren 1938-1953 och som har rätt till inkomstpension och tilläggspension får ta ut pension endast om han eller hon tar ut båda pensionerna i samma utsträckning.
\paragraph*{}
Återkallelse eller minskning av uttag av inkomstpension och tilläggspension gäller endast om anmälan avser båda pensionerna i samma utsträckning.
\subsection*{12 §}
\paragraph*{}
För garantipension finns särskilda bestämmelser om samordning i 66 och 67 kap.
\subsection*{13 §}
\paragraph*{}
Allmän ålderspension ska räknas om från och med månaden efter den då anledning till omräkning uppkom.
\chapter*{57 Innehåll och definitioner}
\subsection*{1 §}
\paragraph*{}
I denna underavdelning finns allmänna bestämmelser om inkomstgrundad ålderspension i 58 kap.
\paragraph*{}
Vidare finns bestämmelser om
\newline - pensionsgrundande inkomst i 59 kap.,
\newline - pensionsgrundande belopp i 60 kap.,
\newline - pensionsrätt och pensionspoäng i 61 kap.,
\newline - inkomstpension i 62 kap.,
\newline - tilläggspension i 63 kap., och
\newline - premiepension i 64 kap.
\subsection*{2 §}
\paragraph*{}
När det gäller inkomstgrundad ålderspension avses med
\newline 1. intjänandeår: det kalenderår den försäkrade har haft en inkomst som är pensionsgrundande respektive det kalenderår för vilket ett pensionsgrundande belopp ska fastställas, och
\newline 2. fastställelseår: kalenderåret efter intjänandeåret.
\paragraph*{}
Om inkomster under beskattningsår som inte sammanfaller med kalenderår finns bestämmelser i 59 kap. 33 § andra stycket.
\chapter*{58 Allmänna bestämmelser om inkomstgrundad ålderspension}
\subsection*{1 §}
\paragraph*{}
I detta kapitel finns grundläggande bestämmelser i 2-5 §§.
\paragraph*{}
Vidare finns bestämmelser om
\newline - underrättelse om beslut i 9 §, och
\newline - indexering, balansering och anslutande definitioner i 10-28 §§.
\subsection*{2 §}
\paragraph*{}
Inkomstpension ska beräknas utifrån de pensionsrätter som fastställts för den försäkrade.
\paragraph*{}
Inkomstpensionens storlek är också beroende av den allmänna inkomstutvecklingen.
\subsection*{3 §}
\paragraph*{}
Premiepension ska beräknas utifrån de pensionsrätter som fastställts för den försäkrade.
\paragraph*{}
Premiepensionens storlek är också beroende av värdeutvecklingen på de medel som fonderats i premiepensionssystemet.
\subsection*{4 §}
\paragraph*{}
Tilläggspensionen ska beräknas utifrån de pensionspoäng som fastställts för den försäkrade.
\paragraph*{}
Tilläggspensionens storlek är också beroende av den allmänna inkomstutvecklingen.
Lag (2014:470).
\subsection*{5 §}
\paragraph*{}
Pensionsrätt och pensionspoäng fastställs årligen och grundas på den försäkrades pensionsunderlag. Detta består av summan av
\newline - fastställd pensionsgrundande inkomst (PGI) och
\newline - fastställda pensionsgrundande belopp (PGB).
\subsection*{9 §}
\paragraph*{}
Om den försäkrade har debiterats slutlig skatt, ska underrättelse om Skatteverkets beslut om pensionsgrundande inkomst enligt 59 kap. anges på beskedet om slutlig skatt. I annat fall ska underrättelse ske genom särskilt besked senast den 15 december fastställelseåret.
\paragraph*{}
Den försäkrade ska skriftligen eller, om den försäkrade begär det, på något annat sätt underrättas om Pensionsmyndighetens beslut om pensionsgrundande belopp, pensionsrätt och pensionspoäng enligt 60 och 61 kap. senast den 31 mars året efter fastställelseåret.
\paragraph*{}
Det ska framgå av en underrättelse enligt första eller andra stycket hur man begär omprövning av beslutet. En underrättelse behöver inte sändas till den som inte är bosatt i Sverige och vars adress är okänd.
Lag (2013:747).
\subsection*{10 §}
\paragraph*{}
Vissa beräkningar som anges i denna balk ska grundas på ett inkomstindex som beräknas för varje år. Inkomstindexet ska visa den allmänna inkomstutvecklingen.
\paragraph*{}
Beräkningarna ska grundas på ett inkomstindex för 1999 om 100,00.
\subsection*{11 §}
\paragraph*{}
/Upphör att gälla U:2025-12-01/
Inkomstindex ska visa den relativa förändringen av genomsnittet av de årliga pensionsgrundande inkomsterna, efter avdrag för allmän pensionsavgift, för försäkrade som under beskattningsåret har fyllt minst 16 år och högst 65 år. Vid beräkningen tillämpas inte 59 kap. 4 § andra stycket.
Lag (2022:878).
\subsection*{11 §}
\paragraph*{}
/Träder i kraft I:2025-12-01/
Inkomstindex ska visa den relativa förändringen av genomsnittet av de årliga pensionsgrundande inkomsterna, efter avdrag för allmän pensionsavgift, för försäkrade som under beskattningsåret har fyllt minst 16 år men ännu inte har uppnått riktåldern för pension. Vid beräkningen tillämpas inte 59 kap. 4 § andra stycket.
\paragraph*{}
När en gällande riktålder ändras ska den nya riktåldern tillämpas första gången vid beräkningen av det inkomstindex som ska gälla två år efter det år riktåldern ändrades.
Lag (2022:879).
\subsection*{12 §}
\paragraph*{}
Förändringen av indextalet mellan två på varandra följande år ska motsvara den beräknade årliga relativa förändringen av de inkomster som anges i 11 § mellan det andra året före det år som indexet avser och året därefter.
Lag (2015:676).
\subsection*{13 §}
\paragraph*{}
Regeringen eller den myndighet som regeringen bestämmer meddelar närmare föreskrifter om inkomstindexet.
\subsection*{14 §}
\paragraph*{}
Vissa beräkningar som anges i denna balk ska grundas på ett balanstal och ett dämpat balanstal. Balanstalet beräknas enligt andra stycket och 15-20 §§. Det dämpade balanstalet beräknas enligt 20 a §.
\paragraph*{}
Balanstalet beräknas för varje år och ska visa kvoten mellan
\newline - summan av fördelningssystemets avgiftstillgång och värdet av tillgångarna hos Första-Fjärde och Sjätte AP-fonderna och
\newline - fördelningssystemets pensionsskuld vid utgången av det andra året före det år balanstalet avser.
\paragraph*{}
Balanstalet ska avrundas till fyra decimaler.
Lag (2015:676).
\subsection*{15 §}
\paragraph*{}
Med avgiftstillgång avses produkten av
\newline - avgiftsinkomsterna till fördelningssystemet och
\newline - medelvärdet av tiden i år räknat från det att en pensionsrätt tjänas in till det att den betalas ut i form av pension (omsättningstiden).
\paragraph*{}
Med värdet av tillgångarna hos Första-Fjärde och Sjätte AP-fonderna avses värdet av de redovisade marknadsvärdena av tillgångarna hos Första-Fjärde och Sjätte AP-fonderna vid utgången av det andra året före det år balanstalet avser.
Lag (2015:676).
\subsection*{16 §}
\paragraph*{}
Med pensionsskuld avses det totala pensionsåtagandet i fördelningssystemet.
\subsection*{17 §}
\paragraph*{}
Med avgiftsinkomsterna avses inkomsterna det andra året före det år balanstalet avser.
Lag (2015:676).
\subsection*{18 §}
\paragraph*{}
Har upphävts genom
lag (2015:676).
\subsection*{19 §}
\paragraph*{}
Med omsättningstid avses omsättningstiden för det tredje året före det år balanstalet avser.
Lag (2015:676).
\subsection*{20 §}
\paragraph*{}
Pensionsskulden beräknas för det andra året före det år balanstalet avser, som summan av
\newline 1. pensionsbehållningar, enligt 62 kap. 5-7 §§, utan beaktande av förändringen av inkomstindex mellan det andra året före det år balanstalet avser och året därefter,
\newline 2. det beräknade värdet av pensionsrätter för inkomstpension enligt 61 kap. 5-10 §§,
\newline 3. utbetalad pension för varje åldersgrupp i december multiplicerad med beräknat antal återstående utbetalningar av ett genomsnittligt pensionsbelopp för samma åldersgrupp, justerat med den räntefaktor som anges i 62 kap. 36 § och
\newline 4. det beräknade värdet av kommande utbetalningar av tilläggspension för dem som inte börjat ta ut sådan pension.
\paragraph*{}
Om ett balansindex har beräknats enligt 22-24 §§ för det år som värdet av pensionsrätter för inkomstpension ska beräknas enligt första stycket 2, ska pensionsrätterna beräknas på det sätt som anges i 62 kap. 5 § andra stycket.
\paragraph*{}
Om ett balansindex har beräknats enligt 22-24 §§ för året före det år balanstalet avser, ska det framräknade beloppet i första stycket 3 multipliceras med det dämpade balanstalet för detta år.
Lag (2015:676).
\subsection*{20 a §}
\paragraph*{}
Under en sådan period som anges i 22 § ska ett dämpat balanstal användas.
\paragraph*{}
Det dämpade balanstalet för ett år utgörs av summan av talet 1 och det tal som motsvarar en tredjedel av differensen mellan balanstalet som har fastställts för samma år och talet 1. Det dämpade balanstalet ska avrundas till fyra decimaler.
Lag (2015:676).
\subsection*{21 §}
\paragraph*{}
Regeringen eller den myndighet som regeringen bestämmer kan med stöd av 8 kap. 7 § regeringsformen meddela närmare föreskrifter om balanstalet och det dämpade balanstalet.
Lag (2015:676).
\subsection*{22 §}
\paragraph*{}
Om balanstalet för ett år understiger 1,0000 ska ett balansindex beräknas som ska användas vid vissa beräkningar enligt denna balk.
\paragraph*{}
Ett balansindex ska därefter beräknas för varje år fram till dess att det når minst samma värde som inkomstindex.
\subsection*{23 §}
\paragraph*{}
När balansindexet första gången bestäms för en sådan period som anges i 22 § ska balansindexet räknas fram som produkten av
\newline - det dämpade balanstalet och
\newline - inkomstindexet för samma år.
Lag (2015:676).
\subsection*{24 §}
\paragraph*{}
Efter det att balansindexet beräknats första gången enligt 23 § ska, för varje därpå följande år under perioden, det beräknade balansindexet multipliceras med kvoten mellan
\newline - inkomstindexet efter årsskiftet och
\newline - inkomstindexet före årsskiftet.
\paragraph*{}
Produkten ska därefter multipliceras med det dämpade balanstalet som ska gälla efter årsskiftet.
Lag (2015:676).
\subsection*{25 §}
\paragraph*{}
Regeringen eller den myndighet som regeringen bestämmer meddelar närmare föreskrifter om balansindexet.
\subsection*{26 §}
\paragraph*{}
Vissa beräkningar enligt denna balk ska grundas på ett inkomstbasbelopp som beräknas för varje år.
\subsection*{27 §}
\paragraph*{}
Inkomstbasbeloppet för ett visst år motsvarar produkten av
\newline - bastalet 43 313 och
\newline - kvoten mellan inkomstindexet för det år inkomstbasbeloppet ska bestämmas för och inkomstindexet för 2005.
\paragraph*{}
Det omräknade beloppet ska avrundas till närmaste hundratal kronor.
\subsection*{28 §}
\paragraph*{}
Regeringen eller den myndighet som regeringen bestämmer meddelar närmare föreskrifter om inkomstbasbeloppet.
\chapter*{59 Pensionsgrundande inkomst}
\subsection*{1 §}
\paragraph*{}
I detta kapitel finns allmänna bestämmelser om pensionsgrundande inkomst i 2-7 §§,
Vidare finns bestämmelser om
\newline - inkomst av anställning i 8-13 §§,
\newline - inkomst av annat förvärvsarbete i 14-21 §§,
\newline - undantag för vissa inkomster i 22-28 §§,
\newline - hur samordnade sociala förmåner är pensionsgrundande i 29-31 §§, och
\newline - beräkning av pensionsgrundande inkomst i 32-38 §§.
\subsection*{2 §}
\paragraph*{}
Pensionsgrundande inkomst beräknas av Skatteverket på inkomster som den försäkrade har haft och som är pensionsgrundande.
\subsection*{3 §}
\paragraph*{}
Följande inkomster är pensionsgrundande:
\newline - inkomster av anställning som avses i 8-13 §§, och
\newline - inkomster av annat förvärvsarbete som avses i 14-21 §§.
\paragraph*{}
Undantag från det som anges i första stycket finns i 22-31 §§.
\subsection*{4 §}
\paragraph*{}
Den försäkrades pensionsgrundande inkomst för ett år (intjänandeåret) utgörs av summan av hans eller hennes inkomster av anställning och inkomster av annat förvärvsarbete för det året.
Vid beräkningen ska det bortses från inkomster av anställning och inkomster av annat förvärvsarbete till den del summan av dessa överstiger 7,5 inkomstbasbelopp under intjänandeåret (intjänandetaket). Det bortses i första hand från inkomster av annat förvärvsarbete.
När fastställs pensionsgrundande inkomst?
\subsection*{5 §}
\paragraph*{}
Pensionsgrundande inkomst fastställs för varje år som en person har varit försäkrad och haft sådana inkomster här i landet som är pensionsgrundande.
\paragraph*{}
Pensionsgrundande inkomst fastställs dock endast om summan av de inkomster som är pensionsgrundande uppgår till minst 42,3 procent av det prisbasbelopp som gäller för intjänandeåret.
\subsection*{6 §}
\paragraph*{}
Pensionsgrundande inkomst fastställs inte för en försäkrad som är född 1937 eller tidigare.
Lag (2011:1075).
\subsection*{7 §}
\paragraph*{}
För det år när den försäkrade har avlidit fastställs pensionsgrundande inkomst endast om pensionsrätt för premiepension ska föras över till den avlidnes make för det året.
\subsection*{8 §}
\paragraph*{}
Som inkomst av anställning räknas lön eller annan ersättning i pengar eller annan avgiftspliktig förmån som en försäkrad har fått som arbetstagare i allmän eller enskild tjänst.
\paragraph*{}
Med lön likställs kostnadsersättning som inte undantas vid beräkning av skatteavdrag enligt 10 kap. 3 § 9 eller 10 skatteförfarandelagen (2011:1244).
Lag (2012:834).
\subsection*{9 §}
\paragraph*{}
Som inkomst av anställning räknas, även om mottagaren inte har varit anställd hos den som betalat ut ersättningen, följande:
\newline 1. ersättning i pengar eller annan avgiftspliktig förmån för utfört arbete, dock inte pension, och
\newline 2. tillfällig förvärvsinkomst av verksamhet som inte bedrivits självständigt.
\paragraph*{}
I fall som anges i första stycket likställs den som har utfört arbetet med en arbetstagare och den som betalat ut ersättningen med en arbetsgivare. När utbetalningen avser statlig ersättning för arbete i etableringsjobb likställs dock den för vars räkning arbetet har utförts med en arbetsgivare.
Lag (2020:475).
\subsection*{10 §}
\paragraph*{}
Som inkomst av anställning räknas skattepliktig intäkt i form av rabatt, bonus eller annan förmån som har lämnats på grund av kundtrohet eller liknande.
\paragraph*{}
Detta gäller dock endast om den som slutligt har stått för de kostnader som ligger till grund för förmånen är någon annan än den som är skattskyldig för förmånen.
\subsection*{11 §}
\paragraph*{}
I vissa fall ska ersättning som avses i 8-10 §§ inte anses som inkomst av anställning. Bestämmelser om detta finns i 15-21 §§.
\subsection*{12 §}
\paragraph*{}
Som inkomst av anställning räknas stipendium (Marie Curie-stipendium) som enligt 11 kap. 46 § inkomstskattelagen (1999:1229) ska tas upp som intäkt i inkomstslaget tjänst.
Detta gäller dock endast om stipendiet har betalats ut av
\newline 1. en fysisk person bosatt i Sverige, eller
\newline 2. en svensk juridisk person.
\paragraph*{}
Den som har betalat ut ett sådant stipendium anses som arbetsgivare.
\subsection*{13 §}
\paragraph*{}
Som inkomst av anställning räknas följande sociala förmåner:
\newline 1. Föräldrapenningsförmåner.
\newline 2. Omvårdnadsbidrag.
\newline 3. Ersättning från Försäkringskassan i form av sjuklönegaranti enligt 20 § lagen (1999:1047) om sjuklön.
\newline 4. Sjukpenning eller motsvarande ersättning enligt denna balk eller annan författning eller på grund av särskilt beslut av regeringen. Detta gäller i den utsträckning ersättningen har trätt i stället för en försäkrads inkomst som arbetstagare i allmän eller enskild tjänst.
\newline 5. Inkomstrelaterad sjukersättning och inkomstrelaterad aktivitetsersättning.
\newline 6. Livränta på grund av arbetsskada eller annan skada som avses i 41-44 kap.
\newline 7. Närståendepenning.
\newline 8. Dagpenning från arbetslöshetskassa.
\newline 9. Aktivitetsstöd till den som deltar i ett arbetsmarknadspolitiskt program.
\newline 10. Ersättning till deltagare i teckenspråksutbildning för vissa föräldrar (TUFF).
\newline 11. Dagpenning till totalförsvarspliktiga som tjänstgör enligt lagen (1994:1809) om totalförsvarsplikt och till andra som får dagpenning enligt de grunder som gäller för totalförsvarspliktiga.
\newline 12. Bidrag från Sveriges författarfond och Konstnärsnämnden i den utsträckning som regeringen föreskriver det.
\newline 13. Omställningsstudiebidrag enligt lagen (2022:856) om omställningsstudiestöd.
Lag (2022:858).
\subsection*{14 §}
\paragraph*{}
Som inkomst av annat förvärvsarbete räknas följande:
\newline 1. Inkomst av en sådan näringsverksamhet som enligt 2 kap. 23 § inkomstskattelagen (1999:1229) utgör aktiv näringsverksamhet.
\newline 2. Tillfällig förvärvsinkomst av självständigt bedriven verksamhet.
\newline 3. Ersättning för arbete för någon annans räkning i pengar eller andra skattepliktiga förmåner.
\newline 4. Sjukpenning eller motsvarande ersättning enligt denna balk eller annan författning eller på grund av särskilt beslut av regeringen. Detta gäller i den utsträckning ersättningen har trätt i stället för inkomst som anges i 1-3.
\newline 5. Stipendium som enligt 11 kap. 46 § inkomstskattelagen ska tas upp som intäkt i inkomstslaget tjänst.
\paragraph*{}
Det som anges i första stycket gäller endast i den utsträckning inkomsten inte ska räknas som inkomst av anställning. Vidare gäller det som anges i 15-21 §§.
\subsection*{15 §}
\paragraph*{}
Som inkomst av annat förvärvsarbete räknas sådan ersättning för utfört arbete som anges i 8-10 §§ och som har betalats ut till en mottagare som var godkänd för F-skatt när ersättningen bestämdes eller betalades ut.
\paragraph*{}
Första stycket gäller inte om ersättningen betalats ut från en semesterkassa.
Lag (2011:1434).
\subsection*{16 §}
\paragraph*{}
Om mottagaren hade ett godkännande för F-skatt med villkor enligt 9 kap. 3 § skatteförfarandelagen (2011:1244), räknas ersättningen som inkomst av annat förvärvsarbete bara om godkännandet har åberopats skriftligen.
Lag (2011:1434).
\subsection*{17 §}
\paragraph*{}
Den som i en handling som upprättas i samband med uppdraget har lämnat uppgift om godkännande för F-skatt ska anses ha haft ett sådant godkännande om handlingen även innehåller följande uppgifter:
\newline 1. utbetalarens och betalningsmottagarens namn och adress eller andra uppgifter som är godtagbara för identifiering, och 2. betalningsmottagarens personnummer, samordningsnummer eller organisationsnummer.
Lag (2011:1434).
\paragraph*{}
Uppgiften om godkännande för F-skatt gäller även som sådant skriftligt åberopande av godkännande som avses i 16 §.
\subsection*{18 §}
\paragraph*{}
Det som anges i 17 § gäller inte om den som har betalat ut ersättningen känt till att uppgiften om godkännande för F-skatt var oriktig.
Lag (2011:1434).
Viss ersättning från privatpersoner
\subsection*{19 §}
\paragraph*{}
Som inkomst av annat förvärvsarbete räknas ersättning för arbete som avses i 8-10 §§ om
\newline 1. utbetalaren var en fysisk person eller ett dödsbo,
\newline 2. den ersättning som betalats ut inte var en utgift i en näringsverksamhet som utbetalaren har bedrivit,
\newline 3. den sammanlagda ersättningen för arbete från samma utbetalare under beskattningsåret var mindre än 10 000 kronor,
\newline 4. utbetalaren och mottagaren inte hade träffat en överenskommelse om att ersättningen ska anses som inkomst av anställning, och
\newline 5. det inte var fråga om sådan ersättning som avses i 12 kap. 16 § föräldrabalken.
Lag (2011:1434).
\subsection*{20 §}
\paragraph*{}
Som inkomst av annat förvärvsarbete räknas ersättning för arbete om ersättningen har betalats ut från
\newline 1. ett handelsbolag till en delägare i handelsbolaget, eller
\newline 2. en europeisk ekonomisk intressegruppering till en medlem i intressegrupperingen.
\subsection*{21 §}
\paragraph*{}
Har upphävts genom
lag (2012:834).
\subsection*{22 §}
\paragraph*{}
Följande ersättningar och inkomster är inte pensionsgrundande:
\newline 1. ersättning som avses i 8-10 §§, om den kommer från en och samma arbetsgivare och är mindre än sammanlagt 1 000 kronor under ett år,
\newline 2. inkomst som avses i 14 § första stycket 1 eller 2, om den är mindre än 1 000 kronor under ett år,
\newline 3. ersättning som avses i 14 § första stycket 3 eller i 15, 16 eller 19 §, om ersättningen från den som arbetet utförts åt är mindre än 1 000 kronor under ett år, och
\newline 4. stipendium som avses i 12 § eller 14 § första stycket 5, om beloppet är mindre än 1 000 kronor under ett år.
\subsection*{23 §}
\paragraph*{}
Som pensionsgrundande inkomst räknas inte sådan intäkt som avses i 10 kap. 3 § 1-4 inkomstskattelagen (1999:1229).
\subsection*{24 §}
\paragraph*{}
Som pensionsgrundande inkomst av anställning räknas inte ersättning som en idrottsutövare får från en ideell förening som har till ändamål att främja idrott och som uppfyller kraven i 7 kap. 4-6 och 10 §§ inkomstskattelagen (1999:1229), om ersättningen från föreningen under ett år är mindre än hälften av det för året gällande prisbasbeloppet.
Lag (2013:949).
\subsection*{25 §}
\paragraph*{}
Som pensionsgrundande inkomst räknas inte följande ersättningar, i den utsträckning dessa utgör underlag för särskild löneskatt:
\newline 1. ersättning som anges i 1 § första stycket 1-5 och fjärde stycket lagen (1990:659) om särskild löneskatt på vissa förvärvsinkomster, och
\newline 2. ersättning enligt gruppsjukförsäkring eller trygghetsförsäkring vid arbetsskada enligt 2 § första stycket lagen om särskild löneskatt på vissa förvärvsinkomster.
\subsection*{26 §}
\paragraph*{}
Som pensionsgrundande inkomst räknas inte ersättning från en stiftelse som har till väsentligt ändamål att tillgodose ekonomiska intressen hos dem som är eller har varit anställda hos en arbetsgivare som har lämnat bidrag till stiftelsen (vinstandelsstiftelse), om följande förutsättningar är uppfyllda: - ersättningen avser en sådan anställd som omfattas av ändamålet med vinstandelsstiftelsen, - ersättningen avser inte betalning för arbete som den anställde utfört åt vinstandelsstiftelsen, och - de bidrag arbetsgivaren har lämnat till vinstandelsstiftelsen har varit avsedda att vara bundna under minst tre kalenderår och att på likartade villkor tillkomma en betydande andel av de anställda.
\paragraph*{}
Detta gäller även ersättning från en annan juridisk person med motsvarande ändamål som en vinstandelsstiftelse.
\subsection*{27 §}
\paragraph*{}
Om arbetsgivaren är ett fåmansföretag eller ett fåmanshandelsbolag gäller det som föreskrivs i 26 § inte ersättning som den juridiska personen har lämnat till företagsledare eller delägare i företaget eller till en person som är närstående till någon av dem.
\paragraph*{}
Med fåmansföretag, fåmanshandelsbolag, företagsledare och närstående avses detsamma som i inkomstskattelagen (1999:1229).
\subsection*{28 §}
\paragraph*{}
När pensionsgrundande inkomst beräknas ska det bortses från sådan ersättning från en vinstandelsstiftelse som härrör från bidrag som arbetsgivaren har lämnat under åren 1988-1991.
Hur är samordnade sociala förmåner pensionsgrundande?
\subsection*{29 §}
\paragraph*{}
Har Försäkringskassan eller en arbetslöshetskassa till en försäkrad för en viss månad betalat ut en ersättning som är pensionsgrundande antingen genom att den utgör pensionsgrundande inkomst, eller genom att det ska beräknas ett pensionsgrundande belopp för den enligt 60 kap. 7 § eller pensionspoäng enligt 61 kap. 20 §, och har den försäkrade senare för samma månad beviljats annan ersättning som är pensionsgrundande och samordnad med den ersättning som tidigare har betalats ut, gäller följande. Den senare beviljade ersättningen är pensionsgrundande endast till den del ersättningen avser tid från och med den månad då den har betalats ut. 30 § Vid tillämpning av 29 § ska det anses som om sjukersättning eller aktivitetsersättning har betalats ut före livränta, om det vid samma tidpunkt beviljas
\newline - inkomstrelaterad sjukersättning eller inkomstrelaterad aktivitetsersättning, och - livränta på grund av arbetsskada eller annan skada som avses i 41-44 kap.
\subsection*{31 §}
\paragraph*{}
Lämnas livränta enligt 41-44 kap. tillsammans med sjukersättning eller aktivitetsersättning är livräntan inte pensionsgrundande till den del den har samordnats med inkomstrelaterad sådan ersättning.
\subsection*{32 §}
\paragraph*{}
Inkomster av anställning och inkomster av annat förvärvsarbete som är pensionsgrundande ska beräknas enligt 33-38 §§ och var för sig avrundas till närmaste lägre hundratal kronor.
\subsection*{33 §}
\paragraph*{}
Till grund för beräkningen av den pensionsgrundande inkomsten ett visst intjänandeår läggs den försäkrades beslut om statlig inkomstskatt för det året.
\paragraph*{}
Om en inkomst som är pensionsgrundande har mottagits under ett beskattningsår som inte sammanfaller med kalenderår ska den inkomsten anses ha mottagits under kalenderåret närmast före det år då beslutet om slutlig skatt fattades.
Lag (2011:1434).
\subsection*{34 §}
\paragraph*{}
Pensionsgrundande inkomst av anställning, som den försäkrade enligt inkomstskattelagen (1999:1229) inte är skattskyldig för i Sverige, bestäms med ledning av en kontrolluppgift eller uppgifter i en arbetsgivardeklaration enligt skatteförfarandelagen (2011:1244). Om underlaget för arbetsgivaravgifter på inkomsten beräknats enligt 2 kap. 25 a § socialavgiftslagen (2000:980), ska den pensionsgrundande inkomsten bestämmas med ledning av detta underlag.
Pensionsgrundande inkomst av annat förvärvsarbete, som den försäkrade enligt inkomstskattelagen inte är skattskyldig för i Sverige, bestäms med ledning av de uppgifter som den försäkrade har lämnat i en sådan inkomstdeklaration som avses i 30 kap. 1 och 5 §§ skatteförfarandelagen.
Lag (2012:834).
\subsection*{35 §}
\paragraph*{}
Skattepliktiga förmåner ska tas upp till ett värde som bestäms i enlighet med 11 kap. 4-11 §§ skatteförfarandelagen (2011:1244).
\paragraph*{}
Avvikelse får ske från det förmånsvärde som Skatteverket bestämt enligt 2 kap. 10 a första stycket 1 och andra stycket samt 10 b-10 d §§ socialavgiftslagen (2000:980) om det finns skäl till detta.
Lag (2011:1434).
\subsection*{36 §}
\paragraph*{}
När den försäkrades pensionsgrundande inkomster av anställning beräknas ska avdrag göras för kostnader som han eller hon har haft i arbetet. Detta gäller i den utsträckning den försäkrades kostnader, minskade med erhållen kostnadsersättning, överstiger 5 000 kronor.
\paragraph*{}
Vid beräkningen ska avdrag också göras för debiterad allmän pensionsavgift som den försäkrade ska betala för dessa inkomster enligt lagen (1994:1744) om allmän pensionsavgift.
\paragraph*{}
Avdraget görs i första hand från inkomst enligt 13 §.
\subsection*{37 §}
\paragraph*{}
När den försäkrades pensionsgrundande inkomster av annat förvärvsarbete beräknas ska avdrag göras för debiterad allmän pensionsavgift som den försäkrade ska betala för dessa inkomster enligt lagen (1994:1744) om allmän pensionsavgift.
\paragraph*{}
Avdraget görs i första hand från inkomst enligt 14 § första stycket 4.
\subsection*{38 §}
\paragraph*{}
När pensionsgrundande inkomster av annat förvärvsarbete beräknas ska underskott i en förvärvskälla inte dras av från inkomst av en annan förvärvskälla.
\chapter*{60 Pensionsgrundande belopp}
\subsection*{1 §}
\paragraph*{}
I detta kapitel finns inledande bestämmelser i 2-6 §§.
\paragraph*{}
Vidare finns bestämmelser om
\newline - pensionsgrundande belopp för sjukersättning eller aktivitetsersättning i 7-16 §§,
\newline - pensionsgrundande belopp för plikttjänstgöring i 17 och 18 §§,
\newline - pensionsgrundande belopp för studier i 19 och 20 §§,
\newline - pensionsgrundande belopp för barnår i 21-36 §§,
\newline - överlåtelse av rätten till pensionsgrundande belopp för barnår i 37-41 §§,
\newline - beräkning av pensionsgrundande belopp för barnår i 42-54 §§, och
\newline - anmälan om tillgodoräknande av pensionsgrundande belopp för barnår i 55-59 §§.
\subsection*{2 §}
\paragraph*{}
Pensionsgrundande belopp beräknas av Pensionsmyndigheten som kompensation för att försäkrade som anges i 3 § kan antas ha gått miste om inkomster som är pensionsgrundande.
\subsection*{3 §}
\paragraph*{}
Pensionsgrundande belopp beräknas för en försäkrad som
\newline 1. har fått inkomstrelaterad sjukersättning eller inkomstrelaterad aktivitetsersättning (pensionsgrundande belopp för sjukersättning eller aktivitetsersättning),
\newline 2. har fullgjort plikttjänstgöring (pensionsgrundande belopp för plikttjänstgöring),
\newline 3. har studerat med studiemedel (pensionsgrundande belopp för studier), eller
\newline 4. har varit småbarnsförälder (pensionsgrundande belopp för barnår).
\paragraph*{}
När det gäller pensionsgrundande belopp som avses i första stycket 2-4 behandlas den som är bosatt i Sverige som försäkrad även om han eller hon under året inte uppfyller förutsättningarna i 6 kap. I 61 kap. 7 och 8 §§ samt 62 kap. 38-41 §§ finns bestämmelser om ett förvärvsvillkor för pensionsrätt för premiepension och för beräkning av inkomstpension på grundval av sådana belopp.
\subsection*{4 §}
\paragraph*{}
Pensionsgrundande belopp ska fastställas för varje år (intjänandeår) som en försäkrad uppfyller förutsättningarna för att tillgodoräknas ett sådant belopp.
\paragraph*{}
Varje pensionsgrundande belopp beräknas och avrundas var för sig till närmaste lägre hundratal kronor, om inte annat följer av bestämmelsen i 5 §.
\subsection*{5 §}
\paragraph*{}
Pensionsgrundande belopp ska fastställas endast i den utsträckning summan av dessa belopp och den försäkrades pensionsgrundande inkomst inte överstiger 7,5 inkomstbasbelopp för intjänandeåret (intjänandetaket). Vid beräkningen ska det bortses från pensionsgrundande belopp i följande turordning:
\newline 1. pensionsgrundande belopp för barnår,
\newline 2. pensionsgrundande belopp för studier,
\newline 3. pensionsgrundande belopp för plikttjänstgöring, och
\newline 4. pensionsgrundande belopp för sjukersättning eller aktivitetsersättning.
\subsection*{6 §}
\paragraph*{}
Pensionsgrundande belopp fastställs inte
\newline 1. för år före det år då den försäkrade fyllt 16 år,
\newline 2. för det år då den försäkrade har avlidit, eller
\newline 3. för försäkrad som är född 1937 eller tidigare.
\subsection*{7 §}
\paragraph*{}
Pensionsgrundande belopp för sjukersättning eller aktivitetsersättning tillgodoräknas en försäkrad för ett år om han eller hon för någon del av året har fått inkomstrelaterad sjukersättning eller inkomstrelaterad aktivitetsersättning.
\paragraph*{}
Det som anges i första stycket gäller endast om inte något annat följer av 14 eller 15 §.
\subsection*{8 §}
\paragraph*{}
Bestämmelser om att inkomstrelaterad sjukersättning och inkomstrelaterad aktivitetsersättning är pensionsgrundande finns i 59 kap. 3 § och 13 § 5.
\subsection*{9 §}
\paragraph*{}
Vid beräkningen av pensionsgrundande belopp för sjukersättning eller aktivitetsersättning ska först den försäkrades reducerade antagandeinkomst räknas fram.
\paragraph*{}
Den reducerade antagandeinkomsten ska motsvara 93 procent av den antagandeinkomst som ligger till grund för den inkomstrelaterade sjukersättningen eller inkomstrelaterade aktivitetsersättningen.
\subsection*{10 §}
\paragraph*{}
När den reducerade antagandeinkomsten beräknats ska det pensionsgrundande beloppet räknas fram för varje månad som inkomstrelaterad sjukersättning eller aktivitetsersättning har lämnats.
\paragraph*{}
När den försäkrade har fått hel inkomstrelaterad sjukersättning eller hel inkomstrelaterad aktivitetsersättning, ska det pensionsgrundande beloppet beräknas som differensen mellan
\newline - en tolftedel av den reducerade antagandeinkomsten och
\newline - den pensionsgrundande inkomsten av den inkomstrelaterade sjukersättningen eller inkomstrelaterade aktivitetsersättningen efter avdrag enligt 59 kap. 36 §.
\subsection*{11 §}
\paragraph*{}
Har den försäkrade för en månad fått en fjärdedels, halv, två tredjedels eller tre fjärdedels inkomstrelaterad sjukersättning eller inkomstrelaterad aktivitetsersättning används vid beräkning enligt 9 och 10 §§ så stor andel av antagandeinkomsten som svarar mot nivån av inkomstrelaterad sjukersättning eller inkomstrelaterad aktivitetsersättning.
\subsection*{12 §}
\paragraph*{}
Om det, enligt bestämmelserna i 106 kap. 16-19 §§, för hel månad betalas ut endast en del av den inkomstrelaterade sjukersättningen eller inkomstrelaterade aktivitetsersättningen, används vid beräkningen enligt 9 och 10 §§ endast så stor andel av antagandeinkomsten som svarar mot den del som betalats ut.
\subsection*{13 §}
\paragraph*{}
Om en inkomstrelaterad sjukersättning har minskats enligt bestämmelserna i 37 kap. 6 och 7 §§ ska det pensionsgrundande beloppet enligt 9-12 §§ beräknas med bortseende från sådan minskning.
\subsection*{14 §}
\paragraph*{}
Om inkomstrelaterad sjukersättning eller inkomstrelaterad aktivitetsersättning, med tillämpning av 42 kap. 2-4 §§ har varit samordnad med pensionsgrundande livränta gäller följande. Den del av antagandeinkomsten som avses i 9 och 10 §§ tillgodoräknas den försäkrade som pensionsgrundande belopp endast om och i den utsträckning den överstiger den försäkrades livränta efter samordning.
\paragraph*{}
Om endast en del av den inkomstrelaterade sjukersättningen eller inkomstrelaterade aktivitetsersättningen har varit samordnad med livräntan gäller följande. Den andel av det belopp som svarar mot den samordnade delen av den inkomstrelaterade sjukersättningen eller inkomstrelaterade aktivitetsersättningen tillgodoräknas den försäkrade som pensionsgrundande belopp endast om och i den utsträckning andelen överstiger den försäkrades livränta efter samordning.
\subsection*{15 §}
\paragraph*{}
Om en pensionsgrundande livränta har minskats med tillämpning av 37 kap. 6 och 7 §§ ska det pensionsgrundande beloppet enligt 14 § beräknas med bortseende från sådan minskning.
\subsection*{16 §}
\paragraph*{}
Om en pensionsgrundande ersättning har betalats ut av Försäkringskassan eller en arbetslöshetskassa till en försäkrad och den försäkrade senare för samma månad beviljas inkomstrelaterad sjukersättning eller inkomstrelaterad aktivitetsersättning som är samordnad med den tidigare utbetalade ersättningen, gäller följande. Den försäkrade tillgodoräknas pensionsgrundande belopp med anledning av den inkomstrelaterade sjukersättningen eller inkomstrelaterade aktivitetsersättningen endast till den del ersättningen avser tid från och med den månad då den har betalats ut.
\subsection*{17 §}
\paragraph*{}
Pensionsgrundande belopp för plikttjänstgöring ska tillgodoräknas en försäkrad för ett år om han eller hon under någon del av året har genomgått grundutbildning enligt lagen (1994:1809) om totalförsvarsplikt.
\paragraph*{}
Pensionsgrundande belopp tillgodoräknas dock den försäkrade endast om tjänstgöringen har pågått sammanlagt minst 120 dagar utan att grundutbildningen avbrutits. Hänsyn tas enbart till dagar som den försäkrade har fått dagersättning för enligt lagen om totalförsvarsplikt.
\subsection*{18 §}
\paragraph*{}
/Upphör att gälla U:2025-12-01/
Det pensionsgrundande beloppet för plikttjänstgöring beräknas för de dagar under året som tjänstgöringen har pågått och som den försäkrade har fått dagersättning för enligt lagen (1994:1809) om totalförsvarsplikt. Beloppet per dag beräknas som kvoten mellan
\newline - hälften av genomsnittet av samtliga pensionsgrundande inkomster som fastställts för intjänandeåret för samtliga försäkrade som under det året har fyllt högst 65 år och
\newline - 365.
\paragraph*{}
Beräkningen av den genomsnittliga summan av samtliga fastställda pensionsgrundande inkomster ska utgå från sådana inkomster som de var bestämda den 1 december fastställelseåret.
Lag (2022:878).
\subsection*{18 §}
\paragraph*{}
/Träder i kraft I:2025-12-01/
Det pensionsgrundande beloppet för plikttjänstgöring beräknas för de dagar under året som tjänstgöringen har pågått och som den försäkrade har fått dagersättning för enligt lagen (1994:1809) om totalförsvarsplikt. Beloppet per dag beräknas som kvoten mellan
\newline - hälften av genomsnittet av samtliga pensionsgrundande inkomster som fastställts för intjänandeåret för samtliga försäkrade som under det året inte har uppnått riktåldern för pension och
\newline - 365.
\paragraph*{}
Beräkningen av den genomsnittliga summan av samtliga fastställda pensionsgrundande inkomster ska utgå från sådana inkomster som de var bestämda den 1 december fastställelseåret.
Lag (2022:879).
\subsection*{19 §}
\paragraph*{}
Pensionsgrundande belopp för studier ska tillgodoräknas en försäkrad för ett år om han eller hon under någon del av året har fått studiemedel i form av studiebidrag enligt 3 kap. studiestödslagen (1999:1395).
\paragraph*{}
Pensionsgrundande belopp tillgodoräknas dock inte för den del som avser tilläggsbidrag.
\subsection*{20 §}
\paragraph*{}
Ett pensionsgrundande belopp för studier ska motsvara 138 procent av det studiebidrag som den försäkrade har fått under året.
\subsection*{21 §}
\paragraph*{}
Pensionsgrundande belopp för barnår ska tillgodoräknas en försäkrad för de år han eller hon har varit småbarnsförälder om villkoren i 30-36 §§ är uppfyllda.
\subsection*{22 §}
\paragraph*{}
Pensionsgrundande belopp för barnår ska för samma år och barn tillgodoräknas endast en av barnets föräldrar. Detsamma gäller om två föräldrar har eller har haft fler än ett gemensamt barn för vilka pensionsgrundande belopp för samma år kan tillgodoräknas någon av föräldrarna.
\paragraph*{}
En förälder får för samma år inte tillgodoräknas mer än ett pensionsgrundande belopp för barnår.
\subsection*{23 §}
\paragraph*{}
Om föräldrarna var för sig kan tillgodoräknas ett pensionsgrundande belopp för barnår får de, genom anmälan till Pensionsmyndigheten, ange vem av dem som ska tillgodoräknas ett sådant belopp.
\subsection*{24 §}
\paragraph*{}
Om anmälan enligt 23 § inte görs ska det pensionsgrundande beloppet för barnår tillgodoräknas den av föräldrarna som, för det år beloppet avser, har lägst pensionsunderlag enligt 61 kap. 5 §.
\paragraph*{}
Har ingen av föräldrarna något pensionsunderlag för det året eller har de lika högt underlag, ska det pensionsgrundande beloppet tillgodoräknas barnets mor eller, om föräldrarna är av samma kön, den äldre av dem.
\subsection*{25 §}
\paragraph*{}
När 24 § tillämpas ska i pensionsunderlaget räknas in följande poster:
\newline 1. pensionsgrundande inkomst enligt 59 kap,
\newline 2. utlandsinkomster enligt 47 § första stycket 1 och 2,
\newline 3. pensionsgrundande belopp för sjukersättning eller aktivitetsersättning enligt 7 §,
\newline 4. pensionsgrundande belopp för plikttjänstgöring enligt 17 §, och
\newline 5. pensionsgrundande belopp för studier enligt 19 §.
\subsection*{26 §}
\paragraph*{}
Pensionsgrundande belopp för barnår får överlåtas enligt bestämmelserna i 37-41 §§.
\subsection*{27 §}
\paragraph*{}
Vid tillämpningen av bestämmelserna om pensionsgrundande belopp för barnår likställs särskilt förordnad vårdnadshavare eller blivande adoptivförälder med förälder.
\subsection*{28 §}
\paragraph*{}
Vid tillämpningen av bestämmelserna i 24 § likställs en kvinna som är särskilt förordnad vårdnadshavare eller blivande adoptivförälder med mor.
\subsection*{29 §}
\paragraph*{}
För en man som enligt 1 kap. föräldrabalken har ansetts vara far till ett barn eller vars faderskap har fastställts genom bekräftelse eller dom, men som genom senare dom som har vunnit laga kraft har förklarats inte vara far till barnet gäller följande. För tiden innan den sistnämnda domen har vunnit laga kraft ska mannen anses som barnets far vid tillämpning av bestämmelserna om pensionsgrundande belopp för barnår.
\paragraph*{}
Första stycket tillämpas också i fråga om föräldraskap enligt 1 kap. 9 § föräldrabalken.
\subsection*{30 §}
\paragraph*{}
/Upphör att gälla U:2025-12-01/
Pensionsgrundande belopp för barnår får tillgodoräknas en förälder endast om
\newline 1. föräldern har varit försäkrad och bosatt i Sverige hela intjänandeåret,
\newline 2. barnet har varit bosatt i Sverige hela intjänandeåret eller, om barnet inte har levt hela det året, den del av året barnet levt,
\newline 3. föräldern har fyllt högst 65 år under intjänandeåret,
\newline 4. föräldern har haft vårdnaden om barnet minst halva intjänandeåret, och
\newline 5. föräldern under minst halva intjänandeåret har bott tillsammans med barnet.
Lag (2022:878).
\subsection*{30 §}
\paragraph*{}
/Träder i kraft I:2025-12-01/
Pensionsgrundande belopp för barnår får tillgodoräknas en förälder endast om
\newline 1. föräldern har varit försäkrad och bosatt i Sverige hela intjänandeåret,
\newline 2. barnet har varit bosatt i Sverige hela intjänandeåret eller, om barnet inte har levt hela det året, den del av året barnet levt,
\newline 3. föräldern inte har uppnått riktåldern för pension under intjänandeåret,
\newline 4. föräldern har haft vårdnaden om barnet minst halva intjänandeåret, och
\newline 5. föräldern under minst halva intjänandeåret har bott tillsammans med barnet.
Lag (2022:879).
\subsection*{31 §}
\paragraph*{}
Den tid under vilken en försäkrad har varit blivande adoptivförälder likställs med tid under vilken en försäkrad har haft vårdnaden om ett barn. I ett sådant fall ska det som föreskrivs i 30 § 2 om barnets bosättning i Sverige gälla endast från och med den tidpunkt då barnet först anlände till Sverige.
\subsection*{32 §}
\paragraph*{}
Om barnet har avlidit under intjänandeåret ska det anses som om föräldern även under resten av året har haft vårdnaden om och bott tillsammans med barnet. Detta gäller dock endast om föräldern när barnet avled hade vårdnaden om och bodde tillsammans med barnet.
\subsection*{33 §}
\paragraph*{}
Om barnet har avlidit under samma år som det fötts och föräldern då hade vårdnaden om och var bosatt tillsammans med barnet, ska ett pensionsgrundande belopp för barnår tillgodoräknas föräldern för det året.
\paragraph*{}
Detta gäller
\newline - även om de förutsättningar som anges i 30 § 4 och 5 inte är uppfyllda, och
\newline - med bortseende från det som föreskrivs i 34 och 35 §§.
\subsection*{34 §}
\paragraph*{}
Pensionsgrundande belopp för barnår får för respektive år tillgodoräknas enligt följande:
\newline 1. Om barnet är fött under någon av månaderna januari-juni tillgodoräknas pensionsgrundande belopp från och med födelseåret till och med det år barnet fyller tre år.
\newline 2. Om barnet är fött under någon av månaderna juli-december tillgodoräknas pensionsgrundande belopp från och med det år barnet fyller ett år till och med det år barnet fyller fyra år.
\subsection*{35 §}
\paragraph*{}
När det gäller blivande adoptivförälder ska den tidpunkt då den försäkrade fick barnet i sin vård anses som tidpunkten för barnets födelse, dock inte vid beräkning av barnets ålder enligt andra stycket.
\paragraph*{}
För barn som avses i första stycket gäller dessutom följande:
\newline 1. Har barnet tagits emot under någon av månaderna januari- juni får ett pensionsgrundande belopp inte tillgodoräknas för år efter det år då barnet fyller nio år.
\newline 2. Har barnet tagits emot under någon av månaderna juli- december får ett pensionsgrundande belopp inte tillgodoräknas för år efter det år då barnet fyller tio år.
\subsection*{36 §}
\paragraph*{}
Pensionsgrundande belopp enligt 35 § får inte tillgodoräknas för mer än fyra år per barn. Om det finns synnerliga skäl får dock pensionsgrundande belopp tillgodoräknas för längre tid.
\subsection*{37 §}
\paragraph*{}
Överlåtelse av rätten att tillgodoräknas ett pensionsgrundande belopp för barnår ska göras genom anmälan till Pensionsmyndigheten.
\subsection*{38 §}
\paragraph*{}
Om en förälder har överlåtit rätten att tillgodoräknas pensionsgrundande belopp för barnår enligt 37 §, har han eller hon inte rätt att för samma år tillgodoräknas ett pensionsgrundande belopp för barnår för helsyskon till det barn som överlåtelsen av rätt till pensionsgrundande belopp avser.
\subsection*{39 §}
\paragraph*{}
Om endast en av föräldrarna enligt 30-36 §§ kan tillgodoräknas ett pensionsgrundande belopp för barnår för ett visst barn och år, får den föräldern överlåta rätten att få tillgodoräkna sig ett sådant belopp till den andra föräldern. Detta gäller endast om skälet till att den andra föräldern inte kan tillgodoräknas beloppet är att han eller hon inte har bott tillsammans med barnet under minst halva året.
\paragraph*{}
Som villkor för överlåtelse enligt första stycket gäller vidare att den förälder som rätten överlåts till
\newline 1. under året har bott tillsammans med barnet i inte obetydlig omfattning, och
\newline 2. inte enligt 30-36 §§ för samma år kan tillgodoräknas ett pensionsgrundande belopp för ett annat barn.
\subsection*{40 §}
\paragraph*{}
Har en förälder avlidit under det år för vilket pensionsgrundande belopp för barnår ska tillgodoräknas eller under ett tidigare år gäller följande. Den andra föräldern får, om han eller hon enligt 30-36 §§ ensam kan tillgodoräknas ett pensionsgrundande belopp för barnår för ett visst barn och år, överlåta rätten att tillgodoräkna sig ett sådant belopp till någon annan försäkrad.
\subsection*{41 §}
\paragraph*{}
/Upphör att gälla U:2025-12-01/
Överlåtelse enligt 40 § får göras endast om mottagaren
\newline 1. under intjänandeåret var gift med eller hade gemensamt barn med överlåtaren,
\newline 2. stadigvarande sammanbodde med överlåtaren under intjänandeåret,
\newline 3. var bosatt i Sverige hela intjänandeåret,
\newline 4. fyllde högst 65 år under intjänandeåret,
\newline 5. under minst halva intjänandeåret bodde tillsammans med barnet, och
\newline 6. inte enligt 30-36 §§ för samma år kan tillgodoräknas ett pensionsgrundande belopp för ett annat barn.
Lag (2022:878).
\subsection*{41 §}
\paragraph*{}
/Träder i kraft I:2025-12-01/
Överlåtelse enligt 40 § får göras endast om mottagaren
\newline 1. under intjänandeåret var gift med eller hade gemensamt barn med överlåtaren,
\newline 2. stadigvarande sammanbodde med överlåtaren under intjänandeåret,
\newline 3. var bosatt i Sverige hela intjänandeåret,
\newline 4. inte har uppnått riktåldern för pension under intjänandeåret,
\newline 5. under minst halva intjänandeåret bodde tillsammans med barnet, och
\newline 6. inte enligt 30-36 §§ för samma år kan tillgodoräknas ett pensionsgrundande belopp för ett annat barn.
Lag (2022:879).
\subsection*{42 §}
\paragraph*{}
Pensionsgrundande belopp för barnår beräknas enligt någon av följande metoder:
\newline 1. beräkningsmetod 1 (individuell jämförelse),
\newline 2. beräkningsmetod 2 (generell jämförelse), och
\newline 3. beräkningsmetod 3 (enhetligt belopp).
\paragraph*{}
Den metod ska användas som ger det högsta pensionsgrundande beloppet för ett visst år.
\subsection*{43 §}
\paragraph*{}
Beräkningsmetod 1 innebär att en förälders individuella jämförelseinkomst och förälderns utfyllnadsinkomst jämförs.
\subsection*{44 §}
\paragraph*{}
Den individuella jämförelseinkomsten beräknas genom att summan av förälderns fastställda pensionsgrundande inkomst och pensionsgrundande belopp för sjukersättning eller aktivitetsersättning för året före barnets födelse räknas om med hänsyn till förändringar av prisbasbeloppet. Det omräknade beloppet ska avrundas till närmaste lägre hundratal kronor.
\subsection*{45 §}
\paragraph*{}
Om föräldern året före barnets födelse hade inkomster av annat förvärvsarbete och inte fullt ut eller i föreskriven tid betalat ålderspensionsavgift och allmän pensionsavgift, ska det vid beräkningen av den individuella jämförelseinkomsten bortses från den del av inkomsten som avgifter inte har betalats för.
\subsection*{46 §}
\paragraph*{}
Med utfyllnadsinkomst avses summan av förälderns pensionsgrundande inkomst och pensionsgrundande belopp för sjukersättning eller aktivitetsersättning, plikttjänstgöring och studier under intjänandeåret.
\subsection*{47 §}
\paragraph*{}
Med pensionsgrundande inkomst under intjänandeåret likställs ersättning, i annan form än pension, som inte är pensionsgrundande enligt denna balk och som en försäkrad under det året har fått för arbete
\newline 1. utfört utomlands, eller
\newline 2. vid en främmande makts beskickning eller lönade konsulat här i landet eller hos en arbetsgivare som tillhör sådan beskickning eller sådant konsulat.
\paragraph*{}
En försäkrad som under intjänandeåret har haft ersättning enligt första stycket och som för det året ska tillgodoräknas ett pensionsgrundande belopp för barnår är skyldig att upplysa Pensionsmyndigheten om ersättningen.
\subsection*{48 §}
\paragraph*{}
Enligt beräkningsmetod 1 ska förälderns pensionsgrundande belopp för barnår beräknas som differensen mellan förälderns
\newline - individuella jämförelseinkomst och
\newline - utfyllnadsinkomst.
\paragraph*{}
En sådan beräkning ska dock göras endast om förälderns utfyllnadsinkomst under intjänandeåret är mindre än förälderns individuella jämförelseinkomst.
\subsection*{49 §}
\paragraph*{}
Om föräldern uppfyller villkoren för att tillgodoräknas ett pensionsgrundande belopp för barnår för fler än ett barn, ska beräkningen enligt 48 § göras med avseende på det barn som ger föräldern den högsta individuella jämförelseinkomsten.
\subsection*{50 §}
\paragraph*{}
Om förälderns individuella jämförelseinkomst överstiger 7,5 inkomstbasbelopp (intjänandetaket) ska det pensionsgrundande beloppet för barnår beräknas som differensen mellan
\newline - intjänandetaket och
\newline - utfyllnadsinkomsten.
\subsection*{51 §}
\paragraph*{}
Beräkningsmetod 2 innebär att en förälders pensionsgrundande belopp för barnår beräknas som differensen mellan
\newline - generell jämförelseinkomst och
\newline - förälderns utfyllnadsinkomst enligt 46 §.
\subsection*{52 §}
\paragraph*{}
/Upphör att gälla U:2025-12-01/
Med generell jämförelseinkomst avses 75 procent av genomsnittet av samtliga för intjänandeåret fastställda pensionsgrundande inkomster för försäkrade som under det året har fyllt högst 65 år.
\paragraph*{}
Beräkningen av genomsnittet av de fastställda pensionsgrundande inkomsterna enligt första stycket ska avse de pensionsgrundande inkomsterna som de var bestämda den 1 december fastställelseåret.
Lag (2022:878).
\subsection*{52 §}
\paragraph*{}
/Träder i kraft I:2025-12-01/
Med generell jämförelseinkomst avses 75 procent av genomsnittet av samtliga för intjänandeåret fastställda pensionsgrundande inkomster för försäkrade som under det året inte har uppnått riktåldern för pension.
\paragraph*{}
Beräkningen av genomsnittet av de fastställda pensionsgrundande inkomsterna enligt första stycket ska avse de pensionsgrundande inkomsterna som de var bestämda den 1 december fastställelseåret.
Lag (2022:879).
\subsection*{53 §}
\paragraph*{}
Beräkningsmetod 3 innebär att en förälder tillgodoräknas ett pensionsgrundande belopp för barnår som motsvarar inkomstbasbeloppet för intjänandeåret.
\subsection*{54 §}
\paragraph*{}
För en försäkrad som under intjänandeåret har fått sådana utlandsinkomster som avses i 47 § första stycket 1 och 2 gäller dock följande. Det inkomstbasbelopp som gäller för intjänandeåret ska minskas om det tillsammans med utfyllnadsinkomst enligt 46 § överstiger 7,5 inkomstbasbelopp (intjänandetaket). Det pensionsgrundande beloppet för barnår ska då beräknas som differensen mellan
\newline - det för intjänandeåret gällande inkomstbasbeloppet och
\newline - det belopp som överstiger intjänandetaket.
\subsection*{55 §}
\paragraph*{}
Anmälan om vem av föräldrarna som ska tillgodoräknas pensionsgrundande belopp för barnår enligt 23 § ska göras skriftligen av föräldrarna gemensamt.
\subsection*{56 §}
\paragraph*{}
Skriftlig anmälan om överlåtelse av rätt till pensionsgrundande belopp för barnår enligt 39 och 40 §§ ska göras gemensamt av överlåtaren och mottagaren.
\subsection*{57 §}
\paragraph*{}
Anmälan enligt 55 eller 56 § ska ha kommit in till Pensionsmyndigheten senast den 31 januari fastställelseåret.
\subsection*{58 §}
\paragraph*{}
Om den som enligt anmälan som avses i 55 eller 56 § ska tillgodoräknas ett pensionsgrundande belopp enligt 30-33 §§ inte kan tillgodoräknas ett sådant belopp, ska Pensionsmyndigheten besluta om pensionsgrundande belopp för barnår som om någon anmälan inte gjorts.
\subsection*{59 §}
\paragraph*{}
Återkallelse av en anmälan som avses i 55 eller 56 § ska vara skriftlig. Återkallelse får inte göras efter den dag då anmälan senast ska ha kommit in till Pensionsmyndigheten.
\chapter*{61 Pensionsrätt och pensionspoäng}
\subsection*{1 §}
\paragraph*{}
I detta kapitel finns inledande bestämmelser i 2 och 3 §§.
\paragraph*{}
Vidare finns bestämmelser om
\newline - fastställande av pensionsrätt i 4 §,
\newline - beräkning av pensionsrätt i 5-10 §§,
\newline - överföring av pensionsrätt för premiepension i 11-16 §§,
\newline - fastställande av pensionspoäng i 17 §,
\newline - beräkning av pensionspoäng i 18-22 §§, och
\newline - år med pensionspoäng vid vård av småbarn (vårdår) i 23-27 §§.
\subsection*{2 §}
\paragraph*{}
Pensionsrätt räknas fram årligen för den försäkrade och ligger till grund för beräkning av inkomstpension och premiepension.
\subsection*{3 §}
\paragraph*{}
Pensionspoäng räknas fram årligen för den försäkrade och ligger till grund för beräkning av tilläggspension.
\subsection*{4 §}
\paragraph*{}
För en försäkrad som är född 1938 eller senare ska det fastställas såväl pensionsrätt för inkomstpension som pensionsrätt för premiepension för varje år som det har fastställts
\newline - pensionsgrundande inkomst, eller
\newline - pensionsgrundande belopp.
\paragraph*{}
För det år då den försäkrade har avlidit fastställs endast pensionsrätt för premiepension.
\subsection*{5 §}
\paragraph*{}
Pensionsrätt för inkomstpension och pensionsrätt för premiepension för ett år ska beräknas var för sig på summan av den försäkrades pensionsunderlag enligt 58 kap. 5 § för det året.
\subsection*{6 §}
\paragraph*{}
Om inte annat följer av bestämmelserna i 7-10 §§, är den försäkrades pensionsrätt
\newline 1. för inkomstpension 16 procent av pensionsunderlaget, och
\newline 2. för premiepension 2,5 procent av pensionsunderlaget.
\paragraph*{}
När pensionsrätten beräknas ska avrundning göras till närmaste lägre hela krontal.
\subsection*{7 §}
\paragraph*{}
För att den försäkrade ska tillgodoräknas pensionsrätt för premiepension för pensionsgrundande belopp för plikttjänstgöring, studier eller barnår krävs att förvärvsvillkoret i 62 kap. 38-41 §§ är uppfyllt vid den tidpunkt när nämnda belopp har fastställts.
\subsection*{8 §}
\paragraph*{}
Om de förutsättningar som avses i 7 § inte är uppfyllda, ska pensionsrätt för premiepension inte beräknas på den del av pensionsunderlaget som motsvarar de pensionsgrundande belopp som anges i samma paragraf. Pensionsrätten för inkomstpension ska då i stället beräknas till 18,5 procent av den delen av pensionsunderlaget.
\paragraph*{}
Första stycket gäller även om pensionsgrundande inkomster och belopp senare ändras så att förvärvsvillkoret i 62 kap. 38 § inte längre är uppfyllt.
\subsection*{9 §}
\paragraph*{}
För att en försäkrad ska tillgodoräknas full pensionsrätt på pensionsgrundande inkomst av annat förvärvsarbete för ett år måste hela ålderspensionsavgiften enligt socialavgiftslagen (2000:980) och hela den allmänna pensionsavgiften enligt lagen (1994:1744) om allmän pensionsavgift för inkomsten vara betald.
\paragraph*{}
Om dessa avgifter inte helt betalats inom den tid som anges i 62 kap. 17 och 18 § skatteförfarandelagen (2011:1244) gäller följande. Pensionsrätt ska beräknas endast på så stor andel av den pensionsgrundande inkomst som härrör från inkomster av annat förvärvsarbete som motsvarar den andel av årets avgifter som betalats inom föreskriven tid.
Lag (2011:1434).
\subsection*{10 §}
\paragraph*{}
För en försäkrad som är född under något av åren 1938-1953 gäller följande. Pensionsrätt för inkomstpension och pensionsrätt för premiepension ska minskas med en tjugondel för varje helt år från och med året efter födelseåret till utgången av 1954.
\paragraph*{}
Första stycket gäller dock inte pensionsrätt för år efter det år då den försäkrade har fyllt 64 år.
\subsection*{11 §}
\paragraph*{}
Hela den pensionsrätt för premiepension som för ett år har fastställts för en försäkrad kan föras över till hans eller hennes make. För sådan överföring, för upphörande av överföring och för återkallelse av anmälan om överföring gäller föreskrifterna i 12-16 §§.
\subsection*{12 §}
\paragraph*{}
Anmälan om överföring av pensionsrätt för premiepension ska skriftligen göras av den make som överföringen ska ske från och ska ha kommit in till Pensionsmyndigheten senast den 30 april det första år som pensionsrätten hänför sig till. Anmälan gäller tills vidare om inget annat anges i anmälan.
Lag (2017:1124).
\subsection*{13 §}
\paragraph*{}
Pensionsrätt för premiepension får föras över till make endast om denne
\newline 1. har varit försäkrad för en bosättningsbaserad eller arbetsbaserad förmån enligt 4-6 kap. någon gång under intjänandeåret, eller
\newline 2. tidigare har tillgodoräknats pensionsrätt för premiepension.
\subsection*{14 §}
\paragraph*{}
Om någon av makarna vill att en överföring som gäller tills vidare ska upphöra måste han eller hon skriftligen anmäla detta till Pensionsmyndigheten senast den 30 april det år från och med vilket överföringen ska upphöra.
Lag (2017:1124).
\subsection*{15 §}
\paragraph*{}
Om makarnas äktenskap har upplösts ska överföringen upphöra från och med det år då äktenskapet upplöstes.
\paragraph*{}
Har äktenskapet upplösts genom att den make som överföringen ska ske från har avlidit, ska dock överföringen gälla även det år då äktenskapet upplöstes, om inte också den andra maken avled samma år.
\subsection*{16 §}
\paragraph*{}
Återkallelse av anmälan om överföring av pensionsrätt för premiepension ska vara skriftlig. Återkallelse får inte göras efter den dag då anmälan senast ska ha kommit in till Pensionsmyndigheten.
\subsection*{17 §}
\paragraph*{}
För en försäkrad som är född 1953 eller tidigare ska det fastställas pensionspoäng för tilläggspension för varje år som det har fastställts pensionsgrundande inkomst.
\paragraph*{}
Pensionspoäng fastställs inte för
\newline 1. år efter det år då den försäkrade fyllt 64 år, och
\newline 2. det år då den försäkrade har avlidit.
\subsection*{18 §}
\paragraph*{}
Den försäkrades pensionspoäng motsvarar kvoten mellan
\newline - den reducerade pensionsgrundande inkomsten och
\newline - det förhöjda prisbasbeloppet.
\paragraph*{}
Pensionspoängen ska beräknas med två decimaler, lägst till en hundradels poäng.
\subsection*{19 §}
\paragraph*{}
Vid beräkningen av pensionspoäng ska den försäkrades reducerade pensionsgrundande inkomst räknas fram som differensen mellan
\newline - den pensionsgrundande inkomsten och
\newline - det förhöjda prisbasbelopp som gällde för intjänandeåret.
\paragraph*{}
Det förhöjda prisbasbeloppet ska i första hand räknas av mot inkomster av anställning.
\subsection*{20 §}
\paragraph*{}
Vid tillämpning av bestämmelserna i 18 och 19 §§ ska pensionsgrundande belopp för sjukersättning eller aktivitetsersättning enligt 60 kap. likställas med pensionsgrundande inkomst.
\subsection*{21 §}
\paragraph*{}
Det som föreskrivs i 9 § om beräkning av pensionsrätt på pensionsgrundande inkomst som härrör från inkomst av annat förvärvsarbete ska även gälla beräkning av pensionspoäng.
\paragraph*{}
Andelsberäkning ska då göras på den pensionsgrundande inkomsten innan det förhöjda prisbasbeloppet dras av.
\subsection*{22 §}
\paragraph*{}
Den pensionspoäng som tillgodoräknas en försäkrad för ett år får sammanlagt inte överstiga den högsta pensionspoäng som någon för det året kan tillgodoräknas på pensionsgrundande inkomst.
\subsection*{23 §}
\paragraph*{}
Vårdår som avses i 24 § ska likställas med år för vilka pensionspoäng har tillgodoräknats den försäkrade, med följande undantag: 1. Vårdår ska inte beaktas vid beräkning enligt 63 kap. 6 § av det medeltal av pensionspoäng som kan tillgodoräknas den försäkrade. 2. Vårdår ska inte beaktas när det bestäms om en försäkrad enligt 63 kap. 3 § har tillgodoräknats pensionspoäng för tillräckligt många år för att ha rätt till tilläggspension.
\subsection*{24 §}
\paragraph*{}
Vårdår ska tillgodoräknas en förälder som är bosatt här i landet och som under större delen av ett kalenderår har vårdat ett barn som är bosatt här och som är yngre än tre år.
För samma barn och år får vårdår tillgodoräknas endast en förälder. 25 § Vid tillämpningen av 24 § likställs med förälder
\newline 1. familjehemsförälder såvitt avser barn som vårdas hos honom eller henne,
\newline 2. förälders make som stadigvarande sammanbor med föräldern, och
\newline 3. förälders sambo som tidigare har varit gift med eller har eller har haft barn med föräldern.
\subsection*{26 §}
\paragraph*{}
Vårdår får inte tillgodoräknas en försäkrad för
\newline 1. år före det år då han eller hon fyllt 16 år,
\newline 2. det år då han eller hon avlidit,
\newline 3. år efter det då han eller hon fyllt 64 år,
\newline 4. år som han eller hon tillgodoräknas pensionspoäng för, eller
\newline 5. år som han eller hon enligt 21 § på grund av bristande eller underlåten avgiftsbetalning inte tillgodoräknas pensionspoäng för.
\subsection*{27 §}
\paragraph*{}
En förälder som vill tillgodoräknas ett vårdår enligt 24 § ska ansöka om det hos Pensionsmyndigheten senast den 31 januari andra året efter det år som vårdår ska tillgodoräknas för.
\chapter*{62 Inkomstpension}
\subsection*{1 §}
\paragraph*{}
I detta kapitel finns allmänna bestämmelser om
\newline - inkomstpension i 2-4 §§, och
\newline - pensionsbehållning i 5-8 §§.
\paragraph*{}
Vidare finns bestämmelser om
\newline - omräkning av pensionsbehållning med hänsyn till arvsvinster i 9-17 §§,
\newline - omräkning med hänsyn till inkomstindex i 18-21 §§,
\newline - omräkning med hänsyn till förvaltningskostnader i 22-26 §§,
\newline - fastställande av arvsvinstfaktorerna och förvaltningskostnadsfaktorn i 27 §,
\newline - beräkning av årlig inkomstpension i 28-37 §§,
\newline - förvärvsvillkor i fråga om vissa pensionsgrundande belopp i 38-41 §§, och
\newline - omräkning av inkomstpension i 42-47 §§.
\subsection*{2 §}
\paragraph*{}
Inkomstpension baseras på den pensionsbehållning som kan tillgodoräknas den försäkrade.
\subsection*{3 §}
\paragraph*{}
En försäkrad som är född 1938 eller senare har rätt till inkomstpension om han eller hon har en pensionsbehållning. 4 § Bestämmelser om uttag av inkomstpension finns i 56 kap.
\subsection*{5 §}
\paragraph*{}
Med pensionsbehållning avses summan av de pensionsrätter för inkomstpension som enligt 61 kap. har fastställts för den försäkrade, omräknade med hänsyn till bestämmelserna om
\newline 1. arvsvinster i 9-17 §§,
\newline 2. indexering i 18-21 §§, och
\newline 3. förvaltningskostnader i 22-26 §§.
\paragraph*{}
Om ett balansindex har beräknats enligt 58 kap. 22-24 §§ för det år som pensionsrätt ska fastställas, ska pensionsrätten för inkomstpension, innan den läggs till pensionsbehållningen, multipliceras med kvoten mellan det balansindex och det inkomstindex som har fastställts för samma år.
Lag (2014:1548).
\subsection*{6 §}
\paragraph*{}
Pensionsbehållningen ska räknas om varje år. När omräkning gjorts ska pensionsbehållningen avrundas till närmaste lägre hela krontal.
\subsection*{7 §}
\paragraph*{}
Om den försäkrade får inkomstpension ska belopp som legat till grund för beräkning av den pensionen inte räknas in i pensionsbehållningen.
\subsection*{8 §}
\paragraph*{}
Om en försäkrad har återkallat eller minskat sitt uttag av inkomstpension ska hans eller hennes pensionsbehållning ökas med produkten av
\newline - det delningstal som skulle ha använts om den försäkrade hade gjort nytt uttag av inkomstpension från och med den månad då pensionsuttaget upphört eller minskat och - den årliga inkomstpension som skulle ha lämnats om återkallelse inte hade skett, respektive den andel av den årliga inkomstpensionen med vilken pensionen minskats.
\paragraph*{}
Omräkning av pensionsbehållning med hänsyn till arvsvinster
\subsection*{9 §}
\paragraph*{}
Pensionsbehållningar för personer som har avlidit (arvsvinster) ska, om inte annat följer av 10 §, fördelas till försäkrade som
\newline 1. är födda samma år som de avlidna,
\newline 2. levde vid utgången av dödsfallsåret, och 3. har tillgodoräknats pensionsrätt för inkomstpension.
\paragraph*{}
Fördelningen ska göras genom en årlig omräkning av de kvarlevandes pensionsbehållningar, med användning av arvsvinstfaktorer, som fastställs varje år.
\subsection*{10 §}
\paragraph*{}
Arvsvinster för personer som har avlidit före det år då de skulle ha fyllt 17 år ska fördelas till kvarlevande som är under 17 år det år personerna avled.
\paragraph*{}
Fördelningen ska göras genom årlig omräkning av de kvarlevandes pensionsbehållningar, med användning av arvsvinstfaktorer.
\subsection*{11 §}
\paragraph*{}
Arvsvinstfaktorerna ska vara desamma för kvinnor och män.
\subsection*{12 §}
\paragraph*{}
/Upphör att gälla U:2025-12-01/
Arvsvinster efter personer som avlidit före det år de skulle ha fyllt 62 år ska fördelas för året efter dödsfallsåret. Arvsvinster efter personer som avlidit det år de fyllt eller skulle ha fyllt 62 år eller senare ska fördelas för dödsfallsåret.
\paragraph*{}
Vid fördelningen ska det bortses från kvarlevandes pensionsbehållning som härrör från pensionsrätt för året för dödsfallen och därefter.
Lag (2022:878).
\subsection*{12 §}
\paragraph*{}
/Träder i kraft I:2025-12-01/
Arvsvinster efter personer som avlidit fem eller fler år före det år de skulle ha uppnått den vid dödsfallet gällande riktåldern för pension ska fördelas för året efter dödsfallsåret. Arvsvinster efter personer som avlidit fyra eller färre år före det år de skulle ha uppnått den vid dödsfallet gällande riktåldern för pension ska fördelas för dödsfallsåret.
\paragraph*{}
Vid fördelningen ska det bortses från kvarlevandes pensionsbehållning som härrör från pensionsrätt för året för dödsfallen och därefter.
Lag (2022:879).
\subsection*{13 §}
\paragraph*{}
/Upphör att gälla U:2025-12-01/
Fördelningen av arvsvinster ska först göras efter personer som avlidit före det år de skulle ha fyllt 62 år. När den fördelningen skett och pensionsrätt för året före det år fördelningen avser har fastställts, ska arvsvinster efter personer som har avlidit det år de fyllt eller skulle ha fyllt 62 år eller avlidit senare fördelas.
Lag (2022:878).
\subsection*{13 §}
\paragraph*{}
/Träder i kraft I:2025-12-01/
Fördelningen av arvsvinster ska först göras efter personer som avlidit fem eller fler år före det år de skulle ha uppnått den vid dödsfallet gällande riktåldern för pension. När den fördelningen skett och pensionsrätt för året före det år fördelningen avser har fastställts, ska arvsvinster efter personer som har avlidit fyra eller färre år före det år de skulle ha uppnått den vid dödsfallet gällande riktåldern för pension fördelas.
Lag (2022:879).
\subsection*{14 §}
\paragraph*{}
Arvsvinstfaktorer som avser fördelning av pensionsbehållningar efter personer som har avlidit före det år då de skulle ha fyllt 17 år ska grundas på kvoten mellan
\newline - summan av pensionsbehållningarna för personer som har avlidit under året före det år omräkningen avser och
\newline - summan av pensionsbehållningarna för de personer under 17 år som levde vid utgången av samma år.
\paragraph*{}
När pensionsbehållningarna bestäms ska det bortses från förändringar efter den 1 december det år fördelningen avser.
\paragraph*{}
/Rubriken upphör att gälla U:2025-12-01/
\paragraph*{}
Arvsvinstfaktorer vid dödsfall efter 16 års ålder men före 62 års ålder
\paragraph*{}
/Rubriken träder i kraft I:2025-12-01/
\subsection*{15 §}
\paragraph*{}
/Upphör att gälla U:2025-12-01/
Arvsvinstfaktorer som avser fördelning av pensionsbehållningar efter personer som har avlidit efter det år då de har fyllt 16 år, men före det år de skulle ha fyllt 62 år, ska grundas på kvoten mellan
\newline - summan av pensionsbehållningarna för personer som har avlidit under året före det år omräkningen avser och
\newline - summan av pensionsbehållningarna för de personer i samma ålder som levde vid utgången av samma år.
\paragraph*{}
När pensionsbehållningarna bestäms ska det bortses från förändringar efter den 1 december det år fördelningen avser.
Lag (2022:878).
\subsection*{15 §}
\paragraph*{}
/Träder i kraft I:2025-12-01/
Arvsvinstfaktorer som avser fördelning av pensionsbehållningar efter personer som har avlidit efter det år då de har fyllt 16 år, men minst fem år före det år de skulle ha uppnått den vid dödsfallet gällande riktåldern för pension, ska grundas på kvoten mellan
\newline - summan av pensionsbehållningarna för personer som har avlidit under året före det år omräkningen avser och
\newline - summan av pensionsbehållningarna för de personer i samma ålder som levde vid utgången av samma år.
\paragraph*{}
När pensionsbehållningarna bestäms ska det bortses från förändringar efter den 1 december det år fördelningen avser.
Lag (2022:879).
\paragraph*{}
/Rubriken upphör att gälla U:2025-12-01/
\subsection*{16 §}
\paragraph*{}
/Upphör att gälla U:2025-12-01/
Arvsvinstfaktorer som avser fördelning av pensionsbehållningar efter personer som har avlidit det år de fyllt eller skulle ha fyllt 62 år eller senare ska grundas på kvoten mellan
\newline - det beräknade antalet personer som har avlidit det år de uppnått eller skulle ha uppnått samma ålder som den person som beräkningen ska göras för och
\newline - det beräknade antalet kvarlevande personer i samma ålder.
Lag (2022:878).
\subsection*{16 §}
\paragraph*{}
/Träder i kraft I:2025-12-01/
Arvsvinstfaktorer som avser fördelning av pensionsbehållningar efter personer som har avlidit fyra eller färre år före det år de skulle ha uppnått den vid dödsfallet gällande riktåldern för pension ska grundas på kvoten mellan
\newline - det beräknade antalet personer som har avlidit det år de uppnått eller skulle ha uppnått samma ålder som den person som beräkningen ska göras för och
\newline - det beräknade antalet kvarlevande personer i samma ålder.
Lag (2022:879).
\subsection*{17 §}
\paragraph*{}
/Upphör att gälla U:2025-12-01/
För en försäkrad som inte har fyllt 66 år ska beräkningen enligt 16 § göras med ledning av officiell statistik över livslängden hos befolkningen i Sverige under femårsperioden närmast före det år personen uppnådde 62 års ålder.
\paragraph*{}
Från och med det år den som avses i första stycket fyller 66 år ska beräkningen enligt 16 § göras med ledning av statistiken för femårsperioden närmast före det år personen uppnådde 65 års ålder.
Lag (2022:878).
\subsection*{17 §}
\paragraph*{}
/Träder i kraft I:2025-12-01/
För en försäkrad som inte har uppnått riktåldern för pension ska beräkningen enligt 16 § göras med ledning av officiell statistik över livslängden hos befolkningen i Sverige under femårsperioden närmast före det fjärde året före det år personen skulle ha uppnått den vid dödsfallet gällande riktåldern för pension.
\paragraph*{}
Från och med det år den som avses i första stycket uppnår riktåldern för pension ska beräkningen enligt 16 § göras med ledning av statistiken för femårsperioden närmast före året före det år personen uppnådde riktåldern för pension.
Lag (2022:879).
\subsection*{18 §}
\paragraph*{}
Pensionsbehållningen ska räknas om med hänsyn till förändringen av inkomstindex om detta index förändras mellan det år arvsvinstomräkningen enligt 9-17 §§ avser och året därefter.
\paragraph*{}
Om arvsvinstomräkning inte ska göras, ska pensionsbehållningen räknas om med hänsyn till förändringen av inkomstindex mellan fastställelseåret och året därefter.
\subsection*{19 §}
\paragraph*{}
För år då balansindex fastställs ska beräkningen göras med hänsyn till detta index i stället för inkomstindexet.
\paragraph*{}
I vilken ordning ska omräkning ske?
\subsection*{20 §}
\paragraph*{}
Omräkningen av pensionsbehållningen med hänsyn till inkomstindex eller balansindex ska göras efter det att pensionsrätt för närmast föregående år har fastställts och arvsvinstomräkningen enligt 9-17 §§ har gjorts.
\subsection*{21 §}
\paragraph*{}
Om den försäkrade har tagit ut inkomstpension under det år arvsvinstomräkningen enligt 9-17 §§ avser eller om uttaget det året har förändrats, ska omräkningen av pensionsbehållningen med avseende på arvsvinster och inkomstindex göras med beaktande av
\newline 1. att den pensionsbehållning som avser pensionsrätt som har tillgodoräknats den försäkrade för åren till och med det andra året före det år arvsvinstomräkningen avser och har räknats om på de sätt som anges i 5 och 6 §§ för åren till och med det närmast föregående året, har uppgått till skilda belopp under året, och
\newline 2. den omräkning som ska göras enligt 42 och 43 §§.
\subsection*{22 §}
\paragraph*{}
Pensionsbehållningen ska minskas med hänsyn till kostnaderna för förvaltningen av försäkringen för inkomstpension och tilläggspension.
\subsection*{23 §}
\paragraph*{}
Minskningen för förvaltningskostnader ska ske genom att pensionsbehållningen multipliceras med den förvaltningskostnadsfaktor som bestämts för det år som arvsvinstomräkningen enligt 9-17 §§ avser. Innan detta görs ska pensionsrätt för närmast föregående år fastställas och omräkning göras enligt 9-21 §§.
\subsection*{24 §}
\paragraph*{}
Den förvaltningskostnadsfaktor som avses i 23 § beräknas för varje år.
\subsection*{25 §}
\paragraph*{}
Förvaltningskostnadsfaktorn ska grundas på kvoten mellan
\newline - kostnaderna för förvaltningen av försäkringen för inkomstpension och tilläggspension det år faktorn avser och
\newline - summan av alla pensionsbehållningar.
\paragraph*{}
När förvaltningskostnadsfaktorn bestäms ska hänsyn också tas till differensen mellan
\newline - det belopp som pensionsbehållningarna minskats med genom föregående års omräkning enligt 22 och 23 §§ och
\newline - de faktiska kostnaderna för förvaltning av försäkringen för inkomstpension och tilläggspension för det året.
\subsection*{26 §}
\paragraph*{}
När förvaltningskostnadsfaktorn för 2011 bestäms avser det som föreskrivs i 25 § 80 procent av kostnaderna för förvaltningen av försäkringen för inkomstpension och tilläggspension. För tid därefter till och med 2021 ska denna andel öka med 2 procentenheter per år.
\subsection*{27 §}
\paragraph*{}
Regeringen eller den myndighet som regeringen bestämmer meddelar närmare föreskrifter om arvsvinstfaktorerna och förvaltningskostnadsfaktorn.
\subsection*{28 §}
\paragraph*{}
Den årliga inkomstpensionen beräknas genom att den försäkrades pensionsbehållning fördelas på det antal år en person i den åldern kan antas ha kvar att leva. Fördelningen sker med användning av ett delningstal.
\subsection*{29 §}
\paragraph*{}
Hel inkomstpension ska för år räknat vara kvoten mellan
\newline - den försäkrades pensionsbehållning vid den tidpunkt från vilken pensionen ska beräknas och - det delningstal som då gäller för den försäkrade.
\subsection*{30 §}
\paragraph*{}
Pensionsbehållning som avses i 28 och 29 §§ ska beräknas på pensionsrätt
\newline 1. som har tillgodoräknats den försäkrade för åren till och med andra året före det år då pensionen tas ut, och
\newline 2. som har omräknats enligt 5 och 6 §§ för åren till och med det år som närmast har föregått pensionsuttaget.
\subsection*{31 §}
\paragraph*{}
För den som har fått folk- eller tilläggspension i form av ålderspension enligt den upphävda lagen (1962:381) om allmän försäkring för tid före den 1 januari 2001 gäller följande. Pensionsbehållningen ska minskas med 0,5 procent för varje månad före 2001 för vilken sådan pension har betalats ut.
\paragraph*{}
Om den pension som tagits ut har begränsats till viss andel av hel förmån (partiellt uttag), ska minskningen av pensionsbehållningen begränsas till motsvarande andel. 32 § Minskning enligt 31 § ska inte göras i fråga om belopp som pensionsbehållningen har ökats med enligt 8 §. 33 § Om inkomstpension tas ut med tre fjärdedelar, hälften eller en fjärdedel av hel förmån, ska beräkningsreglerna i 30-32 §§ tillämpas endast på så stor andel av pensionsbehållningen som motsvarar respektive förmånsnivå.
Lag (2010:1307).
Delningstal
\subsection*{34 §}
\paragraph*{}
Delningstal för beräkning av inkomstpension enligt 28-33 §§ ska vara desamma för kvinnor och män.
\subsection*{35 §}
\paragraph*{}
Delningstal ska beräknas med utgångspunkt i att värdet av pensionsutbetalningarna under genomsnittlig återstående livslängd för personer i den försäkrades ålder från den tidpunkt då pension ska börja lämnas ska motsvara pensionsbehållningen.
\subsection*{36 §}
\paragraph*{}
En kommande månadsutbetalning av inkomstpensionen antas ha ett värde som utgör kvoten mellan
\newline - värdet av en månadsutbetalning vid den tidpunkt då inkomstpension ska börja lämnas och
\newline - en årlig räntefaktor om 1,016 fram till tiden för den kommande pensionsutbetalningen.
\paragraph*{}
Antalet kommande pensionsutbetalningar ska beräknas med ledning av den officiella statistik som anges i 17 §.
\subsection*{37 §}
\paragraph*{}
Regeringen eller den myndighet som regeringen bestämmer meddelar närmare föreskrifter om delningstalet.
\subsection*{38 §}
\paragraph*{}
/Upphör att gälla U:2025-12-01/
Vid beräkningen av inkomstpension ska hänsyn tas till pensionsbehållning som härrör från pensionsgrundande belopp enligt 60 kap. för plikttjänstgöring, studier, eller barnår endast om det förvärvsvillkor som anges i andra stycket är uppfyllt.
\paragraph*{}
Förvärvsvillkoret är uppfyllt om det för den försäkrade, senast det år han eller hon fyller 71 år, har fastställts pensionsgrundande inkomster som för vart och ett av minst fem år uppgått till lägst två gånger det inkomstbasbelopp som gällde för intjänandeåret.
Lag (2022:878).
\subsection*{38 §}
\paragraph*{}
/Träder i kraft I:2025-12-01/
Vid beräkningen av inkomstpension ska hänsyn tas till pensionsbehållning som härrör från pensionsgrundande belopp enligt 60 kap. för plikttjänstgöring, studier, eller barnår endast om det förvärvsvillkor som anges i andra stycket är uppfyllt.
\paragraph*{}
Förvärvsvillkoret är uppfyllt om det för den försäkrade, senast det femte året efter att han eller hon uppnår riktåldern för pension, har fastställts pensionsgrundande inkomster som för vart och ett av minst fem år uppgått till lägst två gånger det inkomstbasbelopp som gällde för intjänandeåret.
Lag (2022:879).
\subsection*{39 §}
\paragraph*{}
Vid bedömningen av om förvärvsvillkoret är uppfyllt, ska pensionsgrundande belopp för sjukersättning och aktivitetsersättning enligt 60 kap. 7 § likställas med pensionsgrundande inkomst.
\subsection*{40 §}
\paragraph*{}
Pension på den pensionsbehållning som avses i 38 § kan lämnas tidigast året efter det fastställelseår då förvärvsvillkoret uppfylldes.
\subsection*{41 §}
\paragraph*{}
Pensionsgrundande inkomst som enligt bestämmelserna om bristande eller underlåten avgiftsbetalning i 61 kap. 9 § inte legat till grund för beräkning av pensionsrätt ska inte beaktas vid bedömning enligt 38-40 §§.
\subsection*{42 §}
\paragraph*{}
Den inkomstpension som en försäkrad har vid ett årsskifte ska räknas om genom följsamhetsindexering.
\subsection*{43 §}
\paragraph*{}
Följsamhetsindexeringen beräknas som produkten av
\newline - den inkomstpension som en försäkrad får vid årsskiftet och
\newline - kvoten mellan inkomstindexet efter årsskiftet och inkomstindexet före årsskiftet, sedan den nämnda kvoten dividerats med talet 1,016.
\paragraph*{}
För år då balansindex fastställs ska beräkningen göras med hänsyn till detta index i stället för inkomstindexet.
\paragraph*{}
/Rubriken upphör att gälla U:2025-12-01/
\subsection*{44 §}
\paragraph*{}
/Upphör att gälla U:2025-12-01/
För den som har tagit ut inkomstpension före det år då han eller hon fyllt 66 år, ska pensionen räknas om från och med det år då den försäkrade uppnår denna ålder om delningstalet ändrats.
Lag (2022:878).
\subsection*{44 §}
\paragraph*{}
/Träder i kraft I:2025-12-01/
För den som har tagit ut inkomstpension före det år då han eller hon uppnått riktåldern för pension, ska pensionen räknas om från och med det år då den försäkrade uppnår denna ålder om delningstalet ändrats.
Lag (2022:879).
\subsection*{45 §}
\paragraph*{}
Omräkningen enligt 44 § ska göras genom att pensionen multipliceras med kvoten mellan
\newline - det delningstal som den pension som lämnas har beräknats efter och
\newline - det delningstal som med ledning av den officiella statistik som avses i 17 § har fastställts för personer i samma ålder som den försäkrade vid den tidpunkt när den lämnade pensionen beräknades.
\subsection*{46 §}
\paragraph*{}
Efter omräkning enligt 42-45 §§ ska den inkomstpension som en person får vid ett visst årsskifte räknas om som om uttaget av pension skulle ha återkallats från och med januari året efter årsskiftet samtidigt som nytt uttag av inkomstpension skulle ha gjorts från och med den månaden.
\paragraph*{}
Det som anges i första stycket gäller dock inte om den försäkrade
\newline 1. har fått hel inkomstpension hela året före nämnda årsskifte, och
\newline 2. inte har tillgodoräknats pensionsrätt för andra året före det årsskiftet.
\subsection*{47 §}
\paragraph*{}
Om den försäkrade ökar eller minskar sitt uttag av inkomstpension, ska en omräkning motsvarande den som anges i 46 § göras för tid från och med månaden för det ändrade uttaget. Det ska då anses som om återkallelse och nytt uttag har gjorts den nämnda månaden.
\chapter*{63 Tilläggspension}
\subsection*{1 §}
\paragraph*{}
I detta kapitel finns allmänna bestämmelser om tilläggspension i 2-5 §§.
\paragraph*{}
Vidare finns bestämmelser om beräkning av årlig tilläggspension i 6-14 §§.
\paragraph*{}
Slutligen finns särskilda bestämmelser för
\newline - försäkrade födda 1938-1953 i 15-22 §§, och
\newline - försäkrade födda före 1938 i 23-29 §§.
\subsection*{2 §}
\paragraph*{}
Tilläggspension baseras på de pensionspoäng som enligt 61 kap. kan tillgodoräknas den försäkrade.
\subsection*{3 §}
\paragraph*{}
En försäkrad som är född 1953 eller tidigare har rätt till tilläggspension om han eller hon har tillgodoräknats pensionspoäng för minst tre år (treårskravet).
\subsection*{4 §}
\paragraph*{}
För en försäkrad som inte är svensk medborgare ska år före 1974, för vilka sjömansskatt betalats, likställas med år för vilka pensionspoäng har beräknats vid bedömningen av om treårskravet i 3 § är uppfyllt för rätt till tilläggspension enligt 6 § första stycket 2.
\subsection*{5 §}
\paragraph*{}
Bestämmelser om uttag av tilläggspension finns i 56 kap.
\subsection*{6 §}
\paragraph*{}
Tilläggspensionen uppgår, om inte annat följer av bestämmelserna i 8 och 10-14 §§, för år räknat till summan av
\newline 1. 60 procent av produkten av det för året gällande prisbasbeloppet och medeltalet av de pensionspoäng som tjänats in av den försäkrade och
\newline 2. 96 procent av det för året gällande prisbasbeloppet för den som är ogift, eller 78,5 procent av prisbasbeloppet för den som är gift.
\paragraph*{}
Om den försäkrade har tjänat in pensionspoäng för mer än 15 år görs beräkningen enligt första stycket 1 på medeltalet av de 15 högsta poängtalen.
\subsection*{7 §}
\paragraph*{}
Bestämmelser om garantitillägg för försäkrade födda något av åren 1938-1953 finns i 18-22 §§.
Avkortning när den försäkrade inte kan tillgodoräknas 30 år med pensionspoäng
\subsection*{8 §}
\paragraph*{}
Om en försäkrad har tjänat in pensionspoäng för färre än 30 år gäller följande vid beräkningen av tilläggspension enligt 6 §. Hänsyn ska tas endast till så stor del av den produkt som anges i 6 § första stycket 1 och till så stor del av det belopp som avses i 6 § första stycket 2 som svarar mot kvoten mellan
\newline - det antal år som pensionspoäng har tjänats in för och
\newline - talet 30.
\paragraph*{}
För en försäkrad som inte är svensk medborgare ska år före 1974, för vilka sjömansskatt betalats, likställas med år för vilka pensionspoäng har beräknats vid beräkningen av tilläggspension enligt 6 § första stycket 2.
\subsection*{9 §}
\paragraph*{}
Vid tillämpning av 6 § första stycket 2 likställs med 1. gift försäkrad en försäkrad som är sambo med någon som han eller hon har varit gift med eller har eller har haft barn med, och
\newline 2. ogift försäkrad en försäkrad som är gift men stadigvarande lever åtskild från sin make, om inte särskilda skäl föranleder något annat.
\subsection*{10 §}
\paragraph*{}
Om tilläggspension tas ut med tre fjärdedelar, hälften eller en fjärdedel av hel förmån, ska beräkningsreglerna i 6, 8 och 11-14 §§ tillämpas endast på så stor andel av tilläggspensionen som motsvarar respektive förmånsnivå.
\subsection*{11 §}
\paragraph*{}
När tilläggspension beräknas för år efter det år då den försäkrade har fyllt 65 år gäller följande. Pensionen ska först beräknas utifrån det prisbasbelopp som gällde för det år då den försäkrade fyllde 65 år. För tiden från och med årsskiftet efter det att den försäkrade har fyllt 65 år ska pensionen följsamhetsindexeras enligt bestämmelserna i 62 kap. 42 och 43 §§.
\subsection*{12 §}
\paragraph*{}
Om tilläggspensionen tas ut tidigare än från och med den månad då den försäkrade fyller 65 år gäller följande.
Pensionen ska minskas med 0,5 procent för varje månad som, från och med den månad när den börjar tas ut, återstår till den månad då den försäkrade fyller 65 år.
\subsection*{13 §}
\paragraph*{}
Om tilläggspension tas ut senare än från och med den månad då den försäkrade fyller 65 år gäller följande. Efter omräkning enligt 11 § ska pensionen ökas med 0,7 procent för varje månad från ingången av den månad då den försäkrade uppnådde 65 års ålder till och med månaden före den när pensionen börjar tas ut. Det ska då bortses från tid efter ingången av den månad då den försäkrade fyller 70 år.
Lag (2011:1075).
\subsection*{14 §}
\paragraph*{}
Beräkningsreglerna i denna paragraf ska användas när
\newline 1. tilläggspension börjar lämnas på nytt efter tidigare återkallelse, eller
\newline 2. uttag av tilläggspension ökas efter tidigare minskning.
\paragraph*{}
När pensionen beräknas ska det bortses från den minskning respektive ökning som tidigare har gjorts enligt 12 och 13 §§.
\paragraph*{}
Efter det att pensionen enligt bestämmelserna i 6, 8 och 11-13 §§ har beräknats på nytt, ska avdrag göras för varje månad för vilken pensionen tidigare har lämnats.
\paragraph*{}
För tid före den månad då den försäkrade har fyllt 65 år ska avdraget motsvara 0,5 procent av pensionen, beräknad enligt 6, 8 och 11 §§. För annan tid ska avdraget motsvara 0,7 procent av pensionen, beräknad enligt 6, 8 och 11 §§.
\subsection*{15 §}
\paragraph*{}
För en försäkrad som är född något av åren 1938-1953 beräknas tillläggspension, utöver vad som tidigare angetts i detta kapitel, enligt bestämmelserna i 16-22 §§.
\subsection*{16 §}
\paragraph*{}
För den som är född under något av åren 1938-1953 ska tilläggspensionen minskas med en tjugondel för varje helt år från och med 1935 till utgången av födelseåret.
\subsection*{17 §}
\paragraph*{}
Om ett balansindex har fastställts för det år den försäkrade fyller 65 år ska pension för den som är född något av åren 1938-1953 från och med den månad den försäkrade fyller 65 år multipliceras med kvoten mellan det balansindex och det inkomstindex som har fastställts för det året.
Lag (2015:676).
Lag (2015:676).
\subsection*{18 §}
\paragraph*{}
En försäkrad som är född under något av åren 1938-1953 har rätt till ett tillägg till tilläggspensionen (garantitillägg) från och med den månad då han eller hon fyller 65 år, om detta följer av 19-22 §§.
\subsection*{19 §}
\paragraph*{}
Garantitillägget ska för år räknat motsvara differensen mellan
\newline 1. hel årlig tilläggspension för den försäkrade vid ingången av år 1995 beräknad enligt 6 och 8 §§, men med beaktande av hans eller hennes civilstånd vid ingången av den månad tillägget avser, samt omräknad enligt 11 § och
\newline 2. summan av den försäkrades inkomstpension och tilläggspension beräknad för ett år.
\paragraph*{}
Vid tillämpningen av första stycket 1 gäller bestämmelserna i 9 § om vem som likställs med gift eller ogift.
\subsection*{20 §}
\paragraph*{}
Har den försäkrade tagit ut inkomstpension eller tilläggspension före den månad han eller hon fyller 65 år ska hel årlig tilläggspension enligt 19 § första stycket 1 minskas med 0,5 procent för varje månad för vilken pensionen lämnats.
\subsection*{21 §}
\paragraph*{}
Om uttag av tilläggspension och inkomstpension avser tre fjärdedelar, hälften eller en fjärdedel av hel pensionsförmån ska det belopp som avses i 19 § första stycket 1 gälla motsvarande andel av det beloppet.
\subsection*{22 §}
\paragraph*{}
Vid beräkning enligt 19 § första stycket 2 ska den försäkrades in-komstpension beräknas som om pensionsrätt för denna pension har utgjort 18,5 procent av pensionsunderlaget före minskning enligt 61 kap. 10 §.
\subsection*{23 §}
\paragraph*{}
För en försäkrad som är född 1937 eller tidigare beräknas tilläggspension, utöver det som föreskrivs i 2-14 §§, enligt bestämmelserna i 24-29 §§.
\subsection*{24 §}
\paragraph*{}
För en försäkrad som är född något av åren 1911-1927 ska tilläggspension beräknas som om pensionspoäng kunnat tillgodoräknas om han eller hon hade fått rätt till förtidspension enligt 13 kap. 1 § i den upphävda lagen (1962:381) om allmän försäkring, i dess lydelse före den 1 januari 2003, från och med januari det år han eller hon uppnått 65 års ålder. Detta gäller dock endast om pensionen därigenom blir större.
\subsection*{25 §}
\paragraph*{}
För den som är född 1935 eller tidigare ska det som föreskrivs i 11 § om prisbasbeloppet för det år då den försäkrade fyllde 65 år i stället avse prisbasbeloppet för 2001.
\subsection*{26 §}
\paragraph*{}
För den som är född 1937 eller tidigare ska det som föreskrivs i 13 § om att bortse från viss tid också gälla tid när den försäkrade har fått folkpension enligt den upphävda lagen (1962:381) om allmän försäkring.
\subsection*{27 §}
\paragraph*{}
För en försäkrad som är svensk medborgare och född under något av åren 1896-1914 ska, vid tillämpning av 8 §, talen 30 bytas ut mot 20.
\subsection*{28 §}
\paragraph*{}
För en försäkrad som är svensk medborgare och född under något av åren 1915-1923 ska, vid tillämpning av 8 §, talen 30 bytas ut mot 20 ökat med ett för varje helt år från och med 1915 till utgången av den försäkrades födelseår.
\subsection*{29 §}
\paragraph*{}
För en försäkrad som inte är svensk medborgare och som är född 1923 eller tidigare, ska vid tillämpning av 8 § från talen 30 räknas av det antal år före 1960, för vilka inkomst som taxerats till statlig inkomstskatt har beräknats för honom eller henne.
\paragraph*{}
Högst tio år får räknas av enligt första stycket. Är den försäkrade född något av åren 1915-1923, ska dock det högsta antal år som får räknas av minskas med ett för varje helt år från och med 1915 till utgången av hans eller hennes födelseår.
\chapter*{64 Premiepension}
\subsection*{1 §}
\paragraph*{}
I detta kapitel finns allmänna bestämmelser om premiepension i 2-6 §§.
\paragraph*{}
Vidare finns bestämmelser om
\newline - uttag av premiepension i 7-14 §§,
\newline - kapitalförvaltning, m.m. i 15-18 §§,
\newline - överföring av medel och byte av fond i 23-27 b §§,
\newline - ändrad pensionsrätt i 28-31 §§,
\newline - skadestånd i 32-36 §§,
\newline - kostnader för premiepensionsverksamheten i 37-45 §§,
\newline - förbud mot marknadsföring och försäljning via telefon i 46 §, och
\newline - samverkan i 47 §.
Lag (2022:761).
\subsection*{2 §}
\paragraph*{}
Premiepensionssystemet innebär att
\newline - medel som motsvarar fastställd pensionsrätt för premiepension fonderas, och
\newline - pensionernas storlek är beroende av värdeutvecklingen på de fonderade medlen.
Lag (2022:761).
\subsection*{2 a §}
\paragraph*{}
Premiepensionssystemet ska erbjuda ett pensionssparande av hög kvalitet som ger en trygg pension. Det innebär att
\newline 1. avkastningen på pensionsspararnas fonderade medel bör vara tydligt högre än förändringen av inkomstindex enligt 58 kap. 10 § men utan garantier om viss avkastning, och
\newline 2. utbetalningarna av premiepensionen bör vara allt mer förutsägbara och stabila, med beaktande av den återstående utbetalningstiden.
Lag (2022:761).
\subsection*{2 b §}
\paragraph*{}
Premiepensionssystemet ska erbjuda valfrihet genom att pensionsspararen ska kunna påverka risknivå och placeringsinriktning för förvaltningen av de medel som fonderas för spararens räkning.
Lag (2022:761).
\subsection*{3 §}
\paragraph*{}
Pensionsmyndigheten är försäkringsgivare för premiepension. Pensionsmyndigheten ska erbjuda pensionsspararna ett urval av upphandlade fonder på ett fondtorg för premiepension (fondtorget för premiepensionen). Myndighetens verksamhet i fråga om premiepension ska bedrivas enligt försäkringsmässiga principer.
\paragraph*{}
Pensionsmyndigheten ska förvalta premiepensionsmedlen genom att
\newline 1. ingå samarbetsavtal med Fondtorgsnämnden,
\newline 2. ingå samarbetsavtal med Sjunde AP-fonden, och
\newline 3. placera medlen i fonder.
\paragraph*{}
Ytterligare bestämmelser om Pensionsmyndighetens verksamhet m.m. finns i lagen (2017:230) om Pensionsmyndighetens försäkringsverksamhet i premiepensionssystemet.
Lag (2022:761).
\subsection*{3 a §}
\paragraph*{}
Pensionsspararna ska ha möjlighet att placera de medel som fonderas för deras räkning i fonder på fondtorget för premiepensionen.
\paragraph*{}
Fondtorget för premiepensionen ska erbjuda en stor bredd i urvalet av för premiepensionssystemet lämpliga fonder med olika risknivå och placeringsinriktning som ger valfrihet för en sparare vid placering enligt första stycket. Fonderna ska vara kostnadseffektiva, hållbara, kontrollerbara och av hög kvalitet.
Lag (2022:761).
\subsection*{3 b §}
\paragraph*{}
Fondtorgsnämnden ska löpande tillhandahålla fonder till fondtorget för premiepensionen genom att upphandla fonder enligt lagen (2022:760) om upphandling av fonder till premiepensionens fondtorg.
Lag (2022:761).
\subsection*{3 c §}
\paragraph*{}
Pensionsmyndigheten förvärvar och löser in andelar i fonderna på fondtorget för premiepensionen. Fondtorgsnämnden ansvarar för kontakterna i övrigt med de fondförvaltare som Fondtorgsnämnden har ingått avtal med enligt lagen (2022:760) om upphandling av fonder till premiepensionens fondtorg (fondavtal).
Lag (2022:761).
\subsection*{4 §}
\paragraph*{}
Med pensionssparare avses var och en för vilken det har
\newline - fastställts pensionsrätt för premiepension enligt 61 kap., och
\newline - förts över medel till förvaltning enligt 18 §.
\paragraph*{}
Med fastställd pensionsrätt likställs pensionsrätt som förts över från pensionsspararens make enligt bestämmelserna i 61 kap. 11-15 §§.
\subsection*{5 §}
\paragraph*{}
För varje pensionssparare ska Pensionsmyndigheten föra ett premiepensionskonto som visar utvecklingen av pensionsspararens tillgodohavande i premiepensionssystemet.
\subsection*{6 §}
\paragraph*{}
Premiepension lämnas i vissa fall till pensionsspararens efterlevande som ett efterlevandeskydd. Bestämmelser om efterlevandeskydd finns i 89, 91 och 92 kap.
\subsection*{7 §}
\paragraph*{}
Pensionsspararen kan välja att få ut sin premiepension antingen i form av uttag från fondförsäkringen enligt 10 § eller i form av livränta med garanterat belopp enligt 11-14 §§.
\subsection*{8 §}
\paragraph*{}
Premiepensionen är livsvarig och ska beräknas lika för kvinnor och män.
\subsection*{9 §}
\paragraph*{}
Ytterligare bestämmelser om uttag av premiepension finns i 56 kap.
\subsection*{10 §}
\paragraph*{}
Premiepension från fondförsäkring ska beräknas med utgångspunkt i tillgodohavandet på pensionsspararens premiepensionskonto.
\subsection*{11 §}
\paragraph*{}
Om pensionsspararen begär det ska premiepensionen lämnas i form av en livränta med garanterade belopp. Den finansiella risken för de tillgångar som motsvarar tillgodohavandet på pensionsspararens premiepensionskonto övergår då på Pensionsmyndigheten.
\subsection*{12 §}
\paragraph*{}
En övergång till livränta får göras tidigast när pensionsspararen börjar ta ut premiepension.
\paragraph*{}
Övergången måste avse hela tillgodohavandet på premiepensionskontot och gäller även för medel som senare tillförs kontot.
\subsection*{13 §}
\paragraph*{}
En övergång som begärs senare än vad som avses i 12 § första stycket får verkan tidigast från och med månaden efter den då begäran kom in till Pensionsmyndigheten.
\paragraph*{}
När Pensionsmyndigheten har påbörjat inlösen av pensionsspararens innehav, kan begäran om övergång till livränta inte återkallas.
\subsection*{14 §}
\paragraph*{}
Vid övergång till premiepension i form av livränta ska pensionen bestämmas med utgångspunkt i tillgångarnas värde vid inlösen av innehavet.
\paragraph*{}
Inlösen ska ske snarast efter det att begäran kom in till Pensionsmyndigheten, dock tidigast tre månader före den månad då livräntan första gången ska betalas ut.
Lag (2018:772).
\subsection*{15 §}
\paragraph*{}
Pensionsmyndigheten ansvarar för förvaltningen av de avgiftsmedel som har förts till Riksgäldskontoret enligt
\newline - 6 § lagen (2000:981) om fördelning av socialavgifter, och
\newline - 8 § lagen (1998:676) om statlig ålderspensionsavgift.
\paragraph*{}
Pensionsmyndigheten ansvarar för förvaltningen av avgiftsmedlen till dess att de förs över enligt 18 § till annan förvaltning.
\subsection*{16 §}
\paragraph*{}
Pensionsmyndigheten ska under den tillfälliga förvaltningen sträva efter att med ett lågt risktagande och med hänsyn till kravet på betalningsberedskap uppnå så god avkastning på medlen som möjligt.
\paragraph*{}
Regeringen meddelar föreskrifter om i vilka tillgångsslag medlen ska placeras.
\subsection*{17 §}
\paragraph*{}
Avkastningen på avgiftsmedlen ska fördelas mellan pensionsspararna i förhållande till storleken av vars och ens fastställda pensionsrätt.
\subsection*{17 a §}
\paragraph*{}
Har upphävts genom
lag (2022:761).
\subsection*{17 b §}
\paragraph*{}
Har upphävts genom
lag (2022:761).
\subsection*{17 c §}
\paragraph*{}
Har upphävts genom
lag (2022:761).
\subsection*{17 d §}
\paragraph*{}
Har upphävts genom
lag (2022:761).
\subsection*{18 §}
\paragraph*{}
När pensionsrätt för premiepension har fastställts för en pensionssparare, ska Pensionsmyndigheten föra över medel som motsvarar pensionsrätten samt avkastningen på dessa medel till förvaltning i
\newline 1. fonder som omfattas av fondavtal, eller
\newline 2. fonder som förvaltas av Sjunde AP-fonden enligt lagen (2000:192) om allmänna pensionsfonder (AP-fonder).
\paragraph*{}
Om pensionsspararen har tagit ut premiepension i form av livränta enligt 11-14 §§ ska medlen i stället föras över till Pensionsmyndighetens livränteverksamhet.
Lag (2022:761).
\subsection*{19 §}
\paragraph*{}
Har upphävts genom
lag (2022:761).
\subsection*{20 §}
\paragraph*{}
Har upphävts genom
lag (2022:761).
\subsection*{21 §}
\paragraph*{}
Har upphävts genom
lag (2022:761).
\subsection*{22 §}
\paragraph*{}
Har upphävts genom
lag (2022:761).
\subsection*{23 §}
\paragraph*{}
Pensionsspararen har rätt att bestämma var de medel som för spararens räkning förs över enligt 18 § ska placeras, utom i det fall som avses i 24 §.
\paragraph*{}
Regeringen eller den myndighet som regeringen bestämmer kan med stöd av 8 kap. 7 § regeringsformen meddela föreskrifter om ett högsta antal fonder som samtidigt får antecknas på ett premiepensionskonto.
\paragraph*{}
Bestämmelser om fördelningen av medel när pensionsspararen bestämmer att hans eller hennes medel ska föras över till en eller flera fonder som förvaltas av Sjunde AP-fonden finns i 5 kap. 1 a § första stycket lagen (2000:192) om allmänna pensionsfonder (AP-fonder).
Lag (2022:1546).
\subsection*{24 §}
\paragraph*{}
När pensionsrätt för premiepension fastställs första gången för en pensionssparare, ska Pensionsmyndigheten föra över de medel som avses i 23 § till sådan förvaltning som avses i 5 kap. 1 a § andra stycket lagen (2000:192) om allmänna pensionsfonder (AP-fonder).
Lag (2022:1546).
\subsection*{25 §}
\paragraph*{}
På begäran av pensionsspararen ska Pensionsmyndigheten placera om spararens fondmedel på det sätt som spararen anger.
\paragraph*{}
Pensionsspararen ska upplysas om rätten att placera om fondmedlen.
\paragraph*{}
En begäran om byte av fond ska vara egenhändigt undertecknad av pensionsspararen.
Lag (2018:772).
\subsection*{26 §}
\paragraph*{}
När pensionsrätt för premiepension för ett senare år än det första året har fastställts för pensionsspararen, ska Pensionsmyndigheten föra över medlen till en eller flera fonder. Fördelningen mellan fonderna ska svara mot vad spararen senast bestämt om placeringen i fond av medlen på premiepensionskontot.
\paragraph*{}
Om pensionsspararen inte har bestämt en sådan placering, ska Pensionsmyndigheten föra över medlen till en eller flera fonder som förvaltas av Sjunde AP-fonden enligt lagen (2000:192) om allmänna pensionsfonder (AP-fonder). Bestämmelser om fördelningen av medlen finns i 5 kap. 1 a § andra stycket samma lag.
Lag (2022:1546).
\subsection*{26 a §}
\paragraph*{}
Med valarkitekturen för premiepensionssystemet avses den presentation av olika val, det stöd och den information som lämnas av Pensionsmyndigheten när en sparare ska ange hur spararens premiepensionsmedel ska förvaltas.
Lag (2022:761).
\subsection*{26 b §}
\paragraph*{}
Valarkitekturen för premiepensionssystemet ska främja att en pensionssparare gör väl övervägda val om hur spararens premiepensionsmedel ska förvaltas.
Lag (2022:761).
\subsection*{26 c §}
\paragraph*{}
Utgångspunkten i valarkitekturen ska vara att pensionsspararens medel förvaltas av Sjunde AP-fonden med fördelning enligt 5 kap. 1 a § andra stycket lagen (2000:192) om allmänna pensionsfonder (AP-fonder).
\paragraph*{}
En pensionssparare som vill bestämma hur spararens premiepensionsmedel ska placeras kan välja att medlen enligt 23 § första stycket förs över till förvaltning i fonder som omfattas av fondavtal eller i fonder som förvaltas av Sjunde AP-fonden enligt lagen om allmänna pensionsfonder (AP-fonder).
Lag (2022:1546).
\subsection*{26 d §}
\paragraph*{}
Pensionsmyndigheten ska skriftligen informera pensionsspararen inför varje val enligt 26 c § andra stycket om innebörden av valet. Spararen ska skriftligen bekräfta att spararen har tagit del av informationen.
Lag (2022:761).
\subsection*{26 e §}
\paragraph*{}
Regeringen eller den myndighet som regeringen bestämmer kan med stöd av 8 kap. 7 § regeringsformen meddela ytterligare föreskrifter om valarkitekturen.
Lag (2022:761).
\subsection*{27 §}
\paragraph*{}
Pensionsmyndigheten ska årligen upplysa pensionsspararen om de åtgärder som myndigheten enligt 26 § ska vidta när pensionsrätt för premiepension har fastställts.
\subsection*{27 a §}
\paragraph*{}
En pensionssparare som har premiepensionsmedel placerade i en fond som omfattas av ett fondavtal som har löpt ut, sagts upp eller hävts, ska av Pensionsmyndigheten ges möjlighet att inom viss tid bestämma att medlen ska placeras i en eller flera fonder som omfattas av ett fondavtal.
\paragraph*{}
Om en pensionssparare inte inom den tid som anges av Pensionsmyndigheten bestämmer hur medlen ska placeras, ska medlen placeras i en fond som i väsentliga avseenden är likvärdig med den tidigare valda fonden. Placeringen av medlen ska göras med tillämpning av den fördelningsordning som har fastställts enligt 2 kap. 23 § lagen (2022:760) om upphandling av fonder till premiepensionens fondtorg.
Lag (2022:761).
\subsection*{27 b §}
\paragraph*{}
Om det inte finns någon likvärdig fond enligt 27 a §, ska medlen placeras i förvaltning av Sjunde AP-fonden enligt 5 kap. 1 a § andra stycket lagen (2000:192) om allmänna pensionsfonder (AP-fonder).
\paragraph*{}
Om det finns särskilda skäl, får Pensionsmyndigheten placera medlen enligt första stycket utan att först ge pensionsspararen möjlighet att bestämma hur medlen ska placeras.
Lag (2022:1546).
\subsection*{28 §}
\paragraph*{}
När ett beslut om pensionsrätt för premiepension har ändrats, ska det belopp med vilket pensionsrätten har ändrats ökas med den avkastning som enligt 17 § har tillförts eller skulle ha tillförts pensionsspararen för detta belopp.
\paragraph*{}
Det ökade beloppet ska därefter räknas upp med basränta enligt 65 kap. 3 § skatteförfarandelagen (2011:1244) för tiden från och med den 1 april året efter fastställelseåret till och med dagen för beslutet om ändring av pensionsrätten.
Lag (2011:1434).
\subsection*{29 §}
\paragraph*{}
Det enligt 28 § uppräknade beloppet ska vid höjning av pensionsrätt för premiepension läggas till och vid sänkning av sådan pensionsrätt dras av från pensionsspararens premiepensionskonto.
\paragraph*{}
När tillägg eller avdrag enligt första stycket har gjorts, tillämpas bestämmelserna i 30 eller 31 §.
\subsection*{30 §}
\paragraph*{}
Om ett beslut innebär att fastställd pensionsrätt för premiepension har höjts, ska Pensionsmyndigheten föra över medel som motsvarar det uppräknade beloppet enligt 28 § från den tillfälliga förvaltningen enligt 15-17 §§ till fonder på det sätt som anges i 26 §.
\paragraph*{}
Om pensionsspararen har börjat ta ut premiepension i form av livränta enligt 11 §, ska beloppet i stället föras över till Pensionsmyndighetens livränteverksamhet.
\subsection*{31 §}
\paragraph*{}
Om ett beslut innebär att fastställd pensionsrätt för premiepension har sänkts, ska Pensionsmyndigheten ta ut medel som motsvarar det uppräknade beloppet enligt 28 § från de fonder där medel har placerats för pensionsspararens räkning.
Fonderna ska tas i anspråk i förhållande till värdet av innehavet i varje fond. Har pensionsspararen tagit ut premiepension i form av livränta enligt 11 §, ska beloppet i stället betalas av Pensionsmyndigheten med omräkning av fastställda livräntebelopp.
\paragraph*{}
Om beloppet inte täcks av tillgodohavandet på pensionsspararens premiepensionskonto, ska det resterande beloppet redovisas som ett underskott på kontot. Om det senare fastställs ny pensionsrätt för spararen, ska underskottet täckas innan medel förs över till förvaltning enligt 18 §.
\subsection*{32 §}
\paragraph*{}
Om någon är skadeståndsskyldig för att ha orsakat att fondandelar som Pensionsmyndigheten har förvärvat för en eller flera pensionssparares räkning gått ned i värde, ska skadeståndet betalas till myndigheten.
\subsection*{33 §}
\paragraph*{}
Talan om skadestånd som avses i 32 § får föras av Pensionsmyndigheten.
\paragraph*{}
Om Fondtorgsnämnden avser att föra talan mot en fondförvaltare i fråga om ett fondavtal som Fondtorgsnämnden har ingått, får Pensionsmyndigheten uppdra åt den myndigheten att samtidigt föra talan om skadestånd enligt 32 § mot samma förvaltare.
\paragraph*{}
En pensionssparare för vars räkning sådana fondandelar som avses i 32 § har förvärvats har också rätt att föra talan om skadestånd till myndigheten. Detta gäller dock endast om Pensionsmyndigheten har förklarat att den
\newline - inte avser att föra talan om skadestånd, och
\newline - inte heller har träffat någon uppgörelse om skadeståndsskyldighet.
Lag (2022:761).
\subsection*{34 §}
\paragraph*{}
Om talan enligt 33 § tredje stycket väcks, får någon uppgörelse om skadeståndsskyldigheten inte träffas utan pensionsspararens samtycke.
Lag (2023:905).
\subsection*{35 §}
\paragraph*{}
En pensionssparare som för talan om skadestånd till Pensionsmyndigheten svarar själv för rättegångskostnaderna.
\paragraph*{}
Pensionsspararen har dock rätt till ersättning för dessa kostnader av Pensionsmyndigheten, i den utsträckning de täcks av vad som har kommit myndigheten till godo genom rättegången. Sådan ersättning ska dras av från skadeståndsbeloppet innan fördelning enligt 36 § görs.
\subsection*{36 §}
\paragraph*{}
När ett skadeståndsbelopp har betalats till Pensionsmyndigheten, ska myndigheten, för varje pensionssparare för vars räkning andelar förvärvats i den berörda fonden, beräkna hur stor del av skadeståndsbeloppet som hänför sig till pensionsspararen.
\paragraph*{}
Myndigheten ska öka tillgodohavandet på pensionsspararens premiepensionskonto med detta belopp och föra över beloppet till fonder på det sätt som anges i 26 §. Om pensionsspararen har börjat ta ut pension i form av livränta enligt 11 §, ska beloppet i stället föras över till myndighetens livränteverksamhet.
\subsection*{37 §}
\paragraph*{}
Pensionsmyndighetens kostnader för skötseln av premiepensionssystemet ska täckas genom avgifter som dras av från tillgodohavandena på pensionsspararnas premiepensionskonton, om de inte täcks på något annat sätt.
\paragraph*{}
Pensionsmyndigheten bestämmer när avgifterna ska tas ut.
Avgifterna disponeras av myndigheten.
\subsection*{38 §}
\paragraph*{}
Avgifterna ska bestämmas så att de beräknas täcka det aktuella årets kostnader med skälig fördelning mellan pensionsspararna. Hänsyn ska då tas till under- eller överskott från det föregående året.
\subsection*{39 §}
\paragraph*{}
Avgifterna ska helt eller delvis anges som en procentsats av tillgodohavandet på varje pensionssparares premiepensionskonto.
\paragraph*{}
När procentsatsen fastställs får Pensionsmyndigheten utgå från en uppskattning av det sammanlagda värdet av tillgodohavandena på pensionsspararnas premiepensionskonton.
\subsection*{40 §}
\paragraph*{}
Pensionsmyndigheten ska ta ut avgifter från fondförvaltare för att täcka myndighetens kostnader för information om fonder.
\paragraph*{}
Bestämmelserna i 37 § andra stycket gäller även för sådana avgifter.
Lag (2022:761).
\subsection*{41 §}
\paragraph*{}
Har upphävts genom
lag (2018:772).
\subsection*{42 §}
\paragraph*{}
Regeringen får meddela föreskrifter om uttaget av avgifter för Pensionsmyndighetens kostnader.
\subsection*{43 §}
\paragraph*{}
Pensionsmyndigheten får bestämma att återkallelse av uttag av premiepension, nytt uttag av pension efter återkallelse samt ändring av den andel av pensionen som tas ut ska bekostas av pensionsspararen. Kostnaden för åtgärden ska då dras av från tillgodohavandet på pensionsspararens premiepensionskonto.
\subsection*{44 §}
\paragraph*{}
De pensionssparare som tagit emot pensionsrätt som överförts från make enligt bestämmelserna i 61 kap. 11-15 §§ ska svara för minskad arvsvinst och övriga kostnader på grund av överföringen. Detta sker genom att kostnaderna, efter skälig fördelning, dras av från tillgodohavandet på spararnas premiepensionskonton.
\paragraph*{}
Ett sådant avdrag ska beräknas lika för män och kvinnor.
\subsection*{45 §}
\paragraph*{}
Pensionsmyndigheten får ta ut avgifter från pensionssparare som genomför fondbyte.
\paragraph*{}
Avgifterna disponeras av myndigheten.
\paragraph*{}
Regeringen eller den myndighet regeringen bestämmer får meddela föreskrifter om uttaget av avgifter.
\subsection*{46 §}
\paragraph*{}
Marknadsföring eller försäljning av tjänster och produkter på premiepensionsområdet får inte ske via telefon.
\paragraph*{}
Ett avtal som har ingåtts i strid med förbudet i första stycket är ogiltigt. Pensionsspararen är då inte skyldig att betala för några tjänster eller produkter.
\paragraph*{}
Ett handlande som strider mot förbudet i första stycket ska vid tillämpning av 5, 23 och 26 §§ marknadsföringslagen (2008:486) anses som otillbörligt mot konsumenter. Ett sådant handlande kan medföra marknadsstörningsavgift enligt bestämmelserna i 29-36 §§ marknadsföringslagen.
Lag (2018:772).
\subsection*{47 §}
\paragraph*{}
Pensionsmyndigheten och Fondtorgsnämnden ska samverka i frågor som rör premiepensionssystemet.
\paragraph*{}
Myndigheterna ska lämna varandra de uppgifter som respektive myndighet behöver för sin verksamhet avseende premiepensionssystemet.
Lag (2022:761).
\chapter*{65 Innehåll och inledande bestämmelser}
\subsection*{1 §}
\paragraph*{}
I denna underavdelning finns bestämmelser om
\newline - garantipension för den som är född 1937 eller tidigare i 66 kap., och
\newline - garantipension för den som är född 1938 eller senare i 67 kap.
\subsection*{2 §}
\paragraph*{}
När det gäller garantipension likställs med gift försäkrad en försäkrad som är sambo med någon som han eller hon har varit gift med, eller har eller har haft barn med.
\subsection*{3 §}
\paragraph*{}
När det gäller garantipension likställs med ogift försäkrad en försäkrad som är gift men stadigvarande lever åtskild från sin make, om inte särskilda skäl föranleder något annat.
\chapter*{66 Garantipension för den som är född 1937 eller tidigare}
\subsection*{1 §}
\paragraph*{}
I detta kapitel finns bestämmelser om
\newline - rätten till garantipension i 2 §,
\newline - förmånsnivåer och samordning i 3 och 4 §§,
\newline - beräkningsunderlag för garantipension i 5-15 §§, och
\newline - beräkning av garantipension i 16-24 §§.
\subsection*{2 §}
\paragraph*{}
Garantipension till en försäkrad som är född 1937 eller tidigare lämnas endast om han eller hon enligt den upphävda lagen (1962:381) om allmän försäkring var berättigad till folkpension i form av ålderspension vid utgången av 2002.
\subsection*{3 §}
\paragraph*{}
Uttag av garantipension får begränsas till tre fjärdedelar, hälften eller en fjärdedel av hel pension.
\subsection*{4 §}
\paragraph*{}
Om den försäkrade har rätt till inkomstgrundad ålderspension i form av tilläggspension lämnas garantipension endast om och till den del han eller hon tar ut tilläggspension för samma tid.
\subsection*{5 §}
\paragraph*{}
Den försäkrades årliga garantipension ska beräknas på det underlag som följer av andra stycket och 6-15 §§ (beräkningsunderlaget).
\paragraph*{}
I beräkningsunderlaget ska det i förekommande fall ingå
\newline 1. inkomstgrundad ålderspension i form av tilläggspension för samma år efter samordning enligt 69 kap. 12 §,
\newline 2. änkepension,
\newline 3. svensk tjänstepension,
\newline 4. ålders-, efterlevande- och förtidspension enligt utländsk lagstiftning samt utländsk tjänstepension,
\newline 5. ett belopp motsvarande den folkpension i form av ålderspension som enligt äldre bestämmelser skulle ha lämnats till den försäkrade, till den del denna pension inte motsvaras av tilläggspension,
\newline 6. ett belopp motsvarande det pensionstillskott som skulle ha lämnats till den försäkrade enligt den upphävda lagen (1969:205) om pensionstillskott och punkten 2 i övergångsbestämmelserna till lagen (1998:705) om ändring i nämnda lag, allt i bestämmelsernas lydelse före den 1 januari 2003, och
\newline 7. andra förmåner som är att anse som folkpension i form av ålderspension.
\subsection*{6 §}
\paragraph*{}
Den tilläggspension som ingår i beräkningsunderlaget enligt 5 § ska alltid uppgå till minst vad den försäkrade skulle ha haft rätt till i form av tillläggspension enligt 63 kap. 6 § första stycket 2, om den delen av tilläggspensionen i stället för att följsamhetsindexeras enligt 63 kap. 11 § hade beräknats med tillämpning av det prisbasbelopp som gällde för det aktuella året.
\paragraph*{}
Om tilläggspensionen har samordnats enligt 69 kap. 12 §, ska bestämmelserna i första stycket inte tillämpas.
\subsection*{7 §}
\paragraph*{}
Om den försäkrade enligt 2 § lagen (2002:125) om överföring av värdet av pensionsrättigheter till och från Europeiska gemenskaperna har överfört värdet av rätt till tilläggspension och någon överföring från gemenskaperna enligt 8 § samma lag därefter inte har skett, ska garantipensionen beräknas som om någon överföring till gemenskaperna inte hade skett.
\subsection*{8 §}
\paragraph*{}
Med tjänstepension enligt 5 § andra stycket 3 och 4 avses följande:
\newline 1. pension som betalas ut på grund av sådan tjänstepensionsförsäkring som avses i 58 kap. 7 § inkomstskattelagen (1999:1229), och
\newline 2. pension som betalas ut på grund av tidigare tjänst på annat sätt än genom försäkring.
\subsection*{9 §}
\paragraph*{}
Tjänstepension ska ingå i beräkningsunderlaget med det belopp som framgår av beslutet om slutlig skatt året före det år garantipensionen avser.
\paragraph*{}
I beräkningsunderlaget ska det även ingå tjänstepension som avses i 5 § första stycket 5 och 7 lagen (1991:586) om särskild inkomstskatt för utomlands bosatta, om skattskyldighet enligt den lagen förelåg kalenderåret två år före det år garantipensionen avser.
Lag (2011:1434).
\subsection*{10 §}
\paragraph*{}
Om en sådan pension till efterlevande som avses i 9 § tillkommer senare, ska beräkningsunderlaget räknas om med hänsyn till den tillkommande pensionen. Omräkningen ska göras från och med månaden efter den månad då Pensionsmyndigheten fick kännedom om pensionen. Detsamma ska gälla när en pension som avses i 9 § tillkommer i samband med att ålderspension beviljas i form av garantipension.
\paragraph*{}
Ett tillkommande belopp enligt första stycket ska, om inte 11 § ska tillämpas, ingå i beräkningsunderlaget till dess uppgifterna vid inkomstbeskattningen avser tjänstepensionen för helt år.
Lag (2011:1434).
\subsection*{11 §}
\paragraph*{}
Upphör eller minskar tjänstepensionen ska beräkningsunderlagets belopp för tjänstepension efter anmälan av den försäkrade justeras från och med den månad då Pensionsmyndigheten fick kännedom om ändringen.
\subsection*{12 §}
\paragraph*{}
Det som anges i 9-11 §§ ska i tillämpliga delar även gälla för ålders-, efterlevande- och förtidspension enligt utländsk lagstiftning enligt 5 § andra stycket 4.
\subsection*{13 §}
\paragraph*{}
Det belopp som motsvarar folkpension enligt 5 § andra stycket 5 ska beräknas enligt
\newline 1. 5 och 6 kap. i den upphävda lagen (1962:381) om allmän försäkring,
\newline 2. punkterna 2 och 3 i övergångsbestämmelserna till lagen (1998:704) om ändring i förstnämnda lag, samt
\newline 3. punkterna 3, 4, 6 och 7 i övergångsbestämmelserna till lagen (1992:1277) om ändring i förstnämnda lag.
\paragraph*{}
Beräkningen ska grundas på den bosättningstid samt de år med pensionspoäng och därmed likställda år som ska tillgodoräknas den försäkrade för tid före den 1 januari 2003. De bestämmelser som anges i första stycket ska tillämpas i sin lydelse före nämnda dag.
\subsection*{14 §}
\paragraph*{}
När det belopp för pensionstillskott som ingår i beräkningsunderlaget enligt 5 § andra stycket 6 ska beräknas gäller följande. Sådan avräkning enligt 3 § första stycket i den upphävda lagen (1969:205) om pensionstillskott som gjordes för tilläggspension i form av ålderspension ska göras för tilläggspension enligt 63 kap. 6 §.
\paragraph*{}
Vid avräkningen ska tilläggspensionen först minskas med det belopp som den försäkrade skulle ha haft rätt till i tilläggspension enligt 63 kap. 6 § första stycket 2, om den delen av tilläggspensionen i stället för att följsamhetsindexeras enligt 63 kap. 11 § hade beräknats med tillämpning av det prisbasbelopp som gällde för det aktuella året.
\subsection*{15 §}
\paragraph*{}
Sådan avräkning som anges i 14 § ska göras för tilläggspension före samordning enligt 69 kap. 12 §.
\paragraph*{}
Om samordning har gjorts enligt 69 kap. 12 §, ska beräkningsunderlaget enligt 5 § uppgå minst till vad som enligt äldre bestämmelser skulle ha lämnats i folkpension och pensionstillskott före avräkning för tilläggspension. Vid beräkningen tillämpas 5 § andra stycket 6 och 13 §.
\subsection*{16 §}
\paragraph*{}
Garantipension ska beräknas enligt 17-21 §§ om beräkningsunderlaget överstiger 0,25 prisbasbelopp men inte överstiger
\newline - 3,16 prisbasbelopp för den som är ogift, eller
\newline - 2,8275 prisbasbelopp för den som är gift.
\paragraph*{}
I andra fall ska garantipensionen beräknas enligt 22-23 §§.
Lag (2019:651).
\subsection*{17 §}
\paragraph*{}
Om beräkningsunderlaget överstiger 0,25 prisbasbelopp men understiger 1,354 prisbasbelopp, multipliceras beräkningsunderlaget med 1,5174. Produkten minskas därefter med 0,1193 prisbasbelopp.
\subsection*{18 §}
\paragraph*{}
För den som är ogift och vars beräkningsunderlag uppgår till minst 1,354 prisbasbelopp men understiger 1,529 prisbasbelopp multipliceras beräkningsunderlaget med 1,343.
Produkten ökas därefter med 0,1168 prisbasbelopp.
\subsection*{19 §}
\paragraph*{}
För den som är ogift och vars beräkningsunderlag uppgår till minst 1,529 prisbasbelopp men inte överstiger 3,16 prisbasbelopp motsvarar beräkningsunderlaget summan av
\newline - 2,17 prisbasbelopp och
\newline - produkten av den del av beräkningsunderlaget som överstiger 1,51 prisbasbelopp och 0,60.
\subsection*{20 §}
\paragraph*{}
För den som är gift och vars beräkningsunderlag uppgår till minst 1,354 prisbasbelopp men inte överstiger 2,8275 prisbasbelopp motsvarar beräkningsunderlaget summan av - 1,935 prisbasbelopp och
\newline - produkten av den del av beräkningsunderlaget som överstiger 1,34 prisbasbelopp och 0,60.
Lag (2011:1075).
\subsection*{20 a §}
\paragraph*{}
Det beräkningsunderlag som har räknats fram enligt 17-20 §§ ska ökas med 0,3 prisbasbelopp. Ökningen ska dock beräknas på samma sätt som anges i 13 § andra stycket första meningen.
Lag (2022:1031).
\subsection*{21 §}
\paragraph*{}
Den årliga garantipensionen motsvarar differensen mellan
\newline - det beräkningsunderlag som räknats upp enligt 17-20 a §§ och
\newline - summan av sådan tilläggspension, tjänstepension, änkepension, utländsk pension och sådana andra förmåner som enligt 5 § andra stycket 7 har ingått i beräkningsunderlaget.
Lag (2019:651).
\subsection*{22 §}
\paragraph*{}
För den vars beräkningsunderlag inte överstiger 0,25 prisbasbelopp gäller följande:
\paragraph*{}
Den årliga garantipensionen motsvarar differensen mellan
\newline - beräkningsunderlaget som först har multiplicerats med 1,04 och därefter har ökats med 0,3 prisbasbelopp som i sin tur har beräknats på samma sätt som anges i 13 § andra stycket första meningen och
\newline - summan av sådan tilläggspension, tjänstepension, änkepension, utländsk pension och sådana andra förmåner som enligt 5 § andra stycket 7 har ingått i beräkningsunderlaget.
Lag (2022:1031).
\subsection*{22 a §}
\paragraph*{}
För den som är ogift och vars beräkningsunderlag överstiger
\paragraph*{}
3,16 prisbasbelopp eller för den som är gift och vars beräkningsunderlag överstiger 2,8275 prisbasbelopp ska beräkningsunderlaget ökas med 0,3 prisbasbelopp. Ökningen ska dock beräknas på samma sätt som anges i 13 § andra stycket första meningen.
\paragraph*{}
För den som det i beräkningsunderlaget inte ingår något belopp enligt 5 § andra stycket 5 ska det belopp som enligt första stycket har ökat beräkningsunderlaget reduceras med 0,4 multiplicerat med den del av beräkningsunderlaget enligt 5 § som överstiger 3,16 prisbasbelopp för ogifta eller 2,8275 prisbasbelopp för gifta.
Lag (2022:1031).
\subsection*{23 §}
\paragraph*{}
För den vars beräkningsunderlag har beräknats enligt 22 a § gäller följande:
\paragraph*{}
Den årliga garantipensionen motsvarar differensen mellan
\newline - beräkningsunderlaget och
\newline - summan av sådan tilläggspension, tjänstepension, änkepension, utländsk pension och sådana andra förmåner som enligt 5 § andra stycket 7 har ingått i beräkningsunderlaget.
Lag (2019:651).
\subsection*{24 §}
\paragraph*{}
Partiell garantipension utgör en så stor andel av garantipension beräknad enligt 21-23 §§ vid helt uttag som svarar mot den andel som tas ut.
\chapter*{67 Garantipension för den som är född 1938 eller senare}
\subsection*{1 §}
\paragraph*{}
I detta kapitel finns bestämmelser om
\newline - rätten till garantipension i 2 §,
\newline - samordning och uttag av garantipension i 3 och 4 §§,
\newline - försäkringstiden i 5, 6, 10-12 och 14 §§,
\newline - beräkningsunderlag för garantipension i 15-20 §§, och
\newline - beräkning av garantipension i 21-26 §§.
Lag (2022:1266).
\subsection*{2 §}
\paragraph*{}
Garantipension till en försäkrad som är född 1938 eller senare kan lämnas om han eller hon enligt 5, 6, 10-12 och 14 §§ har en försäkringstid om minst tre år.
Lag (2022:1266).
\subsection*{3 §}
\paragraph*{}
Garantipension till en försäkrad som har rätt till inkomstgrundad ålderspension i form av inkomstpension eller tilläggspension lämnas endast i den utsträckning som han eller hon tar ut sådan pension för samma tid.
Garantipensionen kan då begränsas till tre fjärdedelar, hälften eller en fjärdedel av hel pension. I annat fall får garantipension inte tas ut i andelar.
\subsection*{4 §}
\paragraph*{}
/Upphör att gälla U:2025-12-01/
Garantipension får tas ut tidigast från och med den månad då den försäkrade fyller 66 år.
Lag (2022:878).
\subsection*{4 §}
\paragraph*{}
/Träder i kraft I:2025-12-01/
Garantipension får tas ut tidigast från och med den månad då den försäkrade uppnår riktåldern för pension.
Lag (2022:879).
\subsection*{5 §}
\paragraph*{}
Som försäkringstid för garantipension räknas tid under vilken en person ska anses ha varit bosatt i Sverige enligt 5 kap.
\subsection*{6 §}
\paragraph*{}
Som försäkringstid för garantipension räknas även tid under vilken en pensionssökande före tidpunkten för bosättning enligt 5 § oavbrutet har vistats i Sverige efter att ha ansökt om uppehållstillstånd.
\subsection*{7 §}
\paragraph*{}
Har upphävts genom
lag (2022:1266).
\subsection*{8 §}
\paragraph*{}
Har upphävts genom
lag (2022:1266).
\subsection*{8 §}
\paragraph*{}
/Träder i kraft I:2025-12-01/
Vid beräkningen enligt 7 § ska en så stor andel av tiden i hemlandet tillgodoräknas som svarar mot kvoten mellan
\newline - den tid under vilken den pensionssökande har varit bosatt i Sverige, inräknad den tid som avses i 6 §, från den första ankomsten till landet till och med kalenderåret före det år då han eller hon uppnådde riktåldern för pension och
\newline - hela tidsrymden från det att den pensionssökande första gången kom till Sverige till och med året före det kalenderår då han eller hon uppnådde riktåldern för pension.
\paragraph*{}
Vid beräkningen ska det bortses från tid för vilken den pensionssökande, vid bosättning i Sverige, har rätt till pension från hemlandet.
Lag (2022:879).
\subsection*{9 §}
\paragraph*{}
Har upphävts genom
lag (2022:1266).
\subsection*{10 §}
\paragraph*{}
Vid beräkning av försäkringstid för garantipension för biståndsarbetare m.fl. som enligt 5 kap. 6 § anses bosatta i Sverige även under vistelse utomlands ska bortses från tid för vilken den utsände, vid bosättning i Sverige, har rätt till sådan pension från det andra landet som inte enligt 15-19 §§ ska ligga till grund för beräkning av garantipension.
\paragraph*{}
/Rubriken upphör att gälla U:2025-12-01/
\subsection*{11 §}
\paragraph*{}
/Upphör att gälla U:2025-12-01/
Vid beräkning av försäkringstid beaktas endast tid från och med det kalenderår då den pensionssökande har uppnått 16 års ålder till och med det kalenderår då han eller hon har fyllt 65 år.
Lag (2022:878).
\paragraph*{}
/Träder i kraft I:2025-12-01/
Vid beräkning av försäkringstid beaktas endast tid från och med det kalenderår då den pensionssökande har uppnått 16 års ålder till och med året före det kalenderår då han eller hon uppnår riktåldern för pension.
Lag (2022:879).
\subsection*{12 §}
\paragraph*{}
Tid från och med det kalenderår då den pensionssökande har fyllt 16 år till och med det kalenderår då han eller hon har fyllt 24 år räknas som försäkringstid endast om det för den pensionssökande för samma tid har fastställts pensionsgrundande inkomst eller pensionsgrundande belopp för sjukersättning eller aktivitetsersättning eller för förtidspension eller sjukbidrag enligt den upphävda lagen (1962:381) om allmän försäkring i dess lydelse före den 1 januari 2003, som per kalenderår uppgår till lägst ett inkomstbasbelopp enligt 58 kap. 26 och 27 §§.
\subsection*{13 §}
\paragraph*{}
Har upphävts genom
lag (2022:1266).
\subsection*{13 §}
\paragraph*{}
/Träder i kraft I:2025-12-01/
För den som omedelbart före den kalendermånad då han eller hon uppnådde riktåldern för pension hade rätt till hel sjukersättning i form av garantiersättning gäller följande. Om det är förmånligare än en beräkning enligt 5-12 §§, räknas som försäkringstid för garantipension den försäkringstid som legat till grund för beräkning av sjukersättningen.
Lag (2022:879).
\subsection*{14 §}
\paragraph*{}
Den sammanlagda försäkringstiden för garantipension ska vid beräkningen sättas ned till närmaste hela antal år.
\subsection*{15 §}
\paragraph*{}
Till grund för beräkning av garantipension enligt 21-25 §§ ska ligga den inkomstgrundade ålderspension som den försäkrade har rätt till för samma år med de ändringar och tillägg som anges i 16-20 §§ (beräkningsunderlaget).
\subsection*{16 §}
\paragraph*{}
Med inkomstgrundad ålderspension enligt 15 § avses
\newline - inkomstgrundad ålderspension enligt denna balk före minskning som anges i 69 kap. 2-11 §§, och
\newline - sådan allmän obligatorisk ålderspension enligt utländsk lagstiftning som inte är att likställa med garantipension enligt denna balk.
\paragraph*{}
Inkomstpension och premiepension ska beräknas som om den försäkrade endast hade tillgodoräknats pensionsrätt för inkomstpension och som om denna pensionsrätt, före minskning enligt 61 kap. 10 §, hade utgjort 18,5 procent av pensionsunderlaget.
Lag (2019:644).
\subsection*{16 a §}
\paragraph*{}
Har upphävts genom
lag (2019:646).
\subsection*{17 §}
\paragraph*{}
I beräkningsunderlaget ska också ingå
\newline - änkepension,
\newline - efterlevandepension enligt utländsk lagstiftning, och
\newline - sådana förmåner som lämnas till den försäkrade enligt utländsk lagstiftning och som motsvarar sjukersättning eller aktivitetsersättning eller som utgör pension vid invaliditet.
Lag (2019:644).
\subsection*{17 a §}
\paragraph*{}
Har upphävts genom
lag (2019:646).
\subsection*{18 §}
\paragraph*{}
När pensionsrätt eller pensionspoäng till följd av bristande eller underlåten avgiftsbetalning inte har tillgodoräknats eller har minskats för den försäkrade ska, vid tillämpning av 15 och 16 §§, hänsyn tas till den inkomstpension eller tilläggspension som skulle ha lämnats om full avgift hade betalats.
\subsection*{19 §}
\paragraph*{}
Om den försäkrade enligt 2 § lagen (2002:125) om överföring av värdet av pensionsrättigheter till och från Europeiska gemenskaperna har överfört värdet av pensionsrätt för inkomstpension, pensionsrätt för premiepension och värdet av rätt till tilläggspension, och någon överföring från gemenskaperna enligt 8 § samma lag därefter inte har skett, ska beräkningsunderlaget bestämmas som om någon överföring inte hade skett.
\subsection*{20 §}
\paragraph*{}
Om den försäkrade är född 1953 eller tidigare och har rätt till inkomstpension eller tilläggspension ska beräkningsunderlaget alltid bestämmas som om inkomstpensionen och tilläggspensionen hade börjat lyftas den kalendermånad då den försäkrade fyllde 65 år.
\paragraph*{}
Om sådant garantitillägg lämnas som avses i 63 kap. 18 och 19 §§ och inkomstpension eller tilläggspension har tagits ut före den kalendermånad då den försäkrade fyllde 65 år ska, när beräkningsunderlaget bestäms, detta tillägg ökas på motsvarande sätt som hel årlig tilläggspension har minskats enligt 63 kap. 20 §.
Lag (2022:878).
\subsection*{20 a §}
\paragraph*{}
Om den försäkrade är född något av åren 1954-1957 och har rätt till inkomstpension, ska beräkningsunderlaget alltid bestämmas som om inkomstpensionen hade börjat lyftas den kalendermånad då den försäkrade fyllde 65 år.
Lag (2022:878).
\paragraph*{}
/Rubriken upphör att gälla U:2025-12-01/
\subsection*{20 b §}
\paragraph*{}
/Upphör att gälla U:2025-12-01/
Om den försäkrade är född 1958 eller senare och har rätt till inkomstpension, ska beräkningsunderlaget alltid bestämmas som om inkomstpensionen hade börjat lyftas den kalendermånad då den försäkrade fyller 66 år.
Lag (2022:878).
\subsection*{20 b §}
\paragraph*{}
/Träder i kraft I:2025-12-01/
Om den försäkrade är född 1958 eller 1959 och har rätt till inkomstpension, ska beräkningsunderlaget alltid bestämmas som om inkomstpensionen hade börjat lyftas den kalendermånad då den försäkrade fyller 66 år.
Lag (2022:879).
\paragraph*{}
/Rubriken träder i kraft I:2025-12-01/
\subsection*{20 c §}
\paragraph*{}
/Träder i kraft I:2025-12-01/
Om den försäkrade är född 1960 eller senare och har rätt till inkomstpension ska beräkningsunderlaget alltid bestämmas som om inkomstpensionen hade börjat lyftas den kalendermånad då den försäkrade uppnår riktåldern för pension.
Lag (2022:879).
\subsection*{21 §}
\paragraph*{}
För den som är ogift och vars beräkningsunderlag inte överstiger 1,26 prisbasbelopp gäller följande.
\paragraph*{}
Den årliga garantipensionen är 2,43 prisbasbelopp (basnivån för ogifta) minskat med beräkningsunderlaget.
Lag (2022:1031).
\subsection*{22 §}
\paragraph*{}
För den som är ogift och vars beräkningsunderlag överstiger 1,26 prisbasbelopp gäller följande.
\paragraph*{}
Den årliga garantipensionen är 1,17 prisbasbelopp minskat med 48 procent av den del av beräkningsunderlaget som överstiger 1,26 prisbasbelopp.
Lag (2022:1031).
\subsection*{23 §}
\paragraph*{}
För den som är gift och vars beräkningsunderlag inte överstiger 1,14 prisbasbelopp gäller följande.
\paragraph*{}
Den årliga garantipensionen är 2,2 prisbasbelopp (basnivån för gifta) minskat med beräkningsunderlaget.
Lag (2022:1031).
\subsection*{24 §}
\paragraph*{}
För den som är gift och vars beräkningsunderlag överstiger 1,14 prisbasbelopp gäller följande.
\paragraph*{}
Den årliga garantipensionen är 1,06 prisbasbelopp minskat med 48 procent av den del av beräkningsunderlaget som överstiger 1,14 prisbasbelopp.
Lag (2022:1031).
\subsection*{25 §}
\paragraph*{}
För den som inte kan tillgodoräknas 40 års försäkringstid för garantipension ska samtliga de prisbasbeloppsanknutna belopp som anges i 21-24 §§ avkortas till så stor andel som svarar mot kvoten mellan
\newline - försäkringstiden och
\newline - talet 40.
\subsection*{26 §}
\paragraph*{}
Partiell garantipension utgör en så stor andel av garantipension beräknad enligt 21-25 §§ vid helt uttag som svarar mot den andel som tas ut.
V Vissa gemensamma bestämmelser om allmän ålderspension
\chapter*{68 Innehåll}
\subsection*{1 §}
\paragraph*{}
I denna underavdelning finns bestämmelser om
\newline - samordning av allmän ålderspension och yrkesskadelivränta i 69 kap.,
\newline - ändring av beräkningsunderlag för allmän ålderspension i 70 kap., och
\newline - utbetalning av allmän ålderspension i 71 kap.
\chapter*{69 Samordning av allmän ålderspension och yrkesskadelivränta}
\subsection*{1 §}
\paragraph*{}
I detta kapitel finns bestämmelser om
\newline - när samordning ska ske i 2-5 §§, och
\newline - hur samordning ska ske i 6-12 §§.
\subsection*{2 §}
\paragraph*{}
Inkomstpension, tilläggspension eller garantipension ska minskas om den försäkrade har rätt till livränta (yrkesskadelivränta) på grund av obligatorisk försäkring enligt följande upphävda lagar:
\newline 1. lagen (1916:235) om försäkring för olycksfall i arbete,
\newline 2. lagen (1929:131) om försäkring för vissa yrkessjukdomar, och
\newline 3. lagen (1954:243) om yrkesskadeförsäkring.
\paragraph*{}
Detsamma gäller om den försäkrade
\newline 1. enligt någon annan författning eller enligt särskilt beslut av regeringen har rätt till annan livränta, som bestäms eller betalas ut av Försäkringskassan, eller
\newline 2. får livränta enligt utländsk lagstiftning om yrkesskadeförsäkring.
\subsection*{3 §}
\paragraph*{}
Inkomstpension, tilläggspension eller garantipension ska inte minskas på grund av livränta på grund av arbetsskada eller annan skada som avses i 41-44 kap.
\subsection*{4 §}
\paragraph*{}
Om samordning ska göras enligt 12 §, ska garantipensionen inte minskas enligt bestämmelserna i 6-11 §§.
\subsection*{5 §}
\paragraph*{}
Om en skada som livränta har börjat lämnas för återigen medför sjukdom som berättigar till sjukpenning, ska det anses som om livränta hade lämnats under sjukdomstiden.
\subsection*{6 §}
\paragraph*{}
Summan av inkomstpension, tilläggspension och garantipension ska minskas med tre fjärdedelar av varje livränta som överstiger en sjättedel av prisbasbeloppet och som den försäkrade får som skadad.
\paragraph*{}
Minskningen ska i första hand göras på garantipensionen och därefter på tilläggspensionen.
\subsection*{7 §}
\paragraph*{}
Avdrag på inkomstpension och på tilläggspension enligt 63 kap. 6 § första stycket 1 får göras endast om den försäkrade kunnat tillgodoräkna sig pensionspoäng för minst ett år när skadan inträffade.
\subsection*{8 §}
\paragraph*{}
Garantipension ska på motsvarande sätt som anges i 6 § minskas med livränta som den försäkrade får som efterlevande i annat fall än som avses i 87 och 88 kap.
Livränta med engångsbelopp 9 § Har livränta eller del av livränta eller livränta för viss tid bytts ut mot ett engångsbelopp, ska det vid beräkningen anses som om livränta lämnas eller som om den livränta som lämnas har höjts med ett belopp som motsvarar engångsbeloppet enligt de försäkringstekniska grunder som tillämpades vid utbytet.
\subsection*{10 §}
\paragraph*{}
Summan av hel inkomstgrundad ålderspension och hel garantipension får aldrig, på grund av bestämmelserna i 2 och 6-8 §§, för månad räknat understiga
\newline - 13 procent av prisbasbeloppet för den som är ogift, och
\newline - 11,5 procent av prisbasbeloppet för den som är gift.
\subsection*{11 §}
\paragraph*{}
Vid beräkningen enligt 10 § ska bestämmelserna i 63 kap. 9 § tillämpas.
\subsection*{12 §}
\paragraph*{}
För en försäkrad som är född 1937 eller tidigare, och som i december 2002 fick såväl ålderspension som sådan yrkesskadelivränta såsom skadad som avses i 2 §, ska tilläggspensionen minskas enligt följande.
\paragraph*{}
Minskningen ska ske med ett belopp som motsvarar det belopp varmed den försäkrades ålderspension i december 2002 minskades för livränta såsom skadad med stöd av 17 kap. 2 § i den upphävda lagen (1962:381) om allmän försäkring. Om livräntan minskas efter utgången av 2002, ska det belopp som minskar tilläggspensionen minskas i motsvarande mån.
\chapter*{70 Ändring av beräkningsunderlag för allmän ålderspension}
\subsection*{1 §}
\paragraph*{}
I detta kapitel finns bestämmelser om
\newline - inkomstgrundad ålderspension i 2 och 3 §§,
\newline - överförd pensionsrätt för premiepension i 4 §, och
\newline - garantipension i 5 §.
\subsection*{2 §}
\paragraph*{}
Ett beslut om inkomstgrundad ålderspension ska ändras om det föranleds av en ändring som gjorts i beräkningsunderlaget för beslutet eller av en ändring av ett beslut enligt skatteförfarandelagen (2011:1244) som avser egenavgift.
Detsamma gäller om det efter ändring av beslutet om slutlig skatt står klart att pensionsrätt för premiepension, till följd av bestämmelserna i 61 kap. 8 §, inte skulle ha tillgodoräknats.
\paragraph*{}
Ett beslut om pensionsrätt och pensionspoäng ska ändras också när det följer av bestämmelserna i 61 kap. 9 eller 21 § om obetalda avgifter.
Lag (2011:1434).
\subsection*{3 §}
\paragraph*{}
Den som ett beslut om ändring enligt 2 § avser ska skriftligen underrättas om ändringen och om hur man begär omprövning av beslutet.
Överförd pensionsrätt för premiepension
\subsection*{4 §}
\paragraph*{}
Överförd pensionsrätt för premiepension ska ändras om ett beslut för den som överfört pensionsrätten har ändrats och detta föranleder en ändring av den överförda pensionsrätten.
\subsection*{5 §}
\paragraph*{}
Ett beslut om garantipension för viss tid för den som är född 1937 eller tidigare ska ändras av Pensionsmyndigheten, om det föranleds av en ändring som gjorts i fråga om den pension eller de belopp för samma tid som ingått i beräkningsunderlaget enligt 66 kap. 5-15 §§.
Ett beslut om garantipension för viss tid för den som är född 1938 eller senare ska ändras av Pensionsmyndigheten, om det föranleds av en ändring som gjorts i fråga om den inkomstgrundade ålderspension, änkepension eller utländska förmån för samma tid som legat till grund för beräkningen av garantipensionen.
\chapter*{71 Utbetalning av allmän ålderspension}
\subsection*{1 §}
\paragraph*{}
I detta kapitel finns bestämmelser om
\newline - utbetalningsperiod i 2 §, och
\newline - avrundning i 3-5 §§.
\subsection*{2 §}
\paragraph*{}
Allmän ålderspension ska betalas ut månadsvis.
\paragraph*{}
Årspension som beräknas uppgå till högst 2 400 kronor ska dock, om det inte finns särskilda skäl, betalas ut i efterskott en eller två gånger per år. Efter överenskommelse med den försäkrade får pensionen även i andra fall betalas ut en eller två gånger per år.
\subsection*{4 §}
\paragraph*{}
Vid beräkning enligt 3 § ska pensioner som betalas ut samtidigt anses som en enda pension.
Om inkomst- eller tilläggspension betalas ut samtidigt med garantipension, ska avrundningen enligt 3 § göras på garantipensionen.
\subsection*{5 §}
\paragraph*{}
Premiepension avrundas månadsvis till närmaste hela krontal, varvid 50 öre avrundas uppåt.
\chapter*{72 Innehåll}
\subsection*{1 §}
\paragraph*{}
I denna underavdelning finns bestämmelser om
\newline - särskilt pensionstillägg i 73 kap.,
\newline - äldreförsörjningsstöd i 74 kap., och
\newline - inkomstpensionstillägg i 74 a kap.
Lag (2020:1239).
\chapter*{73 Särskilt pensionstillägg Innehåll}
\subsection*{1 §}
\paragraph*{}
I detta kapitel finns en inledande bestämmelse i 2 §.
\paragraph*{}
Vidare finns bestämmelser om
\newline - vårdår i 3 §,
\newline - vem som likställs med förälder i 4 §,
\newline - rätten till särskilt pensionstillägg i 5 och 6 §§,
\newline - tillgodoräkning av vårdår i 7-10 §§,
\newline - beräkning av särskilt pensionstillägg i 11-15 §§, och
\newline - likställande med garantipension i 16 §.
\subsection*{2 §}
\paragraph*{}
Särskilt pensionstillägg kan lämnas som tillägg till ålderspension, om en förälder före 1999 långvarigt vårdat ett sjukt eller funktionshindrat barn.
\subsection*{3 §}
\paragraph*{}
Med ett vårdår avses i detta kapitel ett kalenderår före 1999 om
\newline 1. en förälder under större delen av året har vårdat ett sjukt eller funktionshindrat barn, och
\newline 2. barnet under större delen av året har fått hel förtidspension eller helt sjukbidrag tillsammans med handikappersättning enligt den upphävda lagen (1962:381) om allmän försäkring i dess lydelse före den 1 januari 2001.
\subsection*{4 §}
\paragraph*{}
När det gäller särskilt pensionstillägg likställs med förälder den som under vårdåren
\newline 1. har varit särskilt förordnad vårdnadshavare för barnet,
\newline 2. har varit blivande adoptivförälder,
\newline 3. har varit förälderns make och stadigvarande sammanbott med föräldern, eller
\newline 4. har varit sambo med föräldern och haft barn med föräldern.
\paragraph*{}
Den som var särskilt förordnad vårdnadshavare för barnet när barnet fyllde 18 år eller vid den tidigare tidpunkt när barnet ingick äktenskap ska för tid därefter likställas med särskilt förordnad vårdnadshavare enligt första stycket 1.
\subsection*{5 §}
\paragraph*{}
Särskilt pensionstillägg kan lämnas till den som är född 1953 eller tidigare och har rätt till
\newline - inkomstpension,
\newline - tilläggspension, eller
\newline - garantipension.
\subsection*{6 §}
\paragraph*{}
För rätt till särskilt pensionstillägg förutsätts att föräldern har avstått från förvärvsarbete för att vårda ett sjukt eller funktionshindrat barn under minst sex vårdår.
\subsection*{7 §}
\paragraph*{}
Högst femton vårdår får läggas till grund för beräkningen av pensionstillägget.
\subsection*{8 §}
\paragraph*{}
En förälders rätt att tillgodoräknas vårdår påverkas inte av avbrott i vården under den del av dygnet då barnet vistats i förskola, skola, dagcenter eller liknande, eller av kortare avbrott i övrigt i vården som varit utan väsentlig betydelse för förälderns bundenhet till vårduppgiften.
\subsection*{9 §}
\paragraph*{}
Som vårdår räknas inte år för vilket en förälder
\newline 1. under större delen av kalenderåret har fått förtidspension eller sjukbidrag enligt den upphävda lagen (1962:381) om allmän försäkring i dess lydelse före den 1 januari 2001,
\newline 2. har tillgodoräknats pensionspoäng enligt den upphävda lagen om allmän försäkring i dess lydelse före den 1 januari 1999 eller pensionsrätt enligt den upphävda lagen (1998:675) om införande av lagen (1998:674) om inkomstgrundad ålderspension,
\newline 3. inte har tillgodoräknats pensionspoäng på grund av bristande eller underlåten avgiftsbetalning, eller
\newline 4. skulle ha tillgodoräknats pensionspoäng, om inte undantagande från försäkringen för tilläggspension enligt 11 kap. 7 § i den upphävda lagen om allmän försäkring i dess lydelse före den 1 januari 1982 hade gällt för honom eller henne.
\subsection*{10 §}
\paragraph*{}
För år efter det år då en förälder har fyllt 64 år får vårdår inte tillgodoräknas honom eller henne. Detsamma gäller för år då föräldern före den månad när han eller hon fyller 65 år har tagit ut ålderspension enligt 6 kap. i den upphävda lagen (1962:381) om allmän försäkring i dess lydelse före den 1 januari 2001.
\subsection*{11 §}
\paragraph*{}
För sex tillgodoräknade vårdår är det särskilda pensionstillägget per kalenderår 5 procent av prisbasbeloppet. För varje vårdår som tillgodoräknas därutöver ökar tillägget med 5 procent av prisbasbeloppet.
\subsection*{12 §}
\paragraph*{}
För den som börjar få särskilt pensionstillägg före den månad då han eller hon fyller 65 år minskas tillägget enligt 11 § med 0,5 procent för varje månad som, när tillägget börjar lämnas, återstår till den månad när han eller hon fyller 65 år.
\subsection*{13 §}
\paragraph*{}
För den som börjar få särskilt pensionstillägg efter den månad då han eller hon fyller 65 år ökas tillägget enligt 11 § med 0,7 procent för varje månad som, när tillägget börjar lämnas, förflutit från den månad när han eller hon fyllt 65 år, dock längst till och med den månad när han eller hon fyller 70 år.
\subsection*{14 §}
\paragraph*{}
För den som är född 1937 eller tidigare och som inte skulle ha tillgodoräknats oavkortad folkpension enligt bestämmelserna i 5 kap. 3 § andra stycket och 5 § fjärde stycket i den upphävda lagen (1962:381) om allmän försäkring i deras lydelse före den 1 januari 2003 gäller följande.
\paragraph*{}
Det särskilda pensionstillägget avkortas till så stor andel som svarar mot kvoten mellan
\newline - det antal år som pensionspoäng eller bosättningstid skulle ha tillgodoräknats för enligt de bestämmelser som avses i första stycket och
\newline - talet 30 när det gäller pensionspoäng respektive talet 40 när det gäller bosättningstid.
\subsection*{15 §}
\paragraph*{}
För den som är född 1938 eller senare och som inte kan tillgodoräknas 40 års försäkringstid för garantipension gäller följande.
\paragraph*{}
Det särskilda pensionstillägget avkortas till så stor andel som svarar mot kvoten mellan
\newline - försäkringstiden för garantipension och
\newline - talet 40.
\subsection*{16 §}
\paragraph*{}
Om inte annat följer av denna balk ska det som i balken eller i annan författning föreskrivs om garantipension tillämpas på särskilt pensionstilllägg. Detta ska dock inte gälla i fråga om samordning med sådan livränta som avses i 69 kap. Även i fråga om inkomstskatt gäller särskilda föreskrifter.
\chapter*{74 Äldreförsörjningsstöd}
\subsection*{1 §}
\paragraph*{}
I detta kapitel finns inledande bestämmelser i 2 §.
\paragraph*{}
Vidare finns bestämmelser om
\newline - sambor och makar i 3 och 4 §§,
\newline - rätten till äldreförsörjningsstöd i 5 och 6 §§,
\newline - förmånstiden i 7-9 §§,
\newline - beräkning av äldreförsörjningsstöd i 10-18 §§,
\newline - omprövning vid ändrade förhållanden i 19 och 20 §§, och
\newline - utbetalning av äldreförsörjningsstöd i 21 §.
\subsection*{2 §}
\paragraph*{}
Äldreförsörjningsstöd kan lämnas till en försäkrad som inte får sina grundläggande försörjningsbehov tillgodosedda genom allmän ålderspension.
\paragraph*{}
Rätten till äldreförsörjningsstöd respektive stödets storlek är beroende av den försäkrades inkomster och, om han eller hon är gift, även av makens inkomster.
\subsection*{3 §}
\paragraph*{}
Sambor likställs med makar när det gäller äldreförsörjningsstöd. Om det på grund av omständigheterna är sannolikt att två personer är sambor, ska dessa likställas med sambor. Detta gäller inte om den som ansöker om äldreförsörjningsstöd eller den som sådant bidrag betalas ut till visar att de inte är sambor.
\subsection*{4 §}
\paragraph*{}
När det gäller äldreförsörjningsstöd ska en person som är gift men stadigvarande lever åtskild från sin make likställas med en ogift person, om inte särskilda skäl talar mot detta.
\subsection*{5 §}
\paragraph*{}
Den försäkrade ska genom äldreförsörjningsstödet tillförsäkras medel för att täcka en skälig bostadskostnad och uppnå en skälig levnadsnivå i övrigt.
\subsection*{6 §}
\paragraph*{}
Om den försäkrade har rätt till inkomstgrundad ålderspension, garantipension, bostadstillägg eller särskilt bostadstillägg, lämnas äldreförsörjningsstöd endast om den försäkrade tar ut samtliga sådana förmåner som han eller hon är berättigad till för samma tid.
\subsection*{7 §}
\paragraph*{}
/Upphör att gälla U:2025-12-01/
Äldreförsörjningsstöd kan lämnas tidigast från och med den månad då den försäkrade fyller 66 år.
\paragraph*{}
Äldreförsörjningsstöd lämnas från och med den månad då rätt till förmånen har inträtt, dock tidigast från och med ansökningsmånaden.
Lag (2022:878).
\subsection*{7 §}
\paragraph*{}
/Träder i kraft I:2025-12-01/
Äldreförsörjningsstöd kan lämnas tidigast från och med den månad då den försäkrade uppnår riktåldern för pension.
\paragraph*{}
Äldreförsörjningsstöd lämnas från och med den månad då rätt till förmånen har inträtt, dock tidigast från och med ansökningsmånaden.
Lag (2022:879).
\subsection*{8 §}
\paragraph*{}
Stödet kan lämnas under högst tolv sammanhängande månader på grundval av de uppgifter som den försäkrade har lämnat i sin ansökan. För stöd under tid därefter krävs att den försäkrade ger in en ny ansökan.
\subsection*{9 §}
\paragraph*{}
Äldreförsörjningsstöd lämnas till och med den månad då rätten till förmånen har upphört.
\subsection*{10 §}
\paragraph*{}
Äldreförsörjningsstöd ska motsvara differensen mellan
\newline - summan av beloppet för skälig bostadskostnad och beloppet för skälig levnadsnivå i övrigt enligt 15 §, och
\newline - den försäkrades inkomster enligt 16-18 §§.
\subsection*{11 §}
\paragraph*{}
För den som är ogift anses som skälig bostadskostnad högst 7 500 kronor i månaden.
Lag (2021:1244).
\subsection*{12 §}
\paragraph*{}
För den som är gift anses som skälig bostadskostnad högst 3 750 kronor i månaden. Bostadskostnaden för var och en av makarna ska beräknas till hälften av den sammanlagda bostadskostnaden.
Lag (2021:1244).
\subsection*{13 §}
\paragraph*{}
För var och en av de boende i tvåbäddsrum i särskild boendeform anses som skälig bostadskostnad högst 2 850 kronor i månaden.
\subsection*{14 §}
\paragraph*{}
Vid bestämmande av skälig bostadskostnad beaktas inte sådana kostnader som enligt bestämmelserna om bostadstillägg i 101 kap. 7 och 8 §§ inte medför rätt till bostadstillägg.
\subsection*{15 §}
\paragraph*{}
Skälig levnadsnivå i övrigt anses per månad motsvara en tolftedel av
\newline - 1,5357 prisbasbelopp för den som är ogift, och
\newline - 1,2353 prisbasbelopp för den som är gift.
Lag (2021:1244).
\subsection*{16 §}
\paragraph*{}
Med inkomst avses den inkomst, för år räknat, som en försäkrad kan antas komma att få under den närmaste tiden.
\paragraph*{}
Den försäkrades inkomster beräknas enligt bestämmelserna om särskilt bostadstillägg i 102 kap. 29 §.
\subsection*{17 §}
\paragraph*{}
Som inkomst räknas även
\newline - bostadsbidrag, och
\newline - särskilt bostadstillägg.
\subsection*{18 §}
\paragraph*{}
En makes inkomst respektive förmögenhet anses vara hälften av makarnas sammanlagda inkomst och förmögenhet.
\subsection*{19 §}
\paragraph*{}
Äldreförsörjningsstöd ska omprövas när något förhållande som påverkar stödets storlek har ändrats.
\paragraph*{}
Äldreförsörjningsstöd får räknas om utan föregående underrättelse om den del av årsinkomsten ändras som utgörs av
\newline - en förmån som betalas ut av Försäkringskassan eller Pensionsmyndigheten,
\newline - pension enligt utländsk lagstiftning,
\newline - avtalspension eller motsvarande ersättning som följer av kollektivavtal, eller
\newline - överskott i inkomstslaget kapital som avses i 102 kap. 7 § 3.
\paragraph*{}
Det som föreskrivs i andra stycket gäller också när ändring sker av
\newline - sådant belopp som avses i 102 kap. 17-19 §§,
\newline - preliminärt bostadsbidrag enligt 98 kap., eller
\newline - taxeringsvärde för annan fastighet än sådan som avses i 5 § lagen (2009:1053) om förmögenhet vid beräkning av vissa förmåner.
\paragraph*{}
Vid omräkning av äldreförsörjningsstödet med stöd av andra och tredje styckena får en ny beräkning av stödet göras utifrån enbart den ändring som ligger till grund för omräkningen.
Lag (2017:554).
\subsection*{20 §}
\paragraph*{}
Ändring av äldreförsörjningsstöd ska gälla från och med månaden närmast efter den månad när anledningen till ändring uppkom. Gäller det ökning ska även 7 § andra stycket beaktas.
\paragraph*{}
En ändring av äldreförsörjningsstödet ska dock gälla från och med den månad under vilken de förhållanden har uppkommit som föranleder ändringen, om förhållandena avser hela den månaden.
\paragraph*{}
Om Pensionsmyndigheten har hämtat in uppgifter som avses i 102 kap. 7 § 3 direkt från Skatteverket för omräkning av äldreförsörjningsstödet utan föregående underrättelse med stöd av 19 §, ska en ändring av äldreförsörjningsstödet, i stället för vad som följer av första och andra styckena, gälla från och med månaden efter den månad då Pensionsmyndigheten har fått uppgifterna från Skatteverket.
Lag (2017:554).
\subsection*{21 §}
\paragraph*{}
Äldreförsörjningsstödet betalas ut månadsvis.
Årsbeloppet ska avrundas till närmaste hela krontal som är delbart med tolv.
\chapter*{74 a Tillägg till inkomstgrundad ålderspension}
\subsection*{1 §}
\paragraph*{}
I detta kapitel finns en inledande bestämmelse i 2 §.
\paragraph*{}
Vidare finns bestämmelser om
\newline - rätten till inkomstpensionstillägg i 3 §,
\newline - försäkringstid i 4-6 §§,
\newline - uttag av inkomstpensionstillägg m.m. i 7 och 8 §§,
\newline - beräkningsunderlag för inkomstpensionstillägg i 9-12 §§,
\newline - beräkning av inkomstpensionstillägg i 13-22 §§,
\newline - ändring av beräkningsunderlag för inkomstpensionstillägg i 23 §, och
\newline - likställande med allmän ålderspension vid utbetalning i 24 §.
Lag (2020:1239).
\subsection*{2 §}
\paragraph*{}
Ett inkomstpensionstillägg kan lämnas till den inkomstgrundade ålderspensionen enligt 55 kap. 5 och 6 §§.
Lag (2020:1239).
\subsection*{3 §}
\paragraph*{}
Inkomstpensionstillägg kan lämnas till en försäkrad som har en försäkringstid enligt 4-6 §§ om minst ett år och som tar ut inkomstgrundad ålderspension.
Lag (2020:1239).
\subsection*{4 §}
\paragraph*{}
Som försäkringstid för inkomstpensionstillägg för en försäkrad som är född 1938 eller senare räknas kalenderår för vilket det har fastställts pensionsgrundande inkomst enligt 59 kap. Försäkringstiden ska dock endast räknas fram till och med det andra året före det år inkomstpensionstillägget avser.
Lag (2020:1239).
\subsection*{5 §}
\paragraph*{}
Som försäkringstid för inkomstpensionstillägg för en försäkrad som är född 1937 eller tidigare räknas kalenderår för vilket han eller hon har tillgodoräknats pensionspoäng för tilläggspension enligt 61 kap.
Lag (2020:1239).
\subsection*{6 §}
\paragraph*{}
För en försäkrad som inte är svensk medborgare ska år före 1974, för vilka sjömansskatt har betalats, likställas med år för vilka pensionspoäng har beräknats vid bedömningen av försäkringstiden för inkomstpensionstillägg.
Lag (2020:1239).
\subsection*{7 §}
\paragraph*{}
/Upphör att gälla U:2025-12-01/
Inkomstpensionstillägg lämnas tidigast från och med den månad då den försäkrade fyller 66 år.
Lag (2022:878).
\subsection*{7 §}
\paragraph*{}
/Träder i kraft I:2025-12-01/
Inkomstpensionstillägg lämnas tidigast från och med den månad då den försäkrade uppnår riktåldern för pension.
Lag (2022:879).
\subsection*{8 §}
\paragraph*{}
/Upphör att gälla U:2025-12-01/
Det som föreskrivs om allmän ålderspension i 56 kap. i fråga om förmånstid i 4, 5 och 7 §§, ökat uttag i 9 §, återkallelse och minskning av uttag i 10 § och omräkning i 13 § ska även gälla inkomstpensionstillägg.
\paragraph*{}
Inkomstpensionstillägg får lämnas utan ansökan från och med den månad den försäkrade fyller 66 år om den försäkrade fick hel sjukersättning omedelbart före den månaden.
Lag (2022:878).
\subsection*{8 §}
\paragraph*{}
/Träder i kraft I:2025-12-01/
Det som föreskrivs om allmän ålderspension i 56 kap. i fråga om förmånstid i 4, 5 och 7 §§, ökat uttag i 9 §, återkallelse och minskning av uttag i 10 § och omräkning i 13 § ska även gälla inkomstpensionstillägg.
\paragraph*{}
Inkomstpensionstillägg får lämnas utan ansökan från och med den månad den försäkrade uppnår riktåldern för pension om den försäkrade fick hel sjukersättning omedelbart före den månaden.
Lag (2022:879).
\subsection*{9 §}
\paragraph*{}
Till grund för beräkning av inkomstpensionstillägget enligt 13-22 §§ ska ligga den inkomstgrundade ålderspension som den försäkrade har rätt till för samma år med de ändringar och tillägg som anges i 10-12 §§ (beräkningsunderlaget).
Lag (2020:1239).
\subsection*{10 §}
\paragraph*{}
Med inkomstgrundad ålderspension enligt 9 § avses inkomstgrundad ålderspension enligt denna balk före sådan minskning som anges i 69 kap. 2-12 §§.
\paragraph*{}
Inkomstpension och premiepension ska beräknas som om den försäkrade endast hade tillgodoräknats pensionsrätt för inkomstpension och som om denna pensionsrätt, före sådan minskning enligt 61 kap. 10 §, hade utgjort 18,5 procent av pensionsunderlaget.
Lag (2020:1239).
\subsection*{11 §}
\paragraph*{}
Arbetsskadelivränta som betalas ut enligt 4 kap. 40 § lagen (2010:111) om införande av socialförsäkringsbalken ska ingå i beräkningsunderlaget.
Lag (2020:1239).
\subsection*{12 §}
\paragraph*{}
/Upphör att gälla U:2025-12-01/
Om den försäkrade har rätt till inkomstpension eller tilläggspension ska beräkningsunderlaget alltid bestämmas som om inkomstpensionen och tilläggspensionen hade börjat lyftas den kalendermånad då den försäkrade fyller 66 år. Om sådant garantitillägg lämnas som avses i 63 kap. 18 och 19 §§ och inkomstpension eller tilläggspension har tagits ut före den kalendermånad då den försäkrade fyllde 66 år ska, när beräkningsunderlaget bestäms, detta tillägg ökas på motsvarande sätt som hel årlig tilläggspension har minskats enligt 63 kap. 20 §.
\paragraph*{}
Vidare ska
\newline - det som föreskrivs om garantipension i fråga om obetalda avgifter i 67 kap. 18 § tillämpas på inkomstpensionstillägg för en försäkrad som är född 1938 eller senare, och
\newline - det som föreskrivs i 66 kap. 7 § och 67 kap. 19 § om garantipension i fråga om överföring av pensionsrättigheter enligt lagen (2002:125) om överföring av värdet av pensionsrättigheter till och från Europeiska gemenskaperna tillämpas på inkomstpensionstillägg.
Lag (2022:878).
\subsection*{12 §}
\paragraph*{}
/Träder i kraft I:2025-12-01/
Om den försäkrade har rätt till inkomstpension eller tilläggspension ska beräkningsunderlaget alltid bestämmas som om inkomstpensionen och tilläggspensionen hade börjat lyftas den kalendermånad då den försäkrade uppnår riktåldern för pension. Om sådant garantitillägg lämnas som avses i 63 kap. 18 och 19 §§ och inkomstpension eller tilläggspension har tagits ut före den kalendermånad då den försäkrade uppnått riktåldern för pension ska, när beräkningsunderlaget bestäms, detta tillägg ökas på motsvarande sätt som hel årlig tilläggspension har minskats enligt 63 kap. 20 §.
\paragraph*{}
Vidare ska
\newline - det som föreskrivs om garantipension i fråga om obetalda avgifter i 67 kap. 18 § tillämpas på inkomstpensionstillägg för en försäkrad som är född 1938 eller senare, och
\newline - det som föreskrivs i 66 kap. 7 § och 67 kap. 19 § om garantipension i fråga om överföring av pensionsrättigheter enligt lagen (2002:125) om överföring av värdet av pensionsrättigheter till och från Europeiska gemenskaperna tillämpas på inkomstpensionstillägg.
Lag (2022:879).
\subsection*{13 §}
\paragraph*{}
Inkomstpensionstillägg lämnas med följande förmånsbelopp per år i kronor för kalenderår 2021. För efterföljande år räknas inkomstgränserna om på det sätt som anges i 14-16 §§.
\paragraph*{}
Fastställt beräkningsunderlag Förmånsbelopp i
i kronor kronor
Nedre gräns Övre gräns
108 000 109 999 300
110 000 111 999 900
112 000 113 999 1 500
114 000 115 999 2 100
116 000 117 999 2 700
118 000 119 999 3 300
120 000 121 999 3 900
122 000 123 999 4 500
124 000 125 999 5 100
126 000 127 999 5 700
128 000 129 999 6 300
130 000 131 999 6 900
132 000 167 999 7 200
168 000 170 999 6 900
171 000 173 999 6 300
174 000 176 999 5 700
177 000 179 999 5 100
180 000 182 999 4 500
183 000 185 999 3 900
186 000 188 999 3 300
189 000 191 999 2 700
192 000 194 999 2 100
195 000 197 999 1 500
198 000 200 999 900
201 000 203 999 300
Lag (2020:1239).
\subsection*{14 §}
\paragraph*{}
För kalenderåret 2022 och följande år ska inkomstgränserna i 13 § räknas om genom indexering vid varje årsskifte.
Lag (2020:1239).
\subsection*{15 §}
\paragraph*{}
Indexeringen beräknas som produkten av
\newline - den inkomstgräns som gäller före årsskiftet och
\newline - kvoten mellan inkomstindexet efter årsskiftet och inkomstindexet före årsskiftet, sedan den nämnda kvoten dividerats med talet 1,016.
\paragraph*{}
De framräknade inkomstgränserna enligt första stycket avrundas till hela kronor.
\paragraph*{}
För år då balansindex fastställs ska beräkningen göras med hänsyn till detta index i stället för inkomstindexet.
Lag (2020:1239).
\subsection*{16 §}
\paragraph*{}
Regeringen eller den myndighet som regeringen bestämmer kan med stöd av 8 kap. 7 § regeringsformen meddela närmare föreskrifter om inkomstgränserna.
Lag (2020:1239).
\subsection*{17 §}
\paragraph*{}
För en försäkrad född 1945 eller senare som inte kan tillgodoräknas 40 års försäkringstid för inkomstpensionstillägg ska förmånsbeloppet avkortas till så stor andel som svarar mot kvoten mellan
\newline - försäkringstiden, och
\newline - talet 40.
Lag (2020:1239).
\subsection*{18 §}
\paragraph*{}
För en försäkrad född under något av åren 1938-1944 som inte kan tillgodoräknas 35 års försäkringstid för inkomstpensionstillägg ska förmånsbeloppet avkortas till så stor andel som svarar mot kvoten mellan
\newline - försäkringstiden, och
\newline - talet 35.
Lag (2020:1239).
\subsection*{19 §}
\paragraph*{}
För en försäkrad född under något av åren 1924-1937 som inte kan tillgodoräknas 30 års försäkringstid för inkomstpensionstillägg ska förmånsbeloppet avkortas till så stor andel som svarar mot kvoten mellan
\newline - försäkringstiden, och
\newline - talet 30.
Lag (2020:1239).
\subsection*{20 §}
\paragraph*{}
För en försäkrad född under något av åren 1915-1923 som inte kan tillgodoräknas försäkringstid för inkomstpensionstillägg motsvarande 20 år
\paragraph*{}
ökat med 1 för varje helt år från och med 1915 till utgången av den försäkrades födelseår, ska förmånsbeloppet avkortas till så stor andel som svarar mot kvoten mellan
\newline - försäkringstiden, och
\newline - talet 20 ökat med 1 för varje år från och med 1915 till utgången av den försäkrades födelseår.
Lag (2020:1239).
\subsection*{21 §}
\paragraph*{}
För en försäkrad född 1914 eller tidigare som inte kan tillgodoräknas 20 års försäkringstid för inkomstpensionstillägg ska förmånsbeloppet avkortas till så stor andel som svarar mot kvoten mellan
\newline - försäkringstiden, och
\newline - talet 20.
Lag (2020:1239).
\subsection*{22 §}
\paragraph*{}
Uttag av inkomstpensionstillägg får endast göras i samma utsträckning som den försäkrade tar ut inkomstpension eller tilläggspension för samma tid.
\paragraph*{}
Partiellt inkomstpensionstillägg utgör en så stor andel av inkomstpensionstillägget beräknat enligt 13-21 §§ vid helt uttag som svarar mot den andel som tas ut.
Lag (2020:1239).
\subsection*{23 §}
\paragraph*{}
Ett beslut om inkomstpensionstillägg ska ändras av Pensionsmyndigheten om det föranleds av en ändring som har gjorts i fråga om den inkomstgrundade ålderspensionen eller den försäkringstid som har legat till grund för beräkningen av inkomstpensionstillägget.
\paragraph*{}
Den försäkrade ska skriftligen underrättas om ändringen och om möjligheten att begära omprövning av beslutet.
Lag (2020:1239).
\subsection*{24 §}
\paragraph*{}
Det som föreskrivs om utbetalningsperiod om allmän ålderspension i 71 kap. 2 § och om inkomst- och tilläggspension i 71 kap. 3 och 4 §§ ska tillämpas på inkomstpensionstillägg.
Lag (2020:1239).
\part*{F FÖRMÅNER TILL EFTERLEVANDE}
\chapter*{75 Innehåll, definitioner och förklaringar}
\subsection*{1 §}
\paragraph*{}
I avdelning F finns bestämmelser om socialförsäkringsförmåner till efterlevande (efterlevandeförmåner).
\subsection*{2 §}
\paragraph*{}
Efterlevandeförmåner enligt denna avdelning är
\newline 1. efterlevandepension i form av
\newline - barnpension,
\newline - omställningspension till efterlevande make (änka eller änkling),
\newline - änkepension till efterlevande maka (änka), och
\newline - garantipension till omställningspension,
\newline 2. efterlevandestöd som garantiförmån till barn,
\newline 3. efterlevandeförmåner vid arbetsskada och vissa andra skador i form av
\newline - begravningshjälp,
\newline - barnlivränta, och
\newline - omställningslivränta till efterlevande make, samt
\newline 4. efterlevandeskydd i form av premiepension, om pensionsspararen har ansökt om ett sådant skydd.
\subsection*{3 §}
\paragraph*{}
I detta kapitel finns inledande bestämmelser om efterlevandeförmåner.
\paragraph*{}
Vidare finns bestämmelser om
\newline - efterlevandepension och efterlevandestöd i 76-85 kap.,
\newline - efterlevandeförmåner vid arbetsskada och vissa andra skador i 86-88 kap., och
\newline - efterlevandeskydd i form av premiepension i 89, 91 och 92 kap.
\subsection*{4 §}
\paragraph*{}
En förmån enligt denna avdelning lämnas endast efter en avliden som var försäkrad för förmånen enligt 4 och 6 kap.
eller till den som själv har ett gällande försäkringsskydd för förmånen enligt 4 och 5 kap.
\paragraph*{}
Bestämmelser om anmälan och ansökan samt vissa gemensamma bestämmelser om förmåner och handläggning finns i 104-117 kap. (avdelning H).
\subsection*{5 §}
\paragraph*{}
Ärenden som avser förmåner enligt denna avdelning handläggs av Pensionsmyndigheten.
\chapter*{76 Innehåll}
\subsection*{1 §}
\paragraph*{}
I denna underavdelning finns allmänna bestämmelser om efterlevandepension och efterlevandestöd i 77 kap.
\paragraph*{}
Vidare finns bestämmelser om
\newline - barnpension i 78 kap.,
\newline - efterlevandestöd i 79 kap.,
\newline - omställningspension i 80 kap.,
\newline - garantipension till omställningspension i 81 kap.,
\newline - beräkningsunderlag för inkomstgrundad efterlevandepension i 82 kap.,
\newline - änkepension i 83 kap., och
\newline - beräkning av änkepension i 84 kap.
\paragraph*{}
Slutligen finns vissa gemensamma bestämmelser i 85 kap.
\chapter*{77 Allmänna bestämmelser om efterlevandepension och efterlevandestöd}
\subsection*{1 §}
\paragraph*{}
I detta kapitel finns inledande bestämmelser i 2-6 §§.
\paragraph*{}
Vidare finns bestämmelser om
\newline - rätten till efterlevandepension och efterlevandestöd i 7-10 §§,
\newline - samordning av omställningspension och änkepension i 11 och 12 §§, och
\newline - förmånstiden i 13-18 §§.
\subsection*{2 §}
\paragraph*{}
Efterlevandepension som är relaterad till de inkomster som den avlidne har haft lämnas i form av
\newline - barnpension,
\newline - omställningspension, och
\newline - änkepension.
\subsection*{3 §}
\paragraph*{}
Till efterlevande barn kan som ett bosättningsbaserat grundskydd lämnas efterlevandestöd.
\paragraph*{}
Till efterlevande vuxna kan som ett bosättningsbaserat grundskydd lämnas efterlevandepension i form av garantipension till omställningspension.
\paragraph*{}
Garantipensionen är även beroende av den avlidnes försäkringstid.
\subsection*{4 §}
\paragraph*{}
Barnpension och omställningspension lämnas som inkomstgrundad efterlevandepension enligt 78 respektive 80 kap. Denna beräknas på ett underlag enligt 82 kap.
\subsection*{5 §}
\paragraph*{}
Änkepension lämnas som en inkomstrelaterad tilläggspension enligt 83 kap. Denna beräknas enligt 84 kap.
\subsection*{6 §}
\paragraph*{}
Efterlevandestöd lämnas enligt 79 kap. som tillägg till eller i stället för barnpension.
\paragraph*{}
Garantipension till omställningspension lämnas enligt 81 kap.
som tillägg till eller i stället för omställningspension.
\subsection*{7 §}
\paragraph*{}
Efterlevandepension och efterlevandestöd kan lämnas till ett barn vars ena eller båda föräldrar har avlidit.
\paragraph*{}
Efterlevandepension till en vuxen kan lämnas till en man eller kvinna vars make har avlidit.
\subsection*{8 §}
\paragraph*{}
Vem som ska vara försäkrad och andra villkor för rätten till en förmån anges särskilt för varje förmån i 78-81 och 83 kap.
\subsection*{9 §}
\paragraph*{}
Rätt till barnpension, omställningspension eller änkepension föreligger endast om ett inkomstrelaterat underlag kan beräknas för den avlidne enligt 82 eller 84 kap.
\subsection*{10 §}
\paragraph*{}
Om någon har försvunnit och det kan antas att han eller hon har avlidit, gäller samma rätt till efterlevandepension eller efterlevandestöd som vid dödsfall. Visar det sig senare att den försvunna personen lever, upphör rätten till förmånen.
\subsection*{11 §}
\paragraph*{}
En efterlevande kvinna som är född 1945 eller senare får i första hand omställningspension och garantipension till sådan pension enligt 80 och 81 kap. Hon har dock, om förutsättningarna för det är uppfyllda, också rätt till änkepension enligt 83 kap.
\subsection*{12 §}
\paragraph*{}
Om en efterlevande kvinna för samma månad har rätt till såväl änkepension som omställningspension eller garantipension till omställningspension, ska endast omställningspension och garantipension till omställningspension lämnas.
\paragraph*{}
Om änkepensionen är större än summan av omställningspensionen och garantipensionen till omställningspensionen, lämnas dock änkepensionen i den utsträckning den överstiger omställningspensionen och garantipensionen till sådan pension. Minskning ska i första hand göras på 90- procentstillägget enligt 84 kap. 6 § 2.
\subsection*{13 §}
\paragraph*{}
Efterlevandepension och efterlevandestöd får lämnas utan ansökan.
Lag (2012:599).
\subsection*{14 §}
\paragraph*{}
Efterlevandepension och efterlevandestöd lämnas från och med den månad då dödsfallet inträffat. Efterlevandepension och efterlevandestöd lämnas dock från och med månaden efter månaden för dödsfallet om den avlidne vid sin död fick allmän ålderspension, sjukersättning eller aktivitetsersättning.
\subsection*{15 §}
\paragraph*{}
Om rätten till efterlevandepension eller efterlevandestöd inträder vid en annan tidpunkt än den som avses i 14 §, ska förmånen lämnas från och med den månad när rätten inträder.
\subsection*{16 §}
\paragraph*{}
En annan efterlevandepension än barnpension får inte lämnas för längre tid tillbaka än tre månader före ansökningsmånaden. Barnpension får inte lämnas för längre tid tillbaka än två år före ansökningsmånaden.
\paragraph*{}
Efterlevandestöd får inte lämnas för längre tid tillbaka än en månad före ansökningsmånaden.
Lag (2019:1294).
\subsection*{17 §}
\paragraph*{}
Ändring av efterlevandepension och efterlevandestöd ska gälla från och med månaden efter den månad då anledning till ändringen uppkom.
\subsection*{18 §}
\paragraph*{}
/Upphör att gälla U:2025-12-01/
Efterlevandepension och efterlevandestöd lämnas till och med den månad då rätten till förmånen upphör. Dock lämnas följande förmåner längst till och med månaden före den då den efterlevande fyller 66 år:
\newline 1. omställningspension,
\newline 2. garantipension till omställningspension, och
\newline 3. änkepension i form av 90-procentstillägg.
Lag (2022:878).
\subsection*{18 §}
\paragraph*{}
/Träder i kraft I:2025-12-01/
Efterlevandepension och efterlevandestöd lämnas till och med den månad då rätten till förmånen upphör. Dock lämnas följande förmåner längst till och med månaden före den då den efterlevande uppnår riktåldern för pension:
\newline 1. omställningspension,
\newline 2. garantipension till omställningspension, och
\newline 3. änkepension i form av 90-procentstillägg.
Lag (2022:879).
\chapter*{78 Barnpension}
\subsection*{1 §}
\paragraph*{}
I detta kapitel finns bestämmelser om
\newline - rätten till barnpension i 2-6 §§, och
\newline - beräkning av barnpension i 7-14 §§.
\subsection*{2 §}
\paragraph*{}
Ett barn har rätt till barnpension om barnets ena eller båda föräldrar har avlidit.
\paragraph*{}
Detta gäller om den avlidne var försäkrad för sådan förmån enligt 4 och 6 kap.
\subsection*{3 §}
\paragraph*{}
Om en blivande adoptivförälder vårdar ett barn som inte är svensk medborgare och som inte var bosatt här i landet när han eller hon fick det i sin vård ska barnet, vid tillämpningen av bestämmelserna om barnpension, anses som barn till den föräldern.
\subsection*{4 §}
\paragraph*{}
Rätten till barnpension upphör när barnet fyller 18 år, om inte något annat följer av 5 och 6 §§.
\subsection*{5 §}
\paragraph*{}
Ett barn som har fyllt 18 år och som bedriver studier har rätt till barnpension även för studietiden, dock längst till och med juni det år då barnet fyller 20 år. Detta gäller dock endast om studierna ger rätt till
\newline 1. förlängt barnbidrag enligt 15 kap., eller
\newline 2. studiehjälp enligt 2 kap. studiestödslagen (1999:1395).
\paragraph*{}
Undervisning som omfattar kortare tid än åtta veckor ger inte rätt till barnpension.
\subsection*{6 §}
\paragraph*{}
Med tid för studier enligt 5 § ska likställas
\newline - tid för ferier, och
\newline - tid då barnet på grund av sjukdom inte kan bedriva sina studier.
\subsection*{7 §}
\paragraph*{}
Barnpension efter en förälder till ett enda barn som inte har fyllt 12 år motsvarar för år räknat 35 procent av underlaget enligt 82 kap.
\subsection*{8 §}
\paragraph*{}
Har flera barn, varav det yngsta inte har fyllt 12 år, rätt till barnpension efter samma förälder, ska det procenttal som anges i 7 § ökas med talet 25 för varje barn utöver det första oavsett dess ålder. Det sammanlagda barnpensionsbeloppet ska därefter fördelas lika mellan barnen.
\subsection*{9 §}
\paragraph*{}
Barnpension efter en förälder till ett enda barn som har fyllt 12 år motsvarar för år räknat 30 procent av underlaget enligt 82 kap.
\subsection*{10 §}
\paragraph*{}
Har flera barn, som alla har fyllt 12 år, rätt till barnpension efter samma förälder, ska det procenttal som anges i 9 § ökas med talet 20 för varje barn utöver det första. Det sammanlagda barnpensionsbeloppet ska därefter fördelas lika mellan barnen.
\subsection*{11 §}
\paragraph*{}
Om ett barns båda föräldrar har avlidit motsvarar barnpensionen efter vardera föräldern 35 procent av underlaget enligt 82 kap. även om barnet har fyllt 12 år.
\paragraph*{}
Är i sådant fall som avses i första stycket flera barn berättigade till barnpension efter samma förälder, ska det procenttal som anges i första stycket ökas med talet 25 för varje barn utöver det första. Det sammanlagda barnpensionsbeloppet ska därefter fördelas lika mellan barnen.
\subsection*{12 §}
\paragraph*{}
Om flera barnpensioner ska lämnas efter en avliden förälder och dessa pensioner, beräknade enligt 8 eller 10 § eller 11 § andra stycket, sammantagna skulle överstiga underlaget enligt 82 kap., ska pensionerna sättas ned proportionellt så att de tillsammans motsvarar det underlaget.
\subsection*{13 §}
\paragraph*{}
De sammanlagda barnpensionerna efter en avliden förälder ska lämnas med högst 80 procent av underlaget enligt 82 kap., om det efter föräldern förutom barnpension ska lämnas omställningspension eller änkepension.
\paragraph*{}
Om barnpensionerna sammantaget överstiger den högsta nivån enligt första stycket, ska de sättas ned proportionellt så att de tillsammans motsvarar den nivån.
\subsection*{14 §}
\paragraph*{}
Vid tillämpning av 12 eller 13 § får inte barnpensionen för ett barn sänkas på den grunden att barnpensionen för ett annat barn ska räknas om enligt 11 §.
\chapter*{79 Efterlevandestöd}
\subsection*{1 §}
\paragraph*{}
I detta kapitel finns bestämmelser om
\newline - rätten till efterlevandestöd i 2 §, och
\newline - beräkning av efterlevandestöd i 3-7 §§.
\subsection*{2 §}
\paragraph*{}
Ett barn som är försäkrat enligt 4 och 5 kap. och som uppfyller villkoren för rätt till barnpension i 78 kap. 2 § första stycket och 3-6 §§ har rätt till efterlevandestöd.
\subsection*{3 §}
\paragraph*{}
Efterlevandestöd till ett barn motsvarar för år räknat 40 procent av prisbasbeloppet.
\paragraph*{}
Om barnets båda föräldrar har avlidit, motsvarar efterlevandestödet 40 procent av prisbasbeloppet efter vardera föräldern.
\subsection*{4 §}
\paragraph*{}
Om barnet får barnpension enligt 78 kap. efter den avlidne föräldern, lämnas efterlevandestöd endast i den utsträckning pensionen inte uppgår till 40 procent av prisbasbeloppet. Har båda föräldrarna avlidit lämnas efterlevandestöd i den utsträckning pensionen efter föräldrarna sammanlagt inte uppgår till 80 procent av prisbasbeloppet.
\subsection*{5 §}
\paragraph*{}
Vid tillämpning av 4 § ska med barnpension likställas barnlivränta enligt 87 och 88 kap., efter samordning enligt 88 kap. 18-20 §§.
\subsection*{6 §}
\paragraph*{}
Vid tillämpning av 4 § ska med barnpension likställas sådan efterlevandepension som barnet har rätt till enligt utländsk lagstiftning. Det gäller dock inte utländsk efterlevandepension som är att likställa med efterlevandestöd enligt detta kapitel.
\subsection*{7 §}
\paragraph*{}
Om ett barn för samma månad har rätt till såväl efterlevandestöd som sjukersättning i form av garantiersättning eller aktivitetsersättning i form av garantiersättning, lämnas endast den till beloppet största av förmånerna.
Lag (2016:1291).
\chapter*{80 Omställningspension}
\subsection*{1 §}
\paragraph*{}
I detta kapitel finns bestämmelser om
\newline - förmånsformer i 2 §,
\newline - rätten till allmän omställningspension i 3-5 §§,
\newline - rätten till förlängd omställningspension i 6-8 §§,
\newline - förmåner efter flera avlidna i 9 §, och
\newline - beräkning av omställningspension i 10 §.
\subsection*{2 §}
\paragraph*{}
Inkomstgrundad efterlevandepension i form av omställningspension kan lämnas som
\newline - allmän omställningspension, och
\newline - förlängd omställningspension.
\paragraph*{}
Efterlevandepension i form av garantipension till omställningspension kan enligt 81 kap. lämnas som tillägg till eller i stället för omställningspension.
\subsection*{3 §}
\paragraph*{}
En efterlevande make, som stadigvarande sammanbodde med sin make vid dennes död, har rätt till allmän omställningspension om denne var försäkrad för sådan förmån enligt 4 och 6 kap. och den efterlevande
\newline 1. vid dödsfallet stadigvarande sammanbodde med barn under 18 år, som stod under vårdnad av makarna eller en av dem, eller
\newline 2. oavbrutet hade sammanbott med maken under en tid av minst fem år fram till dödsfallet.
\subsection*{4 §}
\paragraph*{}
När det gäller allmän omställningspension likställs med efterlevande make en ogift person som är sambo med en annan ogift person vid hans eller hennes död och som
\newline 1. tidigare har varit gift med den avlidne,
\newline 2. har eller har haft barn med den avlidne, eller
\newline 3. vid dödsfallet väntade barn med den avlidne.
\subsection*{5 §}
\paragraph*{}
Allmän omställningspension kan lämnas under tolv månader från tidpunkten för dödsfallet.
\subsection*{6 §}
\paragraph*{}
En efterlevande som har haft rätt till allmän omställningspension har, för tid efter det att rätten till denna upphört, rätt till förlängd omställningspension om han eller hon har vårdnaden om och stadigvarande sammanbor med barn under 18 år som vid dödsfallet stadigvarande vistades i makarnas hem.
\subsection*{7 §}
\paragraph*{}
Rätt till förlängd omställningspension föreligger inte
\newline 1. om den efterlevande gifter sig, eller
\newline 2. om den efterlevande är sambo med någon som han eller hon har varit gift med eller har eller har haft barn med.
\subsection*{8 §}
\paragraph*{}
Förlängd omställningspension kan lämnas under tolv månader eller till den senare tidpunkt när det yngsta barnet fyller tolv år.
\subsection*{9 §}
\paragraph*{}
Är någon för samma månad berättigad till omställningspension efter flera avlidna, lämnas endast pensionen jämte garantipension efter den sist avlidne eller, om pensionsbeloppen efter de avlidna är olika höga, det högsta beloppet.
\subsection*{10 §}
\paragraph*{}
Omställningspension motsvarar för år räknat 55 procent av det underlag som anges i 82 kap.
\chapter*{81 Garantipension till omställningspension}
\subsection*{1 §}
\paragraph*{}
I detta kapitel finns bestämmelser om
\newline - rätten till garantipension till omställningspension i 2 och 3 §§,
\newline - försäkringstid för den avlidne i 4, 5, 7 och 8 §§, och
\newline - beräkning av garantipension till omställningspension i 9-12 §§.
Lag (2022:1266).
\subsection*{2 §}
\paragraph*{}
En efterlevande make som är försäkrad enligt 4 och 5 kap. och som uppfyller villkoren för rätt till omställningspension har rätt till garantipension till pensionen.
\paragraph*{}
Det som föreskrivs i första stycket gäller även om kravet i 80 kap. 3 § på försäkring enligt 4 och 6 kap. för den avlidne inte är uppfyllt men endast om en försäkringstid om minst tre år kan tillgodoräknas för denne enligt 4, 5, 7 och 8 §§.
Lag (2022:1266).
\subsection*{3 §}
\paragraph*{}
Garantipension till omställningspension lämnas under samma tid som omställningspension.
\subsection*{4 §}
\paragraph*{}
/Upphör att gälla U:2025-12-01/
Garantipension till omställningspension ska beräknas med hänsyn till den försäkringstid som kan tillgodoräknas för den avlidne till och med året före dödsfallet (faktisk försäkringstid).
\paragraph*{}
Som försäkringstid för garantipension till omställningspension ska tid tillgodoräknas även från och med det år då dödsfallet inträffade till och med det år då den avlidne skulle ha fyllt 65 år (framtida försäkringstid).
Lag (2022:878).
\subsection*{4 §}
\paragraph*{}
/Träder i kraft I:2025-12-01/
Garantipension till omställningspension ska beräknas med hänsyn till den försäkringstid som kan tillgodoräknas för den avlidne till och med året före dödsfallet (faktisk försäkringstid).
\paragraph*{}
Som försäkringstid för garantipension till omställningspension ska tid tillgodoräknas även från och med det år då dödsfallet inträffade till och med året före det år då den avlidne skulle ha uppnått den vid dödsfallet gällande riktåldern för pension (framtida försäkringstid).
Lag (2022:879).
\subsection*{5 §}
\paragraph*{}
Som faktisk försäkringstid ska sådan tid räknas som utgör försäkringstid för garantipension enligt 67 kap. 5, 6, 10 och 11 §§.
Lag (2022:1266).
\subsection*{6 §}
\paragraph*{}
Har upphävts genom
lag (2022:1266).
\subsection*{7 §}
\paragraph*{}
/Upphör att gälla U:2025-12-01/
Om det för den avlidne inte kan tillgodoräknas försäkringstid med minst fyra femtedelar av tiden från och med det år då han eller hon fyllde 16 år till och med året före dödsfallet gäller följande.
\paragraph*{}
Den framtida försäkringstiden ska beräknas som produkten av
\newline - tidsperioden från och med året för dödsfallet till och med det år då den avlidne skulle ha fyllt 65 år, och
\newline - kvoten mellan den försäkringstid som kan tillgodoräknas den avlidne från och med det år han eller hon fyllde 16 år till och med året före dödsfallet och fyra femtedelar av hela den faktiska tidsperioden från och med det år då han eller hon fyllde 16 år till och med året före dödsfallet.
\paragraph*{}
Vid beräkningen ska försäkringstiden sättas ned till närmaste hela antal månader.
Lag (2022:878).
\subsection*{7 §}
\paragraph*{}
/Träder i kraft I:2025-12-01/
Om det för den avlidne inte kan tillgodoräknas försäkringstid med minst fyra femtedelar av tiden från och med det år då han eller hon fyllde 16 år till och med året före dödsfallet gäller följande.
\paragraph*{}
Den framtida försäkringstiden ska beräknas som produkten av
\newline - tidsperioden från och med året för dödsfallet till och med året före det år då den avlidne skulle ha uppnått den vid dödsfallet gällande riktåldern för pension, och
\newline - kvoten mellan den försäkringstid som kan tillgodoräknas den avlidne från och med det år han eller hon fyllde 16 år till och med året före dödsfallet och fyra femtedelar av hela den faktiska tidsperioden från och med det år då han eller hon fyllde 16 år till och med året före dödsfallet.
\paragraph*{}
Vid beräkningen ska försäkringstiden sättas ned till närmaste hela antal månader.
Lag (2022:879).
\subsection*{8 §}
\paragraph*{}
Den sammanlagda försäkringstiden för garantipension ska sättas ned till närmaste hela antal år.
\subsection*{9 §}
\paragraph*{}
Beräkningen av garantipension ska grunda sig på den omställningspension som den efterlevande har rätt till.
\paragraph*{}
Med omställningspension avses även sådan efterlevandepension enligt utländsk lagstiftning som inte kan likställas med garantipension enligt detta kapitel.
Lag (2019:644).
\subsection*{9 a §}
\paragraph*{}
Har upphävts genom
lag (2019:646).
\subsection*{10 §}
\paragraph*{}
Den årliga garantipensionen till omställningspension motsvarar 2,13 prisbasbelopp (basnivån), om inte något annat anges i 12 §.
\subsection*{11 §}
\paragraph*{}
Basnivån ska minskas med omställningspension om den försäkrade har rätt till sådan pension.
\paragraph*{}
Minskning enligt första stycket ska göras efter det att basnivån avkortats enligt 12 §.
\subsection*{12 §}
\paragraph*{}
Om det för den avlidne inte kan tillgodoräknas 40 års försäkringstid för garantipension enligt 4, 5, 7 och 8 §§ gäller följande.
\paragraph*{}
Garantipensionens basnivå ska avkortas till så stor andel som svarar mot kvoten mellan
\newline - försäkringstiden och
\newline - talet 40.
Lag (2022:1266).
\chapter*{82 Beräkningsunderlag för inkomstgrundad efterlevandepension}
\subsection*{1 §}
\paragraph*{}
/Upphör att gälla U:2025-12-01/
I detta kapitel finns inledande bestämmelser i 2 och 3 §§.
\paragraph*{}
Vidare finns bestämmelser om
\newline - efterlevandepensionsunderlag efter avlidna som var födda 1938 eller senare i 4-8 §§,
\newline - pensionsbehållning efter avlidna som var födda 1938 eller senare och som avlidit före 66 års ålder i 9-19 §§,
\newline - pensionsbehållning efter avlidna som var födda 1938 eller senare och som avlidit efter 66 års ålder i 20 och 21 §§, och
\newline - efterlevandepensionsunderlag efter avlidna som var födda 1937 eller tidigare i 22-26 §§.
Lag (2022:878).
\subsection*{1 §}
\paragraph*{}
/Träder i kraft I:2025-12-01/
I detta kapitel finns inledande bestämmelser i 2 och 3 §§.
\paragraph*{}
Vidare finns bestämmelser om
\newline - efterlevandepensionsunderlag efter avlidna som var födda 1938 eller senare i 4-8 §§,
\newline - pensionsbehållning efter avlidna som var födda 1938 eller senare och som avlidit före det år då de skulle ha uppnått den vid dödsfallet gällande riktåldern för pension i 9-19 §§,
\newline - pensionsbehållning efter avlidna som var födda 1938 eller senare och som avlidit efter att ha uppnått riktåldern för pension i 20 och 21 §§, och
\newline - efterlevandepensionsunderlag efter avlidna som var födda 1937 eller tidigare i 22-26 §§.
Lag (2022:879).
\subsection*{2 §}
\paragraph*{}
Den inkomstgrundade efterlevandepensionen beräknas på ett efterlevandepensionsunderlag.
\subsection*{3 §}
\paragraph*{}
Efterlevandepensionsunderlaget bestäms på olika sätt beroende på om den avlidne var född
\newline - 1938 eller senare, eller
\newline - 1937 eller tidigare.
\subsection*{4 §}
\paragraph*{}
/Upphör att gälla U:2025-12-01/
Till grund för beräkning av efterlevandepension efter avlidna som var födda 1938 eller senare ska läggas den avlidnes
\newline - faktiska pensionsbehållning, och
\newline - antagna pensionsbehållning.
\paragraph*{}
Pensionsbehållning beräknas på olika sätt beroende på om dödsfallet inträffat före eller efter 66 års ålder.
Lag (2022:878).
\subsection*{4 §}
\paragraph*{}
/Träder i kraft I:2025-12-01/
Till grund för beräkning av efterlevandepension efter avlidna som var födda 1938 eller senare ska läggas den avlidnes
\newline - faktiska pensionsbehållning, och
\newline - antagna pensionsbehållning.
\paragraph*{}
Pensionsbehållning beräknas på olika sätt beroende på om dödsfallet inträffat före eller efter uppnådd riktålder för pension.
Lag (2022:879).
\subsection*{5 §}
\paragraph*{}
Med faktisk pensionsbehållning avses den avlidnes pensionsbehållning för inkomstpension, beräknad antingen enligt 9-13 §§ eller enligt 20 och 21 §§.
\subsection*{6 §}
\paragraph*{}
/Upphör att gälla U:2025-12-01/
Om den avlidne har tagit ut inkomstpension före den månad då han eller hon fyllt eller skulle ha fyllt 66 år (tidigt uttag) gäller vid beräkning av den faktiska pensionsbehållningen det som anges i andra stycket.
\paragraph*{}
Oavsett när dödsfallet inträffar ska den avlidnes pensionsbehållning, beräknad som om något uttag av inkomstpension inte hade gjorts, läggas till grund för beräkningen av faktisk pensionsbehållning.
Lag (2022:878).
\subsection*{6 §}
\paragraph*{}
/Träder i kraft I:2025-12-01/
Om den avlidne har tagit ut inkomstpension före den månad då han eller hon uppnått eller skulle ha uppnått riktåldern för pension (tidigt uttag) gäller vid beräkning av den faktiska pensionsbehållningen det som anges i andra stycket.
\paragraph*{}
Oavsett när dödsfallet inträffar ska den avlidnes pensionsbehållning, beräknad som om något uttag av inkomstpension inte hade gjorts, läggas till grund för beräkningen av faktisk pensionsbehållning.
Lag (2022:879).
\subsection*{7 §}
\paragraph*{}
/Upphör att gälla U:2025-12-01/
Med antagen pensionsbehållning avses ett belopp som enligt 14-19 §§ kan tillgodoräknas för den avlidne för tiden från och med det år då dödsfallet inträffade till och med det år då den avlidne skulle ha fyllt 65 år.
Lag (2022:878).
\subsection*{7 §}
\paragraph*{}
/Träder i kraft I:2025-12-01/
Med antagen pensionsbehållning avses ett belopp som enligt 14-19 §§ kan tillgodoräknas för den avlidne för tiden från och med det år då dödsfallet inträffade till och med året före det år då den avlidne skulle ha uppnått den vid dödsfallet gällande riktåldern för pension.
Lag (2022:879).
\subsection*{8 §}
\paragraph*{}
/Upphör att gälla U:2025-12-01/
Efterlevandepensionsunderlaget ska beräknas som kvoten mellan
\newline - summan av faktisk pensionsbehållning och antagen pensionsbehållning och
\newline - det delningstal som enligt 62 kap. 34-37 §§ ska tillämpas för beräkning av inkomstpension för försäkrade som i januari det år dödsfallet inträffade fyllde 66 år.
Lag (2022:878).
\subsection*{8 §}
\paragraph*{}
/Träder i kraft I:2025-12-01/
Efterlevandepensionsunderlaget ska beräknas som kvoten mellan
\newline - summan av faktisk pensionsbehållning och antagen pensionsbehållning och
\newline - det delningstal som enligt 62 kap. 34-37 §§ ska tillämpas för beräkning av inkomstpension för försäkrade som i januari det år dödsfallet inträffade uppnådde riktåldern för pension.
Lag (2022:879).
\paragraph*{}
/Rubriken upphör att gälla U:2025-12-01/
\subsection*{9 §}
\paragraph*{}
Faktisk pensionsbehållning är summan av de pensionsrätter för inkomstpension, beräknade enligt 61 kap. 5, 6 och 9 §§, som kan tillgodoräknas för den avlidne till och med den 31 december året före det år då dödsfallet inträffade, sedan dessa pensionsrätter räknats om enligt 62 kap. 5 och 6 §§.
\subsection*{10 §}
\paragraph*{}
Pensionsrätterna ska beräknas som om det för den avlidne tillgodoräknats endast pensionsrätt för inkomstpension och som om denna pensionsrätt hade utgjort 18,5 procent av pensionsunderlaget.
\subsection*{11 §}
\paragraph*{}
Vid beräkning av den faktiska pensionsbehållningen ska hänsyn tas till pensionsbehållning som avser pensionsgrundande belopp enligt 60 kap. för plikttjänstgöring, barnår och studier endast om villkoret i andra stycket är uppfyllt.
\paragraph*{}
För den avlidne ska till och med den 31 december året före det år då dödsfallet inträffade, eller på grund av beräkning enligt 13 §, kunna tillgodoräknas pensionsgrundande inkomst som för vart och ett av minst fem år har uppgått till lägst 2 gånger det för respektive intjänandeår gällande inkomstbasbeloppet.
\subsection*{12 §}
\paragraph*{}
Med pensionsgrundande inkomst ska, när 11 § tillämpas, likställas pensionsgrundande belopp enligt 60 kap. för sjukersättning och aktivitetsersättning.
\subsection*{13 §}
\paragraph*{}
Vid bedömningen av om villkoret i 11 § är uppfyllt ska hänsyn även tas till sådana år efter dödsfallet för vilka antagen pensionsbehållning enligt 14-19 §§ ska beräknas.
Beräkning av antagen pensionsbehållning
\subsection*{14 §}
\paragraph*{}
/Upphör att gälla U:2025-12-01/
En antagen pensionsbehållning ska beräknas för den avlidne enligt 15-19 §§ om
\newline 1. dödsfallet har inträffat före det år då den avlidne skulle ha fyllt 66 år, och
\newline 2. pensionsrätt för inkomstpension enligt 61 kap. har fastställts för den avlidne för minst tre av de fem kalenderåren närmast före året för dödsfallet.
Lag (2022:878).
\subsection*{14 §}
\paragraph*{}
/Träder i kraft I:2025-12-01/
En antagen pensionsbehållning ska beräknas för den avlidne enligt 15-19 §§ om
\newline 1. dödsfallet har inträffat före det år då den avlidne skulle ha uppnått den vid dödsfallet gällande riktåldern för pension, och
\newline 2. pensionsrätt för inkomstpension enligt 61 kap. har fastställts för den avlidne för minst tre av de fem kalenderåren närmast före året för dödsfallet.
Lag (2022:879).
\subsection*{15 §}
\paragraph*{}
Hänsyn ska tas till pensionsrätt som avser pensionsgrundande belopp enligt 60 kap. för plikttjänstgöring, barnår och studier endast om villkoret i andra stycket är uppfyllt.
\paragraph*{}
För den avlidne ska det till och med den 31 december året före det år då dödsfallet inträffade kunna tillgodoräknas pensionsgrundande inkomst som för vart och ett av minst fem år uppgått till lägst 2 gånger det för respektive intjänandeår gällande inkomstbasbeloppet. Vid beräkningen tillämpas bestämmelsen i 12 §.
\subsection*{16 §}
\paragraph*{}
/Upphör att gälla U:2025-12-01/
Den antagna pensionsbehållningen ska motsvara summan av de årliga pensionsrätter som kan tillgodoräknas för den avlidne för tiden från och med det år då dödsfallet inträffade till och med det år då den avlidne skulle ha fyllt 65 år.
Lag (2022:878).
\subsection*{16 §}
\paragraph*{}
/Träder i kraft I:2025-12-01/
Den antagna pensionsbehållningen ska motsvara summan av de årliga pensionsrätter som kan tillgodoräknas för den avlidne för tiden från och med det år då dödsfallet inträffade till och med året före det år då den avlidne skulle ha uppnått den vid dödsfallet gällande riktåldern för pension.
Lag (2022:879).
\subsection*{17 §}
\paragraph*{}
Pensionsrätten för vart och ett av de år som anges i 16 § ska beräknas på ett pensionsunderlag som svarar mot medeltalet av de pensionsunderlag enligt 61 kap. 5 § som kan tillgodoräknas för den avlidne för vart och ett av de fem åren närmast före det år då dödsfallet inträffade.
När medeltalet beräknas ska det bortses från de två år då pensionsunderlaget är högst respektive lägst.
\subsection*{18 §}
\paragraph*{}
Vid beräkning av den antagna pensionsbehållningen ska pensionsrätter enligt 17 § beräknas som om all pensionsrätt skulle ha varit pensionsrätt för inkomstpension och som om denna pensionsrätt hade utgjort 18,5 procent av pensionsunderlaget.
\subsection*{19 §}
\paragraph*{}
Om inkomstindex enligt 58 kap. 10-12 §§ för något eller några av de fem år närmast före dödsfallet som avses i 17 § avviker från inkomstindex för dödsfallsåret, ska pensionsunderlaget för det eller de åren räknas om med hänsyn till förändringen av inkomstindex för dödsfallsåret före beräkningen av det medeltal som anges i 17 §.
\paragraph*{}
/Rubriken upphör att gälla U:2025-12-01/
\subsection*{20 §}
\paragraph*{}
/Upphör att gälla U:2025-12-01/
Om dödsfallet inträffat efter det år då den avlidne fyllde 66 år ska till grund för beräkning av efterlevandepension läggas den avlidnes faktiska pensionsbehållning, beräknad enligt 9-13 §§, som tjänats in till och med det år då den avlidne fyllde 65 år.
Lag (2022:878).
\subsection*{20 §}
\paragraph*{}
/Träder i kraft I:2025-12-01/
Om dödsfallet inträffat efter det år då den avlidne uppnådde riktåldern för pension ska till grund för beräkning av efterlevandepension läggas den avlidnes faktiska pensionsbehållning, beräknad enligt 9-13 §§, som tjänats in till och med året före det år då den avlidne uppnådde riktåldern för pension.
Lag (2022:879).
\subsection*{21 §}
\paragraph*{}
/Upphör att gälla U:2025-12-01/
Den faktiska pensionsbehållningen enligt 20 § ska räknas om med förändringar av inkomstindex enligt 58 kap. 10-12 §§ fram till ingången av det år den avlidne fyllde 66 år.
\paragraph*{}
Från och med årsskiftet därefter ska den faktiska pensionsbehållningen följsamhetsindexeras. Det görs genom att den räknas om i enlighet med 62 kap. 42 och 43 §§ till ingången av det år då dödsfallet inträffade.
Lag (2022:878).
\subsection*{21 §}
\paragraph*{}
/Träder i kraft I:2025-12-01/
Den faktiska pensionsbehållningen enligt 20 § ska räknas om med förändringar av inkomstindex enligt 58 kap. 10-12 §§ fram till ingången av det år den avlidne uppnådde riktåldern för pension.
\paragraph*{}
Från och med årsskiftet därefter ska den faktiska pensionsbehållningen följsamhetsindexeras. Det görs genom att den räknas om i enlighet med 62 kap. 42 och 43 §§ till ingången av det år då dödsfallet inträffade.
Lag (2022:879).
\subsection*{22 §}
\paragraph*{}
För avlidna födda 1937 eller tidigare ska efterlevandepension beräknas på ett efterlevandepensionsunderlag som motsvarar 60 procent av produkten av
\newline - det prisbasbelopp som gällde det år då den avlidne fyllde 65 år och
\newline - medeltalet av de pensionspoäng för tilläggspension som enligt 15 § i den upphävda lagen (1998:675) om införande av lagen (1998:674) om inkomstgrundad ålderspension tillgodoräknats för den avlidne efter det att denna medelpoäng ökats med talet ett.
\subsection*{23 §}
\paragraph*{}
Om den avlidne har tillgodoräknats pensionspoäng för mer än 15 år, ska vid beräkningen av medeltalet hänsyn tas enbart till de 15 år för vilka de högsta poängtalen tillgodoräknats.
Vid beräkningen ska 6 kap. 18 § lagen (2010:111) om införande av socialförsäkringsbalken tillämpas.
\subsection*{24 §}
\paragraph*{}
Från och med årsskiftet efter det att den avlidne fyllde 65 år fram till ingången av det år dödsfallet inträffade ska efterlevandepensionsunderlaget följsamhetsindexeras. Detta sker genom att underlaget räknas om enligt 62 kap. 42 och 43 §§.
\subsection*{25 §}
\paragraph*{}
För en avliden som var född 1935 eller tidigare ska det som föreskrivs i 22 och 24 §§ om det år då den avlidne fyllde 65 år i stället avse 2001.
\subsection*{26 §}
\paragraph*{}
Om den avlidne har tillgodoräknats pensionspoäng för tilläggspension för färre än 30 år gäller vid beräkningen enligt 22-25 §§ följande. Hänsyn ska tas endast till så stor del av den i 22 § angivna produkten som motsvarar kvoten mellan
\newline - det antal år för vilka pensionspoäng har tillgodoräknats och
\newline - talet 30.
\paragraph*{}
Vid beräkningen ska även följande bestämmelser tillämpas:
\newline - 63 kap. 24 och 27-29 §§, och
\newline - 6 kap. 12-17 §§ lagen (2010:111) om införande av socialförsäkringsbalken.
\chapter*{83 Änkepension}
\subsection*{1 §}
\paragraph*{}
I detta kapitel finns bestämmelser om
\newline - förmånsformer i 2 §,
\newline - rätten till änkepension på grundval av pensionspoäng i 3-5 §§,
\newline - rätten till 90-procentstillägg i 6 och 7 §§, och
\newline - vad som gäller om en änka gifter sig i 8 §.
\subsection*{2 §}
\paragraph*{}
Änkepension består av
\newline 1. änkepension på grundval av den avlidne makens intjänade pensionspoäng för tilläggspension, och
\newline 2. 90-procentstillägg, som är ett tillägg motsvarande 90 procent av det prisbasbelopp som gäller för dödsfallsåret.
\subsection*{3 §}
\paragraph*{}
En änka har rätt till änkepension enligt 2 § första stycket 1 om den avlidne maken var försäkrad enligt 4 och 6 kap. och vid sin död
\newline 1. hade rätt till inkomstgrundad ålderspension, eller
\newline 2. skulle ha haft rätt till förtidspension enligt 13 kap. 1 § i den upphävda lagen (1962:381) om allmän försäkring i dess lydelse före den 1 januari 2003, om pensionsfall enligt samma paragraf hade förelegat vid tidpunkten för dödsfallet.
\paragraph*{}
För rätt till änkepension krävs det vidare att änkan uppfyller villkoren i 4 eller 5 §. Dessutom krävs det att det för den avlidne maken kan tillgodoräknas pensionspoäng för tilläggspension för minst tre år. Vid bedömning av om detta krav är uppfyllt tillämpas bestämmelserna i 63 kap. 4 § angående sjömän som inte är svenska medborgare, om änkan är född 1944 eller tidigare.
\subsection*{4 §}
\paragraph*{}
En änka som är född 1944 eller tidigare har rätt till änkepension enligt 2 § första stycket 1 om hon var gift med den avlidne vid utgången av 1989 och vid tidpunkten för dödsfallet. Vidare krävs att
\newline 1. äktenskapet vid tidpunkten för dödsfallet hade varat minst fem år och ingåtts senast den dag då den avlidne fyllde 60 år, eller
\newline 2. den avlidne efterlämnar barn som också är barn till änkan.
\subsection*{5 §}
\paragraph*{}
En änka som är född 1945 eller senare har rätt till änkepension enligt 2 § första stycket 1 om hon var gift med den avlidne vid utgången av 1989 och fortlöpande fram till dödsfallet. Vidare krävs att
\newline 1. äktenskapet vid utgången av 1989 hade varat minst fem år och ingåtts senast den dag då mannen fyllde 60 år eller mannen vid utgången av 1989 hade barn som också var barn till kvinnan, och
\newline 2. något av de villkor som anges i 1 var uppfyllt även vid dödsfallet.
\paragraph*{}
Rätten till 90-procentstillägg
\subsection*{6 §}
\paragraph*{}
En änka som är född 1945 eller senare har rätt till 90- procentstillägg om hon, utöver vad som krävs enligt 3 och 5 §§, uppfyller villkoren i 7 §.
\subsection*{7 §}
\paragraph*{}
En änka som är född 1945 eller senare har rätt till 90- procentstillägg om hon såväl vid utgången av 1989 som vid dödsfallet 1. har vårdnaden om och stadigvarande sammanbor med barn under 16 år, som vid makens död stadigvarande vistades i makarnas hem eller hos änkan, eller
\newline 2. har fyllt 36 år och varit gift med den avlidne i minst fem år.
\paragraph*{}
Om villkoret enligt första stycket 1 inte längre är uppfyllt, upphör rätten till 90-procentstillägg enligt denna punkt. Vid bedömningen av rätten till 90-procentstillägg enligt första stycket 2 ska det i så fall anses som om mannen avlidit då rätten till 90-procentstillägg enligt första stycket 1 upphörde och som om äktenskapet varat till den tidpunkten.
\subsection*{8 §}
\paragraph*{}
Änkepension får inte lämnas om änkan gifter sig. Upplöses äktenskapet inom fem år, lämnas dock änkepension på nytt.
\chapter*{84 Beräkning av änkepension}
\subsection*{1 §}
\paragraph*{}
I detta kapitel finns bestämmelser om
\newline - änkor som är födda 1944 eller tidigare i 2-5 §§,
\newline - änkor som är födda 1945 eller senare i 6-11 §§, och
\newline - samordning med inkomstgrundad ålderspension i 12-17 §.
\paragraph*{}
Änkor som är födda 1944 eller tidigare
\subsection*{2 §}
\paragraph*{}
Änkepension till en änka som är född 1944 eller tidigare ska för år räknat motsvara 40 procent eller, om det efter den avlidne finns barn som har rätt till barnpension efter honom enligt 78 kap., 35 procent av
\newline 1. hel förtidspension enligt 4 § för den avlidne, eller
\newline 2. inkomstgrundad ålderspension enligt 5 § för den avlidne.
\subsection*{3 §}
\paragraph*{}
Underlaget för änkepensionen ska beräknas med tillämpning av 2 § 2 om den avlidne vid tidpunkten för dödsfallet
\newline 1. fick inkomstgrundad ålderspension, eller
\newline 2. hade fyllt 65 år och hade rätt till inkomstgrundad ålderspension.
\paragraph*{}
I annat fall ska 2 § 1 tillämpas.
\subsection*{4 §}
\paragraph*{}
Vid beräkningen av förtidspension som avses i 2 § 1 ska bestämmelserna i 13 kap. 2 § första stycket första och andra meningarna och andra stycket samt 3 § i den upphävda lagen (1962:381) om allmän försäkring, i bestämmelsernas lydelse före den 1 januari 2003, tillämpas.
\paragraph*{}
Förtidspensionen ska beräknas som om
\newline 1. den avlidne skulle ha haft rätt till förtidspension om de bestämmelser som anges i första stycket fortfarande hade gällt, och
\newline 2. rätten till förtidspensionen hade inträtt vid tidpunkten för dödsfallet eller, om den avlidne var född 1954 eller senare, vid ingången av 2003.
\subsection*{5 §}
\paragraph*{}
Vid beräkningen av inkomstgrundad ålderspension som avses i 2 § 2 ska följande bestämmelser tillämpas:
\newline - 63 kap. 6 § första stycket 1 och andra stycket,
\newline - 63 kap. 8, 11, 24, 25 och 27-29 §§, samt
\newline - 6 kap. 12-18 §§ lagen (2010:111) om införande av socialförsäkringsbalken.
\subsection*{6 §}
\paragraph*{}
/Upphör att gälla U:2025-12-01/
Änkepension till en änka som är född 1945 eller senare lämnas med ett belopp som motsvarar det som skulle ha lämnats till kvinnan om mannen hade avlidit vid utgången av 1989. Änkepensionen ska för år räknat motsvara
\newline 1. 40 procent eller, om det efter den avlidne finns barn som har rätt till barnpension efter honom enligt 78 kap., 35 procent av
a) hel förtidspension enligt 7 § för den avlidne, eller
b) tilläggspension i form av ålderspension enligt 7 § för den avlidne, och
\newline 2. för tid före den månad då änkan fyller 66 år, 90 procent av det prisbasbelopp som gällde för dödsfallsåret (90- procentstillägg).
Lag (2022:878).
\subsection*{6 §}
\paragraph*{}
/Träder i kraft I:2025-12-01/
Änkepension till en änka som är född 1945 eller senare lämnas med ett belopp som motsvarar det som skulle ha lämnats till kvinnan om mannen hade avlidit vid utgången av 1989. Änkepensionen ska för år räknat motsvara
\newline 1. 40 procent eller, om det efter den avlidne finns barn som har rätt till barnpension efter honom enligt 78 kap., 35 procent av
a) hel förtidspension enligt 7 § för den avlidne, eller
b) tilläggspension i form av ålderspension enligt 7 § för den avlidne, och
\newline 2. för tid före den månad då änkan uppnår riktåldern för pension, 90 procent av det prisbasbelopp som gällde för dödsfallsåret (90-procentstillägg).
Lag (2022:879).
\subsection*{7 §}
\paragraph*{}
Beloppet enligt 6 § 1 ska beräknas på grundval av den förtidspension som skulle ha lämnats till mannen om denne hade fått rätt till hel sådan pension vid utgången av 1989 och om pensionen då hade beräknats med tillämpning av 13 kap. 3 § i den upphävda lagen (1962:381) om allmän försäkring.
Om mannen vid utgången av 1989 hade rätt till tilläggspension i form av ålderspension, ska dock beloppet enligt 6 § 1 beräknas med tillämpning av 12 kap. 2 § första och andra styckena i den upphävda lagen om allmän försäkring i deras lydelse vid utgången av 1989.
Minskning av 90-procentstillägg för änkor som inte hade fyllt 50 år vid makens död
\subsection*{8 §}
\paragraph*{}
Om en änka med tillämpning av 83 kap. 7 § första stycket 2 har rätt till 90-procentstillägg och inte hade fyllt 50 år vid mannens död eller vid den tidpunkt som avses i 83 kap. 7 § andra stycket, ska 90-procentstillägget minskas enligt följande.
\paragraph*{}
90-procentstillägget ska minskas med en femtondel för varje år som änkan är yngre än 50 år vid
\newline 1. mannens död, eller
\newline 2. den tidpunkt som avses i 83 kap. 7 § andra stycket.
\subsection*{9 §}
\paragraph*{}
Om det för den avlidne inte kan tillgodoräknas pensionspoäng för tillläggspension för minst 30 år, ska vid beräkningen av 90-procentstillägget hänsyn tas endast till så stor del av beloppet som motsvarar kvoten mellan
\newline - det antal år för vilka pensionspoäng tillgodoräknas den avlidne och
\newline - talet 30.
\paragraph*{}
Vid beräkningen enligt första stycket ska hänsyn tas endast till det antal år med pensionspoäng som tillgodoräknats för den avlidne till och med 1989.
\subsection*{10 §}
\paragraph*{}
Vid tillämpning av 9 § ska också följande bestämmelser tillämpas:
\newline - 63 kap. 24 och 27-29 §§, och
\newline - 6 kap. 12-17 §§ lagen (2010:111) om införande av socialförsäkringsbalken.
\subsection*{11 §}
\paragraph*{}
Avkortning av 90-procentstillägget enligt 9 och 10 §§ ska göras efter minskning av 90-procentstillägget enligt 8 §.
\subsection*{12 §}
\paragraph*{}
Om en änka, som är född något av åren 1930-1953, har rätt till änkepension och för samma månad får inkomstgrundad ålderspension i form av tilläggspension, ska änkepension lämnas endast i den utsträckning den efter eventuell samordning med omställningspension enligt 77 kap. 12 § överstiger ålderspensionen.
\paragraph*{}
Vid tillämpningen av första stycket ska änkans ålderspension i form av tilläggspension beräknas enligt bestämmelserna i 63 kap. 6 § första stycket 1 och andra stycket, 8, 11 och 12 §§ samt i förekommande fall 17 och 25 §§.
\subsection*{13 §}
\paragraph*{}
Om en änka som avses i 12 § är född något av åren 1930-1944 har hon, oavsett det som anges där, rätt till änkepension med sådant belopp att summan av änkepensionen och den inkomstgrundade ålderspensionen enligt 12 § ska motsvara en viss andel av summan av hennes ålderspension i form av tilläggspension, beräknad enligt samma paragraf, och den tilläggspension för den avlidne mannen som avses i 2 §.
Änkepensionen får dock inte lämnas med högre belopp än vad som följer av 2 §.
\paragraph*{}
Den andel som avses i första stycket uppgår till
\newline 1. 60 procent om änkan är född 1930,
\newline 2. 58 procent om änkan är född 1931,
\newline 3. 56 procent om änkan är född 1932,
\newline 4. 54 procent om änkan är född 1933,
\newline 5. 52 procent om änkan är född 1934, och
\newline 6. 50 procent om änkan är född något av åren 1935-1944.
\subsection*{14 §}
\paragraph*{}
Vid tillämpning av 12 och 13 §§ ska, om änkan tar ut inkomstgrundad ålderspension som avses i nämnda paragrafer vid en senare tidpunkt än från och med den månad då hon fyllt 65 år, hänsyn tas till den inkomstgrundade ålderspension i form av tilläggspension, beräknad enligt 12 §, som skulle ha betalats ut om hon haft rätt till pensionen från och med den månaden.
\subsection*{15 §}
\paragraph*{}
Vid tillämpning av 12 och 13 §§ ska, om pensionspoäng till följd av bristande eller underlåten avgiftsbetalning inte har tillgodoräknats eller har minskats, hänsyn tas till den ålderspension i form av tilläggspension som skulle ha betalats ut om full avgift hade betalats.
Änkor födda 1954 eller senare
\subsection*{16 §}
\paragraph*{}
/Upphör att gälla U:2025-12-01/
Om en änka som är född 1954 eller senare har rätt till änkepension och för samma månad får inkomstgrundad ålderspension, lämnas änkepensionen endast i den utsträckning den, efter eventuell samordning med omställningspension enligt 77 kap. 12 §, överstiger ålderspensionen.
\paragraph*{}
Inkomstpension och premiepension ska i detta sammanhang beräknas som om änkan endast hade tillgodoräknats pensionsrätt för inkomstpension och som om denna pensionsrätt hade utgjort 18,5 procent av pensionsunderlaget.
\paragraph*{}
Om änkan tar ut inkomstpension vid en senare tidpunkt än från och med den månad då hon fyllde 66 år, ska hänsyn tas till den inkomstpension som skulle ha betalats ut om hon hade haft rätt till sådan pension från och med den månaden.
\paragraph*{}
Om pensionsrätt till följd av bristande eller underlåten avgiftsbetalning inte har tillgodoräknats eller minskats, ska hänsyn tas till den inkomstpension som skulle ha betalats ut om full avgift hade betalats.
Lag (2022:878).
\subsection*{16 §}
\paragraph*{}
/Träder i kraft I:2025-12-01/
Om en änka som är född 1954 eller senare har rätt till änkepension och för samma månad får inkomstgrundad ålderspension, lämnas änkepensionen endast i den utsträckning den, efter eventuell samordning med omställningspension enligt 77 kap. 12 §, överstiger ålderspensionen.
\paragraph*{}
Inkomstpension och premiepension ska i detta sammanhang beräknas som om änkan endast hade tillgodoräknats pensionsrätt för inkomstpension och som om denna pensionsrätt hade utgjort 18,5 procent av pensionsunderlaget.
\paragraph*{}
Om änkan tar ut inkomstpension vid en senare tidpunkt än från och med den månad då hon har uppnått riktåldern för pension ska hänsyn tas till den inkomstgrundade ålderspension som skulle ha betalats ut om hon haft rätt till sådan pension från och med den månaden.
\paragraph*{}
Om pensionsrätt till följd av bristande eller underlåten avgiftsbetalning inte har tillgodoräknats eller minskats, ska hänsyn tas till den inkomstpension som skulle ha betalats ut om full avgift hade betalats.
Lag (2022:879).
\subsection*{17 §}
\paragraph*{}
Vid beräkning enligt 16 § ska ett tillägg om en pensionspoäng göras till det underlag som beräkningen av änkepensionen ska grunda sig på. Änkepension får emellertid inte betalas ut med högre belopp än vad som följer av 6 §.
\chapter*{85 Vissa gemensamma bestämmelser om efterlevandepension och efterlevandestöd}
\subsection*{1 §}
\paragraph*{}
I detta kapitel finns bestämmelser om
\newline - följsamhetsindexering i 2 §,
\newline - samordning av efterlevandepension och efterlevandestöd med yrkesskadelivränta i 3-10 §§,
\newline - ändring av beslut om efterlevandepension och efterlevandestöd i 11-12 §§, och
\newline - utbetalning av efterlevandepension och efterlevandestöd i 13-15 §§.
\subsection*{2 §}
\paragraph*{}
Vid ingången av varje kalenderår ska barnpension, omställningspension och änkepension räknas om genom följsamhetsindexering enligt 62 kap. 42 § och 43 § första stycket.
\paragraph*{}
Om barnpension, omställningspension eller änkepension ska lämnas först från ett senare år än året för dödsfallet, ska pensionen beräknas som om den hade lämnats redan dödsfallsåret.
\subsection*{3 §}
\paragraph*{}
Efterlevandepension och efterlevandestöd ska minskas enligt 4-10 §§ om den efterlevande har rätt till livränta (yrkesskadelivränta) på grund av obligatorisk försäkring enligt någon av följande upphävda lagar:
\newline 1. lagen (1916:235) om försäkring för olycksfall i arbete,
\newline 2. lagen (1929:131) om försäkring för vissa yrkessjukdomar, och
\newline 3. lagen (1954:243) om yrkesskadeförsäkring.
\paragraph*{}
Detsamma gäller om den efterlevande
\newline 1. enligt någon annan författning eller enligt särskilt beslut av regeringen har rätt till annan livränta, som bestäms eller betalas ut av Försäkringskassan, eller
\newline 2. får livränta enligt utländsk lagstiftning om yrkesskadeförsäkring.
\subsection*{4 §}
\paragraph*{}
Efterlevandepension och efterlevandestöd ska inte minskas på grund av livränta som lämnas enligt 87 och 88 kap.
\paragraph*{}
Bestämmelser om hur efterlevandelivränta enligt arbetsskadeförsäkringen ska minskas på grund av efterlevandepension och efterlevandestöd finns i 88 kap.
\subsection*{5 §}
\paragraph*{}
Har livränta eller del av livränta eller livränta för viss tid bytts ut mot ett engångsbelopp, ska det vid beräkning enligt 6-10 §§ anses som om livränta lämnas eller som om den livränta som lämnas har höjts med ett belopp som svarar mot engångsbeloppet enligt de försäkringstekniska grunder som tillämpas vid utbytet.
\subsection*{6 §}
\paragraph*{}
Summan av barnpension och efterlevandestöd ska minskas med tre fjärdedelar av varje livränta som överstiger en sjättedel av prisbasbeloppet och som barnet för samma tid har rätt till såsom efterlevande.
\paragraph*{}
Minskningen ska i första hand göras på efterlevandestödet.
Avdrag på barnpension får göras endast om den avlidne kunnat tillgodoräkna sig pensionspoäng för minst ett år när skadan inträffade.
\subsection*{7 §}
\paragraph*{}
Barnpension får aldrig, på grund av bestämmelserna i 6 §, tillsammans med efterlevandestöd understiga en fjärdedel av det belopp som anges i 79 kap. 3 §.
\subsection*{8 §}
\paragraph*{}
Summan av omställningspension och garantipension till sådan pension ska minskas med tre fjärdedelar av varje livränta som överstiger en sjättedel av prisbasbeloppet och som den pensionsberättigade för samma tid har rätt till såsom efterlevande.
\paragraph*{}
Minskningen ska i första hand göras på garantipensionen.
Avdrag på omställningspension får göras endast om den avlidne kunnat tillgodoräkna sig pensionspoäng för minst ett år när skadan inträffade.
\subsection*{9 §}
\paragraph*{}
Änkepension ska minskas med tre fjärdedelar av varje livränta som överstiger en sjättedel av prisbasbeloppet och som den pensionsberättigade för samma tid har rätt till såsom efterlevande.
\paragraph*{}
Minskningen ska i första hand göras på sådan del av änkepension som avses i 84 kap. 6 § 2. Avdrag på sådan del av änkepension som avses i 84 kap. 2 § och 6 § 1 får göras endast om den avlidne kunnat tillgodoräkna sig pensionspoäng för minst ett år när skadan inträffade.
\subsection*{10 §}
\paragraph*{}
Änkepension får aldrig, på grund av bestämmelserna i 9 §, för månad räknat understiga 9 procent av prisbasbeloppet.
\subsection*{11 §}
\paragraph*{}
Ett beslut om efterlevandepension ska ändras om det föranleds av en ändring som har gjorts beträffande den pensionsbehållning eller de pensionspoäng som legat till grund för beräkningen av pensionen.
\subsection*{12 §}
\paragraph*{}
Ett beslut om garantipension till omställningspension eller om efterlevandestöd ska ändras om storleken av garantipensionen eller efterlevandestödet påverkas av en ändring som gjorts i fråga om den inkomstgrundade efterlevandepension som legat till grund för beräkningen av garantipensionen eller efterlevandestödet.
\subsection*{13 §}
\paragraph*{}
Efterlevandepension och efterlevandestöd ska betalas ut månadsvis. Årspension och efterlevandestöd som beräknas uppgå till högst 2 400 kronor ska dock, om det inte finns särskilda skäl, betalas ut i efterskott en eller två gånger per år.
Efter överenskommelse med den efterlevande får utbetalning även i annat fall ske en eller två gånger per år.
\subsection*{14 §}
\paragraph*{}
Om den efterlevande vill avbryta uttaget av efterlevandepension, ska han eller hon skriftligen anmäla detta till Pensionsmyndigheten. Sådan anmälan ska ha kommit in till Pensionsmyndigheten senast månaden före den månad som ändringen avser.
\subsection*{15 §}
\paragraph*{}
När efterlevandepension och efterlevandestöd beräknas ska den totala årliga ersättningen avrundas till närmaste hela krontal som är delbart med tolv.
\paragraph*{}
Om både garantipension eller efterlevandestöd och inkomstgrundad efterlevandepension eller änkepension betalas ut samtidigt ska avrundningen ske på garantipensionen eller efterlevandestödet.
\chapter*{86 Innehåll}
\subsection*{1 §}
\paragraph*{}
I denna underavdelning finns allmänna bestämmelser om arbetsskadeersättning m.m. vid dödsfall i 87 kap.
\paragraph*{}
Vidare finns bestämmelser om efterlevandelivränta i 88 kap.
\chapter*{87 Allmänna bestämmelser om arbetsskadeersättning m.m.}
\subsection*{1 §}
\paragraph*{}
I detta kapitel finns hänvisningar till bestämmelser om arbetsskadeförsäkring, statligt personskadeskydd och krigsskadeersättning i 2 §.
\paragraph*{}
Vidare finns bestämmelser om
\newline - ersättningsformer och grundvillkor i 3 §,
\newline - begravningshjälp i 4 och 5 §§,
\newline - livränta efter försvunna försäkrade i 6 §, och
\newline - förlust av ersättning i 7 §.
\subsection*{2 §}
\paragraph*{}
I 39-42 kap. finns bestämmelser om ersättning från arbetsskadeförsäkringen till försäkrade förvärvsarbetande och vissa studerande vid arbetsskada. Dessa bestämmelser gäller också i tillämpliga delar i fråga om förmåner till efterlevande enligt detta kapitel och 88 kap.
\paragraph*{}
Det som föreskrivs i första stycket gäller på motsvarande sätt i fråga om bestämmelserna om ersättning till efterlevande i form av personskadeersättning eller krigsskadeersättning i 43 respektive 44 kap.
\subsection*{3 §}
\paragraph*{}
Om en försäkrad har avlidit till följd av arbetsskada eller skada som avses i 43 eller 44 kap. kan ersättning lämnas i form av
\newline - begravningshjälp enligt 4 §, och
\newline - efterlevandelivränta enligt 88 kap.
\paragraph*{}
Innan Pensionsmyndigheten fattar beslut om ersättning, ska myndigheten inhämta Försäkringskassans bedömning om dödsfallet har inträffat till följd av en sådan skada som avses i första stycket samt, om det är fråga om efterlevandelivränta, Försäkringskassans beräkning av det ersättningsunderlag som avses i 88 kap. 14-16 §§.
\subsection*{4 §}
\paragraph*{}
Begravningshjälp lämnas med belopp som motsvarar 30 procent av prisbasbeloppet vid tiden för dödsfallet.
\subsection*{5 §}
\paragraph*{}
Begravningshjälpen lämnas till den som enligt lag ska förvalta den avlidnes egendom.
\subsection*{6 §}
\paragraph*{}
Om en försäkrad har försvunnit och det kan antas att han eller hon har avlidit till följd av arbetsskada eller skada som avses i 43 eller 44 kap., gäller samma rätt till livränta som vid dödsfall. Visar det sig senare att den försäkrade är vid liv eller att han eller hon har avlidit av någon annan orsak än sådan skada, upphör rätten till förmånen.
\subsection*{7 §}
\paragraph*{}
Rätten till ersättning går förlorad om en ansökan om efterlevandeförmåner inte görs inom sex år
\newline - i fråga om begravningshjälp från dagen för dödsfallet, och
\newline - i fråga om efterlevandelivränta från den dag ersättningen avser.
\chapter*{88 Efterlevandelivränta}
\subsection*{1 §}
\paragraph*{}
I detta kapitel finns bestämmelser om
\newline - rätten till barnlivränta i 2-5 §§,
\newline - rätten till omställningslivränta i 6-9 §§,
\newline - förmånstiden i 10 och 11 §§,
\newline - beräkning av efterlevandelivränta i 12 och 13 §§,
\newline - ersättningsunderlag för efterlevandelivränta i 14-16 §§,
\newline - begränsningsregler för flera livräntor i 17 §, och
\newline - samordning av efterlevandelivränta med efterlevandepension i 18-20 §§.
\subsection*{2 §}
\paragraph*{}
Ett barn till en avliden som avses i 87 kap. 3 § har rätt till barnlivränta.
\subsection*{3 §}
\paragraph*{}
Rätten till barnlivränta upphör när barnet fyller 18 år, om inte något annat följer av 4 §.
\subsection*{4 §}
\paragraph*{}
Ett barn som har fyllt 18 år och som bedriver studier har rätt till barnlivränta även för studietiden, dock längst till och med juni det år då barnet fyller 20 år. Detta gäller dock endast om studierna och studietiden uppfyller de villkor som anges i 78 kap. 5 och 6 §§.
\subsection*{5 §}
\paragraph*{}
Bestämmelsen i 78 kap. 3 § om barn som vårdas av en blivande adoptivförälder gäller även i fråga om barnlivränta.
\subsection*{6 §}
\paragraph*{}
Omställningslivränta till efterlevande make till en avliden som avses i 87 kap. 3 § kan lämnas i form av
\newline - allmän omställningslivränta, och
\newline - förlängd omställningslivränta.
\paragraph*{}
Det som föreskrivs om make i detta kapitel tillämpas också i fråga om sådan efterlevande som avses i 80 kap. 4 §.
\subsection*{7 §}
\paragraph*{}
En efterlevande make har rätt till allmän omställningslivränta under de närmare förutsättningar som anges i 80 kap. 3 §.
\subsection*{8 §}
\paragraph*{}
En efterlevande make har rätt till förlängd omställningslivränta under de närmare förutsättningar som anges i 80 kap. 6 och 7 §§.
\subsection*{9 §}
\paragraph*{}
Är någon för samma månad berättigad till omställningslivränta efter flera avlidna, lämnas livränta endast efter den sist avlidne eller, om livräntebeloppen efter de avlidna är olika höga, det högsta beloppet.
\subsection*{10 §}
\paragraph*{}
Efterlevandelivränta får lämnas utan ansökan.
\paragraph*{}
Bestämmelserna i 77 kap. 16-18 §§ om barnpension och omställningspension tillämpas även i fråga om barnlivränta respektive omställningslivränta.
Lag (2012:599).
\subsection*{11 §}
\paragraph*{}
Allmän omställningslivränta lämnas för den tid som anges i fråga om allmän omställningspension i 80 kap. 5 §.
\paragraph*{}
Förlängd omställningslivränta lämnas för tid som anges i fråga om förlängd omställningspension i 80 kap. 8 §.
\subsection*{12 §}
\paragraph*{}
Barnlivränta för ett efterlevande barn motsvarar för år räknat 40 procent av det ersättningsunderlag som anges i 14-16 §§.
\paragraph*{}
Har flera barn rätt till barnlivränta efter den avlidne, ska procenttalet ökas med talet 20 för varje barn utöver det första. Det sammanlagda barnlivräntebeloppet ska därefter fördelas lika mellan barnen.
\subsection*{13 §}
\paragraph*{}
Omställningslivränta motsvarar, när den avlidne även efterlämnar barn som har rätt till livränta efter honom eller henne, för år räknat 20 procent av det ersättningsunderlag som anges i 14-16 §§.
\paragraph*{}
Om den avlidne inte efterlämnar barn som har rätt till livränta efter honom eller henne, motsvarar livräntan 45 procent av ersättningsunderlaget.
\subsection*{14 §}
\paragraph*{}
Om den avlidne fick livränta enligt 41-44 kap. med anledning av förlust av arbetsförmågan till följd av skadan, ska livränta efter honom eller henne grundas på hans eller hennes egen livränta.
\paragraph*{}
Livränta efter någon annan avliden ska grundas på vad som enligt 41-44 kap. skulle ha utgjort livränta till honom eller henne vid förlust av arbetsförmågan, om sådan livränta skulle ha börjat lämnas vid tiden för dödsfallet.
\subsection*{15 §}
\paragraph*{}
I den utsträckning det är skäligt ska till grund för beräkningen av livränta till efterlevande läggas även följande ersättningar som lämnades till den avlidne med anledning av tidigare arbetsskada eller skada som avses i 43 eller 44 kap.:
\newline 1. pension,
\newline 2. livränta, och
\newline 3. annan ersättning som har trätt i stället för arbetsinkomst.
\subsection*{16 §}
\paragraph*{}
Efterlevandelivränta får inte grundas på högre belopp än som motsvarar 7,5 gånger prisbasbeloppet vid dödsfallet.
\subsection*{17 §}
\paragraph*{}
Om livräntorna till de efterlevande efter en avliden sammantagna överstiger det ersättningsunderlag som anges i 14-16 §§, ska livräntorna sättas ned proportionellt så att de tillsammans motsvarar detta underlag.
\subsection*{18 §}
\paragraph*{}
Efterlevandelivränta ska minskas om den efterlevande samtidigt har rätt till efterlevandepension med anledning av den inkomstförlust som föranlett livräntan.
\paragraph*{}
Minskningen görs genom att efterlevandelivräntan betalas ut endast i den utsträckning som den överstiger efterlevandepensionen.
\subsection*{19 §}
\paragraph*{}
Efterlevandelivräntan ska även minskas enligt 18 § i fråga om pension som enligt utländskt system för social trygghet lämnas med anledning av arbetsskadan eller sådan skada som avses i 43 eller 44 kap.
\subsection*{20 §}
\paragraph*{}
Om bestämmelserna om beräkning av pensionspoäng vid underlåten eller bristande avgiftsbetalning i 11 kap. 6 § första stycket i den upphävda lagen (1962:381) om allmän försäkring, i detta lagrums lydelse före den 1 januari 1999, 4 kap. 8 § andra stycket i den upphävda lagen (1998:674) om inkomstgrundad ålderspension eller 61 kap. 21 § har tillämpats för år efter det då arbetsskadan inträffade, ska vid tillämpning av 18 § hänsyn tas till den efterlevandepension som skulle ha lämnats om full avgift hade betalats.
\chapter*{89 Innehåll och inledande bestämmelser}
\subsection*{1 §}
\paragraph*{}
I denna underavdelning finns bestämmelser om
\newline - efterlevandeskydd under pensionstiden i 91 kap., och
\newline - uttag och utbetalning m.m. av premiepension till efterlevande i 92 kap.
\subsection*{2 §}
\paragraph*{}
Bestämmelser om premiepension finns i 53-71 kap. (avdelning E) och i lagen (2017:230) om Pensionsmyndighetens försäkringsverksamhet i premiepensionssystemet. I denna underavdelning finns bestämmelser om efterlevandeskydd inom premiepensionssystemet.
Lag (2017:232).
\subsection*{3 §}
\paragraph*{}
Efterlevandeskydd i form av premiepension kan efter ansökan av pensionsspararen hos Pensionsmyndigheten meddelas som efterlevandeskydd under pensionstiden.
\chapter*{91 Efterlevandeskydd under pensionstiden}
\subsection*{1 §}
\paragraph*{}
I detta kapitel finns allmänna bestämmelser i 2 och 3 §§.
\paragraph*{}
Vidare finns bestämmelser om
\newline - rätten att teckna efterlevandeskydd i 4 och 5 §§,
\newline - premiepensionens storlek i 6 §,
\newline - giltighetstiden för efterlevandeskyddet i 7 och 8 §§, och
\newline - vem som får premiepensionen i 9 och 10 §§.
\subsection*{2 §}
\paragraph*{}
Efterlevandeskydd under pensionstiden innebär att det vid pensionsspararens död till hans eller hennes efterlevande make eller sambo betalas ut livsvarig premiepension till efterlevande.
\subsection*{3 §}
\paragraph*{}
En ansökan om efterlevandeskydd under pensionstiden ska göras samtidigt med att pensionsspararen första gången begär att få ut premiepension.
\paragraph*{}
Om pensionsspararen beviljats premiepension utan ansökan enligt 56 kap. 4 a §, får ansökan om efterlevandeskydd göras inom två månader från Pensionsmyndighetens beslut om pension.
\paragraph*{}
Om pensionsspararen senare gifter sig eller inleder ett sådant samboförhållande som avses i 4 §, får ansökan göras inom tre månader från det att äktenskapet ingicks eller samboförhållandet inleddes. Detta gäller dock inte om pensionsspararen var gift eller sambo med samma person när spararen första gången begärde att få ut premiepension.
Lag (2013:747).
\subsection*{4 §}
\paragraph*{}
Efterlevandeskydd för pensionsspararen ska meddelas om pensionsspararen
\newline - är gift, eller
\newline - är ogift men sambo med någon som är ogift och som spararen tidigare har varit gift med eller har eller har haft barn med.
\subsection*{5 §}
\paragraph*{}
Efterlevandeskyddet gäller bara om pensionsspararen vid sin död var gift eller sambo med den person med vilken han eller hon var gift eller sambo vid ansökan om skyddet.
\paragraph*{}
I fråga om sambor krävs också att inte någon av dem var gift med någon annan vid dödsfallet.
\subsection*{6 §}
\paragraph*{}
Omfattningen av efterlevandeskyddet ska beräknas med utgångspunkt i tillgodohavandet på pensionsspararens premiepensionskonto. Pensionsspararens egen premiepension ska då räknas om.
\paragraph*{}
Beräkningarna ska göras så att pensionsspararens egen premiepension och premiepensionen till efterlevande blir lika stora när samma andel av pensionen tas ut.
\subsection*{7 §}
\paragraph*{}
Efterlevandeskyddet börjar i fall som avses i 3 § första och andra styckena gälla vid ingången av den första månad för vilken premiepension ska lämnas till pensionsspararen.
\paragraph*{}
I fall som avses i 3 § tredje stycket börjar efterlevandeskyddet gälla vid det första månadsskiftet ett år efter det att ansökan kom in till Pensionsmyndigheten.
Lag (2013:747).
\subsection*{8 §}
\paragraph*{}
Om äktenskapet eller samboförhållandet upplöses på annat sätt än genom dödsfall, ska Pensionsmyndigheten på begäran av pensionsspararen räkna om försäkringen till att gälla bara på spararens liv.
\subsection*{9 §}
\paragraph*{}
Premiepension till efterlevande vid förordnande om efterlevandeskydd under pensionstiden lämnas endast till efterlevande make eller sådan sambo som avses i 4 §.
\subsection*{10 §}
\paragraph*{}
För den som fått rätt till premiepension till efterlevande enligt 9 §, gäller sedan pensionsspararen avlidit de bestämmelser som avser pensionsspararen med undantag för
\newline - 56 kap. 3 § om tidigaste tidpunkt för uttag av pension, och
\newline - 107 kap. 8 §.
\paragraph*{}
Vidare saknar den som har fått premiepension enligt 9 § möjlighet att ansöka om efterlevandeskydd under pensionstiden såvitt avser denna premiepension.
\chapter*{92 Uttag och utbetalning m.m. av premiepension till efterlevande}
\subsection*{1 §}
\paragraph*{}
I detta kapitel finns bestämmelser om
\newline - uttag av premiepension i 2 §,
\newline - ändring av beräkningsunderlag och pensionsrätt i 3 §, och
\newline - utbetalning av premiepension i 4 §.
\subsection*{2 §}
\paragraph*{}
Premiepension till efterlevande lämnas utan ansökan från och med månaden efter den då pensionsspararen har avlidit.
\paragraph*{}
Den som har fått rätt till premiepension till efterlevande får återkalla sitt uttag av premiepension och ändra den andel av pensionen som tas ut. I sådana fall gäller 56 kap. 10 §.
Vid nytt uttag efter återkallelse och vid ökat uttag gäller bestämmelserna om ansökan i 110 kap. 4 § och bestämmelserna om förmånstid efter ansökan i 56 kap. 4 §.
\subsection*{3 §}
\paragraph*{}
I ärenden om premiepension till efterlevande tillämpas bestämmelserna i
\newline - 70 kap. 2-4 §§ om ändring av beräkningsunderlag för allmän ålderspension, och
\newline - 64 kap. 28-31 §§ om ändrad pensionsrätt.
\subsection*{4 §}
\paragraph*{}
Bestämmelserna i 71 kap. 2 och 5 §§ om utbetalning av allmän ålderspension gäller även premiepension till efterlevande.
\part*{G BOSTADSSTÖD}
\chapter*{93 Innehåll, definitioner och förklaringar}
\subsection*{1 §}
\paragraph*{}
I avdelning G finns bestämmelser om socialförsäkringsförmåner i form av bidrag som är relaterade till kostnader för bostad (bostadsstöd).
\subsection*{2 §}
\paragraph*{}
Förmåner enligt denna avdelning är
\newline - bostadsbidrag till barnfamiljer eller personer i åldern 18-28 år,
\newline - bostadstillägg till den som får sjukersättning, pension eller vissa liknande förmåner, och
\newline - boendetillägg till den som har fått tidsbegränsad sjukersättning eller aktivitetsersättning.
Lag (2011:1514).
\subsection*{3 §}
\paragraph*{}
I detta kapitel finns inledande bestämmelser om bostadsstöd.
\paragraph*{}
Vidare finns bestämmelser om
\newline - bostadsbidrag i 94-98 kap.,
\newline - bostadstillägg i 99-103 kap., och
\newline - boendetillägg i 103 a-103 e kap.
Lag (2011:1513).
\subsection*{4 §}
\paragraph*{}
/Upphör att gälla U:2024-07-01/
En förmån enligt denna avdelning lämnas endast till den som har ett gällande försäkringsskydd för förmånen enligt 4 och 5 kap.
\paragraph*{}
Bestämmelser om anmälan och ansökan samt vissa gemensamma bestämmelser om förmåner och handläggning finns i 104-117 kap. (avdelning H). Bostadsbidrag i form av tilläggsbidrag till barnfamiljer lämnas dock utan ansökan.
Lag (2022:1041).
\subsection*{4 §}
\paragraph*{}
/Träder i kraft I:2024-07-01/
En förmån enligt denna avdelning lämnas endast till den som har ett gällande försäkringsskydd för förmånen enligt 4 och 5 kap.
\paragraph*{}
Bestämmelser om anmälan och ansökan samt vissa gemensamma bestämmelser om förmåner och handläggning finns i 104-117 kap. (avdelning H).
Lag (2022:1042).
\subsection*{5 §}
\paragraph*{}
Ärenden som avser bostadsbidrag och boendetillägg handläggs av Försäkringskassan.
\paragraph*{}
Ärenden som avser bostadstillägg handläggs av Pensionsmyndigheten. Om den försäkrade eller, i förekommande fall, hans eller hennes make inte har någon annan förmån som kan ligga till grund för bostadstillägg än sjukersättning, aktivitetsersättning eller utländsk invaliditetsförmån, handläggs ärendet dock av Försäkringskassan.
\paragraph*{}
Bedömningen enligt andra stycket ska i fall som avses i 101 kap. 4 § göras som om förmånen lämnats.
Lag (2011:1513).
\chapter*{94 Innehåll}
\subsection*{1 §}
\paragraph*{}
I denna underavdelning finns allmänna bestämmelser om bostadsbidrag i 95 kap.
\paragraph*{}
Vidare finns bestämmelser om
\newline - rätten till bostadsbidrag i 96 kap.,
\newline - beräkning av bostadsbidrag i 97 kap., och
\newline - särskilda handläggningsregler för bostadsbidrag i 98 kap.
\chapter*{95 Allmänna bestämmelser om bostadsbidrag}
\subsection*{1 §}
\paragraph*{}
I detta kapitel finns inledande bestämmelser i 2-4 §§.
\paragraph*{}
Vidare finns bestämmelser om
\newline - definitioner i 5 §,
\newline - sambor och makar i 6 och 7 §§, och
\newline - ansökan i 8 §.
\subsection*{2 §}
\paragraph*{}
/Upphör att gälla U:2024-07-01/
Bostadsbidrag lämnas i form av
\newline 1. bidrag till kostnader för bostad,
\newline 2. särskilt bidrag för hemmavarande barn,
\newline 3. särskilt bidrag för barn som bor växelvis,
\newline 4. umgängesbidrag till den som på grund av vårdnad eller umgänge tidvis har barn boende i sitt hem, och
\newline 5. tilläggsbidrag till barnfamiljer.
\paragraph*{}
Med barn som bor växelvis avses ett barn som bor varaktigt hos båda sina föräldrar som inte bor tillsammans.
Lag (2022:1041).
\subsection*{2 §}
\paragraph*{}
/Träder i kraft I:2024-07-01/
Bostadsbidrag lämnas i form av
\newline 1. bidrag till kostnader för bostad,
\newline 2. särskilt bidrag för hemmavarande barn,
\newline 3. särskilt bidrag för barn som bor växelvis, och
\newline 4. umgängesbidrag till den som på grund av vårdnad eller umgänge tidvis har barn boende i sitt hem.
\paragraph*{}
Med barn som bor växelvis avses ett barn som bor varaktigt hos båda sina föräldrar som inte bor tillsammans.
Lag (2022:1042).
\subsection*{3 §}
\paragraph*{}
Rätten till bostadsbidrag och bidragets storlek är beroende av den försäkrades bidragsgrundande inkomst och de andra omständigheter som anges i detta kapitel och 96-98 kap.
\subsection*{4 §}
\paragraph*{}
/Upphör att gälla U:2024-07-01/
Bostadsbidrag enligt 2 § första stycket 1-4 betalas ut löpande som preliminärt bidrag, beräknat efter en uppskattad bidragsgrundande inkomst. Sådant bidrag bestäms slutligt i efterhand på grundval av den fastställda bidragsgrundande inkomsten.
\paragraph*{}
Bostadsbidrag enligt 2 § första stycket 5 beräknas inte på grundval av en bidragsgrundande inkomst. Sådant bidrag betalas ut löpande med ett belopp som bestäms enligt 97 kap. 23 a §.
Lag (2022:1041).
\subsection*{4 §}
\paragraph*{}
/Träder i kraft I:2024-07-01/
Bostadsbidrag betalas ut löpande som preliminärt bidrag, beräknat efter en uppskattad bidragsgrundande inkomst. Bostadsbidraget bestäms slutligt i efterhand på grundval av den fastställda bidragsgrundande inkomsten.
Lag (2022:1042).
\subsection*{5 §}
\paragraph*{}
När det gäller bostadsbidrag avses med
\newline 1. hushåll: barnfamiljer, makar utan barn och ensamstående utan barn,
\newline 2. barn: den som är under 18 år och den som får förlängt barnbidrag eller får studiehjälp enligt 2 kap.
studiestödslagen (1999:1395), och
\newline 3. makar: makar som lever tillsammans.
\subsection*{6 §}
\paragraph*{}
Sambor likställs med makar när det gäller bostadsbidrag.
\paragraph*{}
Om det på grund av omständigheterna är sannolikt att två personer är sambor, ska de likställas med sambor. Detta gäller inte om den som ansöker om bostadsbidrag eller den som bidraget betalas ut till visar att de inte är sambor.
\subsection*{7 §}
\paragraph*{}
När det gäller bostadsbidrag ska personer som är gifta med varandra anses bo tillsammans om inte den som ansöker om bostadsbidrag eller den som får bidraget visar något annat.
\subsection*{8 §}
\paragraph*{}
Ansökan som avser makars bostad ska vara gemensam om det inte finns särskilda skäl för att ansökan görs av den ena maken.
\chapter*{96 Rätten till bostadsbidrag}
\subsection*{1 §}
\paragraph*{}
/Upphör att gälla U:2024-07-01/
I detta kapitel finns allmänna bestämmelser i 2-3 a §§.
\paragraph*{}
Vidare finns bestämmelser om
\newline - barnfamiljer i 4-9 §§,
\newline - hushåll utan barn i 10 och 11 §§, och
\newline - förmånstiden i 12-15 §§.
Lag (2022:1041).
\subsection*{1 §}
\paragraph*{}
/Träder i kraft I:2024-07-01/
I detta kapitel finns allmänna bestämmelser i 2 och 3 §§.
\paragraph*{}
Vidare finns bestämmelser om
\newline - barnfamiljer i 4-9 §§,
\newline - hushåll utan barn i 10 och 11 §§, och
\newline - förmånstiden i 12-15 §§.
Lag (2022:1042).
\subsection*{2 §}
\paragraph*{}
Bostadsbidrag i form av bidrag till kostnader för bostad kan lämnas till
\newline - barnfamiljer, och
\newline - hushåll utan barn.
\paragraph*{}
Sådant bidrag lämnas endast till kostnader för en bostad där den försäkrade är bosatt och folkbokförd. Vidare krävs att han eller hon äger eller innehar bostaden med hyres- eller bostadsrätt. Om det finns särskilda skäl, får bidrag dock lämnas till kostnaderna för en bostad där den försäkrade inte är folkbokförd.
\subsection*{3 §}
\paragraph*{}
Bostadsbidrag i form av särskilt bidrag för hemmavarande barn kan lämnas till familjer med barn som avses i 4, 7 och 8 §§.
\paragraph*{}
Bostadsbidrag i form av särskilt bidrag för barn som bor växelvis kan lämnas till familjer med barn som avses i 5 a §.
\paragraph*{}
Bostadsbidrag i form av umgängesbidrag kan lämnas till den som tidvis bor med barn i fall som avses i 6 §. Sådant bidrag lämnas endast till en försäkrad som bor i en bostad som han eller hon äger eller innehar med hyres- eller bostadsrätt.
Lag (2017:1123).
\subsection*{3 a §}
\paragraph*{}
Har upphävts genom
lag (2022:1042).
\subsection*{4 §}
\paragraph*{}
Bostadsbidrag kan lämnas till en försäkrad som har vårdnaden om och varaktigt bor tillsammans med barn.
\paragraph*{}
Bostadsbidrag kan lämnas även för barn som för vård eller undervisning inte varaktigt bor hemma men vistas hemma under minst så lång tid varje år som motsvarar normala skolferier.
\subsection*{5 §}
\paragraph*{}
Om ett barns föräldrar inte lever tillsammans men har gemensam vårdnad om barnet, har den förälder som barnet är folkbokfört hos rätt till bostadsbidrag enligt 4 och 8 §§.
\subsection*{5 a §}
\paragraph*{}
Bostadsbidrag kan lämnas till en försäkrad som har vårdnaden om ett barn som bor växelvis hos den försäkrade.
Lag (2017:1123).
\subsection*{6 §}
\paragraph*{}
Bostadsbidrag kan lämnas till en försäkrad som på grund av vårdnad eller umgänge tidvis har barn boende i sitt hem.
\paragraph*{}
Bostadsbidrag enligt första stycket lämnas endast för bostäder som
\newline - omfattar minst två rum utöver kök eller kokvrå, och
\newline - har en bostadsyta som uppgår till minst 40 kvadratmeter.
\subsection*{7 §}
\paragraph*{}
Bostadsbidrag kan lämnas till en försäkrad som efter en myndighets beslut har tagit emot barn för vård i familjehem, om barnet beräknas bo i hemmet under minst tre månader.
\subsection*{8 §}
\paragraph*{}
När ett barn bor i ett familjehem, stödboende eller hem för vård eller boende, kan bidrag lämnas även till föräldern eller föräldrarna, om det finns särskilda skäl.
Lag (2015:983).
\subsection*{9 §}
\paragraph*{}
Bestämmelserna i 4-7 §§ gäller endast i tillämpliga delar i fråga om barn som är över 18 år och som får förlängt barnbidrag eller studiehjälp enligt 2 kap. studiestödslagen (1999:1395).
\subsection*{10 §}
\paragraph*{}
Bostadsbidrag till hushåll utan barn kan lämnas till en försäkrad som har fyllt 18 år.
\subsection*{11 §}
\paragraph*{}
Bostadsbidrag enligt 10 § kan inte lämnas om den försäkrade eller dennes make
\newline 1. har fyllt 29 år,
\newline 2. har barn som berättigar till bostadsbidrag till barnfamiljer, eller
\newline 3. får bostadstillägg eller på grund av bestämmelser om inkomstprövning inte får sådant tillägg.
\paragraph*{}
En försäkrad som på grund av vårdnad eller umgänge tidvis har barn boende i sitt hem har rätt till bostadsbidrag till hushåll utan barn utan hinder av första stycket 2, om bostadsbidrag till barnfamiljer inte kan lämnas på grund av att kraven i 6 § andra stycket på bostadens storlek och utformning inte är uppfyllda.
\subsection*{12 §}
\paragraph*{}
Bidrag lämnas från och med månaden efter den då rätten till bidrag har uppkommit till och med den månad då rätten till bidrag har ändrats eller upphört.
\paragraph*{}
Om rätten till bidrag uppkommit eller upphört den första dagen i en månad, ska dock bidraget lämnas eller upphöra från och med den månaden.
\subsection*{13 §}
\paragraph*{}
Ett beslut om bidrag får avse längst tolv månader.
\subsection*{14 §}
\paragraph*{}
Bidrag får inte lämnas för längre tid tillbaka än ansökningsmånaden. Bidrag får dock lämnas för en längre tid tillbaka, om en ansökan om bidrag avseende höjning av hyra eller avgift ges in till Försäkringskassan inom en månad från den dag då den försäkrade fick kännedom om hyres- eller avgiftshöjningen.
\subsection*{15 §}
\paragraph*{}
Om ett barn avlider i en barnfamilj som avses i 4-9 §§ och familjen redan har ansökt om eller får bostadsbidrag får bidraget lämnas som om barnet alltjämt levde till och med den sjätte månaden efter dödsfallet. Detta gäller längst till och med den månad då familjen flyttar från bostaden.
\chapter*{97 Beräkning av bostadsbidrag}
\subsection*{1 §}
\paragraph*{}
/Upphör att gälla U:2024-07-01/
I detta kapitel finns bestämmelser om
\newline - bidragsgrundande inkomst i 2-13 §§,
\newline - kostnader för bostad i 14 §,
\newline - beräkning av bostadsbidrag till barnfamiljer i 15-18 och 20-23 a §§,
\newline - beräkning av bostadsbidrag till hushåll utan barn i 24-28 §§, och
\newline - undantag när bidragsbehov saknas i 29 §.
Lag (2022:1041).
\subsection*{1 §}
\paragraph*{}
/Träder i kraft I:2024-07-01/
I detta kapitel finns bestämmelser om
\newline - bidragsgrundande inkomst i 2-13 §§,
\newline - kostnader för bostad i 14 §,
\newline - beräkning av bostadsbidrag till barnfamiljer i 15-18 och 20-23 §§,
\newline - beräkning av bostadsbidrag till hushåll utan barn i 24-28 §§, och
\newline - undantag när bidragsbehov saknas i 29 §.
Lag (2022:1042).
\subsection*{2 §}
\paragraph*{}
Den försäkrades bidragsgrundande inkomst är summan av
\newline 1. överskott i inkomstslaget tjänst enligt 10 kap. 16 § inkomstskattelagen (1999:1229),
\newline 2. överskott i inkomstslaget näringsverksamhet beräknat enligt 4 §,
\newline 3. överskott i inkomstslaget kapital beräknat enligt 5 §,
\newline 4. del av egen och barns förmögenhet enligt 6-10 §§,
\newline 5. del av barns inkomst av kapital enligt 11 och 12 §§, och
\newline 6. vissa andra inkomster enligt 13 §.
\paragraph*{}
Vid beräkning av inkomster som avses i första stycket 6 ska studiebidrag enligt 3 kap. studiestödslagen (1999:1395) och studiestartsstöd enligt lagen (2017:527) om studiestartsstöd ingå med 80 procent.
Lag (2018:1628).
\subsection*{3 §}
\paragraph*{}
Den bidragsgrundande inkomsten ska avse samma kalenderår som bostadsbidraget och ska anses vara lika fördelad på varje månad under kalenderåret.
\subsection*{4 §}
\paragraph*{}
Överskott av en näringsverksamhet enligt 14 kap. 21 § inkomstskattelagen (1999:1229) ska
\paragraph*{}
ökas med
\newline 1. avdrag enligt 16 kap. 32 § inkomstskattelagen för utgift för egen pension intill ett belopp motsvarande ett halvt prisbasbelopp,
\newline 2. avdrag för avsättning till periodiseringsfond enligt 30 kap. inkomstskattelagen,
\newline 3. avdrag för avsättning till expansionsfond enligt 34 kap.
inkomstskattelagen, och
\newline 4. avdrag för underskott från tidigare beskattningsår enligt 40 kap. inkomstskattelagen,
samt minskas med
\newline 5. återfört avdrag för avsättning till periodiseringsfond, och
\newline 6. återfört avdrag för avsättning till expansionsfond.
\paragraph*{}
Underskott av en näringsverksamhet som avses i första stycket ska minskas med avdrag som avses i första stycket 1-4 samt ökas med återförda avdrag som avses i första stycket 5 och 6.
\subsection*{5 §}
\paragraph*{}
Överskott i inkomstslaget kapital enligt 41 kap. 12 § inkomstskattelagen (1999:1229) ska ökas med gjorda avdrag i inkomstslaget, dock inte med
\newline 1. avdrag för kapitalförluster till den del de motsvarar kapitalvinster som tagits upp som intäkt enligt 42 kap. 1 § inkomstskattelagen,
\newline 2. avdrag för uppskovsbelopp enligt 47 kap. inkomstskattelagen vid byte av bostad, och
\newline 3. avdrag för negativ räntefördelning enligt 33 kap.
inkomstskattelagen.
\paragraph*{}
Underskott i inkomstslaget kapital ska minskas med gjorda avdrag i inkomstslaget, dock inte med avdrag som avses i första stycket 1-3.
\paragraph*{}
Uppkommer ett överskott vid beräkningen ska detta minskas med schablonintäkt enligt 42 kap. 36 och 43 §§ samt 47 kap. 11 b § inkomstskattelagen.
Lag (2011:1288).
\subsection*{6 §}
\paragraph*{}
Tillägg till den bidragsgrundande inkomsten ska göras med 15 procent av den sammanlagda förmögenhet som överstiger 100 000 kronor, avrundad nedåt till helt tiotusental kronor.
\paragraph*{}
I den sammanlagda förmögenheten ingår förmögenhet för varje försäkrad samt för varje barn som avses i 96 kap. 4, 5 a och 9 §§. I fråga om barn som avses i 96 kap. 5 a § ska dock endast hälften av den förmögenhet som tillhör barnet ingå i den sammanlagda förmögenheten.
Lag (2017:1123).
\subsection*{7 §}
\paragraph*{}
Tillägg för makars förmögenhet ska fördelas på makarna i förhållande till deras respektive andel av den sammanlagda förmögenheten.
Tillägg för barns förmögenhet ska fördelas lika mellan makar.
\subsection*{8 §}
\paragraph*{}
Förmögenheten beräknas enligt lagen (2009:1053) om förmögenhet vid beräkning av vissa förmåner, om inte annat följer av 9 och 10 §§.
Förmögenheten beräknas per den 31 december det år som bostadsbidraget avser.
\subsection*{9 §}
\paragraph*{}
Vid beräkning av ett barns förmögenhet ska undantas ett belopp som motsvarar ersättning som barnet 1. har fått med anledning av personskada eller kränkning, om ersättningen inte avser ersättning för kostnader, eller
\newline 2. har fått från en försäkring med anledning av att barnet ådragit sig skada genom olycksfall eller sjukdom, om ersättningen inte avser ersättning för sakskada.
\paragraph*{}
Första stycket gäller även en fordran som avser sådana ersättningar. Om ersättningen till någon del är skattepliktig enligt inkomstskattelagen (1999:1229) gäller undantaget beloppet före beskattning.
\subsection*{10 §}
\paragraph*{}
Vid beräkningen av ett barns förmögenhet ska sådana tillgångar undantas som barnet har förvärvat genom
\newline 1. gåva eller testamente,
\newline 2. förmånstagarförordnande vid försäkring eller pensionssparande enligt lagen (1993:931) om individuellt pensionssparande, och
\newline 3. avkastning av och sådant som trätt i stället för sådan egendom som avses i 1 eller 2.
\paragraph*{}
Ett undantag enligt första stycket förutsätter dock att förvärvet har skett från någon annan än barnets förmyndare och att det är förenat med villkor
\newline - som innebär att egendomen ska stå under förvaltning av någon annan än förmyndaren, utan bestämmanderätt för denne, och
\newline - som anger vem som i stället ska utöva förvaltningen.
\subsection*{11 §}
\paragraph*{}
Om ett eller flera barn som avses i 96 kap. 4, 5 a och 9 §§ har överskott i inkomstslaget kapital, ska till den försäkrades bidragsgrundande inkomst läggas varje barns överskott i inkomstslaget kapital i den utsträckning som överskottet överstiger 1 000 kronor. I fråga om barn som avses i 96 kap. 5 a § ska dock endast hälften av varje barns överskott läggas till den bidragsgrundande inkomsten i den utsträckning överskottet överstiger 1 000 kronor.
\paragraph*{}
Vid beräkning av överskott i inkomstslaget kapital tillämpas 5 §.
Lag (2017:1123).
\subsection*{12 §}
\paragraph*{}
Tillägg enligt 11 § ska fördelas lika mellan makar.
\subsection*{13 §}
\paragraph*{}
Med vissa andra inkomster enligt 2 § första stycket 6 avses följande:
\newline 1. inkomst som på grund av 3 kap. 9-13 §§ inkomstskattelagen (1999:1229) eller skatteavtal inte ska tas upp som intäkt i inkomstslaget näringsverksamhet, tjänst eller kapital,
\newline 2. studiemedel i form av studiebidrag enligt 3 kap. studiestödslagen (1999:1395) och studiestartsstöd enligt lagen (2017:527) om studiestartsstöd, utom de delar som avser tilläggsbidrag,
\newline 3. skattefria stipendier över 3 000 kronor per månad,
\newline 4. skattepliktiga inkomster enligt 5 § lagen (1991:586) om särskild inkomstskatt för utomlands bosatta, och
\newline 5. etableringsersättning för vissa nyanlända invandrare.
\paragraph*{}
Inkomster som avses i första stycket 2 och 3 och som i slutet av ett år betalas ut i förskott som hänförliga till nästföljande år ska beaktas för det senare året.
Lag (2018:1628).
\subsection*{14 §}
\paragraph*{}
Regeringen eller den myndighet som regeringen bestämmer meddelar föreskrifter om vilka bostadskostnader som beaktas när det gäller bostadsbidrag.
\subsection*{15 §}
\paragraph*{}
/Upphör att gälla U:2024-07-01/
Bostadsbidrag till barnfamiljer beräknas enligt 18-23 a §§.
\paragraph*{}
Om den bidragsgrundande inkomsten överstiger 150 000 kronor för en försäkrad, eller för makar 75 000 kronor för var och en av dem, ska bidraget enligt 95 kap. 2 § första stycket 1-4 minskas med 20 procent av den överskjutande inkomsten.
Lag (2022:1041).
\subsection*{15 §}
\paragraph*{}
/Träder i kraft I:2024-07-01/
Bostadsbidrag till barnfamiljer beräknas enligt 18-23 §§.
\paragraph*{}
Om den bidragsgrundande inkomsten överstiger 150 000 kronor för en försäkrad, eller för makar 75 000 kronor för var och en av dem, ska bidraget minskas med 20 procent av den överskjutande inkomsten.
Lag (2022:1042).
\subsection*{16 §}
\paragraph*{}
Bidrag som beräknas till mindre belopp än 1 200 kronor för helt år betalas inte ut.
\subsection*{17 §}
\paragraph*{}
Bostadsbidrag får inte lämnas för fler än tre barn.
\subsection*{18 §}
\paragraph*{}
Bostadsbidrag lämnas månadsvis som bidrag till kostnader för bostad med 50 procent av den del av bostadskostnaden per månad som för familjer med
\newline - ett barn överstiger 1 400 kronor men inte 5 300 kronor,
\newline - två barn överstiger 1 400 kronor men inte 5 900 kronor, och
\newline - tre eller flera barn överstiger 1 400 kronor men inte 6 600 kronor.
\paragraph*{}
Högre bostadskostnad än som anges i första stycket får beaktas, om någon medlem av familjen är funktionshindrad.
Lag (2011:1520).
\subsection*{19 §}
\paragraph*{}
Har upphävts genom
lag (2011:1520).
\subsection*{20 §}
\paragraph*{}
Vid tillämpning av bestämmelserna om bostadskostnader i 18 § ska Försäkringskassan bortse från bostadskostnader som avser bostadsytor som överstiger - 80 kvadratmeter för hushåll med ett barn,
\newline - 100 kvadratmeter för hushåll med två barn,
\newline - 120 kvadratmeter för hushåll med tre barn,
\newline - 140 kvadratmeter för hushåll med fyra barn, och
\newline - 160 kvadratmeter för hushåll med fem eller flera barn.
\paragraph*{}
Större bostadsytor än som anges i första stycket får beaktas, om någon medlem av familjen är funktionshindrad.
\subsection*{21 §}
\paragraph*{}
Försäkringskassan ska beakta bostadskostnader upp till följande belopp per månad, utan hinder av bestämmelserna i 20 § första stycket:
\newline - 3 000 kronor för hushåll med ett barn,
\newline - 3 300 kronor för hushåll med två barn,
\newline - 3 600 kronor för hushåll med tre barn,
\newline - 3 900 kronor för hushåll med fyra barn, och
\newline - 4 200 kronor för hushåll med fem eller flera barn.
\subsection*{22 §}
\paragraph*{}
Bostadsbidrag lämnas månadsvis som särskilt bidrag för hemmavarande barn med
\newline - 1 500 kronor till familjer med ett barn,
\newline - 2 000 kronor till familjer med två barn, och
\newline - 2 650 kronor till familjer med tre eller flera barn.
Lag (2017:1123).
\subsection*{22 a §}
\paragraph*{}
Bostadsbidrag lämnas månadsvis som särskilt bidrag för barn som bor växelvis med
\newline - 1 300 kronor till familjer med ett barn,
\newline - 1 600 kronor till familjer med två barn, och
\newline - 2 100 kronor till familjer med tre eller flera barn.
\paragraph*{}
Om det i hushållet finns såväl hemmavarande barn som barn som bor växelvis, lämnas särskilt bidrag för barn som bor växelvis med
\newline - 300 kronor till familjer med ett hemmavarande barn och ett barn som bor växelvis,
\newline - 500 kronor till familjer med två hemmavarande barn och ett barn som bor växelvis, och
\newline - 800 kronor till familjer med ett hemmavarande barn och två barn som bor växelvis.
Lag (2017:1123).
\subsection*{23 §}
\paragraph*{}
Bostadsbidrag lämnas månadsvis som umgängesbidrag med
\newline - 500 kronor för ett barn,
\newline - 600 kronor för två barn, och
\newline - 700 kronor för tre eller flera barn.
\paragraph*{}
Om det i hushållet förutom hemmavarande barn eller barn som bor växelvis även finns sådana barn som avses i 95 kap. 2 § 4 lämnas umgängesbidrag med 100 kronor per månad för varje barn som berättigar till umgängesbidrag.
Lag (2017:1123).
\subsection*{23 a §}
\paragraph*{}
Har upphävts genom
lag (2022:1042).
\subsection*{24 §}
\paragraph*{}
Bostadsbidrag till hushåll utan barn beräknas enligt 26-28 §§.
\paragraph*{}
Om den bidragsgrundande inkomsten överstiger 41 000 kronor för en försäkrad, eller 58 000 kronor för makar, ska bidraget minskas med en tredjedel av den överskjutande inkomsten. 25 § Bidrag som beräknas till mindre belopp än 1 200 kronor för helt år betalas inte ut.
\subsection*{26 §}
\paragraph*{}
Bostadsbidrag i form av bidrag till kostnaderna för bostaden lämnas med 90 procent av den del av bostadskostnaden per månad som överstiger 1 800 kronor men inte 2 600 kronor.
Lag (2011:1520)
. 27 § Om bostadskostnaden överstiger 2 600 kronor lämnas bostadsbidrag med 65 procent av den överskjutande bostadskostnaden per månad upp till 3 600 kronor.
\paragraph*{}
Högre bostadskostnad än som anges i första stycket får beaktas, om den försäkrade eller hans eller hennes make är funktionshindrad.
Lag (2011:1520).
\subsection*{28 §}
\paragraph*{}
Vid tillämpning av bestämmelserna om bostadskostnader i 26 och 27 §§ ska Försäkringskassan bortse från bostadskostnader som avser bostadsytor som överstiger 60 kvadratmeter.
\paragraph*{}
Större bostadsyta än som anges i första stycket får beaktas, om den försäkrade eller hans eller hennes make är funktionshindrad.
\subsection*{29 §}
\paragraph*{}
Om det är uppenbart att den försäkrade på grund av hushållets inkomst eller förmögenhet eller någon annan omständighet inte behöver det bostadsbidrag som kan beräknas enligt bestämmelserna i denna balk, får Försäkringskassan efter särskild utredning avslå en ansökan om bidrag eller dra in eller sätta ned bidraget.
\paragraph*{}
Ett sådant beslut får fattas även om viss inkomst eller förmögenhet inte ska räknas in i den bidragsgrundande inkomsten.
\chapter*{98 Särskilda handläggningsregler för bostadsbidrag}
\subsection*{1 §}
\paragraph*{}
/Upphör att gälla U:2024-07-01/
I detta kapitel finns bestämmelser om
\newline - preliminärt bostadsbidrag i 2-4 §§,
\newline - slutligt bostadsbidrag i 5-7 §§,
\newline - omprövning vid ändrade förhållanden i 8 och 9 §§,
\newline - utbetalning till annan än den försäkrade i 10 och 11 §§, och
\newline - tilläggsbidrag till barnfamiljer i 12 §.
Lag (2022:1041).
\subsection*{1 §}
\paragraph*{}
/Träder i kraft I:2024-07-01/
I detta kapitel finns bestämmelser om
\newline - preliminärt bostadsbidrag i 2-4 §§,
\newline - slutligt bostadsbidrag i 5-7 §§,
\newline - omprövning vid ändrade förhållanden i 8 och 9 §§, och
\newline - utbetalning till annan än den försäkrade i 10 och 11 §§.
Lag (2022:1042).
\subsection*{2 §}
\paragraph*{}
Preliminärt bostadsbidrag beräknas efter en uppskattad bidragsgrundande inkomst och ska så nära som möjligt motsvara det slutliga bostadsbidrag som kan antas komma att bestämmas.
\subsection*{3 §}
\paragraph*{}
Vid tillämpningen av 96 kap. 15 § ska bostadsbidraget beräknas som om barnet fortfarande levde och grundas på barnets ekonomiska förhållanden månaden före dödsfallet.
\subsection*{4 §}
\paragraph*{}
Preliminärt bostadsbidrag betalas ut med ett månadsbelopp, som avrundas nedåt till närmaste jämna hundratal kronor.
\subsection*{5 §}
\paragraph*{}
Slutligt bostadsbidrag bestäms för varje kalenderår under vilket preliminärt bidrag har betalats ut. Slutligt bostadsbidrag bestäms efter det att beslut om slutlig skatt enligt 56 kap. 2 § skatteförfarandelagen (2011:1244) har meddelats.
Lag (2018:1628).
\subsection*{6 §}
\paragraph*{}
Bestäms det slutliga bostadsbidraget till högre belopp än det som för samma år har betalats ut i preliminärt bidrag, ska skillnaden betalas ut. Bestäms det slutliga bostadsbidraget till lägre belopp än det som för samma år har betalats ut i preliminärt bidrag, ska skillnaden betalas tillbaka enligt vad som anges i 108 kap. 9, 11-14 och 22 §§.
\paragraph*{}
Belopp under 1 200 kronor ska varken betalas ut eller betalas tillbaka.
\subsection*{7 §}
\paragraph*{}
Belopp som ska betalas ut enligt 6 § första stycket ska ökas med ett tilllägg. Tillägget på det överskjutande beloppet beräknas med ledning av den statslåneränta som gällde vid utgången av det kalenderår som bidraget avser.
\paragraph*{}
På belopp som ska betalas tillbaka enligt 6 § första stycket ska en avgift betalas. Avgiften på återbetalningsbeloppet beräknas med ledning av den statslåneränta som gällde vid utgången av det kalenderår som bidraget avser.
Regeringen eller den myndighet som regeringen bestämmer meddelar närmare föreskrifter om beräkningen av tillägget och avgiften.
Omprövning vid ändrade förhållanden
\subsection*{8 §}
\paragraph*{}
Det preliminära bidraget ska omprövas om något har inträffat som påverkar storleken av bidraget.
Försäkringskassan får avstå från att besluta om ändring, om det som har inträffat endast i liten utsträckning påverkar bidraget.
\subsection*{9 §}
\paragraph*{}
Om den försäkrades beslut om slutlig skatt, beslut om studiemedel i form av studiebidrag eller beslut om studiestartsstöd ändras efter det att slutligt bostadsbidrag bestämts och ändringen innebär att bostadsbidraget skulle ha varit högre eller lägre, ska ett nytt bostadsbidrag bestämmas, om den försäkrade begär det eller om Försäkringskassan självmant tar upp frågan.
\paragraph*{}
En fråga om nytt slutligt bostadsbidrag enligt denna paragraf får inte tas upp efter utgången av sjätte året efter det år beslutet om slutlig skatt avser eller sjätte året efter det att studiebidraget lämnades.
Lag (2017:528).
\subsection*{10 §}
\paragraph*{}
På begäran av den försäkrade får Försäkringskassan besluta att bidraget ska betalas ut till någon annan än honom eller henne.
\subsection*{11 §}
\paragraph*{}
Om det finns synnerliga skäl, får Försäkringskassan på framställning av socialnämnden betala ut bidraget till en lämplig person eller till nämnden att användas för hushållets bästa.
\subsection*{12 §}
\paragraph*{}
Har upphävts genom
lag (2022:1042).
\chapter*{99 Innehåll}
\subsection*{1 §}
\paragraph*{}
I denna underavdelning finns allmänna bestämmelser om bostadstillägg i 100 kap.
\paragraph*{}
Vidare finns bestämmelser om
\newline - rätten till bostadstillägg i 101 kap.,
\newline - beräkning av bostadstillägg i 102 kap., och
\newline - särskilda handläggningsregler för bostadstillägg i 103 kap.
\chapter*{100 Allmänna bestämmelser om bostadstillägg}
\subsection*{1 §}
\paragraph*{}
I detta kapitel finns en inledande bestämmelse i 2 §.
\paragraph*{}
Vidare finns bestämmelser om makar och sambor i 3 och 4 §§.
\subsection*{2 §}
\paragraph*{}
Bostadstillägg är ett inkomstprövat tillägg till vissa andra socialförsäkringsförmåner och lämnas i form av
\newline - bostadstillägg, och
\newline - särskilt bostadstillägg.
\subsection*{3 §}
\paragraph*{}
Sambor likställs med makar när det gäller bostadstillägg.
Om det på grund av omständigheterna är sannolikt att två personer är sambor, ska dessa likställas med sambor. Detta gäller inte om den som ansöker om bostadstillägg eller den som sådant bidrag betalas ut till visar att de inte är sambor.
\subsection*{4 §}
\paragraph*{}
När det gäller bostadstillägg ska en person som är gift men stadigvarande lever åtskild från sin make likställas med en ogift person, om inte särskilda skäl talar mot detta.
Lag (2010:1307).
\chapter*{101 Rätten till bostadstillägg}
\subsection*{1 §}
\paragraph*{}
I detta kapitel finns inledande bestämmelser i 2 §.
\paragraph*{}
Vidare finns bestämmelser om
\newline - förmåner som kan ge rätt till bostadstillägg i 3-5 §§,
\newline - undantag från rätten till bostadstillägg i 6 §,
\newline - bostadskostnader i 7 och 8 §§,
\newline - samordning med bostadsbidrag i 9 §, och
\newline - förmånstiden i 10-11 §§.
Lag (2013:747).
\subsection*{2 §}
\paragraph*{}
Bostadstillägg till vissa andra förmåner kan lämnas enligt 3-6 §§.
\paragraph*{}
Den som får bostadstillägg kan dessutom få särskilt bostadstillägg.
\subsection*{3 §}
\paragraph*{}
Bostadstillägg kan lämnas till den som får
\newline 1. sjukersättning eller aktivitetsersättning,
\newline 2. hel allmän ålderspension,
\newline 3. änkepension, eller
\newline 4. pension eller invaliditetsförmån enligt lagstiftningen i en stat som ingår i Europeiska ekonomiska samarbetsområdet, under förutsättning att förmånen motsvarar svensk pension eller ersättning enligt 1-3.
\paragraph*{}
Särskilt om sjukersättning eller aktivitetsersättning
\subsection*{4 §}
\paragraph*{}
Bostadstillägg kan lämnas även när sjukersättning eller aktivitetsersättning har förklarats vilande enligt 36 kap. 13-15 §§ eller inte lämnas på grund av bestämmelserna i 106 kap. 16 §.
\paragraph*{}
Bostadstillägg kan också lämnas till en person som enbart på grund av bestämmelserna i 37 kap. 6 och 7 §§ inte uppbär sjukersättning.
\subsection*{5 §}
\paragraph*{}
Bostadstillägg kan också lämnas till en änka som enbart på grund av samordningsbestämmelserna i 77 kap. 12 § inte får änkepension.
\subsection*{6 §}
\paragraph*{}
/Upphör att gälla U:2025-12-01/
Följande förmåner kan inte ligga till grund för bostadstillägg:
\newline 1. ålderspension för tid före den månad då den försäkrade fyller 66 år,
\newline 2. änkepension för tid från och med den månad då änkan fyller 66 år, och
\newline 3. änkepension till en änka som är född 1945 eller senare på grund av dödsfall efter 2002.
Lag (2022:878).
\subsection*{6 §}
\paragraph*{}
/Träder i kraft I:2025-12-01/
Följande förmåner kan inte ligga till grund för bostadstillägg:
\newline 1. ålderspension för tid före den månad då den försäkrade uppnår riktåldern för pension,
\newline 2. änkepension för tid från och med den månad då änkan uppnår riktåldern för pension, och
\newline 3. änkepension till en änka som är född 1945 eller senare på grund av dödsfall efter 2002.
Lag (2022:879).
\subsection*{7 §}
\paragraph*{}
Bostadstillägg lämnas endast för den bostad där den försäkrade har sitt huvudsakliga boende (permanentbostaden).
För bostad i särskild boendeform lämnas bostadstillägg endast för boende i lägenhet eller för boende i en- eller tvåbäddsrum.
\subsection*{8 §}
\paragraph*{}
Bostadstillägg lämnas inte för sådan bostadskostnad som fastställs med beaktande av den försäkrades inkomst.
\subsection*{9 §}
\paragraph*{}
Bostadstillägg lämnas inte för bostadskostnad till den del bostadskostnaden motsvaras av preliminärt bostadsbidrag enligt 98 kap.
\subsection*{10 §}
\paragraph*{}
Bostadstillägg lämnas från och med den månad som anges i ansökan, dock inte för längre tid tillbaka än tre månader före ansökningsmånaden, om inte annat anges i 10 a §.
Lag (2013:747).
\subsection*{10 a §}
\paragraph*{}
Bostadstillägg lämnas från och med den månad för vilken sjukersättning eller aktivitetsersättning har beviljats, om en ansökan om bostadstillägg inkommit senast under månaden efter den under vilken beslutet om sjukersättning eller aktivitetsersättning fattades.
Första stycket gäller endast när sjukersättning eller aktivitetsersättning har beviljats efter ansökan.
Lag (2013:747).
\subsection*{11 §}
\paragraph*{}
Bostadstillägg lämnas tills vidare, men får beviljas för viss tid.
Lag (2012:599).
\chapter*{102 Beräkning av bostadstillägg}
\subsection*{1 §}
\paragraph*{}
I detta kapitel finns allmänna bestämmelser i 2-6 §§.
\paragraph*{}
Vidare finns bestämmelser om
\newline - beräkning av bidragsgrundande inkomst i 7-13 och 15 §§,
\newline - beräkning av reduceringsinkomst i 16-19 §§,
\newline - häktade eller intagna m.fl. i 20 §,
\newline - beräkning av bostadstillägg i 21-25 a §§, och
\newline - beräkning av särskilt bostadstillägg i 26-30 §§.
Lag (2019:651).
\subsection*{2 §}
\paragraph*{}
Storleken på bostadstillägget är beroende av den försäkrades och, om den försäkrade är gift, även makens
\newline - bidragsgrundande inkomst, och
\newline - reduceringsinkomst.
\subsection*{3 §}
\paragraph*{}
När det gäller bostadstillägg avses med bidragsgrundande inkomst den inkomst enligt 7-15 §§ för år räknat, som någon kan antas komma att få under den närmaste tiden.
\subsection*{4 §}
\paragraph*{}
I fråga om en försäkrad som är gift beräknas den bidragsgrundande inkomsten för den försäkrade och hans eller hennes make var för sig.
Reduceringsinkomst
\subsection*{5 §}
\paragraph*{}
När det gäller bostadstillägg avses med reduceringsinkomst den bidragsgrundande inkomst som återstår efter det att beräkning gjorts enligt 16-19 §§.
\subsection*{6 §}
\paragraph*{}
I fråga om makar ska reduceringsinkomsten för var och en av dem beräknas utgöra hälften av deras sammanlagda reduceringsinkomst.
Beräkning av bidragsgrundande inkomst
\subsection*{7 §}
\paragraph*{}
Den bidragsgrundande inkomsten är summan av
\newline 1. ett uppskattat överskott i inkomstslaget tjänst enligt 10 kap. 16 § inkomstskattelagen (1999:1229) minskat med inkomstpensionstillägg enligt 74 a kap.,
\newline 2. ett uppskattat överskott i inkomstslaget näringsverksamhet beräknat enligt 8 §,
\newline 3. överskott i inkomstslaget kapital beräknat enligt 9 § utifrån de uppgifter som har legat till grund för Skatteverkets beslut om slutlig skatt, för senast möjliga beskattningsår, som har fattats närmast före den första månad som prövningen av bostadstillägget avser,
\newline 4. del av förmögenhet enligt 10-13 §§, och
\newline 5. vissa andra inkomster enligt 15 §.
Lag (2020:1239).
\subsection*{8 §}
\paragraph*{}
Överskott av en näringsverksamhet beräknat enligt 14 kap. 21 § inkomstskattelagen (1999:1229) ska ökas i enlighet med vad som anges i 97 kap. 4 § första stycket 1 och 4. Underskott av näringsverksamheten ska minskas med avdrag som avses i 97 kap. 4 § första stycket 1 och 4.
Lag (2011:1075).
\subsection*{9 §}
\paragraph*{}
Överskott i inkomstslaget kapital ska beräknas enligt bestämmelserna i 97 kap. 5 §.
\subsection*{10 §}
\paragraph*{}
Förmögenhet enligt 7 § 4 beräknas per den 31 december året före det år då ansökan om bostadstillägg görs eller, vid omprövning av bidragsgrundande inkomst av annan anledning än som anges i 103 kap. 3 §, per den 31 december året före det år bostadstillägget avser.
Lag (2012:599).
\subsection*{11 §}
\paragraph*{}
Tillägg till den bidragsgrundande inkomsten ska göras med 15 procent av den sammanlagda förmögenhet som överstiger 100 000 kronor för den som är ogift och 200 000 kronor för makar. Det framräknade förmögenhetsbeloppet avrundas nedåt till helt tusental kronor.
\subsection*{12 §}
\paragraph*{}
I fråga om makar ska värdet av förmögenhet för var och en av dem beräknas utgöra hälften av deras sammanlagda förmögenhet efter avdrag enligt 11 §.
\subsection*{13 §}
\paragraph*{}
Förmögenheten beräknas enligt lagen (2009:1053) om förmögenhet vid beräkning av vissa förmåner.
\subsection*{15 §}
\paragraph*{}
Med vissa andra inkomster enligt 7 § 5 avses detsamma som i 97 kap. 13 § första stycket 1-3 och 5 samt ersättning från avtalsgruppsjukförsäkringar för sjukdomsfall som inträffat före år 1991.
Lag (2015:758).
\subsection*{16 §}
\paragraph*{}
Reduceringsinkomsten är summan av
\newline 1. sjukersättning, aktivitetsersättning, allmän ålderspension och änkepension,
\newline 2. pension och invaliditetsförmån som lämnas enligt utländsk lagstiftning,
\newline 3. inkomst av kapital enligt 7 § 3,
\newline 4. del av förmögenhet enligt 7 § 4,
\newline 5. 50 procent av de delar av den bidragsgrundande inkomsten som består av arbetsinkomst enligt 67 kap. 6 § första stycket inkomstskattelagen (1999:1229),
\newline 6. 50 procent av de delar av inkomsten enligt 15 § som, bortsett från att skatteplikt inte föreligger, är av motsvarande slag som arbetsinkomst enligt 67 kap. 6 § första stycket inkomstskattelagen, och
\newline 7. 80 procent av övriga delar av den bidragsgrundande inkomsten, minskad med ett fribelopp enligt 17-19 §§.
Lag (2019:651).
\subsection*{16 a §}
\paragraph*{}
/Upphör att gälla U:2025-12-01/
Från och med den månad en försäkrad, som har hel allmän ålderspension eller motsvarande utländsk förmån, fyller 66 år är reduceringsinkomsten summan av
\newline 1. 93 procent av den inkomstgrundade ålderspensionen,
\newline 2. garantipension och änkepension,
\newline 3. 93 procent av den pension och invaliditetsförmån som lämnas enligt utländsk lagstiftning,
\newline 4. inkomst av kapital enligt 7 § 3,
\newline 5. del av förmögenhet enligt 7 § 4,
\newline 6. 93 procent av de delar av den bidragsgrundande inkomsten som består av arbetsinkomst enligt 67 kap. 6 § första stycket inkomstskattelagen (1999:1229),
\newline 7. 93 procent av de delar av den inkomst enligt 15 § som, bortsett från att skatteplikt inte föreligger, är av motsvarande slag som arbetsinkomst enligt 67 kap. 6 § första stycket inkomstskattelagen, och
\newline 8. 93 procent av övriga delar av den bidragsgrundande inkomsten, minskad med ett fribelopp enligt 17 §.
\paragraph*{}
Inkomst enligt första stycket 6 och 7 ska endast beaktas till den del den överstiger 24 000 kronor.
Lag (2022:878).
\subsection*{16 a §}
\paragraph*{}
/Träder i kraft I:2025-12-01/
Från och med den månad en försäkrad, som har hel allmän ålderspension eller motsvarande utländsk förmån, uppnår riktåldern för pension är reduceringsinkomsten summan av
\newline 1. 93 procent av den inkomstgrundade ålderspensionen,
\newline 2. garantipension och änkepension,
\newline 3. 93 procent av den pension och invaliditetsförmån som lämnas enligt utländsk lagstiftning,
\newline 4. inkomst av kapital enligt 7 § 3,
\newline 5. del av förmögenhet enligt 7 § 4,
\newline 6. 93 procent av de delar av den bidragsgrundande inkomsten som består av arbetsinkomst enligt 67 kap. 6 § första stycket inkomstskattelagen (1999:1229),
\newline 7. 93 procent av de delar av den inkomst enligt 15 § som, bortsett från att skatteplikt inte föreligger, är av motsvarande slag som arbetsinkomst enligt 67 kap. 6 § första stycket inkomstskattelagen, och
\newline 8. 93 procent av övriga delar av den bidragsgrundande inkomsten, minskad med ett fribelopp enligt 17 §.
\paragraph*{}
Inkomst enligt första stycket 6 och 7 ska endast beaktas till den del den överstiger 24 000 kronor.
Lag (2022:879).
\subsection*{17 §}
\paragraph*{}
Det fribelopp som avses i 16 och 16 a §§ motsvarar 2,43 prisbasbelopp för den som är ogift och 2,2 prisbasbelopp för den som är gift.
Lag (2022:1031).
\subsection*{18 §}
\paragraph*{}
/Upphör att gälla U:2025-12-01/
Om den försäkrade får sjukersättning eller aktivitetsersättning eller motsvarande utländsk förmån ska i stället för vad som anges i 17 § ett avdrag göras med ett belopp som motsvarar garantinivån för hel sådan ersättning enligt 35 kap. 18 respektive 19 §. Detta gäller dock längst till och med månaden före den då den försäkrade fyller 66 år.
Lag (2022:878).
\subsection*{18 §}
\paragraph*{}
/Träder i kraft I:2025-12-01/
Om den försäkrade får sjukersättning eller aktivitetsersättning eller motsvarande utländsk förmån ska i stället för vad som anges i 17 § ett avdrag göras med ett belopp som motsvarar garantinivån för hel sådan ersättning enligt 35 kap. 18 respektive 19 §. Detta gäller dock längst till och med månaden före den då den försäkrade uppnår riktåldern för pension.
Lag (2022:879).
\subsection*{19 §}
\paragraph*{}
Om den försäkrade får mer än en förmån som enligt 101 kap. 3-5 §§ berättigar till bostadstillägg, ska avdrag göras enligt 17 eller 18 § efter vad som är mest förmånligt för honom eller henne.
\subsection*{20 §}
\paragraph*{}
När den bidragsgrundande inkomsten och reduceringsinkomsten beräknas ska den handläggande myndigheten bortse från sådan förändring i utbetalning av aktivitetsersättning, sjukersättning, pensionsförmåner, äldreförsörjningsstöd och livränta som föranleds av att den försäkrade
\newline - är häktad,
\newline - är intagen i kriminalvårdsanstalt,
\newline - är intagen i ett hem som avses i 12 § lagen (1990:52) med särskilda bestämmelser om vård av unga för verkställighet av sluten ungdomsvård, eller
\newline - på grund av skyddstillsyn med särskild behandlingsplan vistas i ett sådant familjehem, stödboende eller hem för vård eller boende som avses i socialtjänstlagen (2001:453).
Lag (2015:983).
\subsection*{21 §}
\paragraph*{}
Bostadstillägg ska motsvara differensen mellan den försäkrades
\newline - bostadskostnad enligt 22-24 §§ och
\newline - den del av reduceringsinkomsten som anges i 25 §.
\subsection*{22 §}
\paragraph*{}
/Upphör att gälla U:2025-12-01/
Till och med månaden före den månad då den försäkrade fyller 66 år beaktas vid beräkning av bostadstillägg 96 procent av bostadskostnaden per månad av den del som inte överstiger 5 000 kronor för den som är ogift och 2 500 kronor för den som är gift.
\paragraph*{}
Om bostadskostnaden överstiger de belopp som anges i första stycket beaktas 70 procent av den överskjutande bostadskostnaden per månad upp till 7 500 kronor för den som är ogift och 3 750 kronor för den som är gift.
\paragraph*{}
För var och en av makar ska bostadskostnaden beräknas till hälften av deras sammanlagda bostadskostnad.
Lag (2022:878).
\subsection*{22 §}
\paragraph*{}
/Träder i kraft I:2025-12-01/
Till och med månaden före den månad då den försäkrade uppnår riktåldern för pension beaktas vid beräkning av bostadstillägg 96 procent av bostadskostnaden per månad av den del som inte överstiger 5 000 kronor för den som är ogift och 2 500 kronor för den som är gift.
\paragraph*{}
Om bostadskostnaden överstiger de belopp som anges i första stycket beaktas 70 procent av den överskjutande bostadskostnaden per månad upp till 7 500 kronor för den som är ogift och 3 750 kronor för den som är gift.
\paragraph*{}
För var och en av makar ska bostadskostnaden beräknas till hälften av deras sammanlagda bostadskostnad.
Lag (2022:879).
\subsection*{22 a §}
\paragraph*{}
/Upphör att gälla U:2025-12-01/
Från och med den månad en försäkrad fyller 66 år beaktas vid beräkning av bostadstillägg hela bostadskostnaden per månad av den del som inte överstiger 3 000 kronor för den som är ogift och 1 500 kronor för den som är gift.
\paragraph*{}
Om bostadskostnaden överstiger de belopp som anges i första stycket beaktas 90 procent av den överskjutande bostadskostnaden per månad upp till 5 000 kronor för den som är ogift och 2 500 kronor för den som är gift. Därutöver beaktas 70 procent av bostadskostnaden per månad mellan 5 001 kronor och 7 000 kronor för den som är ogift och mellan 2 501 kronor och 3 500 kronor för den som är gift. Vidare beaktas 50 procent av bostadskostnaden per månad mellan 7 001 kronor och 7 500 kronor för den som är ogift och mellan 3 501 kronor och 3 750 kronor för den som är gift.
\paragraph*{}
För var och en av makar ska bostadskostnaden beräknas till hälften av deras sammanlagda bostadskostnad.
\paragraph*{}
Ett belopp om 840 kronor för den som är ogift och 420 kronor för den som är gift ska läggas till den bostadskostnad som har beaktats enligt första-tredje styckena och 23 §.
Lag (2022:1032).
\subsection*{22 a §}
\paragraph*{}
/Träder i kraft I:2025-12-01/
Från och med den månad en försäkrad uppnår riktåldern för pension beaktas vid beräkning av bostadstillägg hela bostadskostnaden per månad av den del som inte överstiger 3 000 kronor för den som är ogift och 1 500 kronor för den som är gift.
\paragraph*{}
Om bostadskostnaden överstiger de belopp som anges i första stycket beaktas 90 procent av den överskjutande bostadskostnaden per månad upp till 5 000 kronor för den som är ogift och 2 500 kronor för den som är gift. Därutöver beaktas 70 procent av bostadskostnaden per månad mellan 5 001 kronor och 7 000 kronor för den som är ogift och mellan 2 501 kronor och 3 500 kronor för den som är gift. Vidare beaktas 50 procent av bostadskostnaden per månad mellan 7 001 kronor och 7 500 kronor för den som är ogift och mellan 3 501 kronor och 3 750 kronor för den som är gift.
\paragraph*{}
För var och en av makar ska bostadskostnaden beräknas till hälften av deras sammanlagda bostadskostnad.
\paragraph*{}
Ett belopp om 840 kronor för den som är ogift och 420 kronor för den som är gift ska läggas till den bostadskostnad som har beaktats enligt första-tredje styckena och 23 §.
Lag (2022:1033).
\subsection*{23 §}
\paragraph*{}
Vid boende i tvåbäddsrum i särskild boendeform beaktas inte bostadskostnader till den del de överstiger 2 250 kronor per månad för var och en av de boende.
\subsection*{24 §}
\paragraph*{}
Vid beräkningen av bostadskostnad enligt 22 §, för makar avses den gemensamma kostnaden, ska även hemmavarande barns andel av bostadskostnaden räknas med under förutsättning att barnet inte har fyllt 20 år och inte är självförsörjande.
Detsamma ska gälla så länge barnet får förlängt barnbidrag eller får studiehjälp enligt 2 kap. studiestödslagen (1999:1395).
\subsection*{25 §}
\paragraph*{}
Den del av reduceringsinkomsten som avses i 21 § är
\newline - 62 procent av inkomsten till den del den inte överstiger ett prisbasbelopp, och
\newline - 50 procent av inkomsten till den del den överstiger ett prisbasbelopp.
\subsection*{25 a §}
\paragraph*{}
/Upphör att gälla U:2025-12-01/
Från och med den månad en försäkrad, som har hel allmän ålderspension eller motsvarande utländsk förmån, fyller 66 år ska sådan reduceringsinkomst som avses i 21 § vara 62 procent av inkomsten.
Lag (2022:878).
\subsection*{25 a §}
\paragraph*{}
/Träder i kraft I:2025-12-01/
Från och med den månad en försäkrad, som har hel allmän ålderspension eller motsvarande utländsk förmån, uppnår riktåldern för pension ska sådan reduceringsinkomst som avses i 21 § vara 62 procent av inkomsten.
Lag (2022:879).
\subsection*{26 §}
\paragraph*{}
Särskilt bostadstillägg ska lämnas med det belopp som den försäkrades inkomster efter avdrag för skälig bostadskostnad understiger en skälig levnadsnivå i övrigt, allt räknat per månad.
\subsection*{27 §}
\paragraph*{}
Som skälig bostadskostnad enligt 26 § anses högst
\newline - 7 500 kronor för den som är ogift, och
\newline - 3 750 kronor för den som är gift.
Lag (2021:1244).
\subsection*{27 a §}
\paragraph*{}
Har upphävts genom
lag (2021:1244).
\subsection*{28 §}
\paragraph*{}
Skälig levnadsnivå i övrigt enligt 26 § anses per månad motsvara en tolftedel av
\newline - 1,5357 prisbasbelopp för den som är ogift, och
\newline - 1,2353 prisbasbelopp för den som är gift.
Lag (2021:1244).
\subsection*{29 §}
\paragraph*{}
/Upphör att gälla U:2025-12-01/
Vid tillämpning av 26 § ska följande inkomster beaktas:
\newline 1. Den del av den bidragsgrundande inkomsten som avses i 7 § 1 och 2 efter avdrag för den skatte- och avgiftssats som följer av tillämplig skattetabell enligt 55 kap. 6 och 8 §§ skatteförfarandelagen (2011:1244). Om det för den försäkrade inte kan beslutas någon tillämplig skattetabell ska den skatte- och avgiftssats användas som följer av den skattetabell som skulle ha varit tillämplig om sådant beslut hade kunnat fattas.
\newline 2. Den del av den bidragsgrundande inkomsten som avses i 7 § 3 efter avdrag med 30 procent.
\newline 3. Den del av den bidragsgrundande inkomsten som avses i 7 § 4.
\newline 4. Den del av den bidragsgrundande inkomsten som avses i 7 § 5. När det gäller sådan del av den bidragsgrundande inkomsten som avses i 97 kap. 13 § 1 ska den inkomsten tas upp efter avdrag för de skatter och avgifter som följer av bestämmelserna i 1 och 2 om inkomsten inte hade undantagits från beskattning i Sverige på grund av 3 kap. 9-13 §§ inkomstskattelagen (1999:1229) eller skatteavtal.
\newline 5. Bostadstillägg enligt 21 §.
\paragraph*{}
Från och med den månad en försäkrad fyller 66 år ska arbetsinkomst som avses i 16 a § 6 och 7 beaktas i sin helhet till den del inkomsten, före avdrag för skatte- och avgiftssats, överstiger 24 000 kronor.
Lag (2022:878).
\subsection*{29 §}
\paragraph*{}
/Träder i kraft I:2025-12-01/
Vid tillämpning av 26 § ska följande inkomster beaktas:
\newline 1. Den del av den bidragsgrundande inkomsten som avses i 7 § 1 och 2 efter avdrag för den skatte- och avgiftssats som följer av tillämplig skattetabell enligt 55 kap. 6 och 8 §§ skatteförfarandelagen (2011:1244). Om det för den försäkrade inte kan beslutas någon tillämplig skattetabell ska den skatte- och avgiftssats användas som följer av den skattetabell som skulle ha varit tillämplig om sådant beslut hade kunnat fattas.
\newline 2. Den del av den bidragsgrundande inkomsten som avses i 7 § 3 efter avdrag med 30 procent.
\newline 3. Den del av den bidragsgrundande inkomsten som avses i 7 § 4.
\newline 4. Den del av den bidragsgrundande inkomsten som avses i 7 § 5. När det gäller sådan del av den bidragsgrundande inkomsten som avses i 97 kap. 13 § 1 ska den inkomsten tas upp efter avdrag för de skatter och avgifter som följer av bestämmelserna i 1 och 2 om inkomsten inte hade undantagits från beskattning i Sverige på grund av 3 kap. 9-13 §§ inkomstskattelagen (1999:1229) eller skatteavtal.
\newline 5. Bostadstillägg enligt 21 §.
\paragraph*{}
Från och med den månad en försäkrad uppnår riktåldern för pension ska arbetsinkomst som avses i 16 a § 6 och 7 beaktas i sin helhet till den del inkomsten, före avdrag för skatte- och avgiftssats, överstiger 24 000 kronor.
Lag (2022:879).
\subsection*{30 §}
\paragraph*{}
Summan av inkomsterna enligt 29 § 1-3 ska alltid anses utgöra lägst en tolftedel av det för den försäkrade gällande fribeloppet enligt 17-19 §§ efter avdrag för beräknad preliminär skatt enligt 29 § 1. I fråga om makar ska inkomsterna för var och en av dem beräknas utgöra hälften av deras sammanlagda inkomster.
\paragraph*{}
För den som är född 1937 eller tidigare ska inkomsterna i stället för de belopp som anges i 17 §, alltid, efter avdrag för beräknad preliminär skatt enligt första stycket, anses utgöra lägst en tolftedel av
\newline - 2,47 prisbasbelopp för den som är ogift, och
\newline - 2,235 prisbasbelopp för den som är gift.
Lag (2022:1856).
\chapter*{103 Särskilda handläggningsregler för bostadstillägg}
\subsection*{1 §}
\paragraph*{}
I detta kapitel finns bestämmelser om
\newline - omprövning vid ändrade förhållanden i 2-4 §§, och
\newline - utbetalning av bostadstillägg i 5 §.
Lag (2013:747).
\subsection*{2 §}
\paragraph*{}
Bostadstillägget ska omprövas när något förhållande som påverkar tillläggets storlek har ändrats.
\subsection*{3 §}
\paragraph*{}
Bostadstillägget får räknas om utan föregående underrättelse om den del av årsinkomsten ändras som utgörs av
\newline - en förmån som betalas ut av Försäkringskassan eller Pensionsmyndigheten,
\newline - pension enligt utländsk lagstiftning,
\newline - avtalspension eller motsvarande ersättning som följer av kollektivavtal, eller
\newline - överskott i inkomstslaget kapital som avses i 102 kap. 7 § 3.
\paragraph*{}
Det som anges i första stycket gäller också när ändring sker av
\newline - sådant belopp som avses i 102 kap. 17-19 §§,
\newline - preliminärt bostadsbidrag enligt 98 kap., eller
\newline - taxeringsvärde för annan fastighet än sådan som avses i 5 § lagen (2009:1053) om förmögenhet vid beräkning av vissa förmåner.
\paragraph*{}
Vid omräkning av bostadstillägget med stöd av första och andra styckena får en ny beräkning av bostadstillägget göras utifrån enbart den ändring som ligger till grund för omräkningen.
Lag (2017:554).
\subsection*{4 §}
\paragraph*{}
En ändring av bostadstillägget ska gälla från och med månaden efter den månad då anledningen till ändring har uppkommit. En ändring av bostadstillägget ska dock gälla från och med den månad under vilken de förhållanden har uppkommit som föranleder ändringen, om förhållandena avser hela den månaden. Gäller det höjning av tillägget ska även 101 kap. 10 § beaktas.
\paragraph*{}
Om Pensionsmyndigheten eller Försäkringskassan har hämtat in uppgifter som avses i 102 kap. 7 § 3 direkt från Skatteverket för omräkning av bostadstillägget utan föregående underrättelse med stöd av 3 §, ska en ändring av bostadstillägget, i stället för vad som följer av första stycket, gälla från och med månaden efter den månad då Pensionsmyndigheten eller Försäkringskassan har fått uppgifterna från Skatteverket.
Lag (2018:670).
\subsection*{5 §}
\paragraph*{}
Bostadstillägg ska betalas ut månadsvis. Årsbeloppet ska avrundas till närmaste hela krontal som är delbart med tolv.
\subsection*{6 §}
\paragraph*{}
Har upphävts genom
lag (2013:747).
\chapter*{103 a Innehåll}
\subsection*{1 §}
\paragraph*{}
I denna underavdelning finns allmänna bestämmelser om boendetillägg i 103 b kap.
\paragraph*{}
Vidare finns bestämmelser om
\newline - rätten till boendetillägg i 103 c kap.,
\newline - beräkning av boendetillägg i 103 d kap., och
\newline - särskilda handläggningsregler för boendetillägg i 103 e kap.
Lag (2011:1513).
\chapter*{103 b Allmänna bestämmelser om boendetillägg}
\subsection*{1 §}
\paragraph*{}
I detta kapitel finns en inledande bestämmelse i 2 §.
\paragraph*{}
Vidare finns bestämmelser om sambor och makar i 3 och 4 §§.
Lag (2011:1513).
\subsection*{2 §}
\paragraph*{}
Boendetillägg är ett kompletterande bidrag till boendet för en försäkrad som har fått tidsbegränsad sjukersättning eller aktivitetsersättning.
Lag (2011:1514).
\subsection*{3 §}
\paragraph*{}
Sambor likställs med makar när det gäller boendetillägg.
\paragraph*{}
Om det på grund av omständigheterna är sannolikt att två personer är sambor, ska dessa likställas med sambor. Detta gäller inte om den som ansöker om boendetillägg eller den som sådant tillägg betalas ut till visar att de inte är sambor.
Lag (2011:1513).
\subsection*{4 §}
\paragraph*{}
När det gäller boendetillägg ska en person som är gift men stadigvarande lever åtskild från sin make likställas med en ogift person, om inte särskilda skäl talar mot detta.
Lag (2011:1513).
\chapter*{103 c Rätten till boendetillägg}
\subsection*{1 §}
\paragraph*{}
I detta kapitel finns inledande bestämmelser i 2 och 3 §§.
\paragraph*{}
Vidare finns bestämmelser om
\newline - förmåner som kan ge rätt till boendetillägg i 4 §,
\newline - undantag från rätten till boendetillägg i 5 §,
\newline - förmånstiden i 6-8 §§,
\newline - behållande av rätten till boendetillägg i 9 §, och
\newline - upphörande av rätten till boendetillägg i 10 §.
Lag (2011:1513).
\subsection*{2 §}
\paragraph*{}
En försäkrad som helt eller delvis har fått tidsbegränsad sjukersättning under det högsta antalet månader som sådan ersättning kan betalas ut enligt 4 kap. 31 § lagen (2010:111) om införande av socialförsäkringsbalken har i de fall och under de närmare förutsättningar som anges i detta kapitel rätt till ett boendetillägg. Detta gäller även för en försäkrad vars rätt till aktivitetsersättning upphör på grund av att han eller hon fyller 30 år.
Lag (2011:1514).
\subsection*{3 §}
\paragraph*{}
Rätten till boendetillägg inträder från och med månaden efter den då rätten till sådan tidsbegränsad sjukersättning eller aktivitetsersättning som avses i 2 § har upphört.
Lag (2011:1514).
\subsection*{4 §}
\paragraph*{}
Boendetillägg kan lämnas till den som omfattas av 2 § och som
\newline 1. får sjukpenning enligt 27-28 a kap., eller
\newline 2. får rehabiliteringspenning enligt 31 eller 31 a kap.
Lag (2015:963).
\subsection*{5 §}
\paragraph*{}
Boendetillägg lämnas inte till den som har rätt till bostadstillägg enligt 101 kap.
Lag (2011:1513).
\subsection*{6 §}
\paragraph*{}
Boendetillägg lämnas från och med den månad den försäkrade får sådan ersättning som anges i 4 § till och med den månad sådan ersättning har lämnats.
Lag (2011:1513).
\subsection*{7 §}
\paragraph*{}
Boendetillägg lämnas från och med den månad som anges i ansökan, dock inte för längre tid tillbaka än tre månader före ansökningsmånaden.
Lag (2011:1513).
\subsection*{8 §}
\paragraph*{}
/Upphör att gälla U:2025-12-01/
Boendetillägg lämnas längst till och med månaden före den när den försäkrade fyller 66 år.
Lag (2022:878).
\subsection*{8 §}
\paragraph*{}
/Träder i kraft I:2025-12-01/
Boendetillägg lämnas längst till och med månaden före den när den försäkrade uppnår riktåldern för pension.
Lag (2022:879).
\subsection*{9 §}
\paragraph*{}
En försäkrad behåller sin rätt till boendetillägg under tid då
\newline 1. han eller hon förvärvsarbetar,
\newline 2. en sådan situation föreligger som anges i 26 kap. 11, 12 eller 14-18 §§ som grund för SGI-skydd, eller
\newline 3. han eller hon deltar i ett arbetsmarknadspolitiskt program och får aktivitetsstöd eller står till arbetsmarknadens förfogande.
\paragraph*{}
Regeringen eller den myndighet som regeringen bestämmer kan med stöd av 8 kap. 7 § regeringsformen meddela
\newline 1. föreskrifter om undantag från kravet på att den som deltar i ett arbetsmarknadspolitiskt program ska få aktivitetsstöd, och
\newline 2. föreskrifter om de villkor som gäller för att den försäkrade ska anses stå till arbetsmarknadens förfogande.
Lag (2015:119).
\subsection*{10 §}
\paragraph*{}
Rätten till boendetillägg upphör när den försäkrade inte längre får sådan ersättning som anges i 4 § och inte heller uppfyller förutsättningarna i 9 §.
Lag (2011:1513).
\chapter*{103 d Beräkning av boendetillägg}
\subsection*{1 §}
\paragraph*{}
I detta kapitel finns bestämmelser om
\newline - ersättningsnivåer i 2-4 §§,
\newline - bidragsgrundande inkomst i 5 och 6 §§,
\newline - minskning av boendetillägg i 7 §,
\newline - andelsberäkning vid partiell ersättning i 8 §, och
\newline - samordning med bostadsbidrag i 9 §.
Lag (2015:963).
\subsection*{2 §}
\paragraph*{}
Helt boendetillägg lämnas för år räknat med högst
\newline - 84 000 kronor till en ogift försäkrad, och
\newline - 42 000 kronor till en gift försäkrad.
Lag (2011:1513).
\subsection*{3 §}
\paragraph*{}
Beloppen enligt 2 § ska höjas med
\newline - 12 000 kronor till hushåll med ett barn,
\newline - 18 000 kronor till hushåll med två barn, och
\newline - 24 000 kronor till hushåll med tre eller flera barn.
\paragraph*{}
Med barn avses sådana barn som enligt 96 kap. berättigar till bostadsbidrag.
Lag (2011:1513).
\subsection*{4 §}
\paragraph*{}
Om boendetillägg lämnas till båda makarna ska beloppen enligt 3 § delas lika mellan dem.
Lag (2011:1513).
\subsection*{5 §}
\paragraph*{}
När det gäller boendetillägg avses med bidragsgrundande inkomst den inkomst enligt 6 § för år räknat, som den försäkrade kan antas komma att få under den närmaste tiden.
Beräkningen av den bidragsgrundande inkomsten ska göras utifrån förhållandena den första dagen som sådan ersättning lämnas som enligt 103 c kap. 4 och 7 §§ kan ge rätt till boendetillägg. Om denna ersättning inte lämnas som hel förmån ska inkomsten beräknas som om hel förmån lämnas.
Lag (2011:1513).
\subsection*{6 §}
\paragraph*{}
Den bidragsgrundande inkomsten är den ersättning som lämnas i form av
\newline 1. sjukpenning enligt 27 och 28 kap.,
\newline 2. sjukpenning i särskilda fall enligt 28 a kap.,
\newline 3. rehabiliteringspenning enligt 31 kap.,
\newline 4. rehabiliteringspenning i särskilda fall enligt 31 a kap., eller
\newline 5. aktivitetsstöd.
\paragraph*{}
Till den bidragsgrundande inkomsten enligt första stycket ska läggas sådan livränta som lämnas till den försäkrade enligt 41 eller 43 kap.
Lag (2011:1513).
\subsection*{7 §}
\paragraph*{}
Boendetillägget enligt 2-4 §§ ska minskas med 70 procent av den bidragsgrundande inkomsten enligt 5 och 6 §§ som överstiger 58 400 kronor.
Lag (2011:1513).
\subsection*{8 §}
\paragraph*{}
Boendetillägget enligt 7 § lämnas med samma andel som den ersättning som grundar rätt till boendetillägg lämnas med.
\paragraph*{}
Om såväl sjukpenning enligt 6 § 1 som sjukpenning i särskilda fall enligt 6 § 2 lämnas samtidigt ska en andelsberäkning göras i förhållande till den sjukpenning som lämnas enligt 6 § 2. Motsvarande gäller i de fall rehabiliteringspenning lämnas enligt 6 § 3 och 4.
Lag (2011:1513).
\subsection*{9 §}
\paragraph*{}
Boendetillägget enligt 8 § ska minskas med det belopp som lämnas som preliminärt bostadsbidrag enligt 98 kap.
\paragraph*{}
Om boendetillägg lämnas till båda makarna ska boendetillägget till vardera maken minskas med halva det preliminära bostadsbidraget enligt första stycket.
Lag (2015:963).
\chapter*{103 e Särskilda handläggningsregler för boendetillägg}
\subsection*{1 §}
\paragraph*{}
I detta kapitel finns bestämmelser om
\newline - omprövning vid ändrade förhållanden i 2-4 §§, och
\newline - utbetalning av boendetillägg i 5 §.
Lag (2011:1513).
\subsection*{2 §}
\paragraph*{}
Boendetillägget ska omprövas när något förhållande som påverkar tilläggets storlek har ändrats.
Lag (2011:1513).
\subsection*{3 §}
\paragraph*{}
Boendetillägget får räknas om utan föregående underrättelse om den bidragsgrundande inkomsten enligt 103 d kap. 6 § ändras.
\paragraph*{}
Det som anges i första stycket gäller också när ändring sker av
\newline - förmånsnivån i de förmåner som anges i 103 c kap. 4 §, eller
\newline - preliminärt bostadsbidrag enligt 98 kap.
Lag (2015:963).
\subsection*{4 §}
\paragraph*{}
En ändring av boendetillägget ska gälla från och med månaden efter den månad då anledningen till ändring har uppkommit. Om ändringar av samma slag har skett flera gånger under månaden ska endast den senaste ändringen beaktas. Gäller det höjning av tillägget ska även 103 c kap. 7 § beaktas.
Lag (2011:1513).
\subsection*{5 §}
\paragraph*{}
Boendetillägg betalas ut månadsvis i efterskott.
Årsbeloppet ska avrundas till närmaste hela krontal som är delbart med tolv.
Lag (2011:1513).
\part*{H VISSA GEMENSAMMA BESTÄMMELSER}
\chapter*{104 Innehåll}
\subsection*{1 §}
\paragraph*{}
I avdelning H finns
\newline - gemensamma bestämmelser om förmåner m.m. i 105-108 kap.,
\newline - gemensamma bestämmelser om handläggning m.m. i 109-115 kap., och
\newline - vissa organisatoriska bestämmelser i 116 och 117 kap.
\chapter*{105 Innehåll}
\subsection*{1 §}
\paragraph*{}
I denna underavdelning finns
\newline - bestämmelser om förmåner vid verkställighet i anstalt eller vård på institution m.m. i 106 kap.,
\newline - andra gemensamma bestämmelser om förmåner i 107 kap., och
\newline - bestämmelser om återkrav och ränta i 108 kap.
\chapter*{106 Förmåner vid verkställighet i anstalt eller vård på institution m.m.}
\subsection*{1 §}
\paragraph*{}
I detta kapitel finns inledande bestämmelser i 2 och 3 §§.
\paragraph*{}
Vidare finns bestämmelser om
\newline - familjeförmåner i 4-11 §§,
\newline - förmåner vid sjukdom eller arbetsskada i 12-22 §§,
\newline - särskilda förmåner vid funktionshinder i 23-25 a §§,
\newline - förmåner vid ålderdom i 26-29 §§,
\newline - förmåner till efterlevande i 30-34 §§, och
\newline - bostadsstöd i 35 och 36 §§.
\paragraph*{}
Slutligen finns gemensamma bestämmelser i 37-40 §§.
Lag (2020:440).
\subsection*{2 §}
\paragraph*{}
I detta kapitel finns bestämmelser om när ersättning enligt denna balk inte lämnas, begränsas till sitt belopp eller betalas ut till någon annan under tid när en person är häktad, är intagen i anstalt, vårdas på institution eller av någon annan anledning än sjukdom är omhändertagen på det allmännas bekostnad eller fullgör plikttjänstgöring.
\paragraph*{}
Vidare finns i detta kapitel bestämmelser om när en person under sådan tid som anges i första stycket ska betala för sitt uppehälle genom att den handläggande myndigheten gör avdrag från ersättningen.
\subsection*{3 §}
\paragraph*{}
Om inte något annat särskilt anges ska den som är häktad, är intagen i kriminalvårdsanstalt eller annars på det allmännas bekostnad är omhändertagen, och som olovligen avviker från placeringen, vid tillämpning av bestämmelserna i detta kapitel fortfarande anses som häktad, intagen respektive omhändertagen.
\paragraph*{}
Det som föreskrivs i första stycket ska också gälla den som vistas utanför anstalt med anledning av permission.
\subsection*{4 §}
\paragraph*{}
Bestämmelserna i 12-14 §§ tillämpas även i fråga om graviditetspenning och tillfällig föräldrapenning.
\subsection*{5 §}
\paragraph*{}
Om en kvinna vid tiden för förlossningen är intagen i kriminalvårdsanstalt eller ett hem som avses i 12 § lagen (1990:52) med särskilda bestämmelser om vård av unga, får Försäkringskassan på framställning av föreståndaren för inrättningen besluta att den föräldrapenning som kvinnan har rätt till ska betalas ut till föreståndaren att användas för kvinnans och barnets nytta.
\subsection*{6 §}
\paragraph*{}
Om socialnämnden begär det får Försäkringskassan besluta att familjehemsförälder till ett barn som har placerats av socialnämnden ska få barnbidraget.
\paragraph*{}
Har ett barn placerats i enskilt hem av någon annan än socialnämnden, får Försäkringskassan besluta att familjehemsföräldern ska få barnbidraget om den som annars skulle få bidraget begär det.
\subsection*{7 §}
\paragraph*{}
Om ett barn vid ingången av en viss månad vårdas i ett stödboende eller hem för vård eller boende inom socialtjänsten, har det kommunala organ som svarar för vårdkostnaden rätt att få barnbidraget för den månaden som bidrag till denna kostnad. Uppkommer överskott, ska detta redovisas till den som annars är berättigad att få bidraget.
Lag (2015:983).
\subsection*{8 §}
\paragraph*{}
Underhållsstöd lämnas inte för sådan kalendermånad då barnet under hela månaden
\newline 1. på statens bekostnad får vård på institution eller annars kost och logi,
\newline 2. bor i familjehem eller bostad med särskild service för barn och ungdomar enligt lagen (1993:387) om stöd och service till vissa funktionshindrade, eller
\newline 3. vårdas i familjehem, stödboende eller hem för vård eller boende inom socialtjänsten.
Lag (2015:983).
\subsection*{9 §}
\paragraph*{}
För den som vårdas på en institution som tillhör eller till vars drift det betalas ut bidrag från staten, en kommun eller en region, lämnas omvårdnadsbidrag endast om vården kan beräknas pågå högst sex månader. Detsamma gäller om han eller hon vårdas utanför institutionen genom dess försorg eller i annat fall vårdas utanför en sådan institution och staten, kommunen eller regionen är huvudman för vården.
Lag (2019:843).
\subsection*{10 §}
\paragraph*{}
Är den beräknade vårdtiden längre än sex månader kan omvårdnadsbidrag, trots det som föreskrivs i 9 §, lämnas för ett svårt sjukt barn under högst tolv månader om en förälder i betydande omfattning regelbundet behöver vara närvarande på institutionen som en del av behandlingen av barnet.
Lag (2018:1265).
\subsection*{11 §}
\paragraph*{}
Om ett barn för vilket omvårdnadsbidrag inte lämnas på grund av 9 § tillfälligt inte vårdas genom huvudmannens försorg, lämnas bidrag för sådan tid om denna uppgår till minst tio dagar per kvartal eller till minst tio dagar i följd.
Lag (2018:1265).
\subsection*{12 §}
\paragraph*{}
Sjukpenning lämnas inte för tid när den försäkrade
\newline 1. fullgör någon annan tjänstgöring enligt lagen (1994:1809) om totalförsvarsplikt än grundutbildning som är längre än 60 dagar,
\newline 2. är intagen i sådant hem som avses i 12 § lagen (1990:52) med särskilda bestämmelser om vård av unga med stöd av 3 § samma lag,
\newline 3. är häktad eller intagen i kriminalvårdsanstalt, eller
\newline 4. i annat fall än som anges i 2 eller 3 av någon annan orsak än sjukdom tagits om hand på det allmännas bekostnad.
\subsection*{13 §}
\paragraph*{}
För varje dag då en försäkrad som får sjukpenning vistas i ett sådant familjehem eller hem för vård eller boende inom socialtjänsten som ger vård och behandling åt missbrukare av alkohol eller narkotika, ska han eller hon betala för sitt uppehälle på begäran av den som svarar för vårdkostnaderna.
\subsection*{14 §}
\paragraph*{}
Utan hinder av bestämmelserna i 12 § lämnas sjukpenning till försäkrad som avses under 3 i den paragrafen vid sjukdom som inträffar när han eller hon får vistas utom anstalt och bereds tillfälle att förvärvsarbeta.
\subsection*{15 §}
\paragraph*{}
Bestämmelserna i 12-14 §§ tillämpas även i fråga om rehabiliteringspenning.
\subsection*{16 §}
\paragraph*{}
Sjukersättning och aktivitetsersättning lämnas inte för tid efter det att den försäkrade sextio dagar i följd varit frihetsberövad på grund av att han eller hon är
\newline 1. häktad eller intagen i anstalt, eller
\newline 2. intagen i ett hem som avses i 12 § lagen (1990:52) med särskilda bestämmelser om vård av unga för verkställighet av sluten ungdomsvård.
\paragraph*{}
Förmånerna lämnas dock åter från och med den trettionde dagen före frigivningen.
\subsection*{17 §}
\paragraph*{}
Utan hinder av 16 § lämnas sjukersättning och aktivitetsersättning för tid under vilken den försäkrade vistas utanför anstalt enligt 11 kap. 3 eller 5 § fängelselagen (2010:610).
Lag (2010:613).
\subsection*{18 §}
\paragraph*{}
Försäkringskassan får medge en nära anhörig, som för sitt uppehälle är beroende av den försäkrade, rätt att helt eller delvis få sjukersättning eller aktivitetsersättning som enligt 16 § annars inte ska lämnas. Utbetalning av del av förmån till nära anhörig ska i första hand ske från inkomstrelaterad ersättning.
\subsection*{19 §}
\paragraph*{}
Om en försäkrad som får sjukersättning eller aktivitetsersättning vistas i ett familjehem, stödboende eller hem för vård eller boende inom socialtjänsten på grund av skyddstillsyn med särskild behandlingsplan, ska han eller hon för varje dag betala för sitt uppehälle när staten bekostar vistelsen. Detsamma gäller för tid när den försäkrade vistas utanför anstalt enligt 11 kap. 3 § fängelselagen (2010:610).
Lag (2015:983).
\subsection*{20 §}
\paragraph*{}
Bestämmelserna i 12-14 §§ tillämpas även i fråga om sjukpenning enligt 40, 43 och 44 kap.
\paragraph*{}
Bestämmelserna i 16-19 §§ tillämpas även i fråga om livränta till den försäkrade enligt 41, 43 och 44 kap.
\subsection*{21 §}
\paragraph*{}
Bestämmelserna i 12-14 §§ tillämpas även i fråga om smittbärarpenning.
\subsection*{22 §}
\paragraph*{}
Bestämmelserna i 12 och 14 §§ tillämpas även i fråga om närståendepenning.
\subsection*{23 §}
\paragraph*{}
Bestämmelserna i 9 § tillämpas även i fråga om merkostnadsersättning.
Lag (2018:1265).
\subsection*{24 §}
\paragraph*{}
Assistansersättning lämnas inte för tid när den funktionshindrade
\newline 1. vårdas på en institution som tillhör staten, en kommun eller en region,
\newline 2. vårdas på en institution som drivs med bidrag från staten, en kommun eller en region,
\newline 3. bor i en gruppbostad, eller
\newline 4. vistas i eller deltar i barnomsorg, skola eller daglig verksamhet enligt 9 § 10 lagen (1993:387) om stöd och service till vissa funktionshindrade.
Lag (2019:843).
\subsection*{25 §}
\paragraph*{}
Om det finns särskilda skäl kan assistansersättning lämnas även under tid när den funktionshindrade vårdas på sjukhus under en kortare tid eller deltar i verksamhet enligt 24 § 4.
\subsection*{25 a §}
\paragraph*{}
Bestämmelserna i 24 § 4 och 25 § gäller inte när den funktionshindrade deltar i barnomsorg eller skola och behöver hjälp med
\newline 1. andning,
\newline 2. åtgärder som är direkt nödvändiga för att hjälp med andning ska kunna ges,
\newline 3. måltider i form av sondmatning, eller
\newline 4. åtgärder som är direkt nödvändiga för förberedelse och efterarbete i samband med sådana måltider.
Lag (2020:440).
\subsection*{26 §}
\paragraph*{}
Bestämmelserna i 27-29 §§ gäller allmän ålderspension, särskilt pensionstillägg, äldreförsörjningsstöd och inkomstpensionstillägg. När det gäller äldreförsörjningsstöd tillämpas även det som föreskrivs om bostadstillägg i 35 §.
Lag (2020:1239).
\subsection*{27 §}
\paragraph*{}
Den försäkrade ska betala för sitt uppehälle för varje dag som han eller hon
\newline 1. är häktad eller intagen i anstalt, eller
\newline 2. vistas i familjehem eller hem för vård eller boende inom socialtjänsten på grund av skyddstillsyn med särskild behandlingsplan.
\paragraph*{}
Vid vistelse i ett familjehem eller ett hem för vård eller boende gäller detta endast när staten bekostar vistelsen.
\subsection*{28 §}
\paragraph*{}
Den försäkrade ska inte betala för sitt uppehälle när han eller hon vistas utanför anstalt enligt 11 kap. 5 § fängelselagen (2010:610).
Lag (2010:613).
\subsection*{29 §}
\paragraph*{}
Den försäkrade ska inte betala för sitt uppehälle när han eller hon är berättigad till sjukersättning eller livränta men förmånen enligt 16 eller 20 § inte ska lämnas.
\subsection*{30 §}
\paragraph*{}
Om den som får garantipension till omställningspension sedan minst trettio dagar i följd är intagen i kriminalvårdsanstalt eller är häktad ska, från och med den trettioförsta dagen, garantipensionen lämnas med högst ett belopp som innebär att denna tillsammans med omställnings- eller änkepensionen per månad uppgår till 4,5 procent av prisbasbeloppet.
\paragraph*{}
Denna begränsning gäller inte för tid under vilken den pensionsberättigade vistas utanför anstalt enligt 11 kap. 3 eller 5 § fängelselagen (2010:610).
Lag (2010:613).
\subsection*{31 §}
\paragraph*{}
Om den som får omställningspension, garantipension till omställningspension eller änkepension vistas i ett familjehem, stödboende eller hem för vård eller boende inom socialtjänsten på grund av skyddstillsyn med särskild behandlingsplan ska han eller hon för varje dag betala för sitt uppehälle när staten bekostar vistelsen. Detsamma gäller för tid när en pensionsberättigad vistas utanför anstalt enligt 11 kap. 3 § fängelselagen (2010:610).
Lag (2015:983).
\subsection*{32 §}
\paragraph*{}
Bestämmelserna i 30 och 31 §§ gäller endast under förutsättning att den pensionsberättigade inte dessutom får allmän ålderspension eller äldreförsörjningsstöd.
\subsection*{33 §}
\paragraph*{}
Efterlevandestöd lämnas inte för sådan kalendermånad då barnet under hela månaden
\newline 1. på statens bekostnad får vård på institution eller annars får kost och logi,
\newline 2. bor i familjehem eller bostad med särskild service för barn och ungdomar enligt lagen (1993:387) om stöd och service till vissa funktionshindrade, eller
\newline 3. vårdas i familjehem, stödboende eller hem för vård eller boende inom socialtjänsten.
Lag (2022:938).
\subsection*{34 §}
\paragraph*{}
Bestämmelserna i 33 § hindrar inte att efterlevandestöd lämnas i fråga om en elev i specialskola för tid när eleven vistas utom skolan. För sådan tid ska 11 § tillämpas på motsvarande sätt.
\subsection*{35 §}
\paragraph*{}
Om en försäkrad som har rätt till bostadstillägg vistas eller bor i en särskild boendeform eller i ett liknande boende är han eller hon berättigad till bostadstillägg för sin ursprungliga permanentbostad under högst sex månader från det att den handläggande myndigheten har bedömt vistelsen eller boendet i den särskilda boendeformen som stadigvarande.
\subsection*{36 §}
\paragraph*{}
I fråga om en försäkrad som har rätt till bostadstillägg och som är intagen i anstalt och avtjänar fängelsestraff som överstiger två år lämnas bostadstillägg under längst två månader efter det att vistelsen i anstalten inleddes eller fängelsestraffet började verkställas på något annat sätt.
Bostadstillägg får dock lämnas för tiden från och med den tredje månaden före den månad då frigivning ska ske eller från och med den månad då den intagne påbörjar en vistelse utanför anstalt enligt 11 kap. 5 § fängelselagen (2010:610).
Lag (2010:613).
\subsection*{37 §}
\paragraph*{}
Ska ersättning betalas ut endast för del av en kalendermånad, beräknas ersättningen för varje dag till en trettiondel av månadsbeloppet och avrundas till närmaste högre krontal. Detsamma gäller om garantipension till omställningspension ska minskas enligt 30 § för del av kalendermånad.
Bestämmelserna i första stycket gäller inte för förmåner som beräknas per dag.
Avdrag för kostnader för uppehälle
\subsection*{38 §}
\paragraph*{}
När en person ska betala för sitt uppehälle enligt detta kapitel ska betalningen ske genom att den handläggande myndigheten, efter skatteavdrag enligt skatteförfarandelagen (2011:1244), gör avdrag från ersättningen när den ska betalas ut.
\paragraph*{}
När det gäller graviditetspenning, tillfällig föräldrapenning, sjukpenning, rehabiliteringspenning och smittbärarpenning ska det innehållna beloppet betalas ut till den som har begärt avdraget.
Lag (2011:1434)
. 39 § Avdrag för kostnad som avses i 38 § ska beräknas till 80 kronor för dag.
När det gäller graviditetspenning, tillfällig föräldrapenning, sjukpenning, rehabiliteringspenning och smittbärarpenning får avdraget utgöra högst en tredjedel av ersättningens belopp efter skatteavdrag. I övriga fall får avdraget utgöra högst en tredjedel av förmånernas månadsbelopp efter skatteavdrag delat med 30. Avdraget på förmånen ska avrundas till närmaste lägre krontal.
\subsection*{40 §}
\paragraph*{}
Om den som ska betala för sitt uppehälle får flera förmåner enligt denna balk för samma tid, ska endast ett avdrag göras från de sammanlagda ersättningarna.
\chapter*{107 Andra gemensamma bestämmelser om förmåner}
\subsection*{1 §}
\paragraph*{}
I detta kapitel finns bestämmelser om
\newline - sammanträffande av förmåner vid retroaktiv utbetalning i 2-5 §§,
\newline - sjukdom, skada eller dödsfall på grund av brott i 6-8 §§,
\newline - utmätning, pantsättning och överlåtelse av fordran i 9-11 §§,
\newline - preskription i 12-16 §§, och
\newline - skadestånd i 17 och 18 §§.
\paragraph*{}
Sammanträffande av förmåner vid retroaktiv utbetalning
\subsection*{2 §}
\paragraph*{}
Om Försäkringskassan eller Pensionsmyndigheten har betalat ut en ersättning enligt denna balk till en försäkrad och någon av myndigheterna senare beviljar den försäkrade en annan ersättning enligt balken retroaktivt för samma tid som den tidigare utbetalade ersättningen avser gäller följande. Avdrag på den retroaktiva ersättningen ska göras med det belopp som överstiger vad som skulle ha betalats ut för perioden om beslut om båda ersättningarna hade fattats samtidigt.
\paragraph*{}
Det som föreskrivs i första stycket gäller också när den först utbetalda ersättningen är
\newline 1. en sådan ersättning enligt någon annan författning som Försäkringskassan, Pensionsmyndigheten eller en arbetslöshetskassa fattar beslut om, eller
\newline 2. omställningsstudiestöd som Centrala studiestödsnämnden fattar beslut om enligt lagen (2022:856) om omställningsstudiestöd.
Lag (2022:858).
\subsection*{3 §}
\paragraph*{}
Vid utbetalning av äldreförsörjningsstöd för förfluten tid till en bidragsberättigad, vars make tidigare har fått en sådan förmån för samma tid, ska den senare förmånen minskas på sådant sätt att det sammanlagda beloppet under denna tid motsvarar de ersättningar som skulle ha betalats ut om beslut om båda ersättningarna hade fattats samtidigt.
\paragraph*{}
Det som föreskrivs i första stycket gäller även bostadstillägg.
\subsection*{4 §}
\paragraph*{}
Vid utbetalning av efterlevandepension för tid före ansökningsmånaden ska beloppet minskas om det för samma tid tidigare har betalats ut efterlevandepension till någon annan efterlevande. Minskningen ska ske med det belopp som överstiger vad som skulle ha betalats ut om beslut om båda ersättningarna hade fattats samtidigt.
Det som föreskrivs i första stycket gäller även efterlevandelivränta.
\subsection*{5 §}
\paragraph*{}
Om någon som har rätt till periodisk ersättning enligt denna balk har fått ekonomiskt bistånd enligt 4 kap. 1 § socialtjänstlagen (2001:453) får retroaktivt beviljad ersättning på begäran av socialnämnden betalas ut till nämnden. Betalning till socialnämnden får göras, till den del den ersättningsberättigade inte har återbetalat biståndet och det återstående beloppet överstiger 1 000 kronor, med det belopp som motsvarar vad socialnämnden sammanlagt har betalat ut till den ersättningsberättigade och dennes familj för eller under den tid som den retroaktiva ersättningen avser.
\paragraph*{}
Det som föreskrivs i första stycket gäller inte barnbidrag och assistansersättning.
Sjukdom, skada eller dödsfall på grund av brott
\subsection*{6 §}
\paragraph*{}
Ersättning på grund av sjukdom eller skada som betalas ut av Försäkringskassan får dras in eller sättas ned, om den som är berättigad till ersättningen har ådragit sig sjukdomen eller skadan vid uppsåtligt brott som han eller hon har dömts för genom dom som har vunnit laga kraft.
\subsection*{7 §}
\paragraph*{}
Den som genom brottslig gärning uppsåtligen har berövat någon livet eller medverkat till brottet enligt 23 kap. 4 eller 5 § brottsbalken har inte rätt till efterlevandepension, efterlevandestöd eller efterlevandelivränta efter den avlidne.
\paragraph*{}
Förmånerna får dras in eller sättas ned, om den efterlevande på något annat sätt än som anges i första stycket har vållat dödsfallet genom en handling för vilken ansvar ådömts honom eller henne genom dom som vunnit laga kraft.
\subsection*{8 §}
\paragraph*{}
Om den som skulle ha haft rätt till premiepension till efterlevande har begått ett brott som avses i 14 kap. 15 § försäkringsavtalslagen (2005:104) mot pensionsspararen eller har medverkat till ett sådant brott på det sätt som anges där, ska den paragrafen gälla för rätten till premiepension till efterlevande.
\subsection*{9 §}
\paragraph*{}
En försäkrads fordran på ersättning som innestår hos Försäkringskassan eller Pensionsmyndigheten får inte utmätas.
En fordran på sådan ersättning får inte heller pantsättas eller överlåtas innan den är tillgänglig för lyftning.
\subsection*{10 §}
\paragraph*{}
Bestämmelserna i 9 § hindrar inte utmätning enligt bestämmelserna i 7 kap. utsökningsbalken.
\subsection*{11 §}
\paragraph*{}
Bestämmelserna i 9 och 10 §§ gäller även för tillgodohavande på premiepensionskonto.
\subsection*{12 §}
\paragraph*{}
Om en ersättning enligt denna balk inte har lyfts före utgången av andra året efter det då beloppet blev tillgängligt för lyftning, är fordran på beloppet preskriberad.
\subsection*{13 §}
\paragraph*{}
Om barnbidrag inte har lyfts före utgången av året efter det då bidraget blev tillgängligt för lyftning är fordran preskriberad.
Underhållsstöd
\subsection*{14 §}
\paragraph*{}
Om underhållsstöd inte har lyfts inom sju månader efter utgången av den månad då beloppet blev tillgängligt för lyftning, är fordran preskriberad.
\subsection*{15 §}
\paragraph*{}
Om bilstöd inte har använts inom sex månader från det att den som beviljats bilstöd fått besked om att stödet kan betalas ut är fordran preskriberad.
\subsection*{16 §}
\paragraph*{}
Preskriptionslagen (1981:130) gäller för en enskilds fordringar enligt balken endast när detta är särskilt angivet.
\subsection*{17 §}
\paragraph*{}
Om någon har rätt till ersättning enligt denna balk, är detta inte något hinder mot att göra gällande anspråk på skadestånd utöver ersättningen.
\subsection*{18 §}
\paragraph*{}
Ersättning enligt denna balk får inte krävas åter från den som är skyldig att betala skadestånd till den ersättningsberättigade.
\chapter*{108 Återkrav och ränta}
\subsection*{1 §}
\paragraph*{}
I detta kapitel finns bestämmelser om
\newline - återbetalning av ersättning i 2-10 §§,
\newline - eftergift i 11-14 §§,
\newline - verkställighet m.m. av beslut om återkrav i 14 a §,
\newline - ränta i 15-18 §§,
\newline - dröjsmålsränta i 19 och 20 §§,
\newline - eftergift av ränta och dröjsmålsränta i 21 §, och
\newline - avdrag på ersättning (kvittning) i 22, 24 och 25 §§.
Lag (2013:747).
\subsection*{2 §}
\paragraph*{}
Försäkringskassan eller Pensionsmyndigheten ska besluta om återbetalning av ersättning som den har beslutat enligt denna balk, om den försäkrade eller, i förekommande fall, den som annars har fått ersättningen har orsakat att denna har lämnats felaktigt eller med ett för högt belopp genom att
\newline 1. lämna oriktiga uppgifter, eller
\newline 2. underlåta att fullgöra en uppgifts- eller anmälningsskyldighet.
\paragraph*{}
Detsamma gäller om ersättning i annat fall har lämnats felaktigt eller med ett för högt belopp och den som fått ersättningen har insett eller skäligen borde ha insett detta.
\subsection*{3 §}
\paragraph*{}
I fråga om efterlevandepension, efterlevandestöd, livränta till efterlevande och premiepension till efterlevande kan beslut om återbetalning meddelas även när den avlidne på sätt som anges i 2 § första stycket har orsakat att ersättning betalats ut felaktigt eller med ett för högt belopp.
\subsection*{4 §}
\paragraph*{}
Har inkomst-, tilläggs- eller efterlevandepension betalats ut och minskas därefter pensionsrätt eller pensionspoäng på grund av bristande eller underlåten avgiftsbetalning enligt 61 kap. 9 respektive 21 § ska Pensionsmyndigheten besluta om återbetalning av för mycket utbetald pension.
\subsection*{5 §}
\paragraph*{}
Om en tjänsteman enligt 2 § lagen (2002:125) om överföring av värdet av pensionsrättigheter till och från Europeiska gemenskaperna har överfört värdet av pensionsrätt för inkomstpension och värdet av rätt till tilläggspension till gemenskaperna och den fastställda pensionsrätten för inkomstpension eller fastställda pensionspoäng därefter har sänkts, ska Pensionsmyndigheten i de fall som anges i 6 § återkräva av honom eller henne det som har överförts för mycket.
\subsection*{6 §}
\paragraph*{}
Återkrav enligt 5 § ska ske om sänkningen beror på att pensionsgrundande inkomst av annat förvärvsarbete har sänkts eller på att pensionsrätt eller pensionspoäng inte borde ha tillgodoräknats till följd av bestämmelserna i 61 kap. 9 respektive 21 § om bristande eller underlåten avgiftsbetalning. I andra fall ska beloppet återkrävas endast om den som överfört värdet av pensionsrättigheter genom att lämna oriktiga uppgifter eller genom att underlåta att fullgöra en uppgifts- eller anmälningsskyldighet har orsakat att pensionsrätt eller pensionspoäng har fastställts felaktigt eller med ett för högt belopp.
\subsection*{7 §}
\paragraph*{}
Pensionsmyndigheten ska besluta om återbetalning enligt 2 § av premiepension bara om felet belastar övriga pensionssparare eller kan innebära en avsevärd påverkan på framtida utbetalningar av pension till pensionsspararen. Detta gäller också i fråga om premiepension till efterlevande.
\subsection*{8 §}
\paragraph*{}
Har underhållsstöd lämnats felaktigt eller med för högt belopp, ska Försäkringskassan besluta om återkrav av beloppet från den som det har betalats ut till, även om förutsättningarna enligt 2 § inte är uppfyllda.
\subsection*{9 §}
\paragraph*{}
Försäkringskassan ska besluta om återbetalning av sjukersättning och bostadsbidrag som ska betalas tillbaka enligt 37 kap. 14 § respektive 98 kap. 6 §.
\paragraph*{}
Försäkringskassan ska också besluta om återbetalning av bostadsbidrag som har lämnats felaktigt eller med för högt belopp, även om förutsättningarna enligt 2 § inte är uppfyllda.
Lag (2014:470).
\subsection*{9 a §}
\paragraph*{}
När det gäller beslut enligt 2 § om återbetalning av ersättning får Försäkringskassan besluta om återkrav av assistansersättning från den försäkrades förmyndare, annan ställföreträdare eller den till vilken ersättningen har betalats ut enligt 51 kap. 19 § i stället för från den försäkrade.
\paragraph*{}
Återkrav mot någon annan än den försäkrade får uppgå till högst det belopp som denne tagit emot.
\paragraph*{}
Om det inte finns särskilda skäl, ska hela beloppet återkrävas från den försäkrades förmyndare i stället för från den försäkrade, om denne är under 18 år. Om det finns flera förmyndare, svarar de solidariskt för återkravet.
Lag (2012:935).
\subsection*{10 §}
\paragraph*{}
Om ett sådant interimistiskt beslut som avses i 112 kap. 2 eller 2 a §§ har fattats och det senare bestäms att ersättning inte ska lämnas eller ska lämnas med lägre belopp föreligger inte skyldighet att betala tillbaka utbetalad ersättning i andra fall än som anges i 2 §.
\paragraph*{}
När det gäller assistansersättning föreligger dock, utöver det som anges i 2 §, återbetalningsskyldighet för ersättning som inte har använts för köp av personlig assistans eller för kostnader för personliga assistenter.
Lag (2017:1305).
\subsection*{11 §}
\paragraph*{}
Om det finns särskilda skäl får den handläggande myndigheten helt eller delvis efterge krav på återbetalning enligt 2-10 §§.
\subsection*{12 §}
\paragraph*{}
Vid bedömningen av om det finns särskilda skäl för eftergift av sådan ersättning som ska betalas tillbaka enligt 9 § ska det särskilt beaktas vilken förmåga den eller de försäkrade har att kunna betala tillbaka ersättningen.
Lag (2014:470).
\subsection*{13 §}
\paragraph*{}
Om den eller de försäkrade medvetet eller av oaktsamhet lämnat felaktiga uppgifter till grund för beräkningen av den preliminära sjukersättningen eller för bedömningen av rätten till bostadsbidrag, får kravet på återbetalning inte efterges.
Lag (2014:470).
\subsection*{14 §}
\paragraph*{}
För att rätten till eftergift ska kunna prövas när det gäller sådan ersättning som ska betalas tillbaka enligt 9 § ska skyldigheten enligt 110 kap. 46 och 47 §§ att anmäla ändrade förhållanden ha fullgjorts. Om någon anmälan inte har gjorts kan frågan om eftergift ändå prövas om den anmälningsskyldige inte skäligen borde ha insett att en sådan anmälan skulle ha gjorts.
Lag (2014:470).
\paragraph*{}
Verkställighet m.m. av beslut om återkrav
\subsection*{14 a §}
\paragraph*{}
Efter det att ett beslut om återkrav har fattats av Pensionsmyndigheten i ett ärende ska den fortsatta handläggningen på grund av beslutet handhas av Försäkringskassan, även om återkravet avser en förmån som ska administreras av Pensionsmyndigheten.
Lag (2013:747).
\subsection*{15 §}
\paragraph*{}
Den som är återbetalningsskyldig enligt bestämmelserna i någon av 2-9 §§ ska betala ränta på det återkrävda beloppet om den handläggande myndigheten avseende det återkrävda beloppet har
\newline 1. träffat avtal med honom eller henne om en avbetalningsplan, eller
\newline 2. medgett honom eller henne anstånd med betalningen.
\paragraph*{}
Räntan ska beräknas från den dag då avtalet om avbetalningsplanen träffades eller anståndet medgavs. Ränta ska dock inte betalas för tid innan återkravet har förfallit till betalning.
Lag (2018:772).
\subsection*{16 §}
\paragraph*{}
Räntan enligt 15 § ska tas ut efter en räntesats som vid varje tidpunkt överstiger statens utlåningsränta med 2 procentenheter.
\subsection*{17 §}
\paragraph*{}
Den som är återbetalningsskyldig för premiepension enligt 2 § ska betala ränta på det felaktigt utbetalda beloppet med basränta enligt 65 kap. 3 § skatteförfarandelagen (2011:1244) från utbetalningsdagen till och med dagen för Pensionsmyndighetens beslut om återkrav.
\paragraph*{}
Första stycket tillämpas inte om återkravet grundar sig på att premiepension har betalats ut för tid efter det att den försäkrade har avlidit.
Lag (2018:772).
\subsection*{18 §}
\paragraph*{}
Ränta ska inte tas ut på sådan avgift som avses i 37 kap. 15 § andra stycket eller 98 kap. 7 § andra stycket.
Lag (2014:470).
\subsection*{19 §}
\paragraph*{}
Om ett belopp som har återkrävts enligt någon av bestämmelserna i 2-9 §§ inte betalas i rätt tid, ska dröjsmålsränta tas ut på beloppet.
\paragraph*{}
För uttag av dröjsmålsränta gäller i tillämpliga delar räntelagen (1975:635).
Lag (2018:772).
\subsection*{20 §}
\paragraph*{}
Dröjsmålsränta ska inte tas ut på sådan avgift som avses i 37 kap. 15 § andra stycket eller 98 kap. 7 § andra stycket.
Lag (2014:470).
\subsection*{21 §}
\paragraph*{}
Om det finns särskilda skäl får den handläggande myndigheten helt eller delvis efterge krav på ränta och dröjsmålsränta enligt 15-17 och 19 §§.
\subsection*{22 §}
\paragraph*{}
Försäkringskassan får, i den utsträckning det är skäligt, göra avdrag på ersättning enligt denna balk om den ersättningsberättigade enligt beslut av Försäkringskassan eller Pensionsmyndigheten är återbetalningsskyldig för en ersättning som lämnats på grund av denna balk eller någon annan författning. Avdrag får även göras för upplupen ränta och avgifter. Avdrag får dock endast göras på ersättning som administreras av samma myndighet som fattat beslutet om återkrav.
\paragraph*{}
Bestämmelserna i första stycket tillämpas inte i fråga om ersättning till vårdgivare enligt lagen (2008:145) om statligt tandvårdsstöd.
Lag (2013:747).
\subsection*{24 §}
\paragraph*{}
Om Pensionsmyndigheten har beslutat om återbetalning av inkomstgrundad efterlevandepension och det till följd av beslutet om sänkt sådan pension beviljas efterlevandestöd eller garantipension till omställningspension retroaktivt för samma tid, ska avdrag för den återkrävda pensionen i första hand göras på den retroaktivt beviljade förmånen, i den utsträckning betalning inte redan har skett.
\subsection*{25 §}
\paragraph*{}
Om Pensionsmyndigheten har beslutat om återbetalning av efterlevandestöd eller garantipension till omställningspension, ska avdrag för den återkrävda förmånen i första hand göras på retroaktivt beviljad inkomstgrundad efterlevandepension som avser samma tid som återkravet, i den utsträckning betalning inte redan har skett.
\chapter*{109 Innehåll m.m.}
\subsection*{1 §}
\paragraph*{}
I denna underavdelning finns bestämmelser om
\newline - handläggning av ärenden i 110 kap., - självbetjäningstjänster via Internet i 111 kap.,
\newline - beslut i 112 kap.,
\newline - ändring, omprövning och överklagande av beslut i 113 kap.,
\newline - behandling av personuppgifter i 114 kap., och
\newline - straffbestämmelser i 115 kap.
\chapter*{110 Handläggning av ärenden}
\subsection*{1 §}
\paragraph*{}
I detta kapitel finns bestämmelser om
\newline - tillämpningsområdet i 2 och 3 §§,
\newline - ansökan och anmälan m.m. i 4-12 §§,
\newline - utredning och uppgiftsskyldighet i 13-18 och 20-30 §§,
\newline - uppgiftsskyldighet för andra än parter i 31-35, 37 och 37 a §§,
\newline - bevisupptagning rörande arbetsskada m.m. vid allmän domstol i 38 §,
\newline - undantag från sekretess i 39-42 a §§,
\newline - anmälan om bosättning eller arbete i Sverige i 43-45 §§,
\newline - skyldighet att anmäla ändrade förhållanden i 46-51 §§, och
\newline - indragning och nedsättning av ersättning i 52-58 §§.
Lag (2022:938).
\subsection*{2 §}
\paragraph*{}
Detta kapitel gäller om inte annat anges vid handläggningen hos Försäkringskassan och Pensionsmyndigheten av ärenden som avser förmåner enligt denna balk.
\subsection*{3 §}
\paragraph*{}
När det är särskilt angivet ska bestämmelserna även tillämpas vid handläggningen hos Skatteverket.
\subsection*{4 §}
\paragraph*{}
Den som vill begära en förmån (sökanden) ska ansöka om den skriftligen. Detsamma gäller begäran om ökning av en förmån.
\paragraph*{}
En ansökan om en förmån ska innehålla de uppgifter som behövs i ärendet och ska vara egenhändigt undertecknad. Uppgifter om faktiska förhållanden ska lämnas på heder och samvete.
\subsection*{5 §}
\paragraph*{}
Ansökan om förlängt underhållsstöd enligt 18 kap. 6 § ska göras av den studerande själv, även om han eller hon ännu inte har fyllt 18 år. En sådan underårig studerande ska också själv lämna de uppgifter som behövs för att bedöma rätten till fortsatt barnpension, efterlevandestöd och barnlivränta enligt 78 kap. 5 §, 79 kap. 2 § respektive 88 kap. 4 §.
\paragraph*{}
Beträffande äldreförsörjningsstöd och bostadstillägg ska, om den försäkrade är gift, uppgifter om faktiska förhållanden lämnas på heder och samvete även av den försäkrades make.
\paragraph*{}
När det gäller assistansersättning ska, i de fall uppgifter lämnas av ett ombud, uppgifter om faktiska förhållanden lämnas på heder och samvete även av ombudet.
\paragraph*{}
När det gäller bostadsbidrag ska ett barn som avses i 96 kap. 4 § eller 5 a § och som är över 18 år på ansökningshandlingen självt intyga de uppgifter som rör barnet, om det inte finns särskilda skäl mot det.
Lag (2017:1123).
\subsection*{5 a §}
\paragraph*{}
Ansökan om föräldrapenning i fall som avses i 12 kap. 4 a § ska göras gemensamt av båda föräldrarna.
Lag (2011:1082).
\subsection*{6 §}
\paragraph*{}
/Upphör att gälla U:2024-07-01/
Ansökan behöver, trots det som föreskrivs i 4 §, inte göras om följande förmåner.
\newline 1. Barnbidrag, i annat fall än som avses i 15 kap. 6 §. Det finns dock särskilda bestämmelser om anmälan som gäller flerbarnstillägg i 16 kap. 12 §.
\newline 2. Sjukpenning i fall där den försäkrade på grund av sjukdomen är förhindrad eller har synnerliga svårigheter att göra en ansökan.
\newline 3. Assistansersättning när kommunen har anmält till Försäkringskassan att det kan antas att den enskilde har rätt till sådan ersättning.
\paragraph*{}
Om omvårdnadsbidrag eller merkostnadsersättning har beviljats för begränsad tid får den tid för vilken förmånen ska lämnas förlängas utan att ansökan om detta har gjorts.
\paragraph*{}
Ytterligare bestämmelser om att förmåner lämnas utan ansökan finns beträffande
\newline 1. sjukersättning och aktivitetsersättning i 36 kap. 25-27 §§,
\newline 2. allmän ålderspension i 56 kap. 4 a och 6 §§,
\newline 3. inkomstpensionstillägg i 74 a kap. 8 § andra stycket,
\newline 4. efterlevandepension och efterlevandestöd i 77 kap. 13 §,
\newline 5. efterlevandelivränta i 88 kap. 10 § första stycket,
\newline 6. premiepension till efterlevande i 92 kap. 2 § första stycket, och
\newline 7. bostadsbidrag i form av tilläggsbidrag till barnfamiljer i 93 kap. 4 § andra stycket.
Lag (2022:1041).
\subsection*{6 §}
\paragraph*{}
/Träder i kraft I:2024-07-01/
Ansökan behöver, trots det som föreskrivs i 4 §, inte göras om följande förmåner.
\newline 1. Barnbidrag, i annat fall än som avses i 15 kap. 6 §. Det finns dock särskilda bestämmelser om anmälan som gäller flerbarnstillägg i 16 kap. 12 §.
\newline 2. Sjukpenning i fall där den försäkrade på grund av sjukdomen är förhindrad eller har synnerliga svårigheter att göra en ansökan.
\newline 3. Assistansersättning när kommunen har anmält till Försäkringskassan att det kan antas att den enskilde har rätt till sådan ersättning.
\paragraph*{}
Om omvårdnadsbidrag eller merkostnadsersättning har beviljats för begränsad tid får den tid för vilken förmånen ska lämnas förlängas utan att ansökan om detta har gjorts.
\paragraph*{}
Ytterligare bestämmelser om att förmåner lämnas utan ansökan finns beträffande
\newline 1. sjukersättning och aktivitetsersättning i 36 kap. 25-27 §§,
\newline 2. allmän ålderspension i 56 kap. 4 a och 6 §§,
\newline 3. inkomstpensionstillägg i 74 a kap. 8 § andra stycket,
\newline 4. efterlevandepension och efterlevandestöd i 77 kap. 13 §,
\newline 5. efterlevandelivränta i 88 kap. 10 § första stycket, och
\newline 6. premiepension till efterlevande i 92 kap. 2 § första stycket.
Lag (2022:1042).
\subsection*{7 §}
\paragraph*{}
Oberoende av det som föreskrivs i 6 § kan uppgifter behöva lämnas enligt 13 §.
\subsection*{8 §}
\paragraph*{}
Bestämmelser om ansökan via Internet finns i 111 kap. 4-7 §§.
\subsection*{9 §}
\paragraph*{}
Bestämmelserna i 4 och 8 §§ gäller även i fråga om en sådan ansökan om något annat än en förmån som ska göras enligt en bestämmelse i balken.
\subsection*{10 §}
\paragraph*{}
Om ansökan är så ofullständig att den inte kan läggas till grund för någon prövning i sak ska den handläggande myndigheten avvisa den.
\paragraph*{}
Om ansökan inte i övrigt uppfyller föreskrifterna i 4 § andra stycket eller det som annars är särskilt föreskrivet, ska myndigheten också avvisa ansökan om inte bristen är av ringa betydelse.
\subsection*{11 §}
\paragraph*{}
Den handläggande myndigheten får inte avvisa ansökan enligt 10 § om sökanden inte först har förelagts att avhjälpa bristen vid påföljd att ansökan annars kommer att avvisas.
Ett sådant föreläggande får delges.
\subsection*{12 §}
\paragraph*{}
Om en ansökan, anmälan eller liknande som gäller bostadstillägg eller ersättning på grund av arbetsskada eller skada som avses i 43 eller 44 kap. och som ska göras hos Försäkringskassan i stället har kommit in till Pensionsmyndigheten, ska den anses inkommen till Försäkringskassan samma dag. Detsamma ska gälla om ansökan, anmälan eller liknande ska göras hos Pensionsmyndigheten och den i stället har kommit in till Försäkringskassan.
\subsection*{13 §}
\paragraph*{}
Den handläggande myndigheten ska se till att ärendena blir utredda i den omfattning som deras beskaffenhet kräver.
\paragraph*{}
Den enskilde är skyldig att lämna de uppgifter som är av betydelse för bedömningen av frågan om ersättning eller i övrigt för tillämpningen av denna balk. För sådant uppgiftslämnande gäller även 4 § andra stycket, om inte särskilda skäl talar mot det.
\subsection*{13 a §}
\paragraph*{}
Ett ärende om sjukpenning eller sjukpenning i särskilda fall får inte avgöras till den försäkrades nackdel utan att den försäkrade har underrättats om innehållet i det kommande beslutet och fått tillfälle att inom en bestämd tid yttra sig över det.
\paragraph*{}
Ett ärende får dock avgöras utan att den försäkrade underrättas om det kommande beslutet om det är uppenbart att han eller hon inte kan komma in med uppgifter som kan påverka avgörandet.
Lag (2017:1305).
\subsection*{14 §}
\paragraph*{}
När det behövs för bedömningen av frågan om ersättning eller i övrigt för tillämpningen av denna balk får den handläggande myndigheten
\newline 1. göra förfrågan hos den försäkrades arbetsgivare, läkare, anordnare av personlig assistans eller någon annan som kan antas kunna lämna behövliga uppgifter,
\newline 2. besöka den försäkrade,
\newline 3. begära ett utlåtande av viss läkare eller någon annan sakkunnig, samt
\newline 4. begära att den försäkrade genomgår undersökning enligt lagen (2018:744) om försäkringsmedicinska utredningar eller någon annan utredning eller deltar i ett avstämningsmöte för bedömning av den försäkrades medicinska tillstånd och arbetsförmåga, behov av hjälp i den dagliga livsföringen samt behovet av och möjligheterna till rehabilitering.
Lag (2018:745).
\subsection*{15 §}
\paragraph*{}
För utbetalning av en förmån utomlands får det krävas bevis om att rätten till förmånen består.
\subsection*{16 §}
\paragraph*{}
När det gäller graviditetspenning får Försäkringskassan begära att kvinnan ger in ett utlåtande från sin arbetsgivare.
\subsection*{17 §}
\paragraph*{}
När det gäller tillfällig föräldrapenning i fall som avses i 13 kap. 18 och 24 §§ får Försäkringskassan kräva att en förälder ger in särskilt intyg av arbetsgivare eller någon annan som kan ge upplysning om arbetsförhållandena.
\paragraph*{}
Försäkringskassan får också kräva att en förälder styrker sin rätt till föräldrapenningsförmån i samband med föräldrautbildning eller deltagande i en behandling av ett sjukt eller funktionshindrat barn genom intyg av den som anordnat utbildningen eller ordinerat behandlingen.
\subsection*{18 §}
\paragraph*{}
När det gäller tillfällig föräldrapenning enligt 13 kap. 22 § ska läkarutlåtande ges in för att styrka det särskilda vård- eller tillsynsbehovet. I fall som avses i 13 kap. 30 § ska barnets sjukdomstillstånd styrkas med ett läkarutlåtande om det inte hos Försäkringskassan redan finns tillräcklig utredning för att bedöma rätten till ersättning.
\subsection*{19 §}
\paragraph*{}
Har upphävts genom
lag (2012:931).
\subsection*{20 §}
\paragraph*{}
Vid tillämpning av 14 § 1 och 2 likställs betalningsskyldiga för underhållsstöd med försäkrade.
\subsection*{21 §}
\paragraph*{}
När det gäller sjukpenning ska den försäkrade, om Försäkringskassan begär det, ge in en skriftlig särskild försäkran avseende nedsättningen av arbetsförmågan på grund av sjukdom. Den särskilda försäkran ska innehålla en utförligare beskrivning av den försäkrades arbetsuppgifter och egna bedömning av arbetsförmågan än det som har uppgetts i ansökan. Uppgifterna i den särskilda försäkran ska lämnas på heder och samvete.
Lag (2017:1306).
\subsection*{22 §}
\paragraph*{}
Har upphävts genom
lag (2018:1265).
\subsection*{23 §}
\paragraph*{}
Försäkringskassan, Pensionsmyndigheten och allmän förvaltningsdomstol har i den utsträckning som det behövs för att bestämma skadeersättning, rätt att hos myndighet som utövar tillsyn över arbetsgivares verksamhet begära undersökning av arbetsförhållanden eller att, om det finns särskilda skäl eller det gäller en annan försäkrad än en arbetstagare, själva göra en sådan undersökning.
\subsection*{24 §}
\paragraph*{}
Till en ansökan om smittbärarpenning som grundas på beslut som avses i 46 kap. 5 § första stycket 1 ska det fogas en kopia av beslutet eller ett intyg med uppgift om beslutets innehåll.
\paragraph*{}
Till en ansökan om smittbärarpenning som grundas på sådant förhållande som anges i 46 kap. 5 § första stycket 2 eller en ansökan om resekostnadsersättning enligt 46 kap. 20 § ska det fogas ett intyg om den företagna åtgärden.
\subsection*{25 §}
\paragraph*{}
Till en ansökan om närståendepenning ska det fogas ett utlåtande av en läkare som ansvarar för den sjukes vård och behandling. Utlåtandet ska innehålla uppgifter om den vårdades sjukdomstillstånd.
\paragraph*{}
Kravet på läkarutlåtande gäller inte om det hos Försäkringskassan finns utredning om det som avses i första stycket och utredningen är tillräcklig för att bedöma ersättningsfrågan.
\subsection*{26 §}
\paragraph*{}
Har upphävts genom
lag (2018:1265).
\subsection*{27 §}
\paragraph*{}
Efter föreläggande av Skatteverket eller allmän förvaltningsdomstol ska den enskilde, i den omfattning och inom den tid som har angetts i föreläggandet, meddela sådana upplysningar som är av betydelse för tillämpningen av bestämmelserna om allmän ålderspension.
\subsection*{28 §}
\paragraph*{}
Som villkor för utbetalning av efterlevandestöd, efterlevandepension eller efterlevandelivränta får det krävas en förklaring på heder och samvete av den efterlevande att denne inte vet om en försvunnen person är i livet och, i fråga om livränta, hur eller varför personen har försvunnit.
\paragraph*{}
Är den efterlevande omyndig får förklaring enligt första stycket krävas även från hans eller hennes förmyndare.
Förklaring kan också infordras från en god man eller förvaltare som har förordnats enligt föräldrabalken.
\subsection*{29 §}
\paragraph*{}
Efterlevande som inte är försäkrade likställs ändå med försäkrade vid tillämpning av 14 § 1 och 2 i fråga om deras rätt till efterlevandeförmåner.
\subsection*{30 §}
\paragraph*{}
Regeringen eller den myndighet som regeringen bestämmer kan med stöd av 8 kap. 7 § regeringsformen meddela föreskrifter om ersättning för kostnader som en försäkrad eller någon annan har med anledning av sådan utredning som avses i 14 §.
\paragraph*{}
Regeringen eller den myndighet som regeringen bestämmer kan med stöd av 8 kap. 7 § regeringsformen även meddela föreskrifter om ersättning för kostnader som en försäkrad har för läkarundersökning eller läkarutlåtande i samband med ansökan om omvårdnadsbidrag, sjukersättning, aktivitetsersättning eller merkostnadsersättning.
Lag (2018:1265).
\subsection*{31 §}
\paragraph*{}
Myndigheter, arbetsgivare och uppdragsgivare, anordnare av personlig assistans samt försäkringsinrättningar ska på begäran lämna Försäkringskassan, Pensionsmyndigheten, Skatteverket och allmän förvaltningsdomstol uppgifter som avser en namngiven person när det gäller förhållanden som är av betydelse för tillämpningen av denna balk.
\paragraph*{}
Arbetsgivare och uppdragsgivare är även skyldiga att lämna sådana uppgifter om arbetet och arbetsförhållandena som behövs i ett ärende om arbetsskadeförsäkring.
Lag (2012:935).
\subsection*{32 §}
\paragraph*{}
Arbetslöshetskassor ska på begäran lämna uppgifter enligt 31 § första stycket till Skatteverket, Pensionsmyndigheten och allmän förvaltningsdomstol i ärenden om allmän ålderspension, särskilt pensionstillägg, efterlevandepension och efterlevandestöd.
\subsection*{33 §}
\paragraph*{}
Banker och andra penninginrättningar ska på begäran lämna uppgifter enligt 31 § första stycket till Försäkringskassan, Pensionsmyndigheten och allmän förvaltningsdomstol i ärenden om äldreförsörjningsstöd, bostadsbidrag och bostadstillägg.
\subsection*{34 §}
\paragraph*{}
Centrala studiestödsnämnden ska till Försäkringskassan lämna de uppgifter som har betydelse för tillämpningen av denna balk och lagen (2022:856) om omställningsstudiestöd och föreskrifter som har meddelats i anslutning till den lagen. Detta gäller dock inte sådana personuppgifter som avses i artikel 9.1 i EU:s dataskyddsförordning (känsliga personuppgifter).
Lag (2022:858).
\subsection*{34 a §}
\paragraph*{}
Skatteverket ska på Försäkringskassans begäran lämna sådana uppgifter om personliga assistenter och dem som bedriver yrkesmässig verksamhet med personlig assistans som behövs för Försäkringskassans kontroll av användningen av assistansersättning.
Lag (2012:935).
\subsection*{35 §}
\paragraph*{}
Totalförsvarets plikt- och prövningsverk ska lämna Pensionsmyndigheten de uppgifter som behövs för beräkning av pensionsgrundande belopp enligt 60 kap. 17 §.
\paragraph*{}
Den som är ansvarig för grundutbildning av en totalförsvarspliktig enligt lagen (1994:1809) om totalförsvarsplikt ska underrätta Totalförsvarets plikt- och prövningsverk om utbetalad dagersättning samt i förekommande fall om att utbildningen har avbrutits.
Lag (2020:1272).
\subsection*{36 §}
\paragraph*{}
Har upphävts genom
lag (2012:931).
\subsection*{37 §}
\paragraph*{}
Huvudmannen för en skola ska lämna de uppgifter som behövs i fråga om förlängt barnbidrag.
\subsection*{37 a §}
\paragraph*{}
Den kommunala nämnd som ansvarar för att ett barn får en insats som avses i 106 kap. 8 § 2 eller 3 eller 33 § 2 eller 3 ska så snart som möjligt lämna uppgifter till Försäkringskassan respektive Pensionsmyndigheten om den beslutade insatsen. Uppgiftsskyldigheten gäller endast insatser till barn som får underhållsstöd eller efterlevandestöd.
\paragraph*{}
Regeringen eller den myndighet som regeringen bestämmer får meddela ytterligare föreskrifter om uppgiftsskyldigheten till Försäkringskassan respektive Pensionsmyndigheten om den beslutade insats som avses i första stycket.
Lag (2022:938).
\paragraph*{}
Bevisupptagning rörande arbetsskada m.m. vid allmän domstol
\subsection*{38 §}
\paragraph*{}
Försäkringskassan och Pensionsmyndigheten får i ärenden om arbetsskadeförsäkring, statligt personskadeskydd och krigsskadeersättning begära att tingsrätt håller förhör med vittne eller sakkunnig. För förhöret ska i tillämpliga delar gälla det som föreskrivs om bevisupptagning utom huvudförhandling.
\paragraph*{}
Ingen är skyldig att inställa sig vid en annan tingsrätt än den inom vars domkrets han eller hon vistas. Ersättning till den som har inställt sig för att höras lämnas med skäligt belopp som rätten bestämmer och betalas av den myndighet som begärt förhöret.
\subsection*{39 §}
\paragraph*{}
Sekretess hindrar inte att Skatteverket, Försäkringskassan, Pensionsmyndigheten och allmän förvaltningsdomstol på begäran får lämna ut uppgifter om ersättningar enligt denna balk eller enligt lagstiftning om annan jämförbar ekonomisk förmån till
\newline 1. en försäkringsinrättning, ett försäkringsbolag eller en arbetsgivare, om uppgiften behövs för samordning med ersättning därifrån, och
\newline 2. ett utländskt socialförsäkringsorgan, om uppgiften behövs där vid tilllämpningen av en internationell överenskommelse som Sverige har anslutit sig till.
\subsection*{40 §}
\paragraph*{}
Sekretess hindrar inte att allmän förvaltningsdomstol på begäran får lämna ut uppgifter som avses i 39 § till en arbetslöshetskassa, om uppgiften behövs för samordning med ersättning därifrån.
\subsection*{41 §}
\paragraph*{}
Sekretess hindrar inte att Skatteverket och Pensionsmyndigheten på begäran får lämna ut uppgifter rörande inkomstgrundad ålderspension eller inkomstrelaterad efterlevandepension till en arbetslöshetskassa, om uppgiften behövs för samordning med ersättning därifrån.
\subsection*{42 §}
\paragraph*{}
Bestämmelserna i 39 och 40 §§ gäller även för utlämnande av uppgift från varje annan myndighet som har till uppgift att handlägga ärenden om ersättningar enligt denna balk eller enligt lagstiftning om annan jämförbar ekonomisk förmån.
\subsection*{42 a §}
\paragraph*{}
Försäkringskassan ska anmäla till Inspektionen för vård och omsorg om det finns anledning att anta att en enskild bedriver yrkesmässig verksamhet med personlig assistans utan tillstånd enligt 23 § lagen (1993:387) om stöd och service till vissa funktionshindrade eller att en tillståndshavares lämplighet för att bedriva sådan verksamhet kan ifrågasättas.
Lag (2021:878).
\subsection*{43 §}
\paragraph*{}
Den som bosätter sig i Sverige och som inte är folkbokförd här ska anmäla sig till Försäkringskassan.
\paragraph*{}
En utsänd statsanställd som avses i 5 kap. 4 § första stycket och som inte är folkbokförd här ska anmäla sig till Försäkringskassan.
\subsection*{44 §}
\paragraph*{}
En anmälan om arbete i Sverige som görs av den som arbetar här utan att vara bosatt i landet ska lämnas till Försäkringskassan.
\subsection*{45 §}
\paragraph*{}
Den som lämnar Sverige för en tid som kan antas ha betydelse för socialförsäkringsskyddet enligt 5 eller 6 kap.
ska anmäla detta till Försäkringskassan.
\subsection*{46 §}
\paragraph*{}
Den som ansöker om, har rätt till eller annars får en förmån enligt denna balk ska anmäla sådana ändrade förhållanden som påverkar rätten till eller storleken av förmånen.
\paragraph*{}
Det som anges i första stycket kan avse
\newline 1. bosättning i Sverige eller utlandsvistelse,
\newline 2. bostadsförhållanden,
\newline 3. civilstånd, vårdnad och sammanboende med vuxen eller barn,
\newline 4. hälsotillstånd,
\newline 5. förvärvsarbete i Sverige eller utomlands,
\newline 6. arbetsförmåga,
\newline 7. inkomstförhållanden,
\newline 8. förmögenhetsförhållanden, och
\newline 9. utländsk socialförsäkringsförmån.
\paragraph*{}
Anmälan som gäller assistansersättning ska även göras av den till vilken assistansersättning har betalats ut enligt 51 kap. 19 §, om denne har kännedom om de ändrade förhållandena.
\paragraph*{}
Anmälan behöver inte göras om den handläggande myndigheten har kännedom om ändringen och därför saknar behov av en anmälan. Anmälan behöver inte heller göras i ett ärende om bostadstillägg eller äldreförsörjningsstöd om ändringen innebär att inkomsterna eller förmögenheten endast har ökat i mindre omfattning.
Lag (2014:470).
\subsection*{47 §}
\paragraph*{}
Anmälan enligt 46 § ska göras så snart som möjligt och senast fjorton dagar efter det att den anmälningsskyldige fick kännedom om förändringen.
\paragraph*{}
Den handläggande myndigheten får, när det anses motiverat, kräva att uppgifterna lämnas på det sätt som föreskrivs i 4 §.
\subsection*{48 §}
\paragraph*{}
För den som är betalningsskyldig för underhållsstöd och som har beviljats anstånd gäller det som anges om anmälningsskyldighet i 46 och 47 §§ i fråga om sådana ändrade förhållanden som påverkar rätten till anstånd eller omfattningen av anståndet.
\subsection*{49 §}
\paragraph*{}
Den som är berättigad till graviditetspenning, tillfällig föräldrapenning, sjukpenning eller smittbärarpenning är skyldig att enligt bestämmelserna i 47 § meddela Försäkringskassan sin vistelseadress när han eller hon under sjukdomsfall eller annat ersättningsfall vistas annat än tillfälligt på annan adress än den som angetts till Försäkringskassan.
\subsection*{50 §}
\paragraph*{}
Den som får sjukersättning ska alltid göra anmälan enligt bestämmelserna i 47 § om arbetsförmågan förbättras.
\subsection*{51 §}
\paragraph*{}
Anmälan enligt 46 § av den som får sjukersättning eller aktivitetsersättning ska göras
\newline 1. om han eller hon avser att börja förvärvsarbeta, innan arbetet påbörjas,
\newline 2. om han eller hon avser att börja förvärvsarbeta i större omfattning än tidigare, innan arbetet utökas, och
\newline 3. om han eller hon avser att fortsätta att förvärvsarbeta efter tid som avses i 36 kap. 13 §, innan arbetet fortsätter.
\paragraph*{}
Bestämmelserna i första stycket gäller också den som får livränta vid arbetsskada eller annan skada som avses i 41-44 kap.
\subsection*{52 §}
\paragraph*{}
Ersättning enligt denna balk får dras in eller sättas ned om den försäkrade eller den som annars får ersättningen
\newline 1. medvetet eller av grov vårdslöshet har lämnat oriktig eller vilseledande uppgift,
\newline 2. inte har lämnat uppgift enligt 13 §, eller
\newline 3. inte har anmält ändrade förhållanden enligt 46, 47, 50 och 51 §§.
\paragraph*{}
Indragningen eller nedsättningen får avse viss tid eller gälla tills vidare. Det som anges i första stycket gäller endast om det är fråga om ett förhållande som är av betydelse för rätten till eller storleken av ersättningen.
\subsection*{53 §}
\paragraph*{}
Ersättning enligt denna balk får dras in eller sättas ned om den som är berättigad till ersättningen utan giltig anledning vägrar att medverka till utredningsåtgärder enligt 14-18, 27 och 28 §§. Då gäller 52 § andra stycket.
Lag (2018:1265).
\subsection*{54 §}
\paragraph*{}
Ersättning enligt denna balk får dras in eller sättas ned, om den som är berättigad till ersättningen vägrar att följa läkares föreskrifter. Då gäller 52 § andra stycket.
\subsection*{55 §}
\paragraph*{}
Sjukpenning får dras in eller sättas ned, om den som är berättigad till ersättningen underlåter att
\newline 1. styrka nedsättning av arbetsförmågan genom läkarintyg inom föreskriven tid, eller
\newline 2. ge in en sådan särskild försäkran som avses i 21 §.
Lag (2017:1306).
\subsection*{56 §}
\paragraph*{}
Graviditetspenning, tillfällig föräldrapenning, sjukpenning och smittbärarpenning får dras in eller sättas ned om den som är berättigad till ersättningen underlåter att meddela Försäkringskassan sin vistelseadress enligt 49 §.
\subsection*{57 §}
\paragraph*{}
Ersättning får dras in eller sättas ned om den försäkrade utan giltig anledning vägrar att medverka till
\newline 1. behandling eller rehabilitering enligt 27 kap. 6 §, eller
\newline 2. rehabiliteringsåtgärder enligt 30 kap. 7 § och 31 kap. 3 §.
\newline 1. omvårdnadsbidrag,
\newline 2. sjukpenning,
\newline 3. sjukersättning och aktivitetsersättning,
\newline 4. sjukvårdsersättning vid arbetsskada,
\newline 5. livränta till den försäkrade vid arbetsskada, och
\newline 6. merkostnadsersättning.
\paragraph*{}
Det som anges i första stycket gäller endast
Det som i andra stycket föreskrivs om arbetsskada tillämpas också på motsvarande ersättningar enligt 43 och 44 kap.
Lag (2018:1265).
\subsection*{57 a §}
\paragraph*{}
Assistansersättning får dras in eller sättas ned om den försäkrade vid upprepade tillfällen utan giltig anledning vägrar att medverka till
\newline 1. besök enligt 14 § 2 när assistansen utförs av någon som är närstående till eller lever i hushållsgemenskap med den försäkrade och som inte är anställd av kommunen, eller
\newline 2. inspektion av Inspektionen för vård och omsorg enligt 51 kap. 16 a §.
Lag (2012:962).
\subsection*{58 §}
\paragraph*{}
För att ersättning ska få dras in eller sättas ned på grund av att den försäkrade vägrar att delta vid avstämningsmöte enligt 14 § 4 eller behandling eller rehabilitering enligt 57 §, eller vägrar att medverka till besök eller inspektion enligt 57 a §, krävs att den försäkrade har informerats om denna påföljd.
Lag (2012:935).
\chapter*{111 Självbetjäningstjänster via Internet}
\subsection*{1 §}
\paragraph*{}
I detta kapitel finns bestämmelser om
\newline - tillämpningsområdet i 2 §,
\newline - vad som är självbetjäningstjänster i 3 §,
\newline - när självbetjäningstjänster kan användas i 4 §,
\newline - elektronisk underskrift i 5 och 6 §§, och
\newline - när en uppgift anses ha kommit in i 7 §.
Lag (2016:563).
\subsection*{2 §}
\paragraph*{}
Detta kapitel gäller i fråga om förmåner enligt denna balk samt andra förmåner och ersättningar som enligt lag eller förordning handläggs av Försäkringskassan eller Pensionsmyndigheten.
\paragraph*{}
Kapitlet är inte tillämpligt i fråga om överklagande av beslut i annat fall än då en begäran om omprövning ska anses som ett överklagande.
\subsection*{3 §}
\paragraph*{}
Med självbetjäningstjänster avses möjligheter att via Internet få tillgång till personuppgifter och annan information samt utföra sådana rättshandlingar som anges i 4 § första stycket.
\subsection*{4 §}
\paragraph*{}
En enskild får, i den utsträckning som framgår av föreskrifter som meddelas av regeringen eller den myndighet som regeringen bestämmer, använda självbetjäningstjänster för att
\newline - lämna uppgifter,
\newline - göra anmälningar eller ansökningar,
\newline - förfoga över rättigheter, och
\newline - utföra andra rättshandlingar.
\paragraph*{}
Sådana rättshandlingar som anges i första stycket har samma rättsverkningar som om de utförts i enlighet med de föreskrifter om formkrav som annars gäller för de förmåner och ersättningar som avses i 2 § första stycket. Föreskrifter om att uppgifter ska lämnas på heder och samvete ska dock alltid iakttas när självbetjäningstjänster används.
\subsection*{5 §}
\paragraph*{}
En enskild som lämnar uppgifter i samband med att han eller hon använder en självbetjäningstjänst ska använda en sådan elektronisk underskrift som avses i artikel 3 i Europaparlamentets och rådets förordning (EU) nr 910/2014 av den 23 juli 2014 om elektronisk identifiering och betrodda tjänster för elektroniska transaktioner på den inre marknaden och om upphävande av direktiv 1999/93/EG, i den ursprungliga lydelsen.
\paragraph*{}
Om det är fråga om att få tillgång till personuppgifter ska certifikat, till vilket en säker identifieringsfunktion är knuten, användas för kontroll av användarens identitet.
Lag (2016:563).
\subsection*{6 §}
\paragraph*{}
Om det finns någon annan metod än som avses i 5 § för identifiering eller skydd mot förvanskning av uppgifter som är tillräckligt säker med hänsyn till risken för integritetsintrång eller annan skada får dock den användas.
\paragraph*{}
Elektronisk underskrift ska dock alltid användas när uppgifter ska lämnas på heder och samvete.
Lag (2016:563).
\subsection*{7 §}
\paragraph*{}
En handling eller uppgift som vid användning av en självbetjäningstjänst har översänts till Försäkringskassan eller Pensionsmyndigheten, ska anses ha kommit in till den myndighet till vilken ärendet hör när den har anlänt till den del av ett system för automatiserad behandling som anvisats som mottagningsställe för självbetjäningstjänsten.
\chapter*{112 Om beslut}
\subsection*{1 §}
\paragraph*{}
I detta kapitel finns bestämmelser om
\newline - interimistiska beslut i 2-4 §§,
\newline - när ett beslut blir gällande i 5 §, och
\newline - beslut genom automatiserad behandling i 6 och 7 §§.
\subsection*{2 §}
\paragraph*{}
För tid till dess att ett ärende har avgjorts får Försäkringskassan och Pensionsmyndigheten besluta i fråga om ersättning från samma myndighet, om
\newline 1. det inte utan betydande dröjsmål kan avgöras om rätt till ersättning föreligger,
\newline 2. det är sannolikt att sådan rätt föreligger, och
\newline 3. det är av väsentlig betydelse för den som begär ersättningen.
\paragraph*{}
Ett beslut enligt första stycket får meddelas även när det står klart att rätt till ersättning föreligger men ersättningens belopp inte kan bestämmas utan betydande dröjsmål.
\subsection*{2 a §}
\paragraph*{}
För tid till dess att ett ärende om sjukpenning eller sjukpenning i särskilda fall har avgjorts slutligt får Försäkringskassan trots 2 § besluta om sådan ersättning, om
\newline 1. den försäkrade begär sjukpenning eller sjukpenning i särskilda fall i anslutning till en sjukperiod som har pågått i minst 15 dagar,
\newline 2. sjukpenning eller sjukpenning i särskilda fall har lämnats tidigare i sjukperioden, och
\newline 3. det saknas skäl som talar emot det.
Lag (2017:1305).
\subsection*{3 §}
\paragraph*{}
Finns det sannolika skäl att dra in eller minska en beslutad ersättning, kan den handläggande myndigheten besluta att ersättningen ska hållas inne eller lämnas med lägre belopp till dess att ärendet avgjorts.
\subsection*{4 §}
\paragraph*{}
Interimistiska beslut om ersättning enligt 2 och 3 §§ får inte fattas i ärenden om adoptionsbidrag, bilstöd eller särskilt pensionstillägg.
\paragraph*{}
Interimistiska beslut enligt 2 § får inte fattas i ärenden om allmän ålderspension, inkomstpensionstillägg, efterlevandepension eller efterlevandestöd.
\paragraph*{}
Interimistiska beslut enligt 3 § får inte fattas i ärenden om assistansersättning.
Lag (2020:1239).
\subsection*{5 §}
\paragraph*{}
En myndighets eller en allmän förvaltningsdomstols beslut enligt denna balk ska gälla omedelbart, om inget annat föreskrivs i beslutet eller bestäms av en domstol som har att pröva beslutet.
\subsection*{6 §}
\paragraph*{}
Skatteverkets beslut om pensionsgrundande inkomst får fattas genom automatiserad behandling om en klargörande motivering för beslutet får utelämnas enligt 32 § första stycket förvaltningslagen (2017:900).
\paragraph*{}
Beslut som avses i första stycket får även sättas upp i form av elektroniska dokument. Med ett elektroniskt dokument avses en upptagning som har gjorts med hjälp av automatiserad behandling och vars innehåll och utställare kan verifieras genom ett visst tekniskt förfarande.
Lag (2018:780).
\subsection*{7 §}
\paragraph*{}
Beslut om allmän ålderspension får fattas genom automatiserad behandling av Pensionsmyndigheten om en klargörande motivering för beslutet får utelämnas enligt 32 § första stycket förvaltningslagen (2017:900). Detsamma gäller i tillämpliga delar för beslut om inkomstpensionstillägg, efterlevandepension och efterlevandestöd.
Lag (2020:1239).
\chapter*{113 Ändring, omprövning och överklagande av beslut}
\subsection*{1 §}
\paragraph*{}
I detta kapitel finns allmänna bestämmelser i 2 §.
\paragraph*{}
Vidare finns bestämmelser om
\newline - ändring av beslut i 3-6 §§,
\newline - omprövning av beslut i 7-9 §§,
\newline - överklagande av Försäkringskassans och Pensionsmyndighetens beslut i 10-15 §§,
\newline - överklagande av allmän förvaltningsdomstols beslut i 16 och 17 §§,
\newline - tidsfrister för begäran om omprövning och överklagande i 19 och 20 §§,
\newline - handlingar till fel myndighet i 20 a §, och
\newline - avvisningsbeslut i 21 §.
\paragraph*{}
Slutligen finns särskilda bestämmelser om
\newline - allmän ålderspension i 22-36 §§,
\newline - inkomstpensionstillägg i 36 a §,
\newline - förmåner till efterlevande i 37-40 §§, och
\newline - beslut som rör fondförvaltare i 41 §.
Lag (2020:1239).
\subsection*{2 §}
\paragraph*{}
Beslut i ärenden om förmåner enligt denna balk får ändras, omprövas och överklagas med tillämpning av bestämmelserna i 3-21 §§, om inget annat följer av bestämmelserna i 22-40 §§.
\paragraph*{}
Första stycket gäller även beslut i ärenden
\newline - enligt 19 kap. om bidragsskyldigas betalningsskyldighet mot Försäkringskassan,
\newline - om utfärdande av intyg för tillämpning av Europaparlamentets och rådets förordning (EG) nr 883/2004 av den 29 april 2004 om samordning av de sociala trygghetssystemen, och
\newline - enligt 22 och 23 §§ lagen (2022:856) om omställningsstudiestöd om bestämmande av sjukpenninggrundande inkomst och årlig inkomst för en person som ansökt om omställningsstudiestöd enligt den lagen.
Lag (2022:858).
\subsection*{3 §}
\paragraph*{}
Försäkringskassan och Pensionsmyndigheten ska ändra ett beslut som har fattats av respektive myndighet och som inte har prövats av domstol, om beslutet
\newline 1. på grund av skrivfel, räknefel eller annat sådant förbiseende innehåller uppenbar oriktighet,
\newline 2. har blivit oriktigt på grund av att det har fattats på uppenbart felaktigt eller ofullständigt underlag, eller
\newline 3. har blivit oriktigt på grund av uppenbart felaktig rättstillämpning eller annan liknande orsak.
\paragraph*{}
Beslutet ska ändras även om omprövning inte har begärts.
Ändring behöver dock inte göras om oriktigheten är av ringa betydelse.
\subsection*{4 §}
\paragraph*{}
Det allmänna ombudet får begära ändring enligt 3 §. Sådan begäran får det allmänna ombudet framställa även till förmån för enskild part.
\subsection*{5 §}
\paragraph*{}
Ett beslut får inte ändras till den försäkrades nackdel med stöd av 3 § när det gäller en förmån som har förfallit till betalning, och inte heller i annat fall om det finns synnerliga skäl mot det.
\subsection*{6 §}
\paragraph*{}
En fråga om ändring enligt 3 § får inte tas upp av den handläggande myndigheten senare än två år efter den dag då beslutet meddelades. Ändring får dock ske även senare än två år efter den dag då beslutet meddelades
\newline 1. om det först därefter kommit fram att beslutet har fattats på uppenbart felaktigt eller ofullständigt underlag, eller
\newline 2. om det finns andra synnerliga skäl.
\subsection*{7 §}
\paragraph*{}
Försäkringskassan och Pensionsmyndigheten ska ompröva ett beslut som har fattats av respektive myndighet om det skriftligen begärs av en enskild som beslutet angår och beslutet inte har meddelats med stöd av 3 §.
\paragraph*{}
Ett avvisningsbeslut som grundas på att en begäran om omprövning eller ett överklagande kommit in för sent får inte omprövas. Inte heller får omprövning avse en fråga som har avgjorts
\newline - efter omprövning, eller
\newline - av domstol.
\paragraph*{}
Bestämmelser om begäran om omprövning genom Internet finns i 111 kap. 4-7 §§.
\subsection*{8 §}
\paragraph*{}
Vid omprövningen får beslutet inte ändras till den enskildes nackdel.
\subsection*{9 §}
\paragraph*{}
Har omprövning begärts av ett beslut och överklagar det allmänna ombudet samma beslut, ska beslutet inte omprövas.
Ärendet ska i stället överlämnas till allmän förvaltningsdomstol. Den enskildes begäran om omprövning ska i sådant fall anses som ett överklagande.
\subsection*{10 §}
\paragraph*{}
Försäkringskassans och Pensionsmyndighetens beslut får överklagas hos allmän förvaltningsdomstol.
Ett beslut får dock inte överklagas av en enskild innan beslutet har omprövats enligt 7 §. En enskilds överklagande av ett sådant beslut innan beslutet har omprövats ska anses som en begäran om omprövning enligt nämnda paragraf. 11 § Det som föreskrivs i 10 § andra stycket gäller inte i fråga om beslut som avses i 7 § andra stycket första meningen och inte heller beslut som har meddelats med stöd av 3 §. Den enskildes begäran om omprövning av ett sådant beslut ska anses som ett överklagande.
\subsection*{12 §}
\paragraph*{}
Ett beslut av Försäkringskassan eller Pensionsmyndigheten får överklagas av det allmänna ombudet.
Det allmänna ombudet får uppdra åt en tjänsteman vid Försäkringskassan eller Pensionsmyndigheten att företräda ombudet i allmän förvaltningsdomstol.
\paragraph*{}
Det allmänna ombudet får föra talan även till förmån för enskild part.
\subsection*{13 §}
\paragraph*{}
Om det allmänna ombudet har överklagat ett beslut av Försäkringskassan eller Pensionsmyndigheten, förs det allmännas talan i allmän förvaltningsdomstol av ombudet.
Rätt domstol
\subsection*{14 §}
\paragraph*{}
Ett beslut som rör en person som är bosatt i Sverige överklagas hos den förvaltningsrätt inom vars domkrets personen hade sin hemortskommun när beslutet i saken fattades.
\paragraph*{}
Beslut i övriga fall överklagas hos den förvaltningsrätt inom vars domkrets det första beslutet i saken fattades.
\subsection*{15 §}
\paragraph*{}
Med hemortskommun avses den kommun där den fysiska personen var folkbokförd den 1 november året före det år då beslutet fattades.
\paragraph*{}
För den som var bosatt här i landet under någon del av det år då beslutet fattades, men som inte var folkbokförd här den 1 november föregående år, avses med hemortskommun den kommun där den fysiska personen först var bosatt.
\subsection*{16 §}
\paragraph*{}
Prövningstillstånd krävs vid överklagande hos kammarrätten.
\subsection*{17 §}
\paragraph*{}
Försäkringskassan, Pensionsmyndigheten och det allmänna ombudet får vid överklagande av domstols beslut föra talan även till förmån för enskild part.
\subsection*{18 §}
\paragraph*{}
Har upphävts genom
lag (2013:82).
\subsection*{19 §}
\paragraph*{}
En begäran om omprövning enligt 7 § ska ha kommit in till den handläggande myndigheten inom två månader från den dag då den enskilde fick del av beslutet.
\subsection*{20 §}
\paragraph*{}
Ett överklagande av Försäkringskassans, Pensionsmyndighetens eller en allmän förvaltningsdomstols beslut ska ha kommit in inom två månader från den dag då klaganden fick del av beslutet. Om det är det allmänna ombudet, Försäkringskassan eller Pensionsmyndigheten som överklagar beslutet, ska tiden dock räknas från den dag då beslutet meddelades.
Handlingar till fel myndighet
\subsection*{20 a §}
\paragraph*{}
Om en begäran om omprövning eller ett överklagande som gäller bostadstillägg eller ersättning på grund av arbetsskada eller skada som avses i 43 eller 44 kap. och som ska göras hos Försäkringskassan i stället har kommit in till Pensionsmyndigheten, ska handlingen anses inkommen till Försäkringskassan samma dag. Detsamma ska gälla om en begäran om omprövning eller ett överklagande ska göras hos Pensionsmyndigheten och handlingen i stället har kommit in till Försäkringskassan.
\subsection*{21 §}
\paragraph*{}
Har en begäran om omprövning eller ett överklagande avvisats ska avvisningsbeslutet, med de undantag som anges i 11 §, överklagas i samma ordning som beslutet i huvudsaken.
\subsection*{22 §}
\paragraph*{}
Bestämmelserna om ändring i 3-6 §§ tillämpas inte i ärenden om allmän ålderspension. I 36 § förvaltningslagen (2017:900) finns bestämmelser om rättelse av skrivfel och liknande.
\paragraph*{}
I fråga om omprövning av Pensionsmyndighetens och Skatteverkets beslut i ärenden om allmän ålderspension tillämpas endast 7 § andra stycket samt 9, 21 och 23-31 §§.
\paragraph*{}
När det gäller omprövning av ett beslut om pensionsgrundande inkomst ska vad som i de i andra stycket angivna bestämmelserna, utom 25 och 28 §§, föreskrivs om Pensionsmyndigheten och det allmänna ombudet i stället avse Skatteverket och det allmänna ombudet hos Skatteverket enligt 67 kap. 3 § skatteförfarandelagen (2011:1244).
Lag (2018:780).
\subsection*{23 §}
\paragraph*{}
Ett beslut om allmän ålderspension ska omprövas om det skriftligen begärs av den som beslutet avser eller om det finns andra skäl.
\subsection*{24 §}
\paragraph*{}
Om den som ett beslut om inkomstgrundad ålderspension avser har avlidit, får även någon annan som är berörd av beslutet skriftligen begära omprövning av detta.
\subsection*{25 §}
\paragraph*{}
Om det vid omprövning av ett beslut om pensionsrätt eller pensionspoäng för inkomstgrundad ålderspension även ska göras en prövning av den pensionsgrundande inkomst som legat till grund för beslutet, ska Pensionsmyndigheten överlämna ärendet i den delen till Skatteverket som ska ompröva beslutet om denna inkomst.
\paragraph*{}
Den enskildes begäran om omprövning ska i fall som avses i första stycket även anses som en begäran om omprövning av pensionsgrundande inkomst.
\subsection*{26 §}
\paragraph*{}
En begäran om omprövning av ett beslut enligt 23 eller 24 § om allmän ålderspension ska ha kommit in inom två månader från det att den som begär omprövning fick del av beslutet.
\paragraph*{}
I fråga om omprövning av ett beslut som avser fastställande av pensionsgrundande inkomst ska begäran om omprövning ha kommit in till Skatteverket före utgången av femte året efter det fastställelseår som beslutet avser.
I fråga om omprövning av ett beslut som avser fastställande av pensionsgrundande belopp, pensionsrätt eller pensionspoäng ska begäran om omprövning ha kommit in till Pensionsmyndigheten före utgången av året efter det fastställelseår som beslutet avser.
\subsection*{27 §}
\paragraph*{}
Gör den som begär omprövning av ett beslut om fastställande av pensionsgrundande inkomst, pensionsgrundande belopp, pensionsrätt och pensionspoäng sannolikt att han eller hon inte inom två månader före utgången av den tid som anges i 26 § andra och tredje styckena har fått kännedom om ett beslut eller annan handling med uppgift om vad som har beslutats, ska begäran om omprövning i stället ha kommit in inom två månader från den dag då den som begär omprövningen fick sådan kännedom.
\subsection*{28 §}
\paragraph*{}
Om en begäran om omprövning av ett beslut om allmän ålderspension har kommit in till Skatteverket, Pensionsmyndigheten eller en allmän förvaltningsdomstol inom den tid som anges i 26 och 27 §§, ska den anses inkommen i rätt tid, även om den myndigheten inte har fattat beslutet.
\paragraph*{}
I fall som avses i första stycket ska handlingen sändas till den myndighet som har meddelat beslutet med uppgift om när den kom in.
\subsection*{29 §}
\paragraph*{}
Ett beslut om allmän ålderspension får omprövas till den enskildes fördel, utan att den enskilde har begärt detta, inom två månader från det att beslutet meddelades. En omprövning till den enskildes nackdel får endast ske när det finns anledning till omprövning enligt 30 eller 31 §.
\paragraph*{}
Det som anges i första stycket gäller inte omprövning av ett beslut om pensionsgrundande inkomst, pensionsgrundande belopp eller fastställande av pensionsrätt eller pensionspoäng. Även ett sådant beslut får omprövas utan att den enskilde begär det. Ett sådant omprövningsbeslut ska meddelas inom ett år efter fastställelseåret.
\subsection*{30 §}
\paragraph*{}
Om ett beslut om allmän ålderspension har blivit felaktigt på grund av att den som beslutet avser har lämnat oriktiga uppgifter eller inte har fullgjort en uppgifts- eller anmälningsskyldighet enligt denna balk eller någon annan lag, får omprövning ske också efter de tidpunkter som anges i 29 §.
\subsection*{31 §}
\paragraph*{}
Omprövning av ett beslut om allmän ålderspension efter de i 29 § angivna tidpunkterna får också göras om
\newline 1. det först efter de i 29 § angivna tidpunkterna har visat sig att beslutet har fattats på uppenbart felaktigt eller ofullständigt underlag eller om det finns andra synnerliga skäl,
\newline 2. beslutet påverkas av en ändring av ett beslut enligt skatteförfarandelagen (2011:1244) om arbetsgivaravgift, eller
\newline 3. beslutet omprövas på grund av att det påverkas av en ändring som har gjorts av ett beslut om pensionsgrundande belopp eller pensionsrätt som avser någon annan än den försäkrade.
Lag (2011:1434).
\subsection*{32 §}
\paragraph*{}
Vid överklagande av Pensionsmyndighetens, Skatteverkets och allmän förvaltningsdomstols beslut i ärenden om allmän ålderspension tillämpas 10-17, 20, 21 och 33-36 §§.
\paragraph*{}
Bestämmelserna om det allmänna ombudet tillämpas dock inte i fråga om beslut som avser premiepension.
\paragraph*{}
När det gäller överklagande av beslut om pensionsgrundande inkomst ska vad som i de i första och andra styckena angivna bestämmelserna, utom 35 §, föreskrivs om Pensionsmyndigheten och det allmänna ombudet i stället avse Skatteverket och det allmänna ombudet hos Skatteverket enligt 67 kap. 3 § skatteförfarandelagen (2011:1244).
Lag (2013:82).
\subsection*{33 §}
\paragraph*{}
Om den som ett beslut om allmän ålderspension avser har avlidit, får även någon annan som är berörd av beslutet överklaga detta.
\subsection*{34 §}
\paragraph*{}
Om det allmänna ombudet överklagar ett beslut av Pensionsmyndigheten om pensionsgrundande belopp, pensionsrätt eller pensionspoäng ska överklagandet ha kommit in före utgången av året efter fastställelseåret.
\paragraph*{}
Om beslutet har meddelats efter den 31 oktober året efter det fastställelseår som beslutet avser, ska överklagandet dock ha kommit in inom två månader från den dag då beslutet meddelades.
\subsection*{35 §}
\paragraph*{}
Bestämmelserna i 28 § om begäran om omprövning ska tillämpas även i fråga om överklagande.
\subsection*{36 §}
\paragraph*{}
Det allmänna ombudet får överklaga ett beslut av allmän förvaltningsdomstol endast om ombudet varit part där.
Pensionsmyndigheten får inte överklaga ett beslut av allmän förvaltningsdomstol som kan överklagas av det allmänna ombudet.
\subsection*{36 a §}
\paragraph*{}
De bestämmelser som anges i 22 § om ändring och omprövning av beslut i ärenden om allmän ålderspension, gäller i tillämpliga delar i ärenden om inkomstpensionstillägg.
Lag (2020:1239).
\subsection*{37 §}
\paragraph*{}
I fråga om ändring, omprövning och överklagande av ett beslut i ärende om efterlevandepension och efterlevandestöd gäller i tillämpliga delar de bestämmelser som anges i 22 och 32 §§ om omprövning och överklagande av beslut i ärenden om allmän ålderspension. Följande bestämmelser är dock inte tillämpliga:
\newline - 24 och 25 §§ om omprövning av beslut i ärenden om allmän ålderspension, samt
\newline - 34 § om överklagande av beslut i ärenden om allmän ålderspension.
\subsection*{38 §}
\paragraph*{}
Bestämmelsen i 30 § tillämpas även i fall då det är den avlidne som har lämnat oriktiga uppgifter eller inte fullgjort en uppgifts- eller anmälningsskyldighet.
\subsection*{39 §}
\paragraph*{}
De bestämmelser som anges i 22 och 32 §§ om ändring, omprövning och överklagande av beslut i ärenden om allmän ålderspension, gäller i tilllämpliga delar i fråga om omprövning och överklagande av beslut i ärenden om premiepension till efterlevande.
\subsection*{40 §}
\paragraph*{}
Bestämmelsen i 30 § ska i fråga om beslut i ärenden om premiepension till efterlevande tillämpas också då det är pensionsspararen eller den som tidigare hade rätt till sådan pension som har lämnat oriktiga uppgifter eller inte fullgjort en uppgifts- eller anmälningsskyldighet.
\subsection*{41 §}
\paragraph*{}
I fråga om rättelse, ändring och överklagande av Pensionsmyndighetens beslut enligt 64 kap. 40 § om att ta ut avgift från fondförvaltare gäller bestämmelserna i 36-49 §§ förvaltningslagen (2017:900).
Lag (2022:761).
\chapter*{114 Behandling av personuppgifter}
\subsection*{1 §}
\paragraph*{}
I detta kapitel finns bestämmelser om
\newline - tillämpningsområdet i 2 §,
\newline - förhållandet till annan reglering i 3 och 4 §§,
\newline - uppgifter om avlidna personer i 5 §,
\newline - begränsning av rätten att göra invändningar i 6 §,
\newline - personuppgiftsansvar i 7 §,
\newline - ändamål för behandling av personuppgifter i 8-10 §§,
\newline - känsliga personuppgifter i 11 och 12 §§,
\newline - tillgång till personuppgifter i 13 §,
\newline - direktåtkomst i 14 §,
\newline - gallring i 15 §,
\newline - avgifter i 16 §, och
\newline - tystnadsplikt i 17 §.
Lag (2024:13).
\subsection*{2 §}
\paragraph*{}
Detta kapitel gäller vid behandling av personuppgifter i verksamhet som avser förmåner enligt denna balk samt andra förmåner och ersättningar som enligt lag eller förordning eller särskilt beslut av regeringen handläggs av Försäkringskassan eller Pensionsmyndigheten.
\paragraph*{}
Kapitlet gäller endast om behandlingen är helt eller delvis automatiserad eller om personuppgifterna ingår i eller kommer att ingå i ett register.
Lag (2024:13).
\subsection*{3 §}
\paragraph*{}
Detta kapitel innehåller bestämmelser som kompletterar Europaparlamentets och rådets förordning (EU) 2016/679 av den 27 april 2016 om skydd för fysiska personer med avseende på behandling av personuppgifter och om det fria flödet av sådana uppgifter och om upphävande av direktiv 95/46/EG (allmän dataskyddsförordning), här benämnd EU:s dataskyddsförordning.
\paragraph*{}
Vid behandling av personuppgifter enligt detta kapitel gäller lagen (2018:218) med kompletterande bestämmelser till EU:s dataskyddsförordning och föreskrifter som har meddelats i anslutning till den lagen, om inte annat följer av detta kapitel eller föreskrifter som har meddelats i anslutning till kapitlet.
Lag (2024:13).
\subsection*{4 §}
\paragraph*{}
I fråga om behandling av personuppgifter inom ramen för den officiella statistiken finns särskilda bestämmelser i lagen (2001:99) om den officiella statistiken och i föreskrifter som har meddelats i anslutning till den lagen.
Lag (2024:13).
\subsection*{5 §}
\paragraph*{}
Vid behandling av uppgifter om avlidna personer ska följande reglering gälla i tillämpliga delar:
\newline 1. detta kapitel och föreskrifter som har meddelats i anslutning till kapitlet,
\newline 2. EU:s dataskyddsförordning, och
\newline 3. lagen (2018:218) med kompletterande bestämmelser till EU:s dataskyddsförordning och föreskrifter som har meddelats i anslutning till den lagen.
Lag (2024:13).
\subsection*{6 §}
\paragraph*{}
Artikel 21.1 i EU:s dataskyddsförordning om rätten att göra invändningar gäller inte vid sådan behandling av personuppgifter som är tillåten enligt detta kapitel eller föreskrifter som har meddelats i anslutning till kapitlet.
Lag (2024:13).
\subsection*{7 §}
\paragraph*{}
Försäkringskassan och Pensionsmyndigheten är personuppgiftsansvariga för behandling av personuppgifter i sina respektive verksamheter.
Lag (2024:13).
\subsection*{8 §}
\paragraph*{}
Försäkringskassan och Pensionsmyndigheten får behandla personuppgifter om det är nödvändigt för att
\newline 1. tillgodose behovet av det underlag som krävs för att den registrerades eller någon annans rättigheter eller skyldigheter i fråga om förmåner och ersättningar ska kunna bedömas eller fastställas,
\newline 2. informera om förmåner och ersättningar,
\newline 3. handlägga ärenden,
\newline 4. vidta kontrollåtgärder som syftar till att förebygga, förhindra och upptäcka felaktiga utbetalningar och bidragsbrott,
\newline 5. planera sina respektive verksamheter,
\newline 6. inom sina respektive verksamheter genomföra resultatstyrning, resultatuppföljning, resultatredovisning, utvärdering eller tillsyn, eller
\newline 7. framställa statistik i fråga om verksamhet enligt 3-6.
Lag (2024:13).
\subsection*{9 §}
\paragraph*{}
Personuppgifter som behandlas för ändamål som anges i 8 § får behandlas av Försäkringskassan och Pensionsmyndigheten även för att fullgöra uppgiftslämnande som
\newline 1. sker i överensstämmelse med lag eller förordning, eller
\newline 2. följer av unionsrätten, åtaganden i samarbetet inom Europeiska ekonomiska samarbetsområdet eller avtal om social trygghet eller utgivande av sjukvårdsförmåner som Sverige ingått med andra stater.
Lag (2024:13).
\subsection*{10 §}
\paragraph*{}
Personuppgifter som behandlas enligt 8 eller 9 § får behandlas även för andra ändamål, under förutsättning att uppgifterna inte behandlas på ett sätt som är oförenligt med det ändamål för vilket uppgifterna samlades in.
Lag (2024:13).
\subsection*{11 §}
\paragraph*{}
Sådana personuppgifter som avses i artikel 9.1 i EU:s dataskyddsförordning (känsliga personuppgifter) får behandlas om uppgifterna har lämnats i ett ärende eller är nödvändiga för handläggning av ett ärende. Känsliga personuppgifter får också behandlas om det är nödvändigt för något av de ändamål som anges i 9 §.
\paragraph*{}
Känsliga personuppgifter om hälsa får även behandlas om det är
\newline 1. nödvändigt för något av de ändamål som anges i 8 § 1 och 2,
\newline 2. nödvändigt för något av de ändamål som anges i 8 § 4-7 och uppgifterna behandlas eller har behandlats för något av de ändamål som anges i 8 § 1-3, eller
\newline 3. absolut nödvändigt för behandling med stöd av 10 § och uppgifterna behandlas eller har behandlats för något av de ändamål som anges i 8 § 1-3.
\paragraph*{}
I andra fall får känsliga personuppgifter inte behandlas.
Lag (2024:13).
\subsection*{12 §}
\paragraph*{}
Sökningar får inte utföras i syfte att få fram ett urval av personer grundat på känsliga personuppgifter.
\paragraph*{}
För de ändamål som anges i 8 § får dock sökningar utföras i syfte att få fram ett urval av personer grundat på känsliga personuppgifter om hälsa, om uppgiften avser förmån eller ersättning eller om urvalet ska användas för att
\newline 1. vidta åtgärder i handläggningen,
\newline 2. planera, följa upp eller utvärdera handläggningen, eller
\newline 3. vidta kontrollåtgärder som syftar till att förebygga, förhindra och upptäcka felaktiga utbetalningar och bidragsbrott.
Lag (2024:13).
\subsection*{13 §}
\paragraph*{}
Tillgången till personuppgifter ska begränsas till det som var och en behöver för att kunna fullgöra sina arbetsuppgifter.
\paragraph*{}
Åtkomsten till personuppgifter ska kontrolleras och följas upp regelbundet.
Lag (2024:13).
\subsection*{14 §}
\paragraph*{}
Direktåtkomst till personuppgifter i sådan verksamhet som avses i 2 § är tillåten endast i den utsträckning som anges i lag eller förordning.
\paragraph*{}
Regeringen eller den myndighet som regeringen bestämmer kan med stöd av 8 kap. 7 § regeringsformen meddela föreskrifter om vem som får medges direktåtkomst och vilka uppgifter som då får omfattas av direktåtkomsten.
Lag (2024:13).
\subsection*{15 §}
\paragraph*{}
Personuppgifter som behandlas automatiserat ska gallras när de inte längre är nödvändiga för de ändamål som anges i 8 §.
\paragraph*{}
Regeringen eller den myndighet som regeringen bestämmer kan med stöd av 8 kap. 7 § regeringsformen meddela föreskrifter om att uppgifter får bevaras för arkivändamål av allmänt intresse, vetenskapliga eller historiska forskningsändamål eller statistiska ändamål.
Lag (2024:13).
\subsection*{16 §}
\paragraph*{}
Avgifter får tas ut för utlämnande av uppgifter och handlingar från sådan verksamhet som avses i 2 §.
\paragraph*{}
Regeringen eller den myndighet som regeringen bestämmer kan med stöd av 8 kap. 7 § regeringsformen meddela föreskrifter om uttagandet av avgifter.
\paragraph*{}
Rätten att ta ut avgifter enligt första och andra styckena får inte innebära någon inskränkning i rätten att ta del av och mot fastställd avgift få en kopia eller utskrift av en allmän handling enligt tryckfrihetsförordningen.
Lag (2024:13).
\subsection*{17 §}
\paragraph*{}
Den som, genom sin befattning med personuppgifter som har inhämtats från sådan verksamhet som avses i 2 § till det för kommunerna och regionerna gemensamma organet för administration av personalpensioner, får kännedom om uppgifter om enskildas ekonomiska och personliga förhållanden får inte obehörigen röja dessa uppgifter.
Lag (2024:13).
\chapter*{115 Straffbestämmelser}
\subsection*{1 §}
\paragraph*{}
I detta kapitel finns bestämmelser om arbetsgivare och uppdragsgivare m.fl. i 3 och 4 §§.
Lag (2013:427).
\subsection*{2 §}
\paragraph*{}
Har upphävts genom
lag (2013:427).
\subsection*{3 §}
\paragraph*{}
Arbetsgivare eller arbetsföreståndare som underlåter att fullgöra anmälningsskyldighet i fråga om arbetsskada enligt 42 kap. 10 § första stycket döms till böter.
\subsection*{4 §}
\paragraph*{}
Arbetsgivare eller uppdragsgivare som underlåter att fullgöra uppgiftsskyldighet enligt 110 kap. 31 § döms till penningböter.
\chapter*{116 Innehåll}
\subsection*{1 §}
\paragraph*{}
I denna underavdelning finns bestämmelser om socialförsäkringsväsendet under krig och krigsfara i 117 kap.
\chapter*{117 Socialförsäkringen under krig och krigsfara}
\subsection*{1 §}
\paragraph*{}
I detta kapitel finns bestämmelser om
\newline - tillämpningen i 2-4 §§,
\newline - socialförsäkringsskyddet i 5 §,
\newline - utbetalning av socialförsäkringsförmåner m.m. i 6-8 §§, och
\newline - avvikande föreskrifter i 9 §.
\subsection*{2 §}
\paragraph*{}
Kommer Sverige i krig, ska 5-9 §§ tillämpas.
\subsection*{3 §}
\paragraph*{}
Regeringen får föreskriva att 5-9 §§ helt eller delvis ska tillämpas från den tidpunkt som regeringen bestämmer
\newline 1. om Sverige är i krigsfara, eller
\newline 2. om det råder sådana utomordentliga förhållanden som är föranledda av att det är krig utanför Sveriges gränser eller av att Sverige har varit i krig eller krigsfara.
\paragraph*{}
Föreskrifter enligt första stycket ska underställas riksdagens prövning inom en månad från det de beslutades.
Sker inte underställning eller godkänner riksdagen inte föreskrifterna inom två månader från det underställning skedde, upphör föreskrifterna att gälla.
\subsection*{4 §}
\paragraph*{}
Under tid då bestämmelserna i 5-9 §§ tillämpas gäller inte i lag eller annan författning meddelade föreskrifter i den mån de strider mot 5-9 §§ eller föreskrifter som har meddelats med stöd av dessa bestämmelser.
\paragraph*{}
Gäller inte längre sådana förhållanden som avses i 2 § eller 3 § första stycket, ska regeringen föreskriva att bestämmelserna i 5-9 §§ inte längre ska tillämpas.
\subsection*{5 §}
\paragraph*{}
Försäkringskassan behöver inte ta ställning till om en person ska omfattas av socialförsäkringsskyddet enligt 4-7 kap. förrän fråga uppkommer om rätten till en förmån.
\paragraph*{}
Om Försäkringskassan beslutar om försäkring för sjukpenning, gäller beslutet i den delen från den tidigare tidpunkt då sådana förhållanden inträtt att den försäkrade skulle ha varit försäkrad för sjukpenning.
\subsection*{6 §}
\paragraph*{}
Bestämmelserna i 7 och 8 §§ ska tillämpas på förmåner som enligt lag eller annan författning eller regeringens beslut månatligen eller för längre tidsperiod ska betalas ut av Försäkringskassan eller Pensionsmyndigheten.
\paragraph*{}
Efter överenskommelse mellan Pensionsmyndigheten och Statens tjänstepensionsverk ska Pensionsmyndigheten svara för utbetalningen av statliga tjänstepensioner. Motsvarande gäller pensioner som administreras av ett för kommunerna och regionerna gemensamt organ. Därvid tillämpas bestämmelserna i 7 och 8 §§ på sådana pensioner.
Lag (2019:843).
\subsection*{7 §}
\paragraph*{}
Den som är berättigad till en förmån som avses i 6 § ska få ett bevis (förmånsbevis), som anger den ersättning som han eller hon är berättigad till.
\paragraph*{}
Den som har anspråk på en förmån men inte är registrerad som förmånsberättigad ska anmäla detta till den handläggande myndigheten. Om myndigheten finner att han eller hon är berättigad till förmånen, ska förmånsbevis utfärdas. Om förutsättningarna för rätten till eller storleken av en förmån ändras, ska myndigheten, i förekommande fall mot återlämnande av tidigare utfärdat förmånsbevis, utfärda nytt förmånsbevis.
\paragraph*{}
Den försäkrade är skyldig att utan dröjsmål till den handläggande myndigheten överlämna det förmånsbevis som tidigare utfärdats om nytt förmånsbevis utfärdas eller om rätten att få förmåner upphör. Detta gäller inte om förmånsbeviset har förkommit eller om hinder möter mot att återlämna förmånsbeviset.
\subsection*{8 §}
\paragraph*{}
Om den handläggande myndighetens datasystem har försatts ur funktion, får förmåner som avses i 6 § betalas ut mot uppvisande av förmånsbeviset.
\paragraph*{}
Inträffar något förhållande som avses i första stycket innan förmånsbevis har utfärdats, får förmånen betalas ut mot uppvisande av en utbetalningshandling som avser närmast föregående utbetalningsperiod.
\paragraph*{}
Om utbetalningshandlingen har förkommit, får den handläggande myndigheten på begäran av förmånstagaren utfärda särskilt bevis om rätten till förmån. Motsvarande gäller om förutsättningarna för rätten till förmånen eller storleken av förmånen har ändrats.
\subsection*{9 §}
\paragraph*{}
Regeringen får i den mån förhållandena kräver det föreskriva att förmåner som administreras av Försäkringskassan eller Pensionsmyndigheten inte ska betalas ut. Regeringen får också föreskriva att sådana förmåner ska bestämmas enligt andra grunder eller i annan ordning än som gäller enligt lag eller annan författning.
\paragraph*{}
Övergångsbestämmelser
2011:1082
\paragraph*{}
Denna lag träder i kraft den 1 januari 2012. Äldre föreskrifter gäller fortfarande för förmåner som avser tid före ikraftträdandet.
\paragraph*{}
2011:1091
\paragraph*{}
Denna lag träder i kraft den 1 november 2011. Den nya bestämmelsen ska dock tillämpas för tid från och med den 1 juli 2011.
\paragraph*{}
2011:1288
\newline 1. Denna lag träder i kraft den 1 januari 2012.
\newline 2. Bestämmelserna i 19 kap. 13 § i sin nya lydelse tillämpas första gången i fråga om betalningsskyldighet för underhållsstöd som avser tid efter den 31 januari 2014.
\newline 3. Äldre bestämmelser i 97 kap. 5 § gäller fortfarande för förmåner som avser tid före ikraftträdandet.
\paragraph*{}
2011:1434
\newline 1. Denna lag träder i kraft den 1 januari 2012.
\newline 2. Lagen tillämpas på beskattningsår som börjar efter den 31 januari 2012.
\newline 3. Bestämmelserna om ränta i 64 kap. 28 § andra stycket och 108 kap. 17 § första stycket tillämpas på ränta som hänför sig till tid från och med den 1 januari 2013. För ränta som hänför sig till tid dessförinnan gäller i stället bestämmelserna om ränta i 19 kap. skattebetalningslagen (1997:483).
\paragraph*{}
2011:1513
\newline 1. Denna lag träder i kraft den 1 juli 2012 i fråga om 28 a kap. 18 §, 31 a kap. 13 § och 103 c kap. 9 och 10 §§ samt i övrigt den 1 januari 2012.
\newline 2. Bestämmelserna i 27 kap. 22 §, 23 § och 24 a § 1 tillämpas även för dagar från och med ikraftträdandet som ingår i en sjukperiod som har påbörjats dessförinnan.
\newline 3. Bestämmelserna i 27 kap. 24 a § 5 tillämpas även för en försäkrad som före ikraftträdandet har fått sjukpenning på fortsättningsnivån för 550 dagar eller tidsbegränsad sjukersättning under det högsta antalet månader som sådan ersättning kan betalas ut enligt 4 kap. 31 § lagen (2010:111) om införande av socialförsäkringsbalken.
\newline 4. Bestämmelserna i 28 a, 31 a och 103 b-103 e kap. tillämpas även för en försäkrad vars rätt till tidsbegränsad sjukersättning har upphört före den 1 januari 2012. Ersättning enligt dessa bestämmelser kan dock lämnas tidigast från och med den 1 januari 2012.
\newline 5. Ersättning enligt 28 a kap. lämnas för dagar i januari 2012, om den försäkrade har gjort en sjukanmälan senast den 31 januari 2012.
\paragraph*{}
2011:1519
\paragraph*{}
Denna lag träder i kraft den 1 januari 2012. De nya bestämmelserna ska tillämpas för tid från och med ikraftträdandet.
\paragraph*{}
2011:1520
\paragraph*{}
Denna lag träder i kraft den 1 januari 2012. Äldre bestämmelser tillämpas fortfarande för bostadsbidrag som avser tid före ikraftträdandet.
\paragraph*{}
2012:98
\paragraph*{}
Denna lag träder i kraft den dag regeringen bestämmer och tillämpas på arbete som utförs efter ikraftträdandet.
\paragraph*{}
2012:256
\newline 1. Denna lag träder i kraft den 1 juli 2012.
\newline 2. De nya bestämmelserna tillämpas även på sjukperioder som har påbörjats före ikraftträdandet. Bestämmelserna tillämpas dock första gången vid prövning av rätt till ersättning för dagar från och med ikraftträdandet.
\paragraph*{}
2012:599
\newline 1. Denna lag träder i kraft den 1 november 2012.
\newline 2. De nya bestämmelserna om efterlevandepension, efterlevandestöd och efterlevandelivränta i 77 kap. 13 § och 88 kap. 10 § tillämpas vid dödsfall som inträffat efter den 31 oktober 2012.
\newline 3. Den nya bestämmelsen i 101 kap. 11 § tillämpas första gången på beslut om bostadstillägg som fattas efter den 31 oktober 2012.
\paragraph*{}
2012:834
\paragraph*{}
Denna lag träder i kraft den 1 januari 2013 och tillämpas på ersättning som betalas ut respektive inkomst som uppbärs efter den 31 december 2012.
\paragraph*{}
2012:896
\newline 1. Denna lag träder i kraft den 1 januari 2013.
\newline 2. De nya föreskrifterna i 18 kap. 21, 24 och 25 §§ tillämpas första gången i fråga om underhållsstöd som avser tid efter den 31 januari 2013.
\newline 3. De nya föreskrifterna i 19 kap. 43 § tredje stycket ska tillämpas när en bidragsskyldig inte i rätt tid har betalat ett fastställt belopp som har förfallit till betalning efter ikraftträdandet.
\paragraph*{}
2012:931
\paragraph*{}
Denna lag träder i kraft den 1 januari 2013. Äldre föreskrifter i 12 kap. 22, 23 och 35 §§ gäller fortfarande för förmåner som avser tid före ikraftträdandet.
\paragraph*{}
2012:932
\newline 1. Denna lag träder i kraft den 1 januari 2013.
\newline 2. En försäkrad som vid ikraftträdandet har fyllt 55 år får senast den 30 april 2013 till Försäkringskassan anmäla försäkring med karenstid på 1 dag enligt bestämmelserna i 27 kap. 31 §.
\newline 3. Vid tillämpning av bestämmelserna i 27 kap. 39 § andra stycket och 27 kap. 39 a § ska, inom tolvmånadersperioden, även beaktas sjukperioder som infallit före ikraftträdandet.
\newline 4. För en försäkrad som omfattas av 27 kap. 27 § första stycket 2 och som gör anmälan om karenstid på 1 dag enligt 27 kap. 29 § och för en försäkrad som omfattas av 27 kap. 28 b § första stycket första meningen ska ett beslut om särskilt högriskskydd, som fattats före ikraftträdandet enligt 27 kap. 40 § avseende de första sju dagarna i en sjukperiod, istället anses gälla för den första dagen i en sjukperiod.
\paragraph*{}
2012:933
\paragraph*{}
Denna lag träder i kraft den 1 januari 2013 och tillämpas på aktivitetsersättning som avser tid efter den 31 januari 2013.
\paragraph*{}
2012:934
\paragraph*{}
Denna lag träder i kraft den 1 januari 2013. De nya bestämmelserna ska tillämpas för tid från och med ikraftträdandet.
\paragraph*{}
2012:935
\newline 1. Denna lag träder i kraft den 1 juli 2013.
\newline 2. Föreskrifterna i 51 kap. 16 § första stycket 1 och andra stycket ska inte tillämpas när en personlig assistent har anställts före ikraftträdandet.
\paragraph*{}
2013:82
\newline 1. Denna lag träder i kraft den 1 juli 2013.
\newline 2. Äldre föreskrifter gäller för beslut som har meddelats före ikraftträdandet.
\paragraph*{}
2013:747
\newline 1. Denna lag träder i kraft den 1 november 2013 i fråga om 4 kap. 5 a § och 37 kap. 11 §, den 1 januari 2015 i fråga om 108 kap. 1, 14 a och 22 §§ samt i övrigt den 1 januari 2014.
\newline 2. Den nya bestämmelsen i 4 kap. 5 a § tillämpas första gången i fall som avses i punkten 3 från och med den 1 januari 2014.
\newline 3. Vid ikraftträdandet av 4 kap. 5 a § kan den som med stöd av förordning (EG) nr 883/2004 om samordning av de sociala trygghetssystemen omfattas av socialförsäkringsbalken vid utgången av oktober 2013, om han eller hon önskar det, kvarstå i svensk försäkring längst till och med den 30 april 2020. En anmälan om att fortsatt få omfattas av svensk försäkring ska ha inkommit till Försäkringskassan senast den 31 december 2013.
\newline 4. De nya bestämmelserna i 26 kap. 22 a § första stycket och 26 kap. 22 b § tillämpas första gången för försäkrade som vid utgången av december månad 2013 har fått livränta enligt 41, 43 eller 44 kap.
\newline 5. Vid tillämpning av bestämmelserna i 27 kap. 17 § första meningen i sin nya lydelse får även beaktas dagar som infallit före ikraftträdandet.
\newline 6. Bestämmelserna i 27 kap. 25 § första stycket i sin nya lydelse tillämpas även i de fall sjukperiodens första dag har infallit före ikraftträdandet.
\newline 7. Bestämmelserna i 27 kap. 51 § i sin nya lydelse tillämpas första gången när en sjukperiod har påbörjats efter ikraftträdandet. Vid bedömningen av om dagar i denna sjukperiod ska läggas samman med dagar i en tidigare sjukperiod, ska även beaktas sjukperioder som har infallit före ikraftträdandet.
\newline 8. Bestämmelserna i 37 kap. 11 § i sin nya lydelse tillämpas på beslut om utbetalning av preliminär sjukersättning som ska göras från och med den 1 januari 2014.
\newline 9. De nya bestämmelserna i 47 kap. 7 § tillämpas även på vård som har påbörjats, men inte anmälts, före ikraftträdandet.
\newline 10. De nya bestämmelserna i 101 kap. 10 a § tillämpas första gången på beslut varigenom sjukersättning eller aktivitetsersättning har beviljats för tid efter ikraftträdandet.
\paragraph*{}
2013:999
\newline 1. Denna lag träder i kraft den 1 januari 2014.
\newline 2. De nya föreskrifterna i 12 kap. 3, 8, 12, 13, 15 a, 15 b, 20, 41 a-41 h och 46 §§ tillämpas på föräldrapenning för ett barn som har fötts efter ikraftträdandet eller, vid adoption, när den som adopterat barnet har fått barnet i sin vård efter ikraftträdandet.
\newline 3. Den nya föreskriften i 12 kap. 12 a § tillämpas första gången på föräldrapenning som avser tid från och med den 1 januari 2015.
\paragraph*{}
2013:1018
\newline 1. Denna lag träder i kraft den 1 mars 2014.
\newline 2. De nya föreskrifterna i 16 kap. 5-9, 12, 14 och 15 §§ tillämpas första gången
\newline - på barn som är födda efter ikraftträdandet, eller, vid adoption, när den som adopterat barnet har fått barnet i sin vård efter ikraftträdandet, och
\newline - på barn som har blivit försäkrade för barnbidrag efter ikraftträdandet.
\paragraph*{}
De nämnda föreskrifterna tillämpas dock också om det efter ikraftträdandet har gjorts en anmälan om till vem barnbidrag ska lämnas.
\newline 3. Vid tillämpning av äldre föreskrifter i 16 kap. 5-8 §§ gäller fortfarande den upphävda 16 kap. 10 §.
\newline 4. Om en förälder har fått flerbarnstillägg med tillämpning av de nya föreskrifterna i 16 kap. 12 eller 14 §, tillämpas alltid de nya föreskrifterna om flerbarnstillägg för den föräldern och förälderns hushåll även om det i det hushållet inte ingår något barn som avses i punkten 2.
\newline 5. Äldre föreskrifter i 16 kap. 18 § gäller fortfarande om socialnämndens begäran om utbetalning har kommit in till Försäkringskassan före ikraftträdandet.
\paragraph*{}
2013:1099
\paragraph*{}
Denna lag träder i kraft den 1 januari 2014. Äldre bestämmelser tillämpas fortfarande för bostadstillägg som avser tid före ikraftträdandet.
\paragraph*{}
2013:1100
\paragraph*{}
Denna lag träder i kraft den 1 januari 2014. Äldre bestämmelser tillämpas fortfarande för bostadsbidrag som avser tid före ikraftträdandet.
\paragraph*{}
2014:470
\newline 1. Denna lag träder i kraft den 1 juli 2014.
\newline 2. De nya bestämmelserna tillämpas första gången på återkrav som har beslutats av Försäkringskassan efter ikraftträdandet.
\paragraph*{}
2014:1548
\newline 1. Denna lag träder i kraft den 1 januari 2015.
\newline 2. Den nya föreskriften i 58 kap. 20 § tillämpas första gången på balanstal som ska beräknas för år 2016.
\newline 3. Den nya föreskriften i 62 kap. 5 § tillämpas första gången för fastställelseåret 2015.
\paragraph*{}
2015:119
\newline 1. Denna lag träder i kraft den 1 april 2015.
\newline 2. De nya bestämmelserna ska dock tillämpas för tid från och med den 1 mars 2015.
\paragraph*{}
2015:452
\newline 1. Denna lag träder i kraft den 1 september 2015.
\newline 2. Bestämmelserna i den nya lydelsen tillämpas första gången i fråga om underhållsstöd och betalningsskyldighet som avser tid efter den 30 september 2015.
\paragraph*{}
2015:453
\newline 1. Denna lag träder i kraft den 1 oktober 2015 i fråga om 34 kap. 12 §, den 7 september 2015 i fråga om 28 kap. 11 § och i övrigt den 1 september 2015.
\newline 2. Äldre föreskrifter gäller fortfarande för förmåner som avser tid före ikraftträdandet.
\paragraph*{}
2015:674
\newline 1. Denna lag träder i kraft den 1 januari 2016.
\newline 2. Äldre föreskrifter gäller fortfarande för föräldrapenning för ett barn som har fötts före ikraftträdandet, eller vid adoption, när den som adopterat barnet har fått barnet i sin vård före ikraftträdandet.
\paragraph*{}
2015:676
\newline 1. Denna lag träder i kraft den 1 januari 2016.
\newline 2. De nya bestämmelserna i 58 kap. tillämpas första gången vid beräkning av inkomstindex, balanstal, dämpat balanstal och balansindex för år 2017 med de undantag som anges i punkterna 3 och 4. När beräkningarna görs ska de balanstal som har fastställts för respektive år användas.
\newline 3. Inkomstindex för år 2017 ska beräknas enligt följande. Det inkomstindex som har fastställts för år 2016 ska multipliceras med kvoten mellan
\newline - beräknad genomsnittlig pensionsgrundande inkomst för år 2016, efter avdrag för allmän pensionsavgift, för försäkrade som under beskattningsåret har fyllt minst 16 år och högst 64 år och
\newline - beräknad genomsnittlig pensionsgrundande inkomst för år 2015, efter avdrag för allmän pensionsavgift, för försäkrade som under beskattningsåret har fyllt minst 16 år och högst 64 år.
\paragraph*{}
Vid beräkningen av inkomsterna tillämpas inte 59 kap. 4 § andra stycket.
\newline 4. Inkomstindex för år 2018 ska beräknas enligt följande. Det inkomstindex som har fastställts för år 2016 ska multipliceras med kvoten mellan
\newline - beräknad genomsnittlig pensionsgrundande inkomst för år 2017, efter avdrag för allmän pensionsavgift, för försäkrade som under beskattningsåret har fyllt minst 16 år och högst 64 år och
\newline - genomsnittlig pensionsgrundande inkomst för år 2015, efter avdrag för allmän pensionsavgift, för försäkrade som under beskattningsåret har fyllt minst 16 år och högst 64 år.
\paragraph*{}
Vid beräkningen av inkomsterna tillämpas inte 59 kap. 4 § andra stycket.
\newline 5. De nya bestämmelserna i 41 kap. 21 § och 63 kap. 17 § tillämpas första gången vid omräkning av livränta respektive beräkning av årlig tilläggspension för år 2017.
\paragraph*{}
2015:755
\newline 1. Denna lag träder i kraft den 1 april 2016.
\newline 2. Bestämmelsen i 18 kap. 9 a § tillämpas första gången på sådana betalningar som har gjorts till Försäkringskassan efter ikraftträdandet.
\newline 3. De äldre bestämmelserna i 18 kap. 29 § och 19 kap. 32 § samt de upphävda bestämmelserna i 19 kap. 19 och 20 §§ gäller fortfarande avseende barns vistelse hos den bidragsskyldige som har inletts före ikraftträdandet.
\newline 4. Den äldre lydelsen av 19 kap. 47 § i fråga om ränta vid anstånd gäller fortfarande för ränta som avser tid före ikraftträdandet.
\paragraph*{}
2015:758
\newline 1. Denna lag träder i kraft den 1 februari 2016.
\newline 2. Äldre föreskrifter gäller fortfarande när kommunalt vårdnadsbidrag har lämnats enligt den upphävda lagen (2008:307) om kommunalt vårdnadsbidrag.
\paragraph*{}
2015:963
\newline 1. Denna lag träder i kraft den 1 januari 2016 i fråga om 103 d kap. 1 och 9 §§ och 103 e kap. 3 § och i övrigt den 1 februari 2016.
\newline 2. Äldre föreskrifter gäller fortfarande för förmåner som avser tid före ikraftträdandet.
\newline 3. För den som före den 1 februari 2016 har påbörjat deltagande i arbetslivsintroduktion gäller 27 kap. 51 § och 103 c kap. 4 § i den äldre lydelsen.
\paragraph*{}
2015:964
\newline 1. Denna lag träder i kraft den 1 januari 2016.
\newline 2. Äldre föreskrifter gäller fortfarande för föräldrapenning som avser tid före ikraftträdandet.
\paragraph*{}
2016:1066
\newline 1. Denna lag träder i kraft den 1 januari 2017.
\newline 2. Äldre föreskrifter gäller fortfarande i ärenden i vilka en ansökan om bilstöd har kommit in till Försäkringskassan före ikraftträdandet.
\newline 3. Föreskrifterna i 52 kap. 22 a och 25 §§ tillämpas endast om anpassningen har utförts efter den 1 juli 2017.
\paragraph*{}
2016:1269
\newline 1. Denna lag träder i kraft den 1 juli 2017.
\newline 2. För anställningar med lönebidrag som har beslutats före ikraftträdandet gäller 33 kap. 28 § i den äldre lydelsen.
\paragraph*{}
2016:1291
\newline 1. Denna lag träder i kraft den 1 februari 2017.
\newline 2. Föreskrifterna i 33 kap. 16 §, 34 kap. 10 §, 35 kap. 18 och 19 §§, 36 kap. 2, 9 a, 15 och 18 §§ samt 79 kap. 7 § tillämpas första gången i fråga om ersättning som avser tid efter den 28 februari 2017.
\newline 3. Om ett beslut om förnyad utredning enligt 33 kap. 17 § har fattats före ikraftträdandet ska en sådan utredning göras även om tidpunkten för utredningen infaller efter ikraftträdandet. När den utredningen har slutförts ska Försäkringskassan, så länge rätt till sjukersättning föreligger, minst vart tredje år göra en uppföljning av den försäkrades arbetsförmåga enligt 33 kap. 17 § i den nya lydelsen.
\paragraph*{}
2016:1292
\newline 1. Denna lag träder i kraft den 1 januari 2017.
\newline 2. Äldre föreskrifter gäller fortfarande för förmåner som avser tid före ikraftträdandet.
\paragraph*{}
2016:1293
\newline 1. Denna lag träder i kraft den 1 juli 2017.
\newline 2. Äldre föreskrifter gäller fortfarande för förmåner som avser tid före ikraftträdandet.
\paragraph*{}
2016:1294
\newline 1. Denna lag träder i kraft den 1 januari 2017.
\newline 2. I fråga om barn som har kommit i föräldrarnas vård före den 1 januari 2017 ska 21 kap. 2 § tillämpas i sin äldre lydelse.
\paragraph*{}
2017:277
\newline 1. Denna lag träder i kraft den 1 juli 2017.
\newline 2. Äldre föreskrifter gäller fortfarande för utbildningsbidrag för doktorander som lämnas senast den 30 juni 2022.
\paragraph*{}
2017:554
\newline 1. Denna lag träder i kraft den 1 augusti 2017 i fråga om 112 kap. 4 § och i övrigt den 1 november 2017.
\newline 2. I fråga om ärenden som avser ansökan om äldreförsörjningsstöd eller bostadstillägg gäller 74 kap. 19 och 20 §§, 102 kap. 7 § samt 103 kap. 3 och 4 §§ i den äldre lydelsen fortfarande för förmåner som avser tid till och med den månad då beslut om slutlig skatt för beskattningsåret 2017 har fattats av Skatteverket. I fråga om ärenden som avser omprövning av äldreförsörjningsstöd och bostadstillägg tillämpas 74 kap. 19 och 20 §§, 102 kap. 7 § samt 103 kap. 3 och 4 §§ i den nya lydelsen första gången i fråga om förmåner som avser tid för vilken överskottet i inkomstslaget kapital ska beräknas utifrån beslut om slutlig skatt för beskattningsåret 2017.
\paragraph*{}
2017:559
\newline 1. Denna lag träder i kraft den 1 juli 2017.
\newline 2. Äldre föreskrifter gäller fortfarande för föräldrapenning som avser dagar före ikraftträdandet.
\newline 3. Om föräldrapenning för ett barn har beviljats för dagar före ikraftträdandet ska 12 kap. 12 § i den äldre lydelsen tillämpas även för föräldrapenning som avser dagar efter ikraftträdandet.
\paragraph*{}
2017:585
\newline 1. Denna lag träder i kraft den 1 januari 2018.
\newline 2. Äldre föreskrifter gäller fortfarande när en nyanländ invandrare har en etableringsplan enligt den upphävda lagen (2010:197) om etableringsinsatser för vissa nyanlända invandrare, så länge den planen gäller.
\paragraph*{}
2017:995
\newline 1. Denna lag träder i kraft den 1 januari 2018.
\newline 2. De nya bestämmelserna tillämpas första gången i fråga om underhållsstöd och betalningsskyldighet som avser februari 2018.
\paragraph*{}
2017:1123
\newline 1. Denna lag träder i kraft den 1 mars 2018.
\newline 2. De upphävda bestämmelserna i 17 kap. 4 § och 18 kap. 5, 16, 25, 31, 37 och 39 §§ samt de äldre bestämmelserna i 18 kap. 2, 20 och 35 §§ och 19 kap. 4 § ska fortfarande gälla för underhållsstöd som avser tid före april månad 2018.
\newline 3. De upphävda bestämmelserna i 17 kap. 4 § och 18 kap. 5, 16, 25, 31, 37 och 39 §§ samt de äldre bestämmelserna i 18 kap. 2, 20 och 35 §§ och 19 kap. 4 § ska utöver vad som föreskrivs i andra punkten fortfarande gälla för underhållsstöd som avser tid före februari månad 2021, om det för mars månad 2018 har lämnats underhållsstöd för barnet vid växelvist boende. I stället för det belopp som avses i den upphävda 18 kap. 25 § första stycket ska underhållsstöd då lämnas för var och en av föräldrarna med hälften av 1 173 kronor när stödet avser en månad som infaller under tiden april 2018 - januari 2019, med hälften av 773 kronor när stödet avser en månad som infaller under tiden februari 2019 - januari 2020 och med hälften av 373 kronor när stödet avser en månad som infaller under tiden februari 2020 - januari 2021. Från och med månaden efter den då barnet har fyllt 15 år ska respektive belopp före halveringen höjas med 150 kronor.
\newline 4. Äldre bestämmelser i 95 kap. 2 §, 96 kap. 3 §, 97 kap. 6, 11, 15 och 23 §§ och 110 kap. 5 § ska fortfarande gälla för bostadsbidrag som avser tid före mars månad 2018.
\newline 5. De nya bestämmelserna i 95 kap. 2 §, 96 kap. 3 och 5 a §§ och 97 kap. 6, 11, 15 och 22 a §§ ska tillämpas första gången på bostadsbidrag som avser tid efter februari månad 2018.
\newline 6. I stället för de belopp som anges i 97 kap. 15 § andra stycket ska minskning av bostadsbidraget göras om den bidragsgrundande inkomsten överstiger 135 000 kronor respektive 67 500 kronor när bidraget avser en månad som infaller under tiden mars - december 2018, 142 000 kronor respektive 71 000 kronor när bidraget avser en månad som infaller under år 2019 och 148 000 kronor respektive 74 000 kronor när bidraget avser en månad som infaller under år 2020.
\paragraph*{}
2017:1295
\newline 1. Denna lag träder i kraft den 1 januari 2018.
\newline 2. Äldre bestämmelser gäller fortfarande för förmåner som avser tid före ikraftträdandet.
\paragraph*{}
2017:1305
\newline 1. Denna lag träder i kraft den 1 januari 2018 i fråga om 26 kap. 17 §, 27 kap. 26 §, 108 kap. 10 §, 110 kap. 13 a § och 112 kap. 2 a § samt i övrigt den 1 juli 2018.
\newline 2. Bestämmelserna i 110 kap. 13 a § och 112 kap. 2 a § ska tillämpas första gången vid en begäran om sjukpenning eller sjukpenning i särskilda fall som har kommit in till Försäkringskassan den 1 januari 2018 eller därefter och som avser tid efter ikraftträdandet.
\newline 3. Äldre bestämmelser gäller fortfarande i fråga om förmåner som avser tid före ikraftträdandet.
\paragraph*{}
2017:1306
\newline 1. Denna lag träder i kraft den 1 juli 2018.
\newline 2. Den nya bestämmelsen i 30 kap. 6 § ska tillämpas även i fråga om försäkrade vars arbetsförmåga den 30 juni 2018 har varit nedsatt under minst 60 dagar eller kan antas komma att vara nedsatt under minst 60 dagar räknat från och med dagen då arbetsförmågan blev nedsatt. En plan för återgång i arbete ska då ha upprättats senast den 30 september 2018.
\newline 3. Äldre bestämmelser gäller fortfarande i fråga om sjukersättning och aktivitetsersättning som avser tid före ikraftträdandet.
\paragraph*{}
2017:1308
\newline 1. Denna lag träder i kraft den 1 mars 2018.
\newline 2. Äldre bestämmelser gäller fortfarande för tid före ikraftträdandet.
\paragraph*{}
2018:122
\newline 1. Denna lag träder i kraft den 1 april 2018.
\newline 2. Den äldre bestämmelsen i 51 kap. 12 § gäller fortfarande för omprövningsbeslut som har meddelats före ikraftträdandet.
\paragraph*{}
2018:646
\newline 1. Denna lag träder i kraft den 1 juli 2018.
\newline 2. Äldre föreskrifter gäller fortfarande för ansökan om efterlevandestöd som har kommit in till Pensionsmyndigheten före ikraftträdandet.
\paragraph*{}
2018:647
\newline 1. Denna lag träder i kraft den 1 januari 2019.
\newline 2. Äldre föreskrifter gäller fortfarande för en sjukperiod som har påbörjats före ikraftträdandet.
\newline 3. Vid beräkning av det allmänna högriskskyddet enligt 27 kap. 39 § första stycket ska en karensdag enligt föreskriftens äldre lydelse likställas med ett tillfälle med karensavdrag. Vid beräkning av det allmänna högriskskyddet enligt 27 kap. 39 § andra stycket ska sådana tillfällen då den försäkrade har gått miste om sjukpenning till följd av bestämmelserna i 27 kap. 27 § första stycket 2 i föreskriftens äldre lydelse likställas med tillfällen då den försäkrade går miste om sjukpenning till följd av bestämmelserna i 27 kap. 27 a §.
\newline 4. Karensdag ska likställas med karensavdrag vid återinsjuknande enligt 27 kap. 32 §. I de fall en ny sjukperiod ska anses som en fortsättning på en tidigare sjukperiod, och den tidigare sjukperioden inletts med en karensdag ska 27 kap. 21 § tillämpas i föreskriftens äldre lydelse i den nya sjukperioden.
\newline 5. Karensdag ska likställas med karensavdrag för den som före ikraftträdandet har fått ett beslut om särskilt högriskskydd enligt 27 kap. 40-42 §§.
\paragraph*{}
2018:670
\newline 1. Denna lag träder i kraft den 1 augusti 2018 i fråga om 25 kap. 1, 7 a och 9 §§ och i övrigt den 1 juli 2018.
\newline 2. Den nya bestämmelsen i 25 kap. 7 a § och bestämmelsen i 25 kap. 9 § i den nya lydelsen tillämpas första gången på ersättningsperioder som påbörjas efter ikraftträdandet.
\newline 3. Äldre bestämmelser om sjukpenninggrundande inkomst gäller fortfarande för förmåner som avser tid före ikraftträdandet.
\paragraph*{}
2018:742
\newline 1. Denna lag träder i kraft den 1 augusti 2018.
\newline 2. Den nya bestämmelsen tillämpas första gången i fråga om underhållsstöd som avser september 2018.
\paragraph*{}
2018:743
\newline 1. Denna lag träder i kraft den 1 januari 2019.
\newline 2. De nya bestämmelserna tillämpas första gången i fråga om underhållsstöd och betalningsskyldighet som avser februari 2019.
\paragraph*{}
2018:745
\newline 1. Denna lag träder i kraft den 1 januari 2019.
\newline 2. Den nya bestämmelsen ska tillämpas på försäkringsmedicinska utredningar som Försäkringskassan begär från och med ikraftträdandet.
\paragraph*{}
2018:772
\newline 1. Denna lag träder i kraft den 1 juli 2018 i fråga om 64 kap. 25 och 46 §§ och i övrigt den 1 november 2018.
\newline 2. Äldre bestämmelser gäller fortfarande i fråga om samarbetsavtal som Pensionsmyndigheten har ingått med fondförvaltare före den 1 november 2018.
\paragraph*{}
2018:1265
\newline 1. Denna lag träder i kraft den 1 januari 2019.
\newline 2. Ett beslut om vårdbidrag som har meddelats enligt äldre bestämmelser i 22 kap. ska fortsätta att gälla enligt vad som föreskrivs i beslutet, dock längst till dess att beslutet skulle ha upphört att gälla eller skulle ha omprövats om de äldre bestämmelserna i 22 kap. fortfarande hade varit tillämp- liga. Om ett sådant beslut inte har omprövats trots att det skulle ha gjorts under tid som de äldre bestämmelserna fortfarande gällde, ska beslutet fortsätta att gälla längst till den 31 december 2020, om det inte beslutas att det ska upphöra dessförinnan.
\paragraph*{}
Försäkringskassan får besluta att ett beslut om vårdbidrag som har meddelats enligt äldre bestämmelser i 22 kap. ska fortsätta att gälla även när
\newline 1. det har kommit in en ansökan om omvårdnadsbidrag eller merkostnadsersättning som ännu inte har prövats, och
\newline 2. det skulle ha funnits en rätt till vårdbidraget om de äldre bestämmelserna i 22 kap. fortfarande hade varit tillämpliga.
\paragraph*{}
Beslut enligt andra stycket gäller till dess att Försäkringskassan har prövat eller avskrivit en ansökan om omvårdnadsbidrag eller merkostnadsersättning, dock längst till den 1 juli 2022.
\paragraph*{}
Bestämmelserna i 22 kap. i den äldre lydelsen gäller fortfarande för vårdbidraget om ett beslut om förlängning har meddelats enligt andra stycket. Dock upphör förlängningsbeslutet att gälla om rätten till vårdbidraget ska omprövas enligt 22 kap. 17 § i den äldre lydelsen.
Lag (2020:427).
\newline 3. Äldre bestämmelser om vårdbidrag gäller fortfarande i ärenden där ansökan om vårdbidrag har gjorts före den 1 januari 2019.
\newline 4. Ett beslut om handikappersättning som har meddelats enligt äldre bestämmelser i 50 kap. ska fortsätta att gälla enligt vad som föreskrivs i beslutet, dock längst till dess att beslutet skulle ha upphört att gälla eller skulle ha omprövats om de äldre bestämmelserna i 50 kap. fortfarande hade varit tillämpliga.
\newline 5. Äldre bestämmelser om handikappersättning gäller fortfarande i de fall ett beslut om handikappersättning som har meddelats före ikraftträdandet upphör att gälla eller skulle ha omprövats före den 1 juli 2021 om de äldre bestämmelserna i 50 kap. fortfarande hade varit tillämpliga. I sådana fall får ett nytt beslut om handikappersättning meddelas enligt de äldre bestämmelserna för en tidsperiod om högst 18 månader.
\newline 6. Äldre bestämmelser om handikappersättning gäller fortfarande avseende ärenden där ansökan om handikappersättning har gjorts före den 1 januari 2019. Äldre bestämmelser gäller även avseende ärenden där rätten till handikappersättning skulle ha omprövats enligt de äldre bestämmelserna i 50 kap. 14 § 1 i de fall där det har fattats beslut om sjukersättning, aktivitetsersättning eller allmän ålderspension för tid före den 1 januari 2019. Äldre bestämmelser om handikappersättning gäller fortfarande också avseende ärenden om beslut om handikappersättning i fall som avses i 50 kap. 12 § andra meningen där hel sjukersättning, hel aktivitetsersättning eller hel ålderspension har beviljats för tid före den 1 januari 2019.
\newline 7. Äldre bestämmelser i 6 kap. 20 §, 11 kap. 16 §, 37 kap. 9 §, 51 kap. 6 §, 59 kap. 13 § och 106 kap. 9-11 §§ gäller fortfarande för vårdbidrag enligt 22 kap. i dess lydelse före den 1 januari 2019.
\newline 8. Äldre bestämmelser i 106 kap. 23 § och 110 kap. 6 § gäller fortfarande för handikappersättning enligt 50 kap. i dess lydelse före den 1 januari 2019.
\newline 9. Äldre bestämmelser i 110 kap. 30 § andra stycket och 57 § andra stycket 1 och 6 gäller fortfarande för vårdbidrag enligt 22 kap. och handikappersättning enligt 50 kap., i deras lydelse före den 1 januari 2019.
\paragraph*{}
2018:1627
\newline 1. Denna lag träder i kraft den 1 december 2018.
\newline 2. Den nya bestämmelsen i 5 kap. 17 a § och bestämmelserna i 5 kap. 18 §, 67 kap. 16 och 17 §§ och 81 kap. 9 § i den nya lydelsen tillämpas dock för tid från och med den 1 januari 2011.
\newline 3. Den som vid ikraftträdandet får garantipension eller garantipension till omställningspension utbetald till sig har fortsatt rätt att få förmånen även om han eller hon inte uppfyller kraven på försäkringstid i 67 eller 81 kap.
\newline 4. Vid beräkning av garantipension eller garantipension till omställningspension får förmånen på grund av tillämpningen av 67 kap. 16 eller 17 § eller 81 kap. 9 § inte beräknas till ett lägre belopp än det belopp med vilket förmånen tidigare har lämnats till den försäkrade. Förmånen får dock beräknas till ett lägre belopp om de belopp som ingår i beräkningsunderlaget ändras.
\newline 5. Lagen upphör att gälla vid utgången av 2019.
\paragraph*{}
2018:1628
\newline 1. Denna lag träder i kraft den 1 januari 2019.
\newline 2. Äldre bestämmelser gäller fortfarande för föräldrapenningsförmåner och flerbarnstillägg som avser tid före ikraftträdandet.
\paragraph*{}
2018:1952
\newline 1. Denna lag träder i kraft den 1 juli 2019.
\newline 2. Äldre föreskrifter gäller fortfarande för föräldrapenningsförmåner som avser dagar före ikraftträdandet.
\paragraph*{}
2019:644
\newline 1. Denna lag träder i kraft den 31 december 2019.
\newline 2. Lagen tillämpas för garantipension och garantipension till omställningspension som avser tid från och med den 1 januari 2020.
\newline 3. Äldre bestämmelser gäller fortfarande för garantipension och garantipension till omställningspension som avser tid före den 1 januari 2020.
\paragraph*{}
2019:645
\newline 1. Denna lag träder i kraft den 1 januari 2020.
\newline 2. Bestämmelserna tillämpas dock för tid från och med den 1 januari 2011.
\newline 3. Den som med stöd av punkten 3 i ikraftträdande- och övergångsbestämmelserna till lagen (2018:1627) om ändring i socialförsäkringsbalken hade fortsatt rätt att få garantipension eller garantipension till omställningspension, har fortsatt rätt att få förmånen även om han eller hon inte uppfyller kraven på försäkringstid i 67 eller 81 kap.
\newline 4. Vid beräkning av garantipension eller garantipension till omställningspension får förmånen på grund av tillämpningen av 67 kap. 16, 16 a, 17 eller 17 a § eller 81 kap. 9 eller 9 a § inte beräknas till ett lägre belopp än det belopp med vilket förmånen före den 1 december 2018 har lämnats till den försäkrade. Förmånen får dock beräknas till ett lägre belopp om de belopp som ingår i beräkningsunderlaget ändras.
\paragraph*{}
2019:646
\newline 1. Denna lag träder i kraft den 1 januari 2023.
Lag (2022:1036).
\newline 2. De upphävda paragraferna gäller dock fortfarande i fråga om garantipension och garantipension till omställningspension för tid före ikraftträdandet.
Lag (2022:1036).
\paragraph*{}
2019:649
\newline 1. Denna lag träder i kraft den 1 december 2019.
\newline 2. Lagen tillämpas första gången för förmåner som avser tid från och med den 1 januari 2020.
\newline 3. Äldre föreskrifter gäller fortfarande för den som uppnår respektive åldersgräns före den 1 januari 2020.
\paragraph*{}
2019:651
\newline 1. Denna lag träder i kraft den 1 december 2019.
\newline 2. De nya bestämmelserna tillämpas dock för tid från och med den 1 januari 2020.
\newline 3. Äldre bestämmelser gäller fortfarande för tid före utgången av 2019.
\paragraph*{}
2019:1294
\newline 1. Denna lag träder i kraft den 1 januari 2020.
\newline 2. Äldre föreskrifter gäller fortfarande för ansökan om efterlevandestöd som har kommit in till Pensionsmyndigheten före ikraftträdandet.
\paragraph*{}
2020:427
\newline 1. Denna lag träder i kraft den 1 juli 2020.
\newline 2. De nya bestämmelserna i 31 kap. 5 § första stycket 3 och andra stycket tillämpas inte när frånvaron från en arbetslivsinriktad rehabilitering avser tid före ikraftträdandet.
\newline 3. Den upphävda bestämmelsen i 27 kap. 55 a § gäller fortfarande när frånvaron från ett arbetsmarknadspolitiskt program avser tid före ikraftträdandet.
\paragraph*{}
2020:431
\newline 1. Denna lag träder i kraft den 1 juli 2020.
\newline 2. Äldre bestämmelser gäller fortfarande för bostadsbidrag som avser tid före ikraftträdandet.
\paragraph*{}
2020:432
\newline 1. Denna lag träder i kraft den 1 januari 2021.
\newline 2. Äldre bestämmelser gäller fortfarande för bostadsbidrag som avser tid före ikraftträdandet.
\paragraph*{}
2020:440
\newline 1. Denna lag träder i kraft den 1 juli 2020.
\newline 2. Äldre bestämmelser gäller fortfarande för assistansersättning som avser tid före ikraftträdandet.
\paragraph*{}
2020:509
\newline 1. Denna lag träder i kraft den 1 oktober 2020.
\newline 2. Äldre föreskrifter gäller fortfarande i ärenden i vilka en ansökan om bilstöd har kommit in till Försäkringskassan före ikraftträdandet.
\newline 3. Om en försäkrad har fått bilstöd för ett fordon med stöd av föreskrifterna i 52 kap. 8 § i den äldre lydelsen kan det till den försäkrade även fortsättningsvis lämnas bilstöd för det fordonet i fråga om åtgärder som avses i 52 kap. 8 § första stycket 2-4.
\paragraph*{}
2020:1239
\newline 1. Denna lag träder i kraft den 1 februari 2021.
\newline 2. De nya bestämmelserna tillämpas första gången för inkomstpensionstillägg som avser tid från och med den 1 september 2021.
\newline 3. Inkomstpensionstillägg ska lämnas utan ansökan till en försäkrad som har påbörjat uttag av inkomstgrundad ålderspension om förutsättningarna för inkomstpensionstillägg i 74 a kap. i övrigt är uppfyllda när lagen tillämpas första gången.
\newline 4. En ansökan om allmän inkomstpension, tilläggspension eller premiepension som har kommit in till Pensionsmyndigheten när lagen tillämpas första gången ska även anses omfatta en ansökan om inkomstpensionstillägg.
\paragraph*{}
2020:1271
\newline 1. Denna lag träder i kraft den 1 juni 2021.
\newline 2. Den nya bestämmelsen tillämpas första gången i fråga om underhållsstöd som avser juli 2021.
\paragraph*{}
2021:160
\newline 1. Denna lag träder i kraft den 15 mars 2021.
\newline 2. De nya bestämmelserna tillämpas även på sjukperioder som har påbörjats före ikraftträdandet. Bestämmelserna tillämpas dock första gången vid prövning av rätt till ersättning för dagar från och med ikraftträdandet.
\paragraph*{}
2021:585
\newline 1. Denna lag träder i kraft den 1 juli 2021.
\newline 2. Äldre bestämmelser gäller fortfarande för bostadsbidrag som avser tid före ikraftträdandet.
\paragraph*{}
2021:586
\newline 1. Denna lag träder i kraft den 1 januari 2022.
\newline 2. Äldre bestämmelser gäller fortfarande för bostadsbidrag som avser tid före ikraftträdandet.
\paragraph*{}
2021:767
\newline 1. Denna lag träder i kraft den 20 juli 2021.
\newline 2. För en utlänning som har beviljats uppehållstillstånd som övrig skyddsbehövande gäller 35 kap. 8 § och 67 kap. 7 § i den äldre lydelsen.
\paragraph*{}
2021:878
\newline 1. Denna lag träder i kraft den 1 november 2021.
\newline 2. Äldre bestämmelser gäller fortfarande för personlig assistans som har utförts före ikraftträdandet.
\paragraph*{}
2021:991
\newline 1. Denna lag träder i kraft den 1 januari 2022.
\newline 2. Bestämmelserna i 18 kap. 9 a § i den nya lydelsen tillämpas även på betalningar som har gjorts till Försäkringskassan från och med den 1 augusti 2021. Detta gäller dock inte betalningar som före ikraftträdandet har beaktats vid beslut om att lämna underhållsstöd på grund av särskilda skäl.
\newline 3. För betalningar som har gjorts till Försäkringskassan under minst sex månader i följd före ikraftträdandet gäller fortfarande 18 kap. 9 a § i den äldre lydelsen. Om Försäkringskassan efter ikraftträdandet beslutar att lämna underhållsstöd på grund av särskilda skäl i dessa fall, ska dock de nya bestämmelserna i 18 kap. 9 a § andra och tredje styckena tillämpas.
\paragraph*{}
2021:1240
\newline 1. Denna lag träder i kraft den 1 januari 2022 i fråga om 10 kap. 11 §, 12 kap. 25, 27 och 30 §§, 13 kap. 33 §, 25 kap. 5 §, 26 kap. 27 och 31 §§, 27 kap. 49 a § och 35 kap. 18 och 19 §§ och i övrigt den 1 februari 2022.
\newline 2. De nya bestämmelserna om sjukpenning i 27 och 28 kap. tillämpas även på sjukperioder som har påbörjats före ikraftträdandet. Bestämmelserna tillämpas dock första gången vid prövning av rätt till ersättning för dagar från och med ikraftträdandet.
\newline 3. Äldre bestämmelser gäller fortfarande i fråga om förmåner som avser tid före ikraftträdandet.
\newline 4. Den som för den 31 december 2022 har beviljats sjukpenning med stöd av 27 kap. 49 a § har rätt att efter ansökan få frågan om rätten till sjukpenning från och med den 1 januari 2023 prövad utan hinder av att frågan tidigare avgjorts av Försäkringskassan eller domstol genom beslut som fått laga kraft.
Lag (2022:1855).
\paragraph*{}
2021:1244
\newline 1. Denna lag träder i kraft den 1 januari 2022.
\newline 2. Äldre bestämmelser gäller fortfarande för förmåner som avser tid före ikraftträdandet.
\paragraph*{}
2021:1268
\newline 1. Denna lag träder i kraft den 1 juli 2022.
\newline 2. De nya bestämmelserna tillämpas första gången i fråga om underhållsstöd och betalningsskyldighet som avser augusti 2022.
\paragraph*{}
2022:761
\newline 1. Denna lag träder i kraft den 20 juni 2022.
\newline 2. Äldre bestämmelser gäller fortfarande för fondavtal som har ingåtts före ikraftträdandet. När ett sådant fondavtal sägs upp ska dock premiepensionsmedlen i en fond som omfattas av det avtalet omplaceras med tillämpning av 64 kap. 27 a och 27 b §§.
\newline 3. Vid tillämpningen av äldre bestämmelser om granskning i 64 kap. 3 § tredje stycket och om uttag av avgifter för att täcka kostnader för sådan granskning i 64 kap. 40 § första stycket andra strecksatsen ska det som föreskrivs om Pensionsmyndigheten i stället gälla Fondtorgsnämnden. Detta gäller dock inte sådana granskningar enligt 64 kap. 3 § tredje stycket som har inletts före ikraftträdandet.
\paragraph*{}
2022:878
\newline 1. Denna lag träder i kraft den 1 december 2022.
\newline 2. Lagen tillämpas första gången för förmåner som avser tid från och med den 1 januari 2023.
\newline 3. Bestämmelsen om inkomstindex i 58 kap. 11 § i den nya lydelsen tillämpas dock första gången vid beräkning av det inkomstindex som ska gälla från och med den 1 januari 2025.
\newline 4. Äldre föreskrifter gäller fortfarande för den som uppnår respektive åldersgräns före den 1 januari 2023.
\newline 5. Äldre föreskrifter om försäkringstid i 35 kap. 4, 12, 13 och 15 §§ gäller fortfarande för försäkringsfall som fastställs till tid före den 1 januari 2023.
\newline 6. Äldre föreskrifter om beräkningsunderlag för inkomstgrundad efterlevandepension gäller fortfarande för dödsfall som har inträffat före den 1 januari 2023.
\paragraph*{}
2022:879
\newline 1. Denna lag träder i kraft den 1 december 2025.
\newline 2. Lagen tillämpas första gången för förmåner som avser tid från och med den 1 januari 2026.
\newline 3. Bestämmelsen om inkomstindex i 58 kap. 11 § i den nya lydelsen tillämpas dock första gången vid beräkning av det inkomstindex som ska gälla från och med den 1 januari 2028.
\newline 4. Äldre föreskrifter gäller fortfarande för den som uppnår respektive åldersgräns före den 1 januari 2026.
\newline 5. Äldre föreskrifter om försäkringstid i 35 kap. 4, 12, 13 och 15 §§ gäller fortfarande för försäkringsfall som fastställs till tid före den 1 januari 2026.
\newline 6. Äldre föreskrifter om beräkningsunderlag för inkomstgrundad efterlevandepension gäller fortfarande för dödsfall som har inträffat före den 1 januari 2026.
\paragraph*{}
2022:934
\newline 1. Denna lag träder i kraft den 1 juli 2022.
\newline 2. Lagen tillämpas första gången för bostadstillägg som avser augusti 2022.
\newline 3. Äldre bestämmelser gäller fortfarande för bostadstillägg som avser tid före den 1 augusti 2022.
\paragraph*{}
2022:936
\newline 1. Denna lag träder i kraft den 1 december 2022.
\newline 2. Lagen tillämpas första gången för förmåner som avser tid från och med den 1 januari 2023.
\newline 3. Äldre föreskrifter gäller fortfarande för den som uppnår åldersgränsen före den 1 januari 2023.
\paragraph*{}
2022:937
\newline 1. Denna lag träder i kraft den 1 december 2025.
\newline 2. Lagen tillämpas första gången för förmåner som avser tid från och med den 1 januari 2026.
\newline 3. Äldre föreskrifter gäller fortfarande för den som uppnår åldersgränsen före den 1 januari 2026.
\paragraph*{}
2022:938
\newline 1. Denna lag träder i kraft den 1 januari 2023.
\newline 2. Äldre föreskrifter gäller fortfarande för insatser som har påbörjats före ikraftträdandet.
\paragraph*{}
2022:939
\newline 1. Denna lag träder i kraft den 1 september 2022.
\newline 2. Bestämmelsen i 27 kap. 49 b § i den nya lydelsen tillämpas även på sjukperioder som har påbörjats före ikraftträdandet. Bestämmelsen tillämpas dock första gången vid prövning av rätt till ersättning för dagar från och med ikraftträdandet.
\paragraph*{}
2022:1031
\newline 1. Denna lag träder i kraft den 15 juli 2022.
\newline 2. De nya bestämmelserna tillämpas dock för tid från och med den 1 augusti 2022.
\newline 3. Äldre bestämmelser gäller fortfarande för tid före utgången av juli 2022.
\paragraph*{}
2022:1037
\newline 1. Denna lag träder i kraft den 1 oktober 2022.
\newline 2. Äldre bestämmelser gäller dock fortfarande i fråga om garantipension och garantipension till omställningspension för tid före ikraftträdandet.
\paragraph*{}
2022:1041
\newline 1. Denna lag träder i kraft den 1 juli 2022.
\newline 2. Äldre bestämmelser gäller fortfarande för bostadsbidrag som avser tid före ikraftträdandet.
\paragraph*{}
2022:1042
\newline 1. Denna lag träder i kraft den 1 juli 2024. (Lag 2023:907).
\newline 2. Äldre bestämmelser gäller fortfarande för bostadsbidrag som avser tid före ikraftträdandet.
\paragraph*{}
2022:1222
\newline 1. Denna lag träder i kraft den 1 september 2022.
\newline 2. Äldre bestämmelser gäller fortfarande i fråga om sjukersättning som avser tid före ikraftträdandet.
\paragraph*{}
2022:1226
\newline 1. Denna lag träder i kraft den 1 januari 2023.
\newline 2. Äldre bestämmelser gäller fortfarande för assistansersättning som avser tid före ikraftträdandet.
\paragraph*{}
2022:1266
\newline 1. Denna lag träder i kraft den 2 december 2022.
\newline 2. Lagen tillämpas första gången för förmåner som avser tid från och med den 1 januari 2023.
\newline 3. Äldre bestämmelser i fråga om garantipension gäller fortfarande för den som är född 1957 eller tidigare.
\newline 4. Äldre bestämmelser i fråga om garantipension till omställningspension gäller fortfarande för dödsfall som inträffade före den 1 januari 2023.
\paragraph*{}
2022:1291
\newline 1. Denna lag träder i kraft den 2 augusti 2022.
\newline 2. Äldre föreskrifter gäller fortfarande för en anmälan om att avstå dagar med föräldrapenning på grundnivå som har gjorts före ikraftträdandet.
\paragraph*{}
2022:1854
\newline 1. Denna lag träder i kraft den 1 januari 2024.
\newline 2. Den upphävda paragrafen gäller dock fortfarande för sjukpenning som avser tid före utgången av 2023.
\paragraph*{}
2022:1856
\newline 1. Denna lag träder i kraft den 1 januari 2023.
\newline 2. Äldre bestämmelser gäller fortfarande för bostadstillägg som avser tid före utgången av 2022.
\paragraph*{}
2023:441
\newline 1. Denna lag träder i kraft den 1 juli 2023.
\newline 2. Äldre bestämmelser gäller fortfarande för bostadsbidrag som avser tid före ikraftträdandet.
\paragraph*{}
2023:905
\newline 1. Denna lag träder i kraft den 1 januari 2024 i fråga om 64 kap. 34 § och i övrigt den 1 juli 2024.
\newline 2. Äldre föreskrifter gäller fortfarande för föräldrapenning som avser tid före ikraftträdandet.
\printindex
\end{document}
